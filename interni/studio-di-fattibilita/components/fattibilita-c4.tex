\documentclass[../studio-di-fattibilita.tex]{subfiles}
\appendToGraphicspath{../../commons/img/}

\begin{document}
	\subsection{Informazioni generali}
	\label{subsec:informazioni_generali}
	\begin{description}
		\item[Nome] C4-Predire in Grafana
		\item[Proponente] Zucchetti S.p.A.
		\item[Committente] Tullio Vardanega
	\end{description}
	% subsection informazioni_generali (end)


	\subsection{Descrizione}
	\label{subsec:descrizione}
	Nello scenario Dev-Ops, Zucchetti S.p.A. utilizza Grafana come strumento di monitoraggio di sistemi.
	Grafana è un prodotto open-source che possiede diversi elementi, quali sistemi di presentazione del dato e sistemi di allarme, raccolti in dashboard che fanno uso di grafici per rappresentare il flusso di dati analizzati.
	% subsection descrizione (end)


	\subsection{Finalità del progetto}
	\label{subsec:finalita_del_progetto}
	Si vuole realizzare un plugin scritto in Javascript che effettui previsioni sul flusso di dati raccolti per monitorare lo stato del sistema e, conseguentemente, segnalare agli operatori eventuali problemi, o zone, in cui è richiesto un intervento. La Zucchetti propone come soluzione un algoritmo predittivo di apprendimento automatico, il Support Vector Machine (SVM), e un metodo di stima dell'andamento di un insieme di dati, la regressione lineare (RL).

	Nello specifico sono richiesti i seguenti compiti:
	\begin{itemize}
		\item Produrre un file JSON con i parametri per le previsioni che dovrà essere utilizzato dalla metodologia scelta;
		\item leggere il predittore dal file json;
		\item associare i predittori al flusso dati di Grafana;
		\item applicare la previsione e rendere disponibili i dati al sistema Grafana;
		\item rendere consultabili i dati tramite una dashboard contenente grafici esplicativi;
	\end{itemize}
	% subsection finalità_del_progetto (end)

	
	\subsection{Tecnologie e Metodologie di sviluppo interessate}
	\label{subsec:tecnologie_interessate}
	\begin{description}
		\item [Grafana]: il plug-in è sviluppato per la piattaforma Grafana;
		\item [Linguaggio Javascript]: è richiesta la realizzazione di un plug-in che possa integrarsi con la piattaforma Grafana, stata sviluppata in Javascript;
		\item [Librerie Javascript]: sono necessarie per l'implementazione dei modelli di classificazione e regressione (SVM e RL);
		\item [Orange Canvas]: è lo strumento di analisi dei dati consigliato dal proponente.
	\end{description}
	% subsection tecnologie_interessate (end)


	\subsection{Aspetti positivi}
	\label{subsec:aspetti_positivi}
	\begin{itemize}
		\item Le librerie Javascript e la documentazione per SVM e RL sono direttamente fornite dal proponente;
		\item gli algoritmi proposti sono interessanti dal punto di vista formativo;
		\item la possibilità di poter implementare dei meccanismi di apprendimento di flusso in costante adattamento con i dati osservati;
		\item la facilità nell'esecuzione dei test di verifica.
	\end{itemize}
	% subsection aspetti_positivi (end)


	\subsection{Criticità e fattori di rischio}
	\label{subsec:criticita_e_fattori_di_rischio}
	\begin{itemize}
		\item Si devono utilizzare i propri dati. Zucchetti non consente l'uso dei dati di Grafana.
		\item Il gruppo non reputa interessante lo sviluppo di un plug-in Javascript.
	\end{itemize}
	% subsection criticità_e_fattori_di_rischio (end)


	\subsection{Conclusioni}
	\label{subsec:conclusioni}
	Il progetto non richiede un oneroso carico di lavoro in termini di ore di apprendimento e sviluppo per il soddisfacimento dei requisiti obbligatori. È invece richiesto un impegno più consistente nel caso in cui si vogliano soddisfare tutti i requisiti opzionali proposti.
	Il gruppo ha considerato il capitolato come terza scelta.
	% subsection conclusioni (end)

\end{document}