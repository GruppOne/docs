\documentclass[../studio-di-fattibilita.tex]{subfiles}
\appendToGraphicspath{../../commons/img/}

\begin{document}
  \subsection{Informazioni generali}%
  \label{subsec:informazioni_generali}
  \begin{description}
    \item[Nome] C4-Predire in Grafana
    \item[Proponente] Zucchetti S.p.A.
    \item[Committente] prof. Vardanega Tullio, prof. Cardin Riccardo
  \end{description}
  % subsection informazioni_generali (end)


  \subsection{Descrizione}%
  \label{subsec:descrizione}
  Nello scenario Dev Ops, Zucchetti S.p.A. utilizza Grafana come strumento di monitoraggio di sistemi.
  Grafana è un prodotto open-source composto da diversi elementi, ad esempio sistemi di presentazione del dato e sistemi di allarme, raccolti in dashboard che fanno uso di grafici per rappresentare il flusso di dati analizzati.
  % subsection descrizione (end)


  \subsection{Finalità del progetto}%
  \label{subsec:finalita_del_progetto}
  Il software da realizzare consiste nello sviluppo di un plugin, scritto in JavaScript, che permetta di effettuare previsioni sul flusso di dati raccolti,
  per monitorare lo stato del sistema e, conseguentemente, segnalare agli operatori eventuali problemi in cui è richiesto un intervento.
  La soluzione proposta è un metodo predittivo basato sull'apprendimento automatico, il \textit{Support Vector Machine} (SVM),
   e un metodo di stima dell'andamento di un insieme di dati, la \textit{regressione lineare} (RL).

  Nello specifico sono richieste le seguenti funzionalità:
  \begin{itemize}
    \item prelevare i parametri d'interesse per le previsioni, in modo da produrre un file JSON che dovrà essere utilizzato dalla metodologia scelta
    \item leggere il predittore dal file json
    \item associare i predittori al flusso dati di Grafana
    \item applicare la previsione e rendere disponibili i dati al sistema Grafana
    \item rendere consultabili i dati tramite una dashboard contenente grafici esplicativi
  \end{itemize}
  % subsection finalità_del_progetto (end)


  \subsection{Tecnologie e Metodologie di sviluppo interessate}%
  \label{subsec:tecnologie_interessate}
  \begin{description}
    \item [Grafana]: il plug-in è sviluppato per la piattaforma Grafana
    \item [Linguaggio Javascript]: il plug-in deve essere realizzato con lo stesso linguaggio di Grafana, ovvero in Javascript, per facilità d'integrazione
    \item [Librerie Javascript]: fornite dal proponente, sono necessarie per l'implementazione dei modelli di classificazione e regressione (SVM e RL)
    \item [Orange Canvas]: è lo strumento di analisi dei dati consigliato dal proponente.
  \end{description}
  % subsection tecnologie_interessate (end)


  \subsection{Aspetti positivi}%
  \label{subsec:aspetti_positivi}
  \begin{itemize}
    \item Le librerie Javascript e la documentazione per SVM e RL sono direttamente fornite dal proponente
    \item gli algoritmi proposti sono interessanti dal punto di vista formativo
    \item la possibilità di implementare dei meccanismi di apprendimento di flusso in costante adattamento con i dati osservati
    \item la facilità nell'esecuzione dei test di verifica.
  \end{itemize}
  % subsection aspetti_positivi (end)


  \subsection{Criticità e fattori di rischio}%
  \label{subsec:criticita_e_fattori_di_rischio}
  \begin{itemize}
    \item Si devono utilizzare i propri dati. Zucchetti non consente l'uso dei dati di Grafana
    \item il gruppo non ha valutato in modo positivo la possibilità di sviluppare un plug-in Javascript.
  \end{itemize}
  % subsection criticità_e_fattori_di_rischio (end)


  \subsection{Conclusioni}%
  \label{subsec:conclusioni}
  Il progetto non richiede un oneroso carico di lavoro in termini di ore di apprendimento e sviluppo per il soddisfacimento dei requisiti obbligatori. È invece richiesto un impegno più consistente nel caso in cui si vogliano soddisfare tutti i requisiti opzionali proposti.
  Nonostante il progetto non abbia suscitato interesse nel gruppo, il capitolato è stato considerato come terza scelta per il carico di lavoro minore rispetto agli altri capitolati proposti.
  % subsection conclusioni (end)

\end{document}
