\documentclass[../studio-di-fattibilita.tex]{subfiles}
\appendToGraphicspath{../../commons/img/}

\begin{document}
  \subsection{Informazioni generali}%
  \label{subsec:informazioni_generali}
  \begin{description}
    \item[Nome] ThiReMa
    \item[Proponente] NextBI
    \item[Committente] prof. Tullio Vardanega, prof. Riccardo Cardin
  \end{description}
  % subsection informazioni_generali (end)

  \subsection{Descrizione}%
  \label{subsec:descrizione}
  \textit{ThiReMa} si prefigge la creazione di un'applicazione in grado di ricevere misurazioni da sensori eterogenei collocati in aree geografiche diverse, e di accumulare le suddette misurazioni in modo efficiente ed efficace in un database centralizzato. Queste informazioni devono poi essere inoltrate in modo tempestivo nel momento in cui queste ultime comportino azioni da intraprendere da parte degli utenti del software.
  % subsection descrizione (end)

  \subsection{Finalità del progetto}%
  \label{subsec:finalita_del_progetto}
  Il software da realizzare consiste di una web-application che permetta di valutare la correlazione tra dati operativi (le misurazioni) e fattori influenzanti e di effettuare delle previsioni sull'andamento dei dati tramite la definizione di uno o più algoritmi.
  Per ogni tipologia di informazioni rilevate, dovrà essere possibile assegnare il monitoraggio ad un particolare ente.
  La web-application deve essere suddivisa in tre macro-sezioni:
  \begin{itemize}
    \item censimento dei sensori e dei relativi dati;
    \item modulo di analisi di correlazione;
    \item modulo di monitoraggio per ente.
  \end{itemize}
  % subsection finalità_del_progetto (end)


  \subsection{Tecnologie e Metodologie di sviluppo interessate}%
  \label{subsec:tecnologie_interessate}
  I vincoli indicati dal proponente riguardo alle tecnologie da utilizzare sono:
  \begin{description}
    \item[Telegram] come servizio di dispatching, per inoltrare in modo tempestivo le informazioni utili in caso di emergenza;
    \item[Apache Kafka] come cluster che fa da tramite per l'interazione tra sensori e database;
    \item[Java] per realizzare i componenti custom \textit{Kafka} e lo sviluppo della business logic.
  \end{description}
  Le implementazioni database suggerite sono \textbf{PostgreSQL}, \textbf{TimescaleDB} e \textbf{ClickHouse}.
  Non è necessario che misurazioni e metadati risiedano nello stesso database.
  Altre tecnologie consigliate sono:
  \begin{description}
    \item[Bootstrap] per lo styling del front-end;
    \item[Docker] per l'istanziazione di tutti i componenti.
  \end{description}
  % subsection tecnologie_interessate (end)


  \subsection{Aspetti positivi}%
  \label{subsec:aspetti_positivi}
  \begin{itemize}
    \item Aquisire competenze in ambito Internet of Things(IoT) e Big Data.
  \end{itemize}
  % subsection aspetti_positivi (end)


  \subsection{Criticità e fattori di rischio}%
  \label{subsec:criticita_e_fattori_di_rischio}
  \begin{itemize}
    \item Complessità della mole di dati da gestire.
  \end{itemize}
  % subsection criticità_e_fattori_di_rischio (end)


  \subsection{Conclusioni}%
  \label{subsec:conclusioni}
  Il progetto ha suscitato interesse all'interno del gruppo, ma data la forte concorrenza nell'aggiudicazione del capitolato abbiamo deciso di optare per altre proposte.
  % subsection conclusioni (end)

\end{document}