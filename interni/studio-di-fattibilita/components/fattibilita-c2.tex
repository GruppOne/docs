\documentclass[../studio-di-fattibilita.tex]{subfiles}
\appendToGraphicspath{../../commons/img/}

\begin{document}
  \subsection{Informazioni generali}%
  \label{subsec:informazioni_generali}
  \begin{description}
    \item[Nome] Etherless
    \item[Proponente] Red Babel
    \item[Committente] prof. Tullio Vardanega, prof. Riccardo Cardin
  \end{description}
  % subsection informazioni_generali (end)
  \subsection{Descrizione}%
  \label{subsec:descrizione}
  \textit{Etherless} si pone come servizio di calcolo distribuito per funzioni JavaScript, che permetta allo sviluppatore di mettere a disposizione una funzione a pagamento, e all'utente di usufruirne da remoto. È inoltre in grado di automatizzare il processo di pagamento della prestazione e di far pagare all'utente soltanto l'effettivo utilizzo del servizio.
  % subsection descrizione (end)
  \subsection{Finalità del progetto}%
  \label{subsec:finalita_del_progetto}
  Il software da realizzare consisterà in un pacchetto di \textit{Node.js}, installabile tramite \textit{npm} (\textit{node package manager ndr}), utilizzabile tramite interfaccia a riga di comando.
  Il programma deve offrire le seguenti funzionalità:
  \begin{itemize}
    \item inizializzare \textit{Etherless} sulla propria macchina;
    \item configurare il proprio account \textit{Ethereum};
    \item caricare le funzioni javascript da eseguire remotamente;
    \item mostrare la lista delle funzioni a propria disposizione;
    \item utilizzare le funzioni messe a disposizione nel network;
    \item rimuovere le funzioni precendemente caricate.
  \end{itemize}
  Per il server, il requisito più forte è la sua scalabilità, con un accento sulla scelta fra scalabilità orizzontale e verticale, in base al numero di utenti attivi.
  % subsection finalità_del_progetto (end)
  \subsection{Tecnologie e Metodologie di sviluppo interessate}%
  \label{subsec:tecnologie_interessate}
    L'utilizzo di una funzione deve essere automaticamente tracciato e bisogna gestire il corrispondente pagamento in Ethereum.
  \begin{description}
    \item[Blockchain] Ethereum;
    \item[Serverless Tecnology] per la gestione e l'esecuzione delle funzioni;
    \item[Typescript 3.6] per le comunicazioni tra utente e network;
    \item[ESlint] per l'analisi statica del codice;
    \item[Serverless Framework] per l'integrazione Serverless.
  \end{description}
  % subection tecnologie_interessate (end)
  \subsection{Aspetti positivi}%
  \label{subsec:aspetti_positivi}
  \begin{itemize}
    \item Il contesto tecnologico proposto è di grande interesse: in particolare, la tecnologia serverless.
  \end{itemize}
  % subsection aspetti_positivi (end)
  \subsection{Criticità e fattori di rischio}%
  \label{subsec:criticita_e_fattori_di_rischio}
  \begin{itemize}
    \item Il capitolato è di difficile comprensione data la particolarità degli argomenti, questo rende difficile un'analisi precisa:
    \item l'ambito di formazione tecnologico, per quanto interessante, è limitato al contesto delle tecnologie di RedBabel;
    \item la particolarità dell'argomento trattato richiede molta formazione specifica da parte dei partecipanti.
  \end{itemize}
  % subsection criticità_e_fattori_di_rischio (end)
  \subsection{Conclusioni}%
  \label{subsec:conclusioni}
  Il progetto non ha incontrato l'entusiasmo dei partecipanti per via della specificità degli argomenti trattati, unita allo scarso interesse per le tecnologie trattate, questo ha portato il gruppo a orientarsi verso altre scelte.
  % subsection conclusioni (end)
\end{document}