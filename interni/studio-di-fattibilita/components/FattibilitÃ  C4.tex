\documentclass[../studio-di-fattibilita.tex]{subfiles}
\title{C4-Predire in Grafana}
\author{GruppOne}
\date{\today}

\begin{document}
	\pagenumbering{arabic}
	\subsection{Informazioni generali}
	\label{sec:informazioni_generali}
	\begin{description}
		\item[Nome] C4-Predire in Grafana
		\item[Proponente] Zucchetti s.p.a.
		\item[Committente] Tullio Vardanega
	\end{description}
	% section informazioni_generali (end)
	\subsection{Descrizione}
	\label{sec:descrizione}
	Nello scenario Dev-Ops la Zucchetti s.p.a. utilizza Grafana come strumento di monitoraggio di sistemi. Esso è un prodotto open-source che possiede diversi elementi quali sistemi di presentazione del dato e sistemi di allarme raccolti in dashboard che fanno uso di grafici per rappresentare il flusso di dati analizzati.
	
	% section descrizione (end)
	\subsection{Finalita del progetto}
	\label{sec:finalita_del_progetto}
	Si vuole realizzare un plugin scritto in Javascript che effettui previsioni sul flusso di dati raccolti per monitorare lo stato del sistema e conseguentemente segnalare agli operatori eventuali problemi o zone in cui è richiesto un intervento. La Zucchetti propone come soluzione un algoritmo predittivo di apprendimento automatico che è il Support Vector Machine (SVM) e un metodo di stima dell'andamento di un insieme di dati che è la regressione lineare(RL).
	\\
	Nello specifico sono richiesti i seguenti compiti:
	\begin{itemize}
		\item Produrre un file json con i parametri per le previsioni che dovrà essere utilizzato dalla metodologia scelta.
		\item Leggere il predittore dal file json.
		\item Associare i predittori al flusso dati di Grafana.
		\item Applicare la previsione e rendere disponibili i dati al sistema Grafana.
		\item Rendere consultabili i dati tramite una dashboard contenente grafici esplicativi.
	\end{itemize}
	
	% section finalità_del_progetto (end)
	\subsection{Tecnologie interessate}
	\label{sec:tecnologie_interessate}
	\begin{description}
		\item [Grafana]: il plug-in è sviluppato per la piattaforma Grafana.
		\item[Linguaggio Javascript]: è richiesta la realizzazione di un plug-in che possa integrarsi con la piattaforma Grafana che è stata sviluppata in Javascript.
		\item[Librerie Javascript]: sono necessarie per l'implementazione dei modelli di classificazione e regressione (SVM e RL).
		\item[Orange Canvas]: è lo strumento di analisi dei dati consigliato dal proponente.
	\end{description}
	% section tecnologie_interessate (end)
	\subsection{Aspetti positivi}
	\label{sec:aspetti_positivi}
	\begin{itemize}
		\item Le librerie Javascript e la documentazione per SVM e RL sono direttamente fornite dal proponente.
		\item Gli algoritmi proposti sono interessanti dal punto di vista formativo.
		\item la possibilità di poter implementare dei meccanismi di apprendimento di flusso in costante adattamento con i dati osservati.
		\item La facilità nell'esecuzione dei test di verifica.
	\end{itemize}
	% section aspetti_positivi (end)
	\subsection{Criticità e fattori di rischio}
	\label{sec:criticita_e_fattori_di_rischio}
	\begin{itemize}
		\item Si devono utilizzare i propri dati. Zucchetti non consente l'uso dei dati di Grafana.
		\item Il gruppo non reputa interessante lo sviluppo di un plug-in Javascript.
	\end{itemize}
	% section criticità_e_fattori_di_rischio (end)
	\subsection{Conclusioni}
	\label{sec:conclusioni}
	Il progetto non richiede un oneroso carico di lavoro in termini di ore di apprendimento e sviluppo per il soddisfacimento dei requisiti obbligatori. È invece richiesto un impegno più consistente nel caso in cui si vogliano soddisfare tutti i requisiti opzionali proposti. Il gruppo ha considerato il capitolato come terza scelta.
	
	% section conclusioni (end)
\end{document}