\documentclass[../studio-di-fattibilita.tex]{subfiles}
\appendToGraphicspath{../../../commons/img/}

\begin{document}
\subsection{Informazioni generali}%
\label{subsec:informazioni_generali}
\begin{description}
  \item[Nome] Autonomous Highlights Platform
  \item[Proponente] Zero12
  \item[Committente] prof. Tullio Vardanega, prof. Riccardo Cardin
\end{description}
% subsection informazioni_generali (end)


\subsection{Descrizione}%
\label{subsec:Descrizione}
\textit{Autonomous Highlights Platform} si prefigge l'obiettivo di creare una piattaforma web che riceva in input video di eventi sportivi, e crei autonomamente in output un video della durata massima di 5 minuti.
% subsection descrizione (end)


\subsection{Finalità del progetto}%
\label{subsec:finalita_del_progetto}
Il proponente chiede che la piattaforma, al fine di soddisfare i requisiti, sia dotata di un modello di machine learning in grado di estrapolare i momenti salienti di ogni singolo video. L'ordine del flusso di generazione dell'output sarà:
\begin{itemize}
  \item caricamento del video
  \item identificazione dei momenti salienti
  \item estrazione delle corrispondenti parti di video
  \item generazione del video di sintesi.
\end{itemize}
% subsection finalità_del_progetto (end)


\subsection{Tecnologie e Metodologie di sviluppo interessate}%
\label{subsec:tecnologie_interessate}
Il proponente vincola il fornitore all'utilizzo del software Sage Maker e consiglia l'utilizzo della tecnologia di Amazon Web Services, in particolare i servizi:
\begin{description}
  \item[Elastic Container Service o Elastic Kubernetes Service] servizio di orchestrazione di contenitori altamente dimensionabile ad elevate prestazioni
  \item[DynamoDB] Database NoSQL
  \item[AWS Transcode] per la conversione ed elaborazione di diversi formati video
  \item[Sage Maker] per l'apprendimento automatico (\textit{obbligatorio il suo utilizzo})
  \item[AWS Rekognition video] per l'analisi video basata sull'apprendimento approfondito.
\end{description}
I linguaggi di programmazione raccomandati sono:
\begin{description}
  \item[\glossario{NodeJS}] per lo sviluppo di API Restful JSON
  \item[\glossario{Python}] per lo sviluppo delle componenti di Machine Learning.
\end{description}
Inoltre è richiesto che il caricamento dei video da elaborare avvenga tramite riga di comando, e che venga integrata una console web di analisi e controllo degli stati di elaborazione dei video.
% subsection tecnologie_interessate (end)


\subsection{Aspetti positivi}%
\label{subsec:aspetti_positivi}
\begin{itemize}
  \item L'utilizzo di un modello di machine learning, che permette di identificare ogni momento importante dell’evento sportivo
  \item il fornitore può scegliere liberamente lo sport: il proponente non vincola in alcun modo questa scelta.
\end{itemize}
% subsection aspetti_positivi (end)


\subsection{Criticità e fattori di rischio}%
\label{sec:criticita_e_fattori_di_rischio}
\begin{itemize}
  \item È necessario lo studio approfondito di uno specifico sport e l'osservazione attenta di tutti i momenti principali di un evento sportivo. Questo tipo di attività è costosa in termini di tempo.
\end{itemize}
% subsection criticità_e_fattori_di_rischio (end)


\subsection{Conclusioni}%
\label{subsec:conclusioni}
Il progetto non ha suscitato molto entusiasmo all'interno del gruppo, in quanto il tempo da dedicare all'osservazione degli eventi sportivi ed alla segnalazione dei momenti salienti sarebbe maggiore di quello richiesto alla realizzazione stessa del progetto. Inoltre lo sport non è molto interessante per la maggior parte dei membri del gruppo, dunque si è deciso di non considerare questa proposta.
% subsection conclusioni (end)

\end{document}