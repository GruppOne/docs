\documentclass[../studio-di-fattibilita.tex]{subfiles}
\appendToGraphicspath{../../commons/img/}

\begin{document}
  \pagenumbering{arabic}
  \subsection{Informazioni generali}
  \label{subsec:informazioni_generali}
  \begin{description}
    \item[Nome] NaturalAPI
    \item[Proponente] Teal Blue
    \item[Committente] Prof. Tullio Vardanega e Prof. Riccardo Cardin.
  \end{description}
  % subsection informazioni_generali (end)


  \subsection{Descrizione}
  \label{subsec:descrizione}
  \textit{NaturalAPI} mira allo sviluppo di un \textit{proof of concept} toolkit per ridurre il divario tra le specifiche di progetto e le API, consentendo alla prossima generazione di sviluppatori di scrivere API che siano più coerenti, più prevedibili e più mantenibili, permettendo ai programmatori di concentrarsi sullo sviluppo di funzionalità piuttosto che cercare di replicare il modello aziendale.
  % subsection descrizione (end)


  \subsection{Finalità del progetto}
  \label{subsec:finalita_del_progetto}
  Il toolkit deve generare API complete e test automatizzati sfruttando l'elaborazione del linguaggio naturale.
  Grazie al suo linguaggio specifico di dominio altamente specializzato, \textit{NaturalAPI} deve essere in grado di produrre interfacce di programmazione complete e facilmente gestibili, con la relativa integrazione e test unitari, riguardanti frameworks e linguaggi di programmazione popolari.
  
  Piuttosto che fare affidamento su espressioni regolari o tecniche di analisi delle stringhe simili, \textit{NaturalAPI} deve affidarsi a tecniche e attività avanzate di elaborazione del linguaggio naturale (NLP) e conoscenza del dominio aziendale per:
    \begin{enumerate}
      \item Trovare combinazioni di verbi e sostantivi (\textit{predicati ndr}) nel linguaggio naturale \textit{Gherkin};
      \item normalizzare e convertire i predicati in funzioni libere o metodi con i loro argomenti;
      \item trovare argomenti di funzioni ricorrenti e generare oggetti corrispondenti e proprietà.
    \end{enumerate}
  % subsection finalità_del_progetto (end)


  \subsection{Tecnologie e Metodologie di sviluppo interessate}
  \label{subsec:tecnologie_interessate}
  \begin{description}
		\item[Natural Language Processing]: il toolkit utilizza tecniche e attività avanzate di elaborazione del linguaggio naturale;
		\item[Comma Separated Values]: è richiesto l'output dei vocaboli trovati in formato CSV;
		\item[OpenApi JSON]: sono necessarie per l'output di NaturalAPI Design.
  \end{description}
  % subsection tecnologie_interessate (end)


  \subsection{Aspetti positivi}
  \label{subsec:aspetti_positivi}
  \begin{itemize}
    \item Il capitolato è molto dettagliato e in particolare fornisce esplicitamente dei requisiti facilmente traducibili (attraverso Gherkin) in test di accettazione. Sarà quindi più semplice misurare e documentare la conformità del prodotto.
    \item Considerando che non è necessario fornire un'interfaccia grafica, possiamo permetterci di utilizzare relativamente più risorse per sviluppare le funzionalità richieste, compensandone la maggiore difficoltà.
  \end{itemize}
  % subsection aspetti_positivi (end)


  \subsection{Criticità e fattori di rischio}
	\label{subsec:criticita_e_fattori_di_rischio}
	\begin{itemize}
		\item Il gruppo reputa complesso l'argomento trattato, in quanto richiede molto autoapprendimento da parte dei partecipanti.
	\end{itemize}
  % subsection criticità_e_fattori_di_rischio (end)
  

	\subsection{Conclusioni}
	\label{subsec:conclusioni}
  Il progetto richiede un oneroso carico di lavoro in termini di ore di apprendimento e sviluppo per il soddisfacimento dei requisiti obbligatori, ciò nonostante la possibilità di mettersi in gioco è riuscita a stimolare l'interesse del gruppo, che ha considerato il capitolato come seconda scelta.
  % subsection conclusioni (end)
  
\end{document}