\documentclass[../../norme-di-progetto.tex]{subfiles}

\begin{document}

\subsubsection{Finalità}%
\label{subs:fornitura/finalita}

GruppOne istanzia il processo di fornitura per potersi aggiudicare il capitolato attraverso un accordo con il committente che ha valore contrattuale.

\subsubsection{Descrizione}%
\label{subs:fornitura/descrizione}

Il processo di fornitura concerne il rapporto tra il fornitore e il cliente, e l'amministrazione delle procedure e risorse necessarie per lo sviluppo del \textit{Piano di progetto}.

È composto da diverse attività:

\begin{itemize}
  \item Inizializzazione
  \item Preparazione della risposta
  \item Contratto
  \item Pianificazione
  \item Esecuzione e controllo
  \item Revisione
  \item Consegna e completamento.
\end{itemize}

\subsubsection{Attività}%
\label{subs:attivita}

\paragraph{Inizializzazione}%
\label{par:inizializzazione}

Il team effettua delle attente valutazioni per ogni capitolato  e si candida presso il committente per realizzare quello che reputa migliore.

\subparagraph{Studio di Fattibilità}%
\label{subp:studio_di_fattibilita}

GruppOne si impegna a consegnare un documento sintetico contenente una descrizione, i pregi e le criticità riscontrati in ciascuno dei capitolati.

Lo \textit{Studio di fattibilità} è redatto dall'\glossario{amministratore} con l'aiuto degli \glossario{analisti} e ha l'obiettivo di esporre quali capitolati sono presi in considerazione e le relative motivazioni.

Il documento è così articolato:
\begin{description}
  \item [Descrizione] una breve descrizione del problema esposto dal proponente
  \item [Finalità del progetto] che cosa bisogna realizzare
  \item [Tecnologie interessate] le tecnologie da considerare in fase di sviluppo
  \item [Aspetti positivi] i pregi del capitolato
  \item [Criticità e fattori di rischio] difficoltà che potremmo incontrare.
\end{description}

\paragraph{Preparazione della risposta}%
\label{subs:preparazione_della_risposta}

GruppOne si occupa di fornire al committente una proposta di candidatura per il capitolato che possa soddisfare le sue richieste.

\paragraph{Contratto}%
\label{par:contratto}

GruppOne stipula un contratto con il committente per la realizzazione di un prodotto software.

\paragraph{Pianificazione}%
\label{par:pianificazione}

Il team prepara una pianificazione che si occuperà di:
\begin{itemize}
  \item Scegliere il modello di sviluppo che meglio si adatta al progetto.
  \item Organizzare le attività dei processi secondo il modello di sviluppo scelto.
  \item Definire degli obiettivi di qualità che il prodotto dovrà obbligatoriamente soddisfare.
\end{itemize}

\subparagraph{Piano di progetto}%
\label{subp:piano_di_progetto}
Il piano di progetto è redatto dal responsabile e contiene l'organizzazione dell'attività allo scopo di raggiungere l'economicità.
Esso rappresenta, anche mediante l'utilizzo di diagrammi, le risorse fondamentali a disposizione e come andrebbero impiegate per organizzare un efficiente pianificazione delle attività.
Il documento è così strutturato:
\begin{description}
  \item [Introduzione] si presentano lo scopo e la struttura del documento.
  \item [Analisi dei rischi] si realizza un'indagine per scoprire quali sono i maggiori rischi con cui il gruppo deve confrontarsi nel corso della realizzazione del progetto. Si illustra la gravità di ogni rischio, i danni che ognuno potrebbe arrecare, ed eventuali contromisure per riuscire a contrastarli.
  \item [Pianificazione] si stabiliscono quali sono le risorse temporali e umane a disposizione e si decide come suddividerle nelle diverse attività.
  \item [Preventivo e consuntivo] Il preventivo presenta una stima dei costi totali di progetto, mentre il consuntivo illustra un bilancio totale delle attività di processo compiute in un determinato periodo di tempo.
\end{description}

\subparagraph{Piano di qualifica}%
\label{subp:piano_di_qualifica}
Il \textit{Piano di qualifica} è redatto dai progettisti nella sua parte programmatica e dai verificatori nella parte che illustra le verifiche effettuate.
I suoi scopi principali sono:

\begin{itemize}
  \item Definire gli attributi di qualità che il nostro prodotto deve avere, e dettagliare le specifiche dei test di accettazione, sistema, integrazione e unità.
  \item Definire quali sono i valori minimi e quelli ottimali delle metriche riportate nelle \textit{Norme di progetto} per ciascuno dei processi istanziati da GruppOne che sono soggetti al sistema di \glossario{qualità}.
  \item Documentare le strategie e procedure attraverso cui intendiamo perseguire gli obiettivi definiti nei punti precedenti.
  \item Analizzare in retrospettiva i risultati delle prove e verifiche effettuate in accordo con quanto definito.
\end{itemize}

La struttura del documento vuole riflettere le due visioni perpendicolari del sistema qualità.

\begin{description}
  \item [Introduzione] esplora più in dettaglio lo scopo del documento.
  \item [Qualità di prodotto] riguarda la visione verticale e specifica a ciò che stiamo realizzando.
  \item [Qualità di processo] esplicita la visione orizzontale trasversale a processi, attività e compiti istanziati dal team.
        % \item [Report su attività di verifica] contiene brevi riassunti delle attività di verifica
\end{description}

Il \textit{Piano di qualifica} è redatto dai progettisti nella sua parte programmatica e dai verificatori nella parte che illustra le verifiche effettuate. Esso descrive la strategia complessiva di verifica e validazione adottata da GruppOne per garantire qualità ai processi e ai prodotti. Il documento è così strutturato:
\begin{description}
  \item [Introduzione]: si presentano lo scopo del prodotto e del documento.
  \item [Qualità di prodotto]: si delineano gli attributi di qualità che il nostro prodotto deve avere. Per ognuno degli attributi si indicano le metriche definite nelle \textit{Norme di progetto} per quantificare la qualità e si illustrano dei valori soglia da rispettare. Si definiscono inoltre i test di unità, integrazione, sistema e accettazione.
  \item [Qualità di processo]: per ogni processo istanziato nelle \textit{Norme di progetto} si presentano le metriche per misurare la qualità e si descrivono i valori ottimali e minimali.
\end{description}

\paragraph{Esecuzione e controllo}%
\label{par:esecuzione e controllo}

GruppOne si impegna a sviluppare e consegnare il prodotto nei tempi previsti nella pianificazione.

\paragraph{Revisione}%
\label{par:revisione}

Il team organizza riunioni con l'acquirente per chiedere chiarimenti e informarlo sullo stato del progetto.

\subparagraph{Incontri con il proponente}%
\label{subp:incontri_con_il_proponente}

GruppOne intende mantenere uno stretto rapporto di collaborazione con i proponenti del capitolato Stalker.
Tale rapporto si mantiene attraverso incontri che si svolgono fisicamente presso le aule del Dipartimento di Matematica ``Tullio Levi Civita'' e virtualmente utilizzando Google Hangouts.

Gli obiettivi degli incontri sono:
\begin{description}
  \item [Comprensione e perfezionamento dei requisiti] il team discute col proponente i problemi e i dubbi riscontrati durante l'analisi del capitolato in modo da comprendere incrementalmente i requisiti.
  \item [Ricerca e valutazione del software] il team chiede al proponente se le componenti e i software proposti soddisfino le funzionalità richieste.
  \item [Convalida dei documenti e del prodotto] il team si rivolge al proponente per avere conferme sul lavoro svolto siano essi documenti, \glossario{prototipi} appena abbozzati o prodotti software ad uno stato avanzato.
\end{description}
Gli esiti degli argomenti discussi durante gli incontri saranno poi riportati nei verbali.

\subparagraph{Incontri con il committente}%
\label{subp:incontri_con_il_committente}

GruppOne intende fare riferimento al committente del progetto anche al di fuori delle revisioni obbligatorie.
Compatibilmente con la disponibilità del committente, chiederemo di fissare degli incontri per discutere eventuali aspetti che non ci sono chiari.

Ciò che emerge da questi incontri verrà messo a verbale e reso disponibile nella cartella \verb|esterni/verbali/| della documentazione di progetto.

\paragraph{Consegna e completamento}%
\label{par:consegna e completamento}

Il gruppo si impegna a consegnare il prodotto software secondo le modalità definite nel contratto. In particolare, si forniscono anche un manuale utente e un manuale sviluppatore.

% TODO scrivi metriche

% subs:metriche (end)

\end{document}
