\documentclass[../norme-di-progetto.tex]{subfiles}

\begin{document}

\subsection{Descrizione}%
\label{sub:processi_organizzativi/descrizione}

I processi organizzativi stabiliscono le attività interne che il team deve svolgere per garantire l'\glossario{economicità} nel corso dello sviluppo del software.
Rivestono un ruolo fondamentale in quanto permettono di gestire la suddivisione dei ruoli e coordinare i membri del gruppo.
Essi sono trasversali rispetto ai singoli progetti.

I principali processi organizzativi sono:

\begin{itemize}
  \item Gestione dei processi
  \item Gestione delle infrastrutture
  \item Gestione dei rischi
  \item Miglioramento del processo
  \item Formazione del personale.
\end{itemize}

\subsection{Gestione dei rischi}%
\label{sub:gestione_dei_rischi}

\subfile{processi-organizzativi/gestione-dei-rischi}

\subsection{Gestione del personale}%
\label{sub:gestione_del_personale}

\subfile{processi-organizzativi/gestione-del-personale}

\subsection{Formazione del personale}%
\label{sub:formazione_del_personale}

\subfile{processi-organizzativi/formazione-del-personale}

\end{document}
