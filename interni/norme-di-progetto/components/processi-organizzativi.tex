\documentclass[../norme-di-progetto.tex]{subfiles}

\begin{document}

\subsection{Descrizione}%
\label{sub:processi_organizzativi/descrizione}

I processi organizzativi stabiliscono le attività interne che il team deve svolgere per garantire l'\glossario{economicità} nel corso dello sviluppo del software.
Rivestono un ruolo fondamentale in quanto permettono di gestire la suddivisione dei ruoli e coordinare i membri del gruppo.
Essi sono trasversali rispetto ai singoli progetti.

Le attività coinvolte dai processi organizzativi sono:

\begin{itemize}
  \item Gestione dei processi
  \item Gestione delle infrastrutture
  \item Gestione dei rischi
  \item Miglioramento del processo
  \item Formazione del personale.
\end{itemize}

\subsection{Gestione dei rischi}%
\label{sub:gestione_dei_rischi}

\subsubsection{Classificazione dei rischi}%
\label{subs:classificazione_dei_rischi}
La gestione dei rischi consiste nel prevedere i rischi che potrebbero influire negativamente su alcune attività del ciclo di vita del software danneggiando i risultati o la qualità del prodotto intermedio.
Questo paragrafo ha lo scopo di presentare i potenziali rischi all'interno di un progetto, mentre l'attività di analisi dei rischi è descritta dettagliatamente nel PdP.
Si parla di attività di analisi dei rischi dato che la gestione dei rischi potrebbe essere intesa come un vero e proprio processo in quanto racchiude molteplici attività. La gestione dei rischi si divide in diverse fasi:

\begin{description}
  \item [Identificazione dei rischi]: si identificano i rischi per il progetto e il prodotto.
  \item [Analisi dei rischi]: si valutano le probabilità che si verifichino i rischi e si descrivono le conseguenze.
  \item [Pianificazione dei rischi]: si presentano i piani per evitare i rischi e i rimedi per minimizzare gli effetti.
  \item [Monitoraggio dei rischi]: si monitorano i rischi e si aggiornano incrementalmente i piani.
\end{description}

Analisi dei rischi e pianificazione dei rischi saranno trattare nel PdP mentre in questo paragrafo ci si occuperà dell'attività di identificazione dei rischi.
Una prima classificazione dei rischi distingue:

\begin{description}
  \item [Rischi per il progetto]: influiscono sui tempi di realizzazione del progetto. L'assenza di un componente del gruppo per un lungo periodo di tempo potrebbe condizionare la pianificazione di progetto e potrebbe allungare i tempi di consegna.
  \item [Rischi per il prodotto]: influiscono sulla qualità del software che si sta producendo. L'utilizzo di algoritmi e procedure non ottimizzate potrebbe peggiorare le prestazioni del sistema e influire negativamente sulla qualità del nostro prodotto.
\end{description}
Essendo la classificazione soprastante troppo generica, GruppOne ha individuato un elenco di rischi maggiormente dettagliato:
\begin{description}
  \item [Rischi del personale]: riguardano i membri di GruppOne. L'assenza o l'indisponibilità di un componente del gruppo per un lungo periodo di tempo ne è un esempio.
  \item [Rischi dei requisiti]: riguardano il modo in cui GruppOne gestisce requisiti del cliente. Un esempio potrebbe essere il fraintendimento di un requisito del cliente durante l'attività di analisi dei requisiti.
  \item [Rischi tecnologici]: riguardano i software e l'hardware utilizzato per sviluppare il sistema. Un esempio potrebbe essere la mancanza di alcune feature nei software scelti per lo sviluppo del prodotto.
  \item [Rischi degli strumenti]: riguardano i software di supporto impiegati. Un esempio potrebbe essere la presenza di un bug in uno strumento e la necessità di risolverlo.
\end{description}

\subsection{Gestione del personale}%
\label{sub:gestione_del_personale}

\subsubsection{Finalità}%
\label{subs:Gestione_del_personale/finalita}

Gli scopi del processo di gestione del personale sono:

\begin{itemize}
  \item Individuare i ruoli fondamentali all'interno di un progetto
  \item Pianificare le attività da ripartire nel team
  \item Gestire le comunicazioni interne ed esterne
  \item Organizzare come e dove svolgere gli incontri
\end{itemize}

\subsubsection{Pianificazione}%
\label{subs:pianificazione}

\paragraph{Documenti di pianificazione}%
\label{par:documenti_di_pianificazione}

\subparagraph{Piano di progetto}%
\label{subp:piano_di_progetto}
Il piano di progetto è redatto dal responsabile e contiene l'organizzazione dell'attività allo scopo di raggiungere l'economicità. Esso rappresenta, anche mediante l'utilizzo di diagrammi le risorse fondamentali a disposizione e come andrebbero impiegate per organizzare un efficiente pianificazione delle attività. Il documento è così strutturato:
\begin{description}
  \item [Introduzione]: si presentano lo scopo e la struttura del documento.
  \item [Analisi dei rischi]: si realizza un'indagine per scoprire quali sono i maggiori rischi con cui il gruppo deve confrontarsi nel corso della realizzazione del progetto. Si illustra la gravità di ogni rischio, i danni che ognuno potrebbe arrecare ed eventuali contromisure per riuscire a contrastarli.
  \item [Pianificazione]: si stabiliscono quali sono le risorse temporali e umane a disposizione e si decide come suddividerle nelle diverse attività.
  \item [Preventivo e consuntivo]: Il preventivo presenta una stima dei costi totali di progetto, mentre il consuntivo illustra un bilancio totale delle attività di processo compiute in un determinato periodo di tempo.
\end{description}

% TODO piano di qualifica
\subparagraph{Piano di qualifica}%
\label{subp:piano_di_qualifica}

\subsubsection{Ruoli di progetto}%
\label{subs:ruoli_di_progetto}

GruppOne ha deciso che ogni membro del team deve svolgere ogni ruolo a rotazione che verrà effettuata ogni due settimane.
Per tale motivo si realizza un file Google docs per poter rendicontare il numero di ore svolte da ciascun membro del team in ogni ruolo.

\paragraph{Responsabile}%
\label{par:responsabile}
Il responsabile di progetto si occupa di garantire l'amministrazione e la gestione contabile di progetto.
Ha il compito di rappresentare l'azienda presso il proponente e redige il piano di progetto e l'organigramma.
Le sue attività principali sono:

\begin{itemize}
  \item Effettuare l'approvazione finale di ogni documento
  \item Gestire la pianificazione di progetto
  \item Gestire le risorse umane
  \item Coordinare i rapporti con clienti e fornitori
  \item Approvare l'offerta e i relativi allegati.
\end{itemize}

\textbf{costo orario in euro: 30}

\paragraph{Amministratore}%
\label{par:amministratore}
L'amministratore di progetto ha il controllo diretto sull'ambiente di lavoro. È responsabile dell'efficienza e dell'operatività del team di sviluppo e redige le norme di progetto. Le sue attività principali sono:

\begin{itemize}
  \item Gestire controllo di versioni e configurazioni
  \item Amministrare infrastrutture di supporto
  \item Redigere il piano di progetto per conto del responsabile
  \item Controllare le procedure di gestione della qualità
  \item Organizzare o configurare l'archivio di documentazione.
\end{itemize}

\textbf{costo orario in euro: 30}

\paragraph{Analista}%
\label{par:analista}
L'analista svolge l'attività di analisi dei requisiti.
Egli ha esperienza professionale e ha il compito di indagare il dominio applicativo.
Redige l`analisi dei requisiti e lo studio di fattibilità. Le sue attività principali sono:

\begin{itemize}
  \item tradurre i requisiti esposti in un linguaggio comprensibile ai progettisti dopo aver studiato il dominio dell'applicazione.
  \item collaborare con il progettista per cercare soluzioni al problema
  \item Interagire con gli \glossario{stakeholder} attraverso interviste.
\end{itemize}

\textbf{costo orario in euro: 25}

\paragraph{Progettista}%
\label{par:progettista}
Il progettista è responsabile dell'attività di progettazione.
Egli ha competenze tecniche e tecnologiche avanzate.
Redige inoltre la Specifica tecnica, la Definizione di prodotto e parte del Piano di Qualifica.
Le sue attività principali sono:

\begin{itemize}
  \item Ricevere i requisiti che il sistema deve soddisfare
  \item Pianificare l'\glossario{architettura} complessiva
  \item Selezionare i principali componenti coinvolti nella realizzazione del prodotto.
\end{itemize}

\textbf{costo orario in euro: 20}

% TODO risolvere "Underfull \hbox (badness 10000)"
\paragraph{Programmatore}%
\label{par:programmatore}
Il programmatore svolge l'attività di codifica per la realizzazione del prodotto.
Partecipa concretamente alla produzione del software e ha la responsabilità di realizzare i moduli per effettuare i test di unità sulle singole componenti.
Possiede competenze tecnologiche mirate ma settoriali.
\\\newline\textbf{costo orario in euro: 15}

% TODO risolvere "Underfull \hbox (badness 10000)"
\paragraph{Verificatore}%
\label{par:verificatore}
Il verificatore svolge l'attività di verifica e validazione per l'intera durata del progetto.
Ha competenze tecniche, esperienza professionale e conoscenza delle norme.Redige una parte del piano di qualifica.
Svolge un ruolo fondamentale nella fase di revisione del prodotto, dove va a cercare eventuali errori nell'oggetto che sta visionando.
\\\newline\textbf{costo orario in euro: 15}

\subsubsection{Coordinazione del personale}%
\label{subs:coordinazione_del_personale}
Per organizzare le attività di progetto è necessario avere una buona coordinazione nella gestione delle riunioni, delle comunicazioni e accordarsi strumenti da utilizzare.
Nella sezione sottostante si presentano le metodologie e i mezzi adottati da GruppOne per dirigere le comunicazioni interni ed esterne.

\paragraph{Coordinamento incontri}%
\label{par:coordinamento_incontri}

\subparagraph{interni}%
\label{subp:coordinamento_incontri/interni}
Gli incontri del gruppo avvengono in due modalità:
\begin{description}
  \item [Fisicamente]: una volta ogni una o due settimane il team si riunisce per effettuare il punto della situazione. Si discutono, quindi, le problematiche riscontrate nello svolgimento dei compiti assegnati, si definiscono nuove \glossario{{milestone}}.
  \item [Virtualmente]: con la stessa frequenza degli incontri fisici il team si incontra utilizzando software VoIP per coordinarsi nelle attività di maggiore complessità e che richiedono la presenza di tutti i membri del team. GruppOne ha scelto di utilizzare \glossario{Google Hangouts}.
\end{description}
Il responsabile fissa le date degli incontri su \glossario{Google Calendar} dove ogni membro del gruppo deve indicare la propria presenza o assenza.

\subparagraph{esterni}%
\label{subp:coordinamento_incontri/esterni}
Per organizzare gli incontri con il proponente o le revisioni di progetto col committente, GruppOne utilizza la mail ufficiale del gruppo \href{gruppone.swe@gmail.com} che è gestita dal responsabile.

\paragraph{Coordinamento comunicazioni}%
\label{par:coordinamento_comunicazioni}

\subparagraph{interne}%
\label{subp:coordinamento_comunicazioni/interne}
Per le comunicazioni interne GruppOne fa uso di due differenti mezzi di comunicazione:
\begin{description}
  \item [Telegram]: è un software di messaggistica istantanea che viene utilizzato dal team per comunicazioni urgenti, avvisi dell'ultimo minuto o eventi imminenti.
  \item [Slack]: è uno strumento di collaborazione aziendale che permette la creazione di canali diversi suddivisi per argomento. Ha, inoltre, il pregio di potersi connettere con software di terze parti e quindi la possibilità di inviare notifiche non appena ci sono delle novità (es: apertura di nuove issue su GitHub). Per tale motivo GruppOne ha preso in considerazione l'utilizzo di Slack. Come norma generale si è stabilito inoltre che tutti i membri del gruppo dovrebbero avere una scheda del proprio browser con Slack aperto per poter mostrare agli altri che si è online e si sta lavorando.
\end{description}

\subparagraph{esterne}%
\label{subp:esterne}
Per comunicazioni con il proponente o con il committente, GruppOne utilizza la mail ufficiale del gruppo come già illustrato nel paragrafo~\ref{subp:coordinamento_incontri/esterni}.

\subsection{Formazione del personale}%
\label{sub:formazione_del_personale}

\subsubsection{Formazione individuale}%
\label{subs:formazione_individuale}

Il prodotto da sviluppare è decisamente di elevata complessità.
Molti degli strumenti che si utilizzano nel corso del progetto sono sconosciuti alla maggior parte dei componenti del team.
Per tale motivo è fondamentale che ciascun membro del gruppo investa una parte del proprio tempo nell'autoapprendimento di nuove tecnologie.
Nel caso in cui fosse presente qualcuno già a conoscenza delle tecnologie è invitato a condividere le proprie conoscenze con gli altri.

\end{document}
