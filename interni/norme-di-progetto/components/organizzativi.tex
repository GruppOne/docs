\documentclass[../norme-di-progetto.tex]{subfiles}
\begin{document}
\subsection{Descrizione}
I processi organizzativi stabiliscono le attività interne che il team di sviluppo deve svolgere per garantire \glossario{\textit{l'economicità}} nel corso dello sviluppo del software. Rivestono un ruolo fondamentale in quanto gestiscono la suddivisione dei ruoli e coordinano i membri del gruppo. Il team stesso è garante che tali processi siano attivi e funzionali.
\subsection{gestione del personale}
\subsubsection{Finalità}
Gli scopi del processo di gestione del personale sono:
\begin{itemize}
	\item Individuare i ruoli fondamentali all'interno di un progetto
	\item Pianificare le attività da ripartire nel team
	\item Gestire le comunicazioni interne ed esterne
	\item Organizzare come e dove svolgere gli incontri
\end{itemize}
\subsubsection{Ruoli di progetto}
Grupp0ne ha deciso che ogni membro del team deve svolgere ogni ruolo a rotazione che verrà effettuata ogni due settimane. Per tale motivo è stato realizzato un file Google docs per poter rendicontare il numero di ore svolte da ciascun membro del team in ogni ruolo.
\paragraph{Responsabile}
Il responsabile di progetto si occupa di garantire l'amministrazione e la gestione contabile di progetto. Ha il compito di rappresentare l'azienda presso il proponente e redige il piano di progetto e l'organigramma. Le sue attività principali sono:
\begin{itemize}
	\item Effettuare l'approvazione finale di ogni documento
	\item Gestire la pianificazione di progetto
	\item Gestire le risorse umane
	\item Coordinare i rapporti con clienti e fornitori 
	\item Approvare l'offerta e i relativi allegati
\end{itemize}
\textbf{costo orario in euro: 30}
\paragraph{Amministratore}
L'amministratore di progetto ha il controllo diretto sull'ambiente di lavoro. È responsabile dell'efficienza e dell'operatività del team di sviluppo e redige le norme di progetto. Le sue attività principali sono:
\begin{itemize}
	\item Gestire controllo di versioni e configurazioni
	\item Amministrare infrastrutture di supporto
	\item Redigere il piano di progetto per conto del responsabile
	\item Controllare le procedure di gestione della qualità
	\item Organizzare o configurare l'archivio di documentazione
\end{itemize}
\textbf{costo orario in euro: 30}
\paragraph{Analista}
L'analista svolge l'attività di analisi dei requisiti. Egli ha esperienza professionale e ha il compito di indagare il dominio applicativo. Redige l`analisi dei requisiti e lo studio di fattibilità. Le sue attività principali sono:
\begin{itemize}
	\item tradurre i requisiti esposti in un linguaggio comprensibile ai progettisti dopo aver studiato il dominio dell'applicazione.
	\item collaborare con il progettista per cercare soluzioni al problema
	\item Interagire con gli \glossario{\textit{stakeholders}} attraverso interviste
\end{itemize}
\textbf{costo orario in euro: 25}
\paragraph{Progettista}
Il progettista è responsabile dell'attività di progettazione. Egli ha competenze tecniche e tecnologiche avanzate. Redige inoltre la Specifica tecnica, la Definizione di prodotto e parte del Piano di Qualifica. Le sue attività principali sono:
\begin{itemize}
	\item Ricevere i requisiti che il sistema deve soddisfare
	\item Pianificare \glossario{\textit{l'architettura}} complessiva
	\item Selezionare i principali componenti coinvolti nella realizzazione del prodotto
\end{itemize}
\textbf{costo orario in euro: 20}
\paragraph{Programmatore}
Il programmatore svolge l'attività di codifica per la realizzazione del prodotto. Partecipa concretamente alla produzione del software e ha la responsabilità di realizzare i moduli per effettuare i test di unità sulle singole componenenti e possiede competenze tecnologiche mirate ma settoriali.
\\
\newline\textbf{costo orario in euro: 15}
\paragraph{Verificatore}
\subsubsection{Coordinazione del personale}
\paragraph{Coordinamento incontri}
\paragraph{Coordinamento comunicazioni}
\subsection{Formazione personale}
\subsubsection{Formazione individuale}
\end{document}