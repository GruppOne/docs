\documentclass[../norme-di-progetto.tex]{subfiles}
\begin{document}

\subsection{Scopo del documento}
\label{sub:scopo del documento}
Tale documento è rivolto ai membri di Grupp0ne. Si vogliono definire i compiti, le regole e le direttive a cui ogni membro del gruppo deve attenersi per avere un efficace way of working nel corso di ogni attività di processo. Esso verrà aggiornato nel corso del tempo, pertanto è ancora incompleto. 


\subsection{Scopo del prodotto}
\label{sub:scopo del prodotto}
Si vuole realizzare un'applicazione che permetta di monitorare le persone presenti in un determinato luogo. In particolare, l'obiettivo è quello di sviluppare un applicazione software per Android o Ios in grado di segnalare ad un server l'entrata e l'uscita dall'area di interesse.
Si vogliono tracciare:
\begin{itemize}
	\item[$\bullet$]  I dipendenti all'interno di Imola Informatica.
	\item[$\bullet$] Le persone presenti nei padiglioni della fiera di Verona.
\end{itemize}

\subsection{Glossario}
\label{sub:glossario}
L'uso di vocaboli tecnici e facilmente fraintendibili rende necessaria la realizzazione di un glossario che presenterà i termini dai significati più ambigui presenti in ogni documento. Per garantire l'inequivocabilità, le parole che possono assumere un significato ambiguo sono evidenziate (i.e., \glossario{way of working}) e riportate in \textit{Glossario1.0.0.pdf} accompagnati da una definizione.

\subsection{Riferimenti}
\label{sub:riferimenti}
\subsubsection{normativi}
\label{subs:normativi}
\begin{itemize}
	\item Slide gestione di progetto 
	\newline \url{https://www.math.unipd.it/\textasciitilde tullio/IS-1/2019/Dispense/L06.pdf}
	\item Slide amministrazione di progetto
	\newline \url{https://www.math.unipd.it/~tullio/IS-1/2019/Dispense/FC01.pdf}
	\item Standard ISO 12207-1995 
	\newline \url{https://www.math.unipd.it/\textasciitilde tullio/IS-1/2009/Approfondimenti/ISO\textunderscore 12207-1995.pdf}
	\item Capitolato d'appalto C5 
	\newline \url{https://www.math.unipd.it/\textasciitilde tullio/IS-1/2019/Progetto/C5.pdf}
	\item Ingegneria del software - Ian Sommerville - decima edizione
	\newline \url{https://www.pearson.it/opera/pearson/0-6424-ingegneria_del_software}
\end{itemize}
\subsubsection{informativi}
\label{subs:riferimenti informativi}
\begin{itemize}
	\item \LaTeX
	\newline \url{https://www.latex-project.org/help/documentation/}
	\item Subfiles
	\newline \url{https://www.overleaf.com/learn/latex/Multi-file_LaTeX_projects}
	\item Git
	\newline \url{https://git-scm.com/}
	\item Github
	\newline \url{https://github.com/}
	\item Commitizen
	\newline \url{https://github.com/commitizen/cz-cli}
	\item Telegram
	\newline \url{https://desktop.telegram.org/}
	\item Code owners
	\newline \url{https://help.github.com/en/github/creating-cloning-and-archiving-repositories/about-code-owners}
\end{itemize}
\end{document}