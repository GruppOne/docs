\documentclass[../../norme-di-progetto.tex]{subfiles}

% FIXME struttura non conforme
% FIXME attività mancanti
\begin{document}

\subsubsection{Scopo}%
\label{subs:formazione_del_personale/scopo}

Il processo di formazione del personale prevede di avere sempre personale preparato, qualificato e all'altezza dei compiti da svolgere durante tutto il corso del progetto.

\subsubsection{Attività}%
\label{subs:formazione_del_personale/attivita}

\paragraph{Formazione individuale e autoapprendimento}%
\label{par:formazione_individuale_e_autoapprendimento}

Il prodotto da sviluppare presenta un'elevata complessità tecnica e teorica.
Molti degli strumenti che si utilizzano nel corso del progetto sono sconosciuti alla maggior parte dei componenti del team.
Per tale motivo è fondamentale che ciascun membro del gruppo investa una parte del proprio tempo nell'autoapprendimento di nuove tecnologie.
Tali ore, inoltre, non vanno rendicontate come attività di progetto.
Ogni componente del team ha, quindi, la responsabilità di capire autonomamente quanto tempo dedicare a tale attività.
Inoltre, nel caso in cui qualcuno possedesse già una certa familiarità con le tecnologie utilizzate è invitato a condividere le proprie conoscenze con gli altri.

\subparagraph{Materiali utili}%
\label{subp:materiali_utili}

La lista dei materiali utili all'autoapprendimento è fornita nell'introduzione del documento per praticità,~\ref{sub:materiali_consigliati_per_l_autoapprendimento}.

\end{document}
