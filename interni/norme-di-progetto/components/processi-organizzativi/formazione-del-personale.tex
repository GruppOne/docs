\documentclass[../../norme-di-progetto.tex]{subfiles}

\begin{document}

\subsubsection{Finalità}%
\label{subs:formazione_del_personale/finalita}

Il processo di formazione del personale prevede di avere sempre personale preparato, qualificato e all'altezza dei compiti da svolgere durante tutto il corso del progetto.
Viene richiesto ad ogni componente del gruppo di impegnarsi in autonomia ad effettuare attività di autoapprendimento qualora il compito
ad esso assegnato richieda conoscenze ad esso sconosciute. Inoltre il componente è invitato a condividere tali conoscenze.

\subsubsection{Descrizione}%
\label{subs:formazione_del_personale/descrizione}

L'implementazione di questo processo non utilizza attività come definite dall'ISO/IEC 12207:1995 per via della natura particolare del progetto, e del fatto che le ore impegnate nell'apprendimento delle tecnologie necessarie a sviluppare il prodotto non vengono rendicontate.
L'unica attività istanziata da GruppOne è quella di formazione individuale e autoapprendimento.

\subsubsection{Attività}%
\label{subs:formazione_del_personale/attivita}

\paragraph{Formazione individuale e autoapprendimento}%
\label{par:formazione_individuale_e_autoapprendimento}

Il prodotto da sviluppare presenta un'elevata complessità tecnica e teorica e molti degli strumenti che si utilizzano nel corso del progetto sono sconosciuti alla maggior parte dei componenti del team.
Per tale motivo è fondamentale che ciascun membro del gruppo investa una parte del proprio tempo nell'apprendimento autonomo di nuove tecnologie.
Ogni componente del team ha, quindi, la responsabilità di capire autonomamente quanto tempo dedicare a tale attività.
Inoltre, nel caso in cui qualcuno possedesse già una certa familiarità con le tecnologie utilizzate è invitato a condividere le proprie conoscenze con gli altri. A questo scopo GruppOne mantiene una lista interna di membri del team su cui ciascuno può fare affidamento per farsi spiegare eventuali dubbi.

\subparagraph{Materiali utili}%
\label{subp:materiali_utili}

La lista dei materiali utili all'autoapprendimento è fornita nell'introduzione del documento per praticità,~\ref{sub:materiali_consigliati_per_l_autoapprendimento}.
Ogni componente del gruppo è libero di usare ulteriori materiali di autoapprendimento oltre a quelli indicati, purché condivida tale materiale con il resto dei membri.
\end{document}
