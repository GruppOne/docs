\documentclass[../../norme-di-progetto.tex]{subfiles}

\begin{document}

\subsubsection{Finalità}%
\label{subs:gestione_di_processo/finalita}

Gli scopi del processo di gestione di processo sono:

\begin{itemize}
  \item Individuare i ruoli fondamentali all'interno di un progetto
  \item Pianificare le attività da ripartire nel team
  \item Individuare procedure per gestire le comunicazioni interne ed esterne
  \item Organizzare come e dove svolgere gli incontri.
\end{itemize}

\subsubsection{Ruoli di progetto}%
\label{subs:ruoli_di_progetto}

Il committente ha fissato per i membri del team dei ruoli attraverso cui si svolgono le attività di progetto. Ciascun ruolo ha un costo orario correlato alle responsabilità che comporta, e rappresentativo del fatto che ogni processo consuma risorse.

GruppOne si impegna a far sì che i ruoli vengano distribuiti tra i membri del team secondo quanto definito nella pianificazione di progetto, assicurandosi che ogni membro ricopra ciascun ruolo in proporzioni eque.

Oltre che nella pianificazione, terremo traccia delle risorse consumate per ogni ruolo attraverso un foglio di lavoro su \href{https://www.google.it/intl/it/sheets/about/}{Google Sheets}, i cui totali verranno riportati nel consuntivo di periodo da consegnare al committente.

\paragraph{Responsabile}%
\label{par:responsabile}
Il responsabile di progetto si occupa di garantire l'amministrazione e la gestione contabile di progetto.
Ha il compito di rappresentare il gruppo presso il proponente e redige il \textit{Piano di progetto} e l'organigramma.
Le sue attività principali sono:

\begin{itemize}
  \item Effettuare l'approvazione finale di ogni documento
  \item Gestire la pianificazione di progetto
  \item Gestire le risorse umane
  \item Coordinare i rapporti con clienti e fornitori
  \item Approvare l'offerta e i relativi allegati.
\end{itemize}

Consuma risorse pari a 30 €/h.

\paragraph{Amministratore}%
\label{par:amministratore}
L'amministratore di progetto ha il controllo diretto sull'ambiente di lavoro. È responsabile dell'efficienza e dell'operatività del team di sviluppo e redige le \textit{Norme di progetto}. Le sue attività principali sono:

\begin{itemize}
  \item Gestire controllo di versioni e configurazioni
  \item Amministrare infrastrutture di supporto
  \item Redigere il piano di progetto per conto del responsabile
  \item Controllare le procedure di gestione della qualità
  \item Organizzare o configurare l'archivio di documentazione.
\end{itemize}

Consuma risorse pari a 30 €/h.

\paragraph{Analista}%
\label{par:analista}
L'analista svolge l'attività di \textit{Analisi dei requisiti}.
Egli ha esperienza professionale e ha il compito di indagare il dominio applicativo.
Redige l`\textit{Analisi dei requisiti} e lo \textit{Studio di fattibilità}. Le sue attività principali sono:

\begin{itemize}
  \item tradurre i requisiti esposti in un linguaggio comprensibile ai progettisti dopo aver studiato il dominio dell'applicazione.
  \item collaborare con il progettista per cercare soluzioni al problema
  \item Interagire con gli \glossario{stakeholder} attraverso interviste.
\end{itemize}

Consuma risorse pari a 25 €/h.

\paragraph{Progettista}%
\label{par:progettista}
Il progettista è responsabile dell'attività di progettazione.
Egli ha competenze tecniche e tecnologiche avanzate.
Redige inoltre la Specifica tecnica, la Definizione di prodotto e parte del \textit{Piano di Qualifica}.
Le sue attività principali sono:

\begin{itemize}
  \item Ricevere i requisiti che il sistema deve soddisfare
  \item Pianificare l'\glossario{architettura} complessiva
  \item Selezionare i principali componenti coinvolti nella realizzazione del prodotto.
\end{itemize}

Consuma risorse pari a 20 €/h.

\paragraph{Programmatore}%
\label{par:programmatore}
Il programmatore svolge l'attività di codifica al fine di implementare il prodotto, senza avere alcuna autorità decisionale.
Partecipa concretamente alla produzione del software e ha la responsabilità di realizzare i moduli per effettuare i test di unità sulle singole componenti.
Possiede competenze tecnologiche mirate ma settoriali.

Consuma risorse pari a 15 €/h.

\paragraph{Verificatore}%
\label{par:verificatore}
Il verificatore svolge l'attività di verifica per l'intera durata del progetto, e di validazione in preparazione al collaudo.
Ha competenze tecniche, esperienza professionale e conoscenza delle norme. Redige una parte del \textit{piano di qualifica}.
Svolge un ruolo fondamentale nella fase di revisione del prodotto, dove va a cercare eventuali errori nel documento che sta visionando.

Consuma risorse pari a 15 €/h.

\subsubsection{Coordinazione del personale}%
\label{subs:coordinazione_del_personale}
Per organizzare le attività di progetto è necessario avere una buona coordinazione nella gestione delle riunioni, delle comunicazioni e ci si deve accordare sugli strumenti da utilizzare.
Nella sezione sottostante si presentano le metodologie e i mezzi adottati da GruppOne per dirigere le comunicazioni interni ed esterne.

\paragraph{Coordinamento incontri}%
\label{par:coordinamento_incontri}

\subparagraph{interni}%
\label{subp:coordinamento_incontri/interni}
Gli incontri del gruppo avvengono in due modalità:
\begin{description}
  \item [Fisicamente] una volta ogni una o due settimane il team si riunisce per effettuare il punto della situazione. Si discutono, quindi, le problematiche riscontrate nello svolgimento dei compiti assegnati, si definiscono nuove \glossario{milestone}.
  \item [Virtualmente] con la stessa frequenza degli incontri fisici il team si riunisce utilizzando software VoIP per coordinarsi nelle attività di maggiore complessità e che richiedono la presenza di tutti i membri del team. GruppOne ha scelto di utilizzare \glossario{Google Hangouts}.
\end{description}
Il responsabile fissa le date degli incontri su \glossario{Google Calendar} dove ogni membro del gruppo deve indicare la propria presenza o assenza.

\subparagraph{esterni}%
\label{subp:coordinamento_incontri/esterni}
Per organizzare gli incontri con il proponente o le revisioni di progetto col committente, GruppOne utilizza la mail ufficiale del gruppo \href{gruppone.swe@gmail.com}{gruppone.swe@gmail.com} che è gestita dal responsabile.

\paragraph{Coordinamento comunicazioni}%
\label{par:coordinamento_comunicazioni}

\subparagraph{interne}%
\label{subp:coordinamento_comunicazioni/interne}
Per le comunicazioni interne GruppOne fa uso di due differenti mezzi di comunicazione:
\begin{description}
  \item [Telegram] è un software di messaggistica istantanea che viene utilizzato dal team per comunicazioni urgenti, avvisi dell'ultimo minuto o eventi imminenti.
  \item [Slack] è uno strumento di collaborazione aziendale che permette la creazione di canali diversi suddivisi per argomento. Ha il pregio di interfacciarsi con software di terze parti ed è in grado di inviare notifiche al verificarsi di nuovi eventi, come l'apertura di una nuova issue su GitHub, per questo motivo il gruppo ha optato per il suo utilizzo. Come norma desiderabile si è stabilito inoltre che tutti i membri del gruppo tengano il sito internet di Slack in esecuzione in background, in modo che il resto del team capisca chi sta lavorando.
\end{description}

\subparagraph{esterne}%
\label{subp:esterne}
Per comunicazioni con il proponente o con il committente, GruppOne utilizza la mail ufficiale del gruppo, come già illustrato nel paragrafo~\ref{subp:coordinamento_incontri/esterni}.


\subsubsection{Metriche di processo}%
\label{subs:gestione_del_personale/metriche_di_processo}

\paragraph{MPS-VDC\@: Varianza dei costi}%
\label{par:MPS-VDC_varianza_dei_costi}

La metrica varianza dei costi misura se il costo effettivamente maturato al momento del calcolo dell'indice sia in linea con quanto pianificato. Si calcola:
\[
  BCWP-ACWP
\]
dove:
\begin{description}
  \item [BCWP] misura il costo del lavoro svolto alla data attuale.
  \item [ACWP] indica il costo sostenuto alla data corrente.
\end{description}

\paragraph{MPS-VDS\@: Varianza rispetto allo schedule}%
\label{par:MPS-VDS_varianza_rispetto_allo_schedule}

La metrica varianza rispetto allo schedule misura se si è in anticipo o in ritardo rispetto allo schedule delle attività di progetto pianificate nella baseline. Si calcola:
\[
  BCWP - BCWS
\]
dove:
\begin{description}
  \item [BCWP] misura il costo del lavoro svolto alla data attuale.
  \item [BCWS] indica il budget previsto per la realizzazione dell'intero progetto.
\end{description}
Il numero risultante indica:
\begin{itemize}
  \item Una varianza di schedule pari a 0 indica che si sta rispettando la pianificazione.
  \item Una varianza di schedule inferiore a 0 indica che si è in ritardo rispetto alla pianificazione.
  \item Una varianza di schedule superiore a 0 indica che si è in anticipo rispetto alla pianificazione.
\end{itemize}

\end{document}
