\documentclass[../../norme-di-progetto.tex]{subfiles}

\begin{document}

\subsubsection{Finalità}%
\label{subs:gestione_dei_rischi/finalita}

Gli scopi del processo di gestione dei rischi sono:

\begin{itemize}
  \item Individuare i principali rischi associati alle attività degli altri processi.
  \item Classificare i rischi individuati per attribuire loro una priorità.
  \item Identificare dei metodi di prevenzione dei rischi.
  \item Effettuare continui monitoraggi sui processi durante l'andamento del progetto.
  \item Stabilire, per ciascun rischio, delle strategie di contenimento che possano ridurre gli effetti negativi provocati dall'eventuale verificarsi del rischio.
\end{itemize}

\subsubsection{Descrizione}%
\label{subs:gestione_dei_rischi/descrizione}

La gestione dei rischi consiste nel prevedere i rischi che potrebbero influire negativamente su alcune attività del ciclo di vita del software, danneggiando i risultati o la qualità del prodotto intermedio.

Questo processo raggruppa le seguenti attività:

\begin{itemize}
  \item Classificazione dei rischi
  \item Analisi dei rischi
  \item Pianificazione dei rischi
  \item Monitoraggio dei rischi
\end{itemize}

\subsubsection{Attività}%
\label{subs:gestione_dei_rischi/attivita}

\paragraph{Classificazione dei rischi}%
\label{par:classificazione_dei_rischi}

si identificano e si classificano i rischi per il progetto e il prodotto.

\begin{description}
  \item [Rischi per il progetto] influiscono sui tempi di realizzazione del progetto.
        L'assenza di un componente del gruppo per un lungo periodo di tempo potrebbe condizionare la pianificazione di progetto e potrebbe allungare i tempi di consegna.
  \item [Rischi per il prodotto] influiscono sulla qualità del prodotto.
        L'utilizzo di algoritmi e procedure non ottimizzate potrebbe peggiorare le prestazioni del sistema e influire negativamente sulla qualità del prodotto.
\end{description}

Essendo la classificazione soprastante troppo generica, GruppOne ha individuato un elenco di rischi maggiormente dettagliato:
\begin{description}
  \item [Rischi del personale] riguardano i membri di GruppOne.
        L'assenza o l'indisponibilità di un componente del gruppo per un lungo periodo di tempo ne è un esempio.
  \item [Rischi dei requisiti] riguardano il modo in cui GruppOne gestisce i requisiti del cliente.
        Un esempio potrebbe essere il fraintendimento di un requisito del cliente durante l'attività di analisi dei requisiti.
  \item [Rischi tecnologici] riguardano i software e l'hardware utilizzato per sviluppare il sistema.
        Un esempio potrebbe essere la mancanza di alcune feature nei software scelti per lo sviluppo del prodotto.
  \item [Rischi degli strumenti] riguardano i software di supporto impiegati.
        Un esempio potrebbe essere la presenza di un bug in uno strumento e la necessità di risolverlo.
\end{description}

\paragraph{Analisi dei rischi}%
\label{par:analisi__dei_rischi}

si valutano le probabilità che si verifichino i rischi e si descrivono le conseguenze.

L'attività di analisi dei rischi descrive e valuta i rischi individuati durante un progetto software per facilitarne il successivo monitoraggio. Per ogni rischio si individuano:
\begin{description}
  \item[Nome del rischio]: è il nome del rischio che si sta descrivendo.
  \item[Codice del rischio]: è un codice univoco che ha lo scopo di identificare precisamente il rischio che si sta esaminando.
  \item[Descrizione del rischio]: fornisce una breve descrizione delle condizioni nelle quali il rischio potrebbe verificarsi e di quali siano le probabili conseguenze.
  \item[Monitoraggio]: propone tecniche di prevenzione dei rischi o di contenimento dei loro effetti negativi.
  \item[Probabilità]: è un indicatore che misura la probabilità di verificarsi di un rischio.
  \item[Impatto]: è un indicatore che quantifica l'impatto di un rischio sull'andamento del progetto.
\end{description}

\paragraph{Pianificazione dei rischi}%
\label{par:pianificazione_dei_rischi}

I compiti di questa attività sono presentare i piani per evitare i rischi e mostrare i rimedi per minimizzarne gli effetti.
Le misure pratiche messe in atto da GruppOne sono documentate all'interno della sezione apposita del \textit{Piano di progetto (versione \versione)}.
% TODO riferire sezione esatta del doc?

% par:pianificazione_dei_rischi (end)

\paragraph{Monitoraggio dei rischi}%
\label{par:monitoraggio_dei_rischi}

In appendice al \textit{Piano di progetto (versione \versione)} vengono documentati i rischi previsti che sono stati effettivamente riscontrati da GruppOne, e le disposizioni attuate per evitarne la ricomparsa.

% par:monitoraggio_dei_rischi (end)

\subsubsection{Procedure}%
\label{subs:gestione_dei_rischi/procedure}

\paragraph{Identificazione e classificazione dei rischi}%
\label{par:identificazione_e_classificazione_dei_rischi}

Per identificare i rischi senza alcun fraintendimento, il team ha deciso di introdurre dei codici che possano identificarli univocamente.
Ogni rischio ha un codice così costituito:

\begin{center}
  \textbf{RI[Tipo][Codice]}
\end{center}
dove il tipo può essere uno tra i seguenti:
\begin{description}
  \item [PRG]: identifica un rischio di progetto.
  \item [PRD]: identifica un rischio di prodotto.
  \item [REQ]: identifica un rischio dei requisiti.
  \item [PER]: identifica un rischio del personale.
  \item [TEC]: identifica un rischio tecnologico.
  \item [STR]: identifica un rischio di strumento.
\end{description}
Il codice è invece un numero progressivo a tre cifre. Ad esempio il primo rischio di prodotto si identifica con \textit{RIPRD001}.

Utilizziamo inoltre i seguenti indicatori per misurare la pericolosità e i danni che un rischio può arrecare:

\begin{description}
  \item[Probabilità] è una stima indicativa della probabilità che un rischio si verifichi e può assumere i seguenti valori:
        \begin{itemize}
          \item Bassa
          \item Media
          \item Alta
        \end{itemize}
  \item[Impatto] indica il tipo di effetto che l'attuarsi del rischio può avere sull'andamento del progetto. Può essere:
        \begin{itemize}
          \item Basso
          \item Tollerabile
          \item Significativo
        \end{itemize}
\end{description}

% par:identificazione_e_classificazione_dei_rischi (end)

\subsubsection{Metriche di processo}%
\label{subs:gestione_dei_rischi/metriche_di_processo}

\paragraph{MPS-RNP: Rischi incontrati e non preventivati}% in questo caso NON ci va il "\@". chktex 13
\label{par:MPS-RNP_rischi_incontrati_e_non_preventivati}

La metrica misura il numero di rischi incontrati che non sono stati preventivati nell'analisi dei rischi del \textit{Piano di progetto (versione \versione)}. Un valore maggiore di 0 indica una analisi dei rischi effettuata in maniera non ottimale.

\end{document}
