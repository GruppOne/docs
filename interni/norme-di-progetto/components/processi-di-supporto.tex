\documentclass[../norme-di-progetto.tex]{subfiles}

\begin{document}

\subsection{Descrizione}%
\label{sub:processi_di_supporto/descrizione}

I processi di supporto operano a sostegno degli altri processi con lo scopo di garantire il successo e fornire qualità al progetto.
Essi non esistono autonomamente ma hanno la necessità di appoggiarsi ad altri processi.
ISO 11207:1995 ne distingue otto:

\begin{itemize}
  \item Documentazione
  \item Gestione della configurazione
  \item Accertamento della qualità
  \item Verifica
  \item Validazione
  \item Verifica congiunta
  \item Revisione
  \item Risoluzione dei problemi.
\end{itemize}

GruppOne non istanzia il processo di verifica congiunta, in quanto, vista l'inesperienza dei componenti, nessuno può svolgere efficacemente il ruolo di reviewer.

\subsection{Documentazione}%
\label{sub:documentazione}

\subfile{processi-di-supporto/documentazione}

\subsection{Gestione della configurazione}%
\label{sub:gestione_della_configurazione}

\subfile{processi-di-supporto/gestione-della-configurazione}

\subsection{Accertamento della qualità}%
\label{sub:accertamento_della_qualita}

\subfile{processi-di-supporto/accertamento-della-qualita}

\subsection{Verifica}%
\label{sub:verifica}

\subfile{processi-di-supporto/verifica}

\subsection{Validazione}%
\label{sub:validazione}

\subfile{processi-di-supporto/validazione}

\subsection{Revisione}%
\label{sub:revisione}

\subfile{processi-di-supporto/revisione}

\subsection{Risoluzione dei problemi}%
\label{sub:risoluzione_dei_problemi}

\subfile{processi-di-supporto/risoluzione-dei-problemi}

\end{document}
