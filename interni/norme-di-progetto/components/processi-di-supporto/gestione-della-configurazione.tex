\documentclass[../../norme-di-progetto.tex]{subfiles}

\begin{document}

\subsubsection{Finalità}%
\label{subs:gestione_della_configurazione/finalita}

GruppOne istanzia il processo di gestione della configurazione per:
\begin{itemize}
  \item Assicurare completezza e correttezza del prodotto.
  \item Gestire al meglio il lavoro di gruppo.
  \item Identificare procedure per l'organizzazione, gestione e rilascio del prodotto.
  \item Conservare documentazione e software in un luogo accessibile e compatibile con la licenza richiesta dal committente.
\end{itemize}

\subsubsection{Descrizione}%
\label{subs:gestione_della_configurazione/descrizione}

Attraverso il processo di gestione della configurazione vengono tracciati lo stato e l'evoluzione delle singole componenti del prodotto, la relazione complessiva tra le parti e vengono documentate le richieste di modifiche.
Le attività coinvolte sono:

\begin{itemize}
  \item Implementazione del processo
  \item Identificazione della configurazione
  \item Controllo della configurazione
  \item Resoconto dello stato della configurazione
  \item Valutazione della configurazione
  \item Gestione dei rilasci e della consegna.
\end{itemize}

\subsubsection{Attività}%
\label{subs:gestione_della_configurazione/attivita}

\paragraph{Implementazione del processo}%
\label{par:gestione_della_configurazione/implementazione_del_processo}

Verrà sviluppato nella fase iniziale del progetto un piano di gestione della configurazione, espresso attraverso procedure documentate nel presente documento (§~\ref{subs:gestione_della_configurazione/procedure}).
I vari membri del gruppo saranno di volta in volta incaricati della gestione puntuale di aspetti specifici della configurazione del prodotto nel rispetto delle procedure definite.

% par:implementazione_del_processo (end)

\paragraph{Identificazione della configurazione}%
\label{par:identificazione_della_configurazione}

GruppOne stabilisce uno schema delle componenti software e delle loro versioni soggette a controllo (§~\ref{par:controllo_della_configurazione}), identificando per ogni componente la versione attuale associata al codice sorgente che la costituisce e la versione di riferimento della documentazione di progetto come documentato nella procedura §~\ref{par:rilascio_interno_di_versioni_dei_componenti}.

GruppOne identifica in maniera univoca le versioni delle componenti software del prodotto in conformità con la specifica \href{https://semver.org/spec/v2.0.0.html}{Semantic Versioning 2.0.0}.
Ciascuna versione delle componenti software fa inoltre riferimento ad una versione della documentazione di progetto come definita nel sotto-paragrafo §~\ref{subp:schema_di_versionamento_della_documentazione_di_progetto}.

\subparagraph{Schema di versionamento della documentazione di progetto}%
\label{subp:schema_di_versionamento_della_documentazione_di_progetto}

La versione della documentazione di progetto è rappresentata da un numero organizzato su 3 livelli (X.Y.Z) ma con significato non conforme alla specifica di versionamento semantico già citata.

\begin{itemize}
  \item Un aumento di Z corrisponde all'approvazione interna di un documento, che implica necessariamente la verifica da parte di minimo 2 membri del team.
  \item Un aumento di Y corrisponde al raggiungimento di una milestone come definita nel \textit{Piano di progetto (versione \versione)}. Le milestone comprendono quindi sia le revisioni di progetto che gli incrementi.
  \item Un aumento di X corrisponde al rilascio pubblico di una nuova versione \textit{non retro-compatibile} dell'API pubblica del prodotto Stalker. In questo contesto il numero 0 va interpretato come indicativo del fatto che non è ancora avvenuto alcun rilascio dell'API, in quanto incompleta.
\end{itemize}

GruppOne ha inoltre stabilito che il numero di versione così definito è rappresentativo sia della documentazione stessa che del prodotto nel suo complesso, considerato che esso riflette sia l'approvazione di modifiche a documenti che l'evoluzione nel tempo delle componenti software.

% subp:schema_di_versionamento_della_documentazione_di_progetto (end)

% TODO finire schema di versionamento di prodotto

% \subparagraph{Schema di versionamento del prodotto}%
% \label{subp:schema_di_versionamento_del_prodotto}

% Il numero di versione di prodotto che forniamo al committente è rappresentativo dei due aspetti complementari della documentazione di progetto e dell'evoluzione delle componenti software. Adottiamo una convenzione ragionata ed eventualmente soggetta a modifiche applicabili retroattivamente qualora si rivelasse insoddisfacente.

% Ad ogni rilascio esterno di materiale, GruppOne indicherà un numero di versione con la seguente forma:

% \begin{verbatim}
%   x.y.z+Ma.b-Sc.d-We.f
% \end{verbatim}

% % par:identificazione_della_configurazione (end)

\paragraph{Controllo della configurazione}%
\label{par:controllo_della_configurazione}

Verranno identificate e tracciate tutte le richieste di modifiche pervenute da fonti esterne (quali ad es.\ il committente) e interne, insieme all'evoluzione, implementazione e rilascio di tali modifiche.
La procedura è documentata nella sezione §~\ref{par:tracciamento_delle_richieste_di_modifiche}.

% par:controllo_della_configurazione (end)

\paragraph{Resoconto dello stato della configurazione}%
\label{par:resoconto_dello_stato_della_configurazione}

Considerati tutti i repository di progetto creati da GruppOne, la \glossario{baseline} di prodotto consiste in ogni momento in ciò che è presente nelle branch di default dei repository delle componenti e della documentazione di progetto, disponibile agli indirizzi:

\begin{description}
  \item \href{https://github.com/GruppOne/project-docs/tree/master}{GruppOne/project-docs/tree/master}
  \item \href{https://github.com/GruppOne/stalker-mobile-app/tree/master}{GruppOne/stalker-mobile-app/tree/master}
  \item \href{https://github.com/GruppOne/stalker-server/tree/master}{GruppOne/stalker-server/tree/master}
  \item \href{https://github.com/GruppOne/stalker-web-app/tree/master}{GruppOne/stalker-web-app/tree/master}
\end{description}

Dove il numero di versione è reperibile alle pagine indicate nella descrizione della procedura §~\ref{par:rilascio_interno_di_versioni_dei_componenti}.

% par:resoconto_dello_stato_della_configurazione (end)

\paragraph{Valutazione della configurazione}%
\label{par:valutazione_della_configurazione}

Nel corso del progetto verrà documentata all'interno del documento \textit{Piano di qualifica (versione \versione)} la copertura dei requisiti elencati nel documento \textit{Analisi dei requisiti (versione \versione)} rispetto all'implementazione effettiva dei componenti software richiesti.

% par:valutazione_della_configurazione (end)

\paragraph{Gestione dei rilasci e delle consegne}%
\label{par:gestione_dei_rilasci_e_delle_consegne}

Per mantenere una visione d'insieme del prodotto, GruppOne effettuerà in corrispondenza degli incrementi definiti nel \textit{Piano di progetto (versione \versione)} dei rilasci interni dei componenti del prodotto software in via di sviluppo, che andranno a costituire baseline incrementali. La procedura per effettuare i rilasci interni è documentata in §~\ref{par:rilascio_interno_di_versioni_dei_componenti}.

In occasione delle consegne di materiale richieste dal committente, la baseline più recente di ciascun componente software verrà resa disponibile al committente utilizzando la procedura definita in §~\ref{par:rilascio_esterno_di_versioni_del_prodotto}.

Il rilascio e consegna finale del prodotto compiuto avverrà attraverso un repository unico come documentato nella procedura §~\ref{par:consegna_del_prodotto}.

% par:gestione_dei_rilasci_e_della_consegna (end)

% \subsubsection{Metriche}%
% \label{subs:gestione_della_configurazione/metriche}

\subsubsection{Procedure}%
\label{subs:gestione_della_configurazione/procedure}

\paragraph{Tracciamento delle richieste di modifiche}%
\label{par:tracciamento_delle_richieste_di_modifiche}

Ogni richiesta di modifiche pervenuta al team viene documentata sulla piattaforma GitHub, creando una issue sul repository di ogni componente coinvolto dalla richiesta e applicandovi il label \textit{change request}.
Questa procedura è funzionalmente identica, a meno del label effettivamente selezionato, a quanto documentato in §~\ref{par:flusso_di_lavoro_per_la_risoluzione_dei_problemi}.

% par:tracciamento_delle_richieste_di_modifiche (end)

\paragraph{Rilascio interno di versioni dei componenti}%
\label{par:rilascio_interno_di_versioni_dei_componenti}

Il rilascio interno avviene attraverso il meccanismo delle \href{https://git-scm.com/book/en/v2/Git-Basics-Tagging}{git tag}, da apporre determinando il numero di versione che risulta dall'approvazione di un set di modifiche, che coincide con il \textit{merge} sulla branch di default del repository del lavoro effettuato su una \textit{feature/fix branch}. Tali rilasci interni sono visibili agli indirizzi:

\begin{description}
  \item \href{https://github.com/GruppOne/project-docs/tags}{GruppOne/project-docs/tags}
  \item \href{https://github.com/GruppOne/stalker-mobile-app/tags}{GruppOne/stalker-mobile-app/tags}
  \item \href{https://github.com/GruppOne/stalker-server/tags}{GruppOne/stalker-server/tags}
  \item \href{https://github.com/GruppOne/stalker-web-app/tags}{GruppOne/stalker-web-app/tags}
\end{description}

% par:rilascio_interno_di_versioni_dei_componenti (end)

\paragraph{Rilascio esterno di versioni del prodotto}%
\label{par:rilascio_esterno_di_versioni_del_prodotto}

In corrispondenza delle revisioni non bloccanti previste dal committente è necessario consegnare:

\begin{enumerate}
  \item Al committente prof. Tullio Vardanega la documentazione di progetto, attraverso una \href{https://help.github.com/en/enterprise/2.16/user/github/administering-a-repository/about-releases}{release} alla pagina \href{https://github.com/GruppOne/project-docs/releases}{GruppOne/project-docs/releases}.
  \item Al committente prof. Riccardo Cardin le componenti software e, in occasione della Product Baseline, un allegato tecnico.
\end{enumerate}

Ad entrambi i committenti verrà notificata l'avvenuta consegna attraverso email entro le scadenze prefissate.
Resta inteso che in ogni momento lo stato attuale dei componenti software, inclusivo delle modifiche non ancora approvate, è disponibile pubblicamente in conformità con la licenza richiesta a questi indirizzi:

\begin{description}
  \item \href{https://github.com/GruppOne/stalker-mobile-app/branches}{GruppOne/stalker-mobile-app/branches}
  \item \href{https://github.com/GruppOne/stalker-server/branches}{GruppOne/stalker-server/branches}
  \item \href{https://github.com/GruppOne/stalker-web-app/branches}{GruppOne/stalker-web-app/branches}
\end{description}

% par:rilascio_esterno_di_versioni_del_prodotto (end)

\paragraph{Consegna del prodotto}%
\label{par:consegna_del_prodotto}

La consegna finale del prodotto avverrà attraverso un repository unico che definisce univocamente la baseline conclusiva di prodotto e comprenderà dei puntatori a versioni specifiche delle componenti software e della documentazione di progetto, utilizzando il meccanismo dei \href{https://git-scm.com/book/en/v2/Git-Tools-Submodules}{submodule}, che verrà reso disponibile all'indirizzo:

\begin{center}
  \href{https://github.com/GruppOne/stalker}{https://github.com/GruppOne/stalker}
\end{center}

All'interno del file \verb|README.md| presente nella radice di tale repository verranno fornite le istruzioni di installazione del prodotto.

% par:consegna_del_prodotto (end)

\subsubsection{Strumenti di supporto}%
\label{subs:gestione_della_configurazione/strumenti_di_supporto}

\paragraph{Git}%
\label{par:git}

Il versionamento delle componenti del prodotto avviene utilizzando Git, e distribuendo le componenti software e la documentazione su un totale di 4 \glossario{repository} diversi.

Utilizziamo il workflow \href{https://trunkbaseddevelopment.com/}{Trunk Based Development} a livello dei singoli repository sia per la documentazione di progetto che per le componenti software, considerando come \glossario{trunk} la branch di default \textit{master}.

\subparagraph{Integrazione di modifiche nel trunk}%
\label{subp:integrazione_di_modifiche_nel_trunk}

Una parte fondamentale del flusso di lavoro da noi utilizzato è l'integrazione di piccoli gruppi di modifiche all'interno del trunk.
Questa parte viene realizzata isolando ciascun gruppo di modifiche in una branch dedicata e affidando la responsabilità di mantenere aggiornata la branch rispetto al trunk ad un membro del team chiaramente identificato come \textit{assignee} nella Pull request associata alla branch. È quindi necessario che l'\textit{assignee} effettui periodicamente dei \textit{merge} dalla branch \textit{master} verso la branch di sua responsabilità.

Infine, per mantenere linearità nell'evoluzione del trunk e facilitare l'individuazione di ciò che deve essere effettivamente soggetto a verifica, è necessario che l'\textit{assignee}, una volta completata l'implementazione delle modifiche, effettui un \textit{rebase} della branch sulla branch di default.

% subp:integrazione_di_modifiche_nel_trunk (end)

% par:git (end)

\paragraph{GitHub}%
\label{par:github}

I repository vengono condivisi tra i membri del gruppo attraverso la piattaforma online GitHub, in repository raggiungibili a partire dalla pagina dell'organizzazione GruppOne, all'indirizzo \href{https://github.com/GruppOne}{https://github.com/GruppOne}.

\subparagraph{Pull request}%
\label{subp:pull_request}

Le pull request sono uno spazio dove è possibile discutere aspetti e conseguenze delle modifiche introdotte da una branch. Utilizziamo in particolare la distinzione tra pull request e draft pull request (distinguibili dal colore dell'icona, rispettivamente verde e grigio) per esprimere il fatto che una branch ha completato l'implementazione del set di modifiche richieste, ed è dunque pronta per sottostare all'attività di verifica utilizzando lo strumento delle code review descritto in §~\ref{par:code_review}.

% subp:pull_request (end)

\subparagraph{Protezione della branch di default}%
\label{subp:protezione_della_branch_di_default}

GruppOne ha implementato una protezione condizionale sulla branch \textit{master}, che vieta ai membri del team di effettuare dei push diretti. Ogni aggiunta presente su master deve avvenire attraverso il meccanismo delle \glossario{Pull request}, che sono imprescindibilmente soggette sia a verifica effettuata seguendo la procedura documentata in §~\ref{par:code_review} che al superamento di check appropriati alle modifiche effettuate, implementati con il meccanismo descritto in §~\ref{par:github_actions}.

% subp:protezione_della_branch_di_default (end)

% par:github (end)

\paragraph{Conventional Changelog --- Standard Version}%
\label{par:conventional_changelog_standard_version}

Questo strumento, la cui documentazione di utilizzo è reperibile all'indirizzo \href{https://github.com/conventional-changelog/standard-version/blob/master/README.md}{Conventional Changelog - Standard Version}, permette la generazione automatica di un registro delle modifiche per ogni componente software sfruttando la categorizzazione dei commit imposta dalla specifica \href{https://www.conventionalcommits.org/en/v1.0.0/}{Conventional Commits 1.0.0}.

Tale registro è contenuto in un file \verb|CHANGELOG.md| presente nella radice del repository della componente considerata.
Viene aggiornato con una procedura semiautomatica documentata in~\ref{par:gestione_dei_rilasci_e_delle_consegne} dal responsabile che approva ciascuna Pull Request.

% par:conventional_changelog_standard_version (end)

\end{document}
