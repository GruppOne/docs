\documentclass[../../norme-di-progetto.tex]{subfiles}

\begin{document}

\subsubsection{Finalità}%
\label{subs:revisione/finalita}

GruppOne istanzia il processo di revisione per sostenere le revisioni di avanzamento previste nel corso del progetto didattico.

\subsubsection{Descrizione}%
\label{subs:revisione/descrizione}

Il processo viene utilizzato per determinare quanto il progetto sia svolto in allineamento con i requisiti, con la pianificazione e con il contratto.
Richiede che un revisore valuti le attività o i prodotti del gruppo revisionato.
Consiste delle seguenti attività:

\begin{itemize}
  \item Implementazione del processo
  \item Revisione.
\end{itemize}

\subsubsection{Attività}%
\label{subs:revisione/attivita}

\paragraph{Implementazione del processo}%
\label{par:revisione/implementazione_del_processo}

Parte dell'implementazione del processo è di responsabilità del revisore, che per GruppOne è il committente, e di conseguenza non controllabile direttamente.
Sono comunque garantite le seguenti condizioni:

\begin{itemize}
  \item Le revisioni sono svolte a milestone predeterminate specificate nel \textit{Piano di progetto (versione \versione)}.
  \item Il revisore non ha alcuna responsabilità diretta nel progetto di cui revisiona attività e prodotti.
  \item GruppOne è informato dal revisore con sufficiente anticipo riguardo le risorse necessarie alla revisione.
  \item Il revisore ha fissato gli obiettivi della revisione, i prodotti da revisionare, le modalità della revisione e le condizioni di entrata e di uscita della revisione.
  \item I problemi individuati durante la revisione devono essere inseriti nel processo di risoluzione dei problemi (§~\ref{sub:risoluzione_dei_problemi}).
  \item Dopo la revisione, il revisore deve informare GruppOne sui risultati. GruppOne deve riconoscere i problemi individuati e pianificarne la risoluzione.
\end{itemize}

\paragraph{Revisione}%
\label{par:revisione/revisione}
La revisione deve assicurare la validità delle seguenti condizioni:

\begin{itemize}
  \item I prodotti software riflettono la documentazione progettuale.
  \item I requisiti di verifica e test prescritti dalla documentazione sono adeguati.
  \item I risultati dei test sono conformi alle specifiche.
  \item I prodotti software sono stati testati e sono conformi alle specifiche.
  \item I risultati dei test sono corretti, ed eventuali discrepanze con i valori attesi sono state risolte.
  \item La documentazione per gli utenti rispetta gli standard.
  \item Le attività sono state svolte seguendo i requisiti, la pianificazione e il contratto.
  \item I costi rispettano quanto accordato.
\end{itemize}

% \subsubsection{Metriche}%
% \label{subs:revisione/metriche}

\subsubsection{Procedure}%
\label{subs:revisione/procedure}

Per la partecipazione alle revisioni GruppOne deve consegnare la documentazione richiesta dal revisore entro scadenze specificate dallo stesso. Per la consegna, GruppOne segue la procedura seguente:

\begin{enumerate}
  \item L'amministratore crea una nuova release nel repository della documentazione, mantenuto sulla piattaforma GitHub.
  \item L'amministratore associa alla release il tag relativo alla versione corrente dei documenti (se non esistente, lo crea).
  \item L'amministratore dà della release un titolo che indichi il nome della revisione a cui GruppOne è intenzionato a partecipare.
  \item L'amministratore inserisce nella descrizione eventuali informazioni aggiuntive.
  \item L'amministratore allega come asset un archivio in formato zip contenente i documenti in formato pdf, organizzati in directory che rispecchiano la struttura del repository.
  \item Contestualmente alla richiesta di partecipazione alla revisione, il responsabile consegna al committente anche un link alla release.
\end{enumerate}

% \subsubsection{Strumenti di supporto}%
% \label{subs:revisione/strumenti_di_supporto}

\end{document}
