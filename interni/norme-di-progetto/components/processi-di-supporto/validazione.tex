\documentclass[../../norme-di-progetto.tex]{subfiles}

\begin{document}

\subsubsection{Finalità}%
\label{subs:validazione/finalita}

GruppOne istanzia il processo di validazione per assicurarsi che il prodotto finale realizzato soddisfi pienamente le attese del committente.

\subsubsection{Descrizione}%
\label{subs:validazione/descrizione}

Dovendo verificare se il prodotto finale soddisfa i requisiti, anche il processo di validazione fa uso dell'analisi dinamica del codice e quindi di test ad hoc per dimostrare la correttezza del prodotto.

\subsubsection{Attività}%
\label{subs:attivita}

Il processo comprende due attività principali:

\begin{itemize}
\item implementazione del processo
\item validazione.
\end{itemize}

\paragraph{Implementazione del processo}%
\label{subs:validazione/implementazione}
Un piano di implementazione dovrebbe includere almeno:
\begin{itemize}
  \item Scelta dell'oggetto di validazione.
  \item Decisione dei compiti da eseguire per realizzare la validazione.
  \item Determinazione delle risorse e delle responsabilità.
  \item Procedure per fornire al cliente i report di validazione.
\end{itemize}

\paragraph{Validazione}
L'attività di validazione consiste nel preparare test sui requisiti e analizzare i loro risultati.

\paragraph{Test di accettazione}
I test di accettazione vengono effettuati in presenza del proponente in modo da verificare congiuntamente che siano stati soddisfatti tutti i requisiti espliciti o impliciti del prodotto.
Vengono effettuati nel collaudo, utilizzando anche i test di sistema.

\paragraph{Test di sistema}
Verificano, dopo aver passato con successo i test di integrazione, i requisiti del capitolato. I test sono trasversali su più componenti del sistema.
Tali test garantiscono, oltre ai casi d'uso del software, test di performance, stress test o test similari.
Essendo tutti i test derivanti da un gruppo di requisiti che hanno una determinata importanza nel prodotto software il team enumera i Test di Sistema nel modo seguente:
\begin{center}
    [TA][numero][tipo][priorità]
\end{center}
Dove: \textit{priorità} è un valore numerico ad indicare l'importanza del requisito che deve essere soddisfatto, con un numero da 1 a 3 \textit{numero} è un numero identificativo per il test e \textit{tipo} può assumere i seguenti valori:
\begin{description}
  \item [F]: requisito funzionale
  \item [P]: requisito prestazionale
  \item [V]: requisito vincolo
  \item [Q]: requisito qualità.
\end{description}


\end{document}
