\documentclass[../../norme-di-progetto.tex]{subfiles}

\begin{document}

\subsubsection{Finalità}%
\label{subs:verifica/finalita}

GruppOne istanzia il processo di verifica per certificare che il susseguirsi delle attività svolte dai membri del team non comporti l'introduzione di errori nei prodotti intermedi.
Si deve verificare qualsiasi prodotto intermedio, pertanto sia documentazione che codice sono soggetti a verifica.

\subsubsection{Descrizione}%
\label{subs:verifica/descrizione}

Il processo di verifica controlla e garantisce che i prodotti di una determinata attività soddisfino i requisiti e le condizioni che quel prodotto deve avere. Il processo di verifica si integra in toto con gli altri processi, in particolare con quello di sviluppo, e regolamenta revisione, analisi e test. È composto da due attività:

\begin{itemize}
  \item Implementazione del processo
  \item Verifica.
\end{itemize}

\subsubsection{Attività}%
\label{subs:verifica/attivita}

\paragraph{Implementazione del processo}%
\label{par:implementazione_di_processo}

GruppOne si impegna, nel contesto del progetto intrapreso, ad effettuare attività di verifica e a formulare un piano delle attività di verifica quantificando nel \textit{Piano di progetto (versione \versione)} l'impegno orario necessario in ogni fase rilevante del ciclo di vita del prodotto. Tale pianificazione verrà messa in atto utilizzando le procedure definite nel sotto-paragrafo §~\ref{subs:verifica/procedure}.

% par:implementazione_di_processo (end)

\paragraph{Verifica}%
\label{par:verifica}

Vista l'importanza dell'attività, ogni compito verrà descritto in un sotto-paragrafo a parte. I compiti coinvolti dall'attività di verifica sono:

\begin{itemize}
  \item verifica del contratto
  \item verifica del processo
  \item verifica dei requisiti
  \item verifica della progettazione
  \item verifica del codice
  \item verifica dell'integrazione
  \item verifica della documentazione.
\end{itemize}

\subparagraph{Verifica del contratto}%
\label{subp:verifica_del_contratto}

GruppOne effettua verifica dell'accuratezza e fattibilità di quanto delineato nei documenti a valore contrattuale consegnati al committente, in particolare per assicurarsi della soddisfazione dei criteri di accettazione per il prodotto come descritti dal committente nel capitolato d'appalto propostoci.

% subp:verifica_del_contratto (end)

\subparagraph{Verifica del processo}%
\label{subp:verifica_del_processo}

Per verificare l'adeguatezza dei processi istanziati, vista la mancanza di esperienza dei membri del team, GruppOne fa affidamento principalmente al feedback espresso dal committente nei colloqui e nelle revisioni ufficiali per assicurarsi un corretto way of working durante ogni fase del progetto intrapreso.

% subp:verifica_del_processo (end)

\subparagraph{Verifica dei requisiti}%
\label{subp:verifica_dei_requisiti}

GruppOne effettua verifica interna dell'adeguatezza, consistenza, fattibilità e testabilità dei singoli requisiti emersi dall'attività di analisi, come elencati nel documento \textit{Analisi dei requisiti (versione \versione)}.
La procedura è descritta nel paragrafo §~\ref{par:verifica_semantica_dei_prodotti_documentali}.

% subp:verifica_dei_requisiti (end)

\subparagraph{Verifica della progettazione}%
\label{subp:verifica_della_progettazione}

Questo compito serve ad assicurarsi che la progettazione sia ``consistente con'' che ``tracciabile ai'' requisiti del prodotto. Ciò avviene sia attraverso la procedura descritta nel paragrafo §~\ref{par:verifica_semantica_dei_prodotti_documentali} che con la produzione di documenti tecnici nelle fasi avanzate del progetto.

% subp:verifica_della_progettazione (end)

\subparagraph{Verifica del codice}%
\label{subp:verifica_del_codice}

La verifica del codice si divide in automatica e manuale.
La verifica manuale avviene attraverso lo strumento descritto nel paragrafo §~\ref{par:code_review}, mentre quella automatica è subordinata al corretto svolgimento dell'attività di scrittura dei test di unità, sistema e integrazione.
L'esecuzione effettiva dei test scritti è garantita attraverso lo strumento descritto nel paragrafo §~\ref{par:github_actions} e implementato come check obbligatorio per effettuare il \textit{merge} nel trunk, come descritto nel paragrafo §~\ref{subp:protezione_della_branch_di_default}.

% subp:verifica_del_codice (end)

\subparagraph{Verifica dell'integrazione}%
\label{subp:verifica_dell_integrazione}

Nel modello incrementale la verifica dell'integrazione avviene gradualmente attraverso la scrittura di test appositi come documentato nei paragrafi §~\ref{par:test_integrazione} e §~\ref{par:test_sistema}, che va effettuata in concomitanza con l'integrazione di modifiche descritta nel sotto-paragrafo §~\ref{subp:integrazione_di_modifiche_nel_trunk}.

% subp:verifica_dell'integrazione (end)

\subparagraph{Verifica della documentazione}%
\label{subp:verifica_della_documentazione}

GruppOne garantisce consistenza, adeguatezza e completezza della documentazione attraverso le procedure descritte nei paragrafi §~\ref{par:verifica_grammaticale_dei_prodotti_documentali} e §~\ref{par:verifica_semantica_dei_prodotti_documentali}.
Tutti i prodotti documentali sono adeguatamente versionati, come specificato nel paragrafo §~\ref{par:gestione_della_configurazione/implementazione_del_processo}.

% subp:verifica_della_documentazione (end)

% par:verifica (end)

\subsubsection{Procedure}%
\label{subs:verifica/procedure}

\paragraph{Verifica della correttezza grammaticale dei prodotti documentali}%
\label{par:verifica_grammaticale_dei_prodotti_documentali}

Di seguito si riporta una checklist da seguire per effettuare l'analisi statica dei documenti:

\begin{description}
  \item [Errori di ortografia] si devono individuare errori di punteggiatura e ortografia.
  \item [Errori nei verbi] si deve controllare che tutti i verbi siano in forma attiva.
  \item [Errori di sintassi] si deve verificare che non ci siano paragrafi di lunghezza eccessiva e che le frasi siano il più brevi possibile.
  \item [Errori negli elenchi puntati] si deve controllare che ogni elemento degli elenchi puntati inizi con la lettera maiuscola e che vengano rispettate le regole definite nel sotto-paragrafo §~\ref{subp:elenchi}.
  \item [Errori nella strutturazione del documento] si deve verificare che tutti i paragrafi si trovino nella sezione corretta, che le sezioni siano ben strutturate e che i titoli siano significativi.
\end{description}

% par:verifica_grammaticale_dei_prodotti_documentali (end)

\paragraph{Verifica semantica dei prodotti documentali}%
\label{par:verifica_semantica_dei_prodotti_documentali}

Il verificatore deve prestare particolare attenzione ad appurare la correttezza concettuale di quanto da lui letto. Per via della natura complessa di questo aspetto della verifica, GruppOne si raccomanda che il verificatore effettui la compilazione del sorgente \LaTeX{} ed almeno una lettura completa delle sezioni del documento di cui sta svolgendo la verifica, se non del documento nel suo complesso.

% par:verifica_dei_prodotti_documentali (end)

\paragraph{Procedure di verifica del codice}%
\label{par:procedure_di_verifica_del_codice}

% par:procedure_di_verifica_del_codice (end)

\subparagraph{Analisi statica}%
\label{subp:verifica/analisi_statica}

L'analisi statica consiste nell'insieme delle buone pratiche che valutano un componente del sistema sulla base della sua forma, struttura e contenuto.
Consiste in una serie di regole a cui tutti i prodotti intermedi del ciclo di vita del software devono attenersi.
Viene effettuata in maniera automatizzata utilizzando gli strumenti descritti nel paragrafo §~\ref{par:analisi_statica_e_dinamica}. L'adeguata impostazione di tali strumenti è richiesta ai verificatori ed è soggetta a controllo di configurazione.

\subparagraph{Test di regressione}%
\label{subp:test_di_regressione}

I test di regressione servono per accertare che le modifiche apportate non reintroducano errori nel software che erano stati risolti in precedenza.
Consistono nella ripetizione selettiva di test di unità, integrazione, test di sistema.

\subparagraph{Test di unità}%
\label{subp:test_di_unita}
I test di unità hanno lo scopo di verificare il corretto funzionamento di una singola unità software in modo automatico.
Per classificare questa tipologia di test andrà utilizzato un codice univoco come il seguente:
\begin{center}
  \textbf{TU-[identificativo componente]-[numero identificativo]}
\end{center}
dove:
\begin{description}
  \item[identificativo componente] può assumere tre valori distinti: \textbf{W} se il test è riferito alla componente web app, \textbf{M} se il test è riferito alla componente mobile, \textbf{S} se il test è riferito alla componente server.
  \item[numero identificativo] è un numero positivo incrementale che ha lo scopo di identificare univocamente il test di unità.
\end{description}

% subp:test_unita (end)

\subparagraph{Test di integrazione}%
\label{subp:test_di_integrazione}
I test di integrazione hanno lo scopo di verificare se un gruppo di unità software si interfaccia in modo corretto.
Per classificare questa tipologia di test andrà utilizzato un codice univoco come il seguente:
\begin{center}
  \textbf{TI-[numero identificativo]}
\end{center}
dove:
\begin{description}
  \item [numero identificativo] è un numero positivo incrementale che ha lo scopo di identificare univocamente il test di integrazione.
\end{description}

% subp:test_integrazione (end)

% subs:procedure (end)


\subsubsection{Metriche}%
\label{subs:verifica/metriche}

\paragraph{MPS-COC: Copertura del codice}% in questo caso NON ci va il "\@". chktex 13
\label{par:MPS-COC_copertura_codice}

La percentuale di copertura del codice o \textit{code coverage} indica la quantità di codice che è possibile esaminare attraverso una suite di test rispetto al codice totale dell'unità testata.
Nello specifico, la \textit{code coverage} può misurare quattro aspetti:
\begin{itemize}
  \item statement coverage: misura la percentuale di statement percorsi dai test
  \item branch coverage: misura la percentuale di diramazioni percorse dai test
  \item function coverage: misura la percentuale di funzioni percorse dai test
  \item line coverage: misura la percentuale di righe di codice percorsi dai test.
\end{itemize}

Il risultato rilevato ha i seguenti significati:
\begin{itemize}
  \item se il risultato è pari a 0\% significa che il codice scritto non è stato testato.
  \item se il risultato è compreso tra 0\% e 100\% significa che la il codice non è completamente testato e quindi la copertura è parziale.
  \item se il risultato è 100\% allora la copertura dei test è ottima e tutto il codice scritto è stato testato.
\end{itemize}



\paragraph{MPS-TPA: Percentuale di test superati}% in questo caso NON ci va il "\@". chktex 13
\label{par:MPS-TPA_test_passati}

La metrica indica la percentuale di test superati con successo in una specifica fase del progetto fino all data corrente.

La formula che calcola la percentuale dei test superati è la seguente:
\[
 v(t) = (tp / tnp) * 100
\]
dove:
\begin{description}
 \item[v(t):] percentuale di test superati. % chktex 36
 \item[tp:] numero di test superati.
 \item[tnp:] numero di test non superati.

\end{description}

Il risultato rilevato ha i seguenti significati:
\begin{itemize}
  \item se il risultato è pari a 0\% significa che nessun test viene eseguito con successo.
  \item se il risultato è compreso tra 0\% e 100\% significa che non tutti i test vengono superati con successo.
  \item se il risultato è 100\% allora tutti i test realizzati per il software sono andati a buon fine.
\end{itemize}




\paragraph{MPS-TNP: Percentuale di test non superati}% in questo caso NON ci va il "\@". chktex 13
\label{par:MPS-TNP_test_non_passati}

La metrica indica la percentuale di test non superati con successo in una specifica fase del progetto fino all data corrente.

La formula che calcola la percentuale dei test non superati è la seguente:
\[
 v(t) = (tnp / tp) * 100
\]
dove:
\begin{description}
 \item[v(t):] percentuale di test superati. % chktex 36
 \item[tp:] numero di test superati.
 \item[tnp:] numero di test non superati.

\end{description}

Il risultato rilevato ha i seguenti significati:
\begin{itemize}
  \item se il risultato è pari a 0\% allora tutti i test realizzati per il software sono andati a buon fine.
  \item se il risultato è compreso tra 0\% e 100\% significa che non tutti i test vengono superati con successo.
  \item se il risultato è 100\% allora nessun test realizzato per il software è andato a buon fine.
\end{itemize}



% subs:metriche (end)

\subsubsection{Strumenti di supporto}%
\label{subs:verifica/strumenti}

\paragraph{GitHub Code Review}%
\label{par:code_review}

Il processo di verifica avviene principalmente attraverso il meccanismo delle \href{https://github.com/features/code-review/}{Code Review} offerto da GitHub.
Lo stretto legame con lo strumento di versionamento permette di svolgere la verifica in maniera incrementale, procedendo di pari passo con l'aggiunta di contenuto sia ai documenti che al codice sorgente.

La procedura descritta nel sotto-paragrafo §~\ref{subp:integrazione_di_modifiche_nel_trunk} ha come conseguenza diretta il fatto che il set di modifiche disponibile nella scheda \textit{changes} della Pull request corrisponda esattamente con ciò che non è ancora stato verificato.

% par:code_review (end)

\paragraph{GitHub Actions}%
\label{par:github_actions}

GruppOne utilizza la feature \href{https://help.github.com/en/actions/automating-your-workflow-with-github-actions/about-github-actions}{GitHub Actions} al fine di automatizzare quante più attività di verifica possibili.

È infatti possibile definire o riutilizzare delle action per rendere meno onerose sia l'analisi statica che dinamica del codice, e contemporaneamente facilitare molte delle attività e dei compiti svolti dai membri del team.

La configurazione delle action si trova in file \glossario{YAML} posizionati nella cartella \verb|.github/workflows| del repository a cui vengono applicate.

% par:github_actions (end)

% subs:verifica/strumenti (end)

\end{document}
