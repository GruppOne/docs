\documentclass[../../norme-di-progetto.tex]{subfiles}

\begin{document}

\subsubsection{Finalità}%
\label{subs:risoluzione_dei_problemi/finalita}

GruppOne istanzia il processo di risoluzione dei problemi per affrontare in maniera strutturata ed efficiente i problemi riscontrati.

\subsubsection{Descrizione}%
\label{subs:risoluzione_dei_problemi/descrizione}

Il processo fornisce procedure e per analizzare e risolvere i problemi riscontrati e le non-conformità indipendentemente dall'origine.
Consiste nelle seguenti attività:

\begin{itemize}
  \item Implementazione del processo
  \item Risoluzione dei problemi.
\end{itemize}

\subsubsection{Attività}%
\label{subs:risoluzione_dei_problemi/attivita}

\paragraph{Implementazione del processo}%
\label{par:implementazione_del_processo}

Il processo di risoluzione dei problemi istanziato da GruppOne rispetta i seguenti requisiti:

\begin{enumerate}
  \item Ciascuna istanza del processo è isolata, in modo da assicurare che i problemi identificati siano riportati, risolti e che ne sia tracciato lo stato.
  \item È presente uno schema di categorizzazione e prioritizzazione dei problemi, in modo da facilitare l'analisi delle tendenze.
  \item Si effettua una verifica dell'avvenuta risoluzione di ogni problema, della corretta implementazione delle modifiche e dell'assenza di ulteriori problemi introdotti dalla risoluzione.
\end{enumerate}

\paragraph{Risoluzione dei problemi}%
\label{par:risoluzione_dei_problemi}

L'attività di risoluzione dei problemi viene intrapresa ove necessario, secondo le caratteristiche sopra descritte, a partire dalla stesura di un \glossario{ticket} contenente il report del problema, e identificando chiaramente chi ne è responsabile.

\subsubsection{Procedure}%
\label{subs:risoluzione_dei_problemi/procedure}

La procedura stabilita da GruppOne per riportare i problemi consiste in:

\begin{enumerate}
  \item Creare una branch apposita, o posizionarsi su una branch esistente e correlata.
  \item Creare una \textit{issue} che descrive il problema sulla piattaforma GitHub.
  \item Assegnare una o più categorie (\textit{label} e/o \textit{project}) rilevanti al problema.
  \item Selezionare uno o più \textit{assignee} tra i membri del team in base alla categoria individuata.
\end{enumerate}

Per la risoluzione effettiva del problema, il membro del team che ci sta lavorando deve:

\begin{enumerate}
  \item Rimuovere eventuali altri assignee dall'issue su GitHub.
  \item Se l'issue è presente in una o più \textit{Project Board}, marcarlo come \textit{In Progress}.
  \item Risolvere il problema, discutendone con altri membri ove necessario.
  \item Implementare la soluzione nel minimo numero possibile di commit, avendo cura di aggiornare lo stato dell'issue.
\end{enumerate}

La verifica dell'avvenuta risoluzione del problema avviene attraverso la Pull Request associata, come descritto in~\ref{par:code_review}.

I ticket vengono chiusi solo dopo la verifica e approvazione della modifica, quando la Pull Request viene integrata nella baseline.

\subsubsection{Strumenti di supporto}%
\label{subs:risoluzione_dei_problemi/strumenti_di_supporto}

Il tracciamento dei problemi riscontrati avviene direttamente nella piattaforma GitHub.
Lo schema delle categorie e delle priorità individuate da GruppOne è accessibile dalla lista delle issue. Per ogni label è presente una descrizione del suo intento.

\subsubsection{Metriche}%
\label{subs:risoluzione_dei_problemi/metriche}

% TODO sistemare questa metrica
\paragraph{MPS-RDP: Indice di risoluzione dei problemi}% in questo caso NON ci va il "\@". chktex 13
\label{par:MPS-RDP_indice_di_risoluzione_dei_problemi}

L'indice di risoluzione dei problemi misura il tempo di risoluzione dei problemi incontrati nel corso dello svolgimento del progetto.
Esso quantifica il tempo trascorso tra l'apertura della relativa issue su GitHub e la sua chiusura. Tale issue deve essere identificata da una label appropriata. Si calcola:
\[
  CS-AS
\]
dove:
\begin{description}
  \item [CS] giorno di chiusura dell'issue.
  \item [AS] giorno di apertura dell'issue.
\end{description}
% sub:risoluzione_dei_problemi (end)

\end{document}
