\documentclass[../../norme-di-progetto.tex]{subfiles}

\begin{document}

\subsubsection{Finalità}%
\label{subs:processo_di_risoluzione_dei_problemi/finalita}

GruppOne istanzia il processo di risoluzione dei problemi per affrontare sia i rischi che si sono manifestati concretamente che quelli non previsti, cercando di ridurre al minimo questi ultimi attraverso analisi e gestione dei rischi.

\subsubsection{Descrizione}%
\label{subs:processo_di_risoluzione_dei_problemi/descrizione}

Per effettuare il tracciamento dei problemi riscontrati utilizziamo gli issue di GitHub come \glossario{ticket}, facendo attenzione a utilizzare i label corretti (e.g\@. bug, invalid) in modo da distinguerli ed effettuarne una prioritizzazione adeguata.

L'attività di risoluzione dei problemi verrà istanziata ove necessario, identificando chiaramente chi ne è responsabile, che dovrà provvedere a riportare nel ticket corrispondente la soluzione applicata.

\subsubsection{Metriche di processo}%
\label{subs:processo_di_risoluzione_dei_problemi/metriche_di_processo}

% TODO scrivi metriche (velocità di risoluzione?)
\paragraph{MPS-RDP: Indice di risoluzione dei problemi}% in questo caso NON ci va il "\@". chktex 13
\label{par:MPS-RDP_indice_di_risoluzione_dei_problemi}

L'indice di risoluzione dei problemi misura il tempo di risoluzione dei problemi incontrati nel corso dello svolgimento del progetto.
Esso quantifica il tempo trascorso tra l'apertura della relativa issue su GitHub e la sua chiusura. Tale issue deve essere identificata da una label appropriata. Si calcola:
\[
  CS-AS
\]
dove:
\begin{description}
  \item [CS] giorno di chiusura dell'issue.
  \item [AS] giorno di apertura dell'issue.
\end{description}
% sub:risoluzione_dei_problemi (end)

\end{document}
