\documentclass[../norme-di-progetto.tex]{subfiles}

\begin{document}

\subsection{Scopo del documento}%
\label{sub:scopo_del_documento}

Il presente documento si rivolge ai membri di GruppOne.
Si vogliono definire le regole e le norme a cui ogni membro del gruppo deve attenersi al fine di ottenere un efficace way of working nel corso di ogni attività di processo.
Il documento verrà aggiornato nel corso del tempo, pertanto è ancora incompleto.

\subsection{Scopo del prodotto}%
\label{sub:scopo_del_prodotto}

Si vuole realizzare un prodotto software che permetta ad un organizzazione di monitorare le persone presenti in un luogo di interesse attraverso un'applicazione per smartphone.
Questo tracciamento può essere in forma autenticata o anonima a seconda della volontà dell'utente di identificarsi o meno nei confronti dell'organizzazione.

\subsection{Glossario}%
\label{sub:glossario}

L'uso di vocaboli tecnici e facilmente fraintendibili rende necessaria la realizzazione di un glossario che definirà i termini dai significati più travisabili presenti in ogni documento.
Per garantirne l'inequivocabilità, le parole che possono assumere un significato ambiguo sono evidenziate (ad es., \glossario{way of working}) e riportate nel documento \textit{Glossario} accompagnate da una breve definizione.

\subsection{Riferimenti}%
\label{sub:riferimenti}

Questi sono riferimenti generali a risorse online relative a standard e strumenti che abbiamo deciso di utilizzare.
All'interno del documento verranno dati dei riferimenti puntuali ove necessario.

\subsubsection{Normativi}%
\label{subs:riferimenti/normativi}

\begin{itemize}
  \item \href{https://www.math.unipd.it/~tullio/IS-1/2019/Dispense/L06.pdf}{Corso di Ingegneria del Software, slide gestione di progetto}
  \item \href{https://www.math.unipd.it/~tullio/IS-1/2019/Dispense/FC01.pdf}{Corso di Ingegneria del Software, slide amministrazione di progetto}
  \item \href{https://www.math.unipd.it/~tullio/IS-1/2019/Progetto/C5.pdf}{Capitolato d'appalto C5}
  \item \href{https://www.pearson.it/opera/pearson/0-6424-ingegneria_del_software}{Ingegneria del software - Ian Sommerville - decima edizione}.
  \item \href{https://www.math.unipd.it/~tullio/IS-1/2009/Approfondimenti/ISO_12207-1995.pdf}{Standard ISO 12207-1995}
\end{itemize}

\subsubsection{Informativi}%
\label{subs:riferimenti/informativi}

\begin{itemize}
  \item \href{https://www.latex-project.org/help/documentation/}{\LaTeX}
  \item \href{https://www.overleaf.com/learn/latex/Multi-file_LaTeX_projects#The_subfiles_package}{Subfiles}
  \item \href{https://plantuml.com/}{PlantUML}
  \item \href{https://git-scm.com/}{Git}
  \item \href{https://help.github.com/en}{GitHub help pages}
  \item \href{https://commitizen.github.io/cz-cli/}{Commitizen}
  \item \href{https://12factor.net/}{The 12-factor app}
  \item \href{https://www.joelonsoftware.com/2000/08/09/the-joel-test-12-steps-to-better-code/}{The Joel Test}
  \item \href{https://spec.openapis.org/oas/v3.0.2}{OpenAPI specification v3.0.2}
  \item \href{https://code.visualstudio.com/docs}{Visual Studio Code documentation}
  % \item \href{https://telegram.org/}{Telegram}
  \item \href{https://slack.com/intl/en-it/help}{Slack help pages}
  % \item \href{https://hangouts.google.com}{Hangouts}
\end{itemize}

\end{document}
