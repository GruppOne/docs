\documentclass[../norme-di-progetto.tex]{subfiles}

\begin{document}

\subsection{Scopo del documento}%
\label{sub:scopo_del_documento}

Il presente documento si rivolge ai membri di GruppOne.
Si vogliono definire le regole e le norme a cui ogni membro del gruppo deve attenersi al fine di ottenere un efficace way of working nel corso di ogni attività di processo.
Il documento verrà aggiornato nel corso del tempo, pertanto è ancora incompleto.

\subsection{Scopo del prodotto}%
\label{sub:scopo_del_prodotto}

Si vuole realizzare un'applicazione che permetta di monitorare le persone presenti in un determinato luogo che sono in possesso di uno smartphone (Android o Ios) e nelle condizioni di installare un'applicazione.
In particolare, l'obiettivo è quello di sviluppare un applicazione software in grado di segnalare ad un server l'entrata e l'uscita dall'area di interesse di ogni utente. La tracciatura può essere:
\begin{description}
  \item Anonima: se il soggetto è non autenticato all'organizzazione di riferimento.
  \item Autenticata: se il soggetto è autenticato all'organizzazione di riferimento.
\end{description}
Alla registrazione nell'applicazione si effettua una richiesta al server per poter ottenere la lista delle organizzazioni alle quali è possibile registrarsi. L'organizzazione, infatti, è il soggetto di interesse nella tracciatura degli utenti. Essa può possedere uno o più luoghi, ciascuno contrassegnato da opportune coordinate geografiche.

\subsection{Glossario}%
\label{sub:glossario}

L'uso di vocaboli tecnici e facilmente fraintendibili rende necessaria la realizzazione di un glossario che definirà i termini dai significati più travisabili presenti in ogni documento.
Per garantire l'inequivocabilità, le parole che possono assumere un significato ambiguo sono evidenziate (i.e., \glossario{way of working}) e riportate in \textit{Glossario.pdf} accompagnate da una definizione.

\subsection{Riferimenti}%
\label{sub:riferimenti}

\subsubsection{Normativi}%
\label{subs:riferimenti/normativi}

\begin{itemize}
  \item \href{https://www.math.unipd.it/~tullio/IS-1/2019/Dispense/L06.pdf}{Slide gestione di progetto}
  \item \href{https://www.math.unipd.it/~tullio/IS-1/2019/Dispense/FC01.pdf}{Slide amministrazione di progetto}
  \item \href{https://www.math.unipd.it/~tullio/IS-1/2009/Approfondimenti/ISO_12207-1995.pdf}{Standard ISO 12207-1995}
  \item \href{https://www.math.unipd.it/~tullio/IS-1/2019/Progetto/C5.pdf}{Capitolato d'appalto C5}
  \item \href{https://www.pearson.it/opera/pearson/0-6424-ingegneria_del_software}{Ingegneria del software - Ian Sommerville - decima edizione}.
\end{itemize}

\subsubsection{Informativi}%
\label{subs:riferimenti/informativi}

\begin{itemize}
  \item \href{https://www.latex-project.org/help/documentation/}{\LaTeX}
  \item \href{https://www.overleaf.com/learn/latex/Multi-file_LaTeX_projects#The_subfiles_package}{Subfiles}
  \item \href{https://git-scm.com/}{Git}
  \item \href{https://github.com/}{GitHub}
  \item \href{https://commitizen.github.io/cz-cli/}{Commitizen}
  \item \href{https://desktop.telegram.org/}{Telegram}
  \item \href{https://help.github.com/en/github/creating-cloning-and-archiving-repositories/about-code-owners}{Code owners}
  \item \href{https://slack.com/intl/en-it/}{Slack}
  \item \href{https://hangouts.google.com}{Hangouts}
  \item \href{https://plantuml.com/}{PlantUML}.
\end{itemize}

\end{document}
