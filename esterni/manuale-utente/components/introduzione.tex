\documentclass[../manuale-utente.tex]{subfiles}

\begin{document}
\subsection{Scopo del documento}%
\label{sub:scopo_del_documento}
Il presente \glossario{documento} ha lo scopo di illustrare in modo più chiaro possibile le procedure per l'utilizzo della \glossario{mobile app} e della \glossario{web application}.

\subsection{Scopo del prodotto}%
\label{sub:scopo_del_prodotto}
L'obiettivo del progetto è sviluppare un'applicazione mobile distribuita, seguendo il modello client/server.
Il client deve essere in grado di segnalare sia l'ingresso che l'uscita dell'utente dai luoghi (in modalità anonima o meno a seconda delle esigenze), i quali sono definiti dalle organizzazioni.
Il server deve fornire la possibilità di raccogliere ed analizzare i dati relativi alle organizzazioni.
In caso di utenti anonimi l'analisi riguarda solo una stima del numero totale di persone presenti in un dato momento.
In caso di utenti autenticati deve inoltre essere possibile effettuare query di monitoraggio specifiche.
In merito all'ottimizzazione della geolocalizzazione, è richiesto un report che esponga le scelte progettuali, le rispettive motivazioni e i test eseguiti per garantire la rilevazione sufficiente precisa della posizione, considerando le limitazioni dello smartphone.

\subsection{Glossario}%
\label{sub:glossario}
Per garantire che il manuale utente sia chiaro, tutti i termini che richiedono una spiegazione avranno una G a pendice, e saranno riportati con il loro significato in fondo al documento.

\subsection{Premessa}%
\label{sub:glossario}
Il documento non è da considerarsi completo, ma rappresenta solo un prototipo.

\end{document}
