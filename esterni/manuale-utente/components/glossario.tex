\documentclass[../manuale-utente.tex]{subfiles}

\begin{document}

\subsection{A}
\begin{description}
    \item[Action bar:] componente grafica di Android che permette di istanziare una barra orizzontale che contiene voci corrispondenti a singole funzionalità.
    \item[APK:] acronimo di Android Package, è un formato di file utilizzato per la distribuzione e l'installazione di applicazioni mobile sul sistema operativo Android.
\end{description}

\subsection{B}

\subsection{C}

\subsection{D}

\subsection{E}

\subsection{F}
\begin{description}
    \item[File manager] programma di un sistema operativo che fornisce un'interfaccia grafica per lavorare su un file system.
\end{description}

\subsection{G}

\subsection{H}

\subsection{I}

\subsection{J}

\subsection{K}

\subsection{L}
\begin{description}
  \item[LDAP:] acronimo di Lightweight Directory Access Protocol, LDAP è un protocollo d'accesso a servizi di directory, basato sul modello client-server, che opera su TCP/IP o su altre connessioni orientate al servizio.
\end{description}

\subsection{M}
\begin{description}
    \item[Manager:] in italiano \textit{gestore}, è un utente appartenente ad un'organizzazione abilitato a gestire i luoghi dell'organizzazione. Un singolo utente può essere gestore di più organizzazioni distinte.
\end{description}

\subsection{N}

\subsection{O}

\subsection{P}
\begin{description}
    \item[Placeholder:] breve suggerimento posto all'interno di caselle input di tipo testo, che sparisce una volta che la casella riceve il focus.
    \item[Play Store:] servizio di distribuzione digitale gestito e sviluppato da Google LLC\@.
    \item[Pop-up:] finestra a comparsa che compare automaticamente durante la navigazione web in un browser per specifiche operazioni.
\end{description}

\subsection{Q}

\subsection{R}
\begin{description}
    \item[Reverse geocoding:] processo per l'ottenimento dell'indirizzo di un luogo a partire dalle coordinate geografiche.
\end{description}

\subsection{S}
\begin{description}
    \item[Server LDAP:] server che utilizza il protocollo standard LDAP per l'autenticazione degli utenti ad un'organizzazione, se questa la prevede.
\end{description}

\subsection{T}
\begin{description}
  \item[Tap:] azione umana che consiste nel toccare rapidamente uno schermo touch screen per attivare un particolare comando.
  \item[Toggle:] pulsante di attivazione/disattivazione che consente all'utente di modificare un'impostazione binaria, ovvero tra due stati: true o false.
\end{description}

\subsection{U}

\subsection{V}
\begin{description}
    \item[Vendor:] azienda che fornisce un servizio o un prodotto standard.
    \item[Viewer:] in italiano \textit{visualizzatore}, è un utente appartenente ad un'organizzazione abilitato a visualizzare i luoghi dell'organizzazione e i report sui dati. Un singolo utente può essere visualizzatore di più organizzazioni distinte.
\end{description}

\subsection{W}

\subsection{X}

\subsection{Y}

\subsection{Z}

\end{document}
