\documentclass[../piano-di-progetto.tex]{subfiles}

\begin{document}
Per ogni fase svolta, il gruppo presenta il consuntivo delle ore effettivamente lavorate per ruolo, assieme ai corrispettivi costi, indicando le differenze rispetto ai totali riportati nel preventivo e mostrando le cause di tali differenze.
Differenze positive, negative o nulle rappresentano un carico di lavoro superiore, inferiore o uguale a quello preventivato.
\subsection{Fase di Analisi preliminare}%
\label{sub:consuntivo_di_periodo/fase_di_analisi_preliminare}
\begin{table}[H]
  \centering
  \rowcolors{2}{lightgray}{white!80!lightgray!100}
  \renewcommand{\arraystretch}{2}
  \begin{tabular}{c c c}
    \rowcolor{darkgray!90!}\color{white}{\textbf{Ruolo}} & \color{white}{\textbf{Totale ore}} & \color{white}{\textbf{Costo}} \\
    Re&42 (-5)&1 260,00€ (-150,00€)\\
    Am&56 (+3)&1 120,00€ (+60,00€)\\
    An&104 (0)&2 600,00€ (0,00€)\\
    Pt&21 (+9)&462,00€ (+198,00€)\\
    Pr&-&-\\
    Ve&49 (-5)&735,00€ (-75,00€)\\
    \textbf{Totale}&272 (+2)&6 177,00€ (+108,00€)\\
  \end{tabular}
  \caption{consuntivo di periodo per l'Analisi preliminare}%
~~\label{tab:consuntivo_di_periodo_analisi_preliminare}
\end{table}
Le differenze individuate si possono ricondurre all'inesperienza dei membri del gruppo, che ha causato errori di valutazione nel calcolo delle ore a preventivo e nel carico di lavoro da dedicare alle attività da svolgere.
I 108,00€ eccedenti il preventivo non hanno influenza sulla pianificazione delle fasi successive, in quanto compresi nelle fasi di investimento, e quindi non rendicontati.
% sub:fase_di_analisi_preliminare (end)
\subsection{Fase di Preparazione in entrata alla RR}%
\label{sub:fase_di_preparazione_in_entrata_alla_rr}
\begin{table}[H]
  \centering
  \rowcolors{2}{lightgray}{white!80!lightgray!100}
  \renewcommand{\arraystretch}{2}
  \begin{tabular}{c c c}
    \rowcolor{darkgray!90!}\color{white}{\textbf{Ruolo}} & \color{white}{\textbf{Totale ore}} & \color{white}{\textbf{Costo}} \\
    Re&-&-\\
    Am&24 (0)&480,00€ (0,00€)\\
    An&-&-\\
    Pt&-&-\\
    Pr&-&-\\
    Ve&16 (0)&240,00€ (0,00€)\\
    \textbf{Totale}&40 (0)&720,00€ (0,00€)\\
  \end{tabular}
  \caption{consuntivo di periodo per la Preparazione in entrata alla RR}%
~~\label{tab:consuntivo_di_periodo_preparazione_in_entrata_alla_rr}
\end{table}
La breve durata di questa fase ha facilitato il rispetto delle scadenze, permettendo di evitare discostamenti dalle ore pianificate.
% sub:fase_di_preparazione_in_entrata_alla_rr (end)
\subsection{Fase di Progettazione architetturale}%
\label{sub:consuntivo_di_periodo/fase_di_progettazione_architetturale}
\begin{table}[H]
  \centering
  \rowcolors{2}{lightgray}{white!80!lightgray!100}
  \renewcommand{\arraystretch}{2}
  \begin{tabular}{c c c}
    \rowcolor{darkgray!90!}\color{white}{\textbf{Ruolo}} & \color{white}{\textbf{Totale ore}} & \color{white}{\textbf{Costo}} \\
    Re&10 (+4)&300,00€ (+120,00€)\\
    Am&10 (+4)&200,00€ (+80,00€)\\
    An&10 (0)&250,00€ (0,00€)\\
    Pt&72 (+9)&1 584,00€ (+198,00€)\\
    Pr&64 (-7)&960,00€ (-105,00€)\\
    Ve&42 (-10)&630,00€ (-150,00€)\\
    \textbf{Totale}&208 (0)&3 924,00€ (+143,00€)\\
  \end{tabular}
  \caption{consuntivo di periodo per la Progettazione architetturale}%
~~\label{tab:consuntivo_di_periodo_progettazione_architetturale}
\end{table}
\subsubsection{Modifica alla pianificazione}%
\label{subs:modifica_alla_pianificazione}

Dopo la Revisione dei requisiti, il gruppo ha sospeso la propria attività per sostenere i rispettivi esami, portando ad un ritardo rispetto la pianificazione.
In particolare, la conseguenza principale è stata l'impossibilità di progettare e sviluppare il POC in tempo per svolgere il colloquio di Technology Baseline con il committente.
In accordo con il proponente, il giorno 18 febbraio 2020 GruppOne ha deciso di posporre la propria Revisione di progettazione, saltando la revisione di marzo, e ridistribuendo la propria pianificazione sfruttando il mese aggiuntivo.
Il gruppo ha inoltre colto la modifica alla pianificazione come opportunità per correggere l'errore formale sul minimo preventivo imposto dal committente.
Grazie ai tempi più rilassati, il gruppo ha scelto di pianificare gli incrementi ad un ritmo meno serrato, in particolare spostando i primi due al periodo precedente alla Revisione di progettazione.
Inoltre, il gruppo ha pianificato il lavoro su base settimanale, mantenendo i periodi e le fasi indicati in §\ref{sec:pianificazione} come semplice assembramento di alto livello volto a presentare al proponente ed al committente la distribuzione temporale delle attività.
Il gruppo ha infine pianificato degli incontri settimanali atti a fare il punto della situazione ed eventualmente a modificare la pianificazione in caso di discostamenti.

% subs:modifica_alla_pianificazione (end)
\subsubsection{Progettazione architetturale e POC}%
\label{subs:progettazione_architetturale_e_poc}

Nel corso della progettazione architetturale e dello sviluppo del POC, a causa di incomprensioni fra i membri del gruppo, abbiamo speso più tempo del necessario su aspetti che si era pianificato di abbozzare, e successivamente raffinare durante gli incrementi, e non abbastanza su aspetti che invece erano indispensabili alla produzione di un POC funzionante.
La discrepanza fra le ore pianificate e quelle realmente effettuate non è particolarmente accentuata in quanto parte del gruppo ha frainteso gli obiettivi della fase corrente, ma ha comunque svolto lavoro che risulterà utile nelle fasi successive, in particolare durante gli incrementi.
La differenza più consistente per quanto riguarda il ruolo di verificatore si può ricondurre in parte alla minore quantità di codice prodotto, ma è troppo accentuata perché questa sia l'unica causa.
Di conseguenza, per le prossime fasi le ore da verificatore pianificate saranno ridotte.

% subs:progettazione_architetturale_e_poc (end)
% sub:fase_di_progettazione_architetturale (end)
\end{document}
