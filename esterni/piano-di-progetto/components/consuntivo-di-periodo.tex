\documentclass[../piano-di-progetto.tex]{subfiles}

\begin{document}
Per ogni fase svolta, il gruppo presenta il consuntivo delle ore effettivamente lavorate per ruolo, assieme ai corrispettivi costi, indicando le differenze rispetto ai totali riportati nel preventivo. Differenze positive, negative o nulle rappresentano un carico di lavoro superiore, inferiore o uguale a quello preventivato.
\subsection{Fase di Analisi preliminare}%
\label{sub:fase_di_analisi_preliminare}
\begin{table}[H]
  \centering
  \rowcolors{2}{lightgray}{white!80!lightgray!100}
  \renewcommand{\arraystretch}{2}
  \begin{tabular}{c c c}
    \rowcolor{darkgray!90!}\color{white}{\textbf{Ruolo}} & \color{white}{\textbf{Totale ore}} & \color{white}{\textbf{Costo}} \\
    Re&42 (-5)&1 260,00€ (-150,00€)\\
    Am&56 (+3)&1 120,00€ (+60,00€)\\
    An&104 (0)&2 600,00€ (0,00€)\\
    Pt&21 (+9)&462,00€ (+198,00€)\\
    Pr&-&-\\
    Ve&49 (-5)&735,00€ (-75,00€)\\
    \textbf{Totale}&272 (+2)&6 177,00€ (+108,00€)\\
  \end{tabular}
  \caption{consuntivo di periodo per l'Analisi preliminare}%
~~\label{tab:consuntivo_di_periodo_analisi_preliminare}
\end{table}
I 108,00€ eccedenti il preventivo non hanno influenza sulla pianificazione delle fasi successive, in quanto compresi nelle fasi di investimento, e quindi non rendicontati.
% sub:fase_di_analisi_preliminare (end)
\end{document}
