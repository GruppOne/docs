\documentclass[../piano-di-progetto.tex]{subfiles}


\begin{document}
	\subsection{Scopo del documento}
  \label{subsec:scopo_del_documento}
  Questo documento ha come obbiettivo la definizione delle modalità e delle tempistiche con cui il gruppo \textbf{Gruppone} vuole realizzare il progetto \textbf{Stalker}.
  Il documento si compone delle seguenti parti:
  \begin{itemize}
      \item Definizione del Modello di sviluppo;
      \item Analisi dei Rischi;
      \item Pianificazione delle fasi di progetto;
      \item Preventivo delle fasi di progetto;
      \item Consuntivo della fase di analisi e conclusioni.
  \end{itemize}
  In questo documento le parole con un significato specifico verranno contrassegnate con una 'G' a pedice, si rimanda al documento \textit{Glossario} per la definizione delle suddette parole.
  % subsection scopo_del_documento (end)
  \subsection{Riferimenti}
  \label{subsec:riferimenti}

    \subsubsection{Normativi}
    \label{subsubsec:normativi}
    \begin{itemize}
      \item  \textbf{Norme di Progetto:} Norme di Progetto v1.0.0;
      \item  \textbf{Organigramma e specifica tecnico-economica:} \url{https://www.math.unipd.it/~tullio/IS-1/2018/Progetto/RO.html}.
    \end{itemize}
    % subsection normativi (end)
    \subsubsection{Informativi}
    \begin{itemize}
      \item \textbf{Analisi dei Requisiti:} Analisi dei Requisiti 
    \end{itemize}
    \label{subsubsec:informativi}
    % subsection informativi (end)
  % subsection riferimenti (end)
  \subsection{Scandenze fissate}
  \label{subsubsec:scadenze_fissate}
  % subsection scadenze_fissate (end)
  \subsection{Modello di sviluppo}
  \label{subsec:modello_di_sviluppo}
  % subsection modello_di_sviluppo (end)
  
  \end{document}