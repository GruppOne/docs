\documentclass[../piano-di-progetto.tex]{subfiles}

\begin{document}
\subsection{Scopo del documento}%
\label{sub:scopo_del_documento}
Questo documento ha come obiettivo la definizione delle modalità e delle tempistiche con cui il gruppo \textbf{Gruppone} vuole realizzare il progetto \textbf{Stalker}.
Il documento si compone delle seguenti parti:
\begin{itemize}
  \item Analisi dei Rischi.
  \item Definizione del Modello di sviluppo.
  \item Pianificazione delle fasi di progetto.
  \item Preventivo delle fasi di progetto.
  \item Consuntivo della fase di analisi e conclusioni.
\end{itemize}
% sub:scopo_del_documento (end)

\scopoDelProdotto{}

\subsection{Glossario}%
\label{sub:glossario}
Al fine di rendere il documento più chiaro possibile, i termini che possono assumere un significato ambiguo sono evidenziati (i.e., \glossario{server}) e riportati nel \textit{Glossario} accompagnati da una definizione dettagliata.
% sub:glossario (end)
\subsection{Riferimenti}%
\label{sub:riferimenti}
\subsubsection{Normativi}%
\label{subs:normativi}
\begin{itemize}
  \item \textit{Norme di progetto (versione \versione)}.
  \item \href{https://www.math.unipd.it/~tullio/IS-1/2019/Progetto/RO.html}{Organigramma e specifica tecnico-economica}.
\end{itemize}
% subs:normativi (end)
\subsubsection{Informativi}%
\label{subs:informativi}
\begin{itemize}
  \item \href{https://www.math.unipd.it/~tullio/IS-1/2019/Progetto/C5.pdf}{Capitolato d'appalto C5}.
  \item \href{https://www.math.unipd.it/~tullio/IS-1/2019/Dispense/L05.pdf}{Corso di Ingegneria del Software - Lezione 5}, diapositive da 7 a 19.
  \item \href{https://www.math.unipd.it/~tullio/IS-1/2019/Dispense/L06.pdf}{Corso di Ingegneria del Software - Lezione 6}, diapositive da 1 a 35.
  \item \textit{Software Engineering --- Ian Sommerville --- 2015}, capitolo 19 e 20.
\end{itemize}%
% subs:informativi (end)
% sub:riferimenti (end)
\subsection{Scadenze fissate}%
\label{sub:scadenze_fissate}
Con questa sezione il gruppo decide di impegnarsi a partecipare alle seguenti revisioni nello sviluppo del progetto \textit{Stalker}:
  \begin{description}
    \item[Revisione dei Requisiti:] 21/01/2020
    \item[Revisione di Progettazione:] 20/04/2020
    \item[Revisione di Qualifica:] 18/05/2020
    \item[Revisione di Accettazione:] 18/06/2020.
  \end{description}
% sub:scadenze_fissate (end)
\end{document}
