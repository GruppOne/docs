\documentclass[../piano-di-progetto.tex]{subfiles}

\begin{document}
\subsection{Scopo del documento}%
\label{sub:scopo_del_documento}
Questo documento ha come obbiettivo la definizione delle modalità e delle tempistiche con cui il gruppo \textbf{Gruppone} vuole realizzare il progetto \textbf{Stalker}.
Il documento si compone delle seguenti parti:
\begin{itemize}
  \item Definizione del Modello di sviluppo;
  \item Analisi dei Rischi;
  \item Pianificazione delle fasi di progetto;
  \item Preventivo delle fasi di progetto;
  \item Consuntivo della fase di analisi e conclusioni.
\end{itemize}
% sub:scopo_del_documento (end)
\subsection{Glossario}%
\label{sub:glossario}
Al fine di rendere il documento più chiaro possibile, i termini che possono assumere un significato ambiguo sono evidenziati (i.e., \glossario{client}) e riportati in \textit{Glossario1.0.0.pdf} accompagnati da una definizione.
% sub:glossario (end)
\subsection{Riferimenti}%
\label{sub:riferimenti}
\subsubsection{Normativi}%
\label{subs:normativi}
\begin{itemize}
  \item  \textbf{Norme di Progetto:} Norme di Progetto v1.0.0;
  \item  \textbf{Organigramma e specifica tecnico-economica:} \url{https://www.math.unipd.it/~tullio/IS-1/2018/Progetto/RO.html}.
\end{itemize}
% subs:normativi (end)
\subsubsection{Informativi}%
\label{subs:informativi}
\begin{itemize}
  \item \textbf{Analisi dei Requisiti:} Analisi dei Requisiti
\end{itemize}
% subs:informativi (end)
% sub:riferimenti (end)
\subsection{Scadenze fissate}%
\label{sub:scadenze_fissate}

% sub:scadenze_fissate (end)
\end{document}
