\documentclass[../piano-di-progetto.tex]{subfiles}

\begin{document}
Dopo le considerazioni in sezione §~\ref{sec:consuntivo_di_periodo} il gruppo presenta il preventivo a finire alla conclusione di ciascuna fase.
\subsection{Fase di Analisi preliminare}%
\label{sub:fase_di_analisi_preliminare_2}
% il _2 nel label serve per distinguere questa subsection da quella con
% lo stesso nome nel consuntivo di periodo. Provvisorio, finché non
% prendiamo una decisione su queste situazioni.
Viste le cause nelle differenze individuate in sottosezione §~\ref{sub:fase_di_analisi_preliminare}, il preventivo a finire rimane coincidente con quanto dichiarato in sezione §~\ref{sec:preventivo}. In particolare, l'inesperienza dei membri del gruppo rimane un fattore significativo: anche l'esperienza acquisita nel corso di questa fase non permetterebbe di migliorare le stime presentate a preventivo, considerati i diversi ambiti su cui vertono le fasi successive.
Infatti, mentre in questa fase le attività riguardano quasi interamente gli ambiti decisionale ed organizzativo, nelle prossime riguarderanno maggiormente gli ambiti progettuale e produttivo.
Il costo delle ore rendicontate rimane quindi pari a quanto dichiarato a preventivo, mentre il costo totale previsto delle ore è 21~909€.
% sub:fase_di_analisi_preliminare (end)
\subsection{Fase di Preparazione in entrata alla RR}%
\label{sub:preventivo_a_finire/fase_di_preparazione_in_entrata_alla_rr}
Data l'assenza di discostamenti del consuntivo di periodo dal preventivo, il preventivo a finire rimane come indicato per la fase precedente.
% sub:fase_di_preparazione_in_entrata_alla_rr (end)
\subsection{Fase di Progettazione architetturale}%
\label{sub:preventivo_a_finire/fase_di_progettazione_architetturale}

Viste le considerazioni riportate in §\ref{subs:progettazione_architetturale_e_poc}, GruppOne ha deciso di modificare la ripartizione del lavoro fra gli incrementi.
In particolare, è ridotto il carico di lavoro richiesto per gli incrementi 3 e 6.
Per quanto riguarda le ore da verificatore, il gruppo ha deciso di ridurre il carico di ciascun componente di 0,5 ore per incremento, portando ad una riduzione di 60,00€ per incremento, per un risparmio totale di 420,00€.
Considerati i problemi insorti nella fase corrente, GruppOne ha deciso di mantenere comunque il preventivo a finire dichiarato per la fase precedente e mantenere questo credito, risultante in 277,00€, per mitigare simili evenienze in futuro.
Ferma restante l'inesperienza dei membri del gruppo, se questo credito non sarà esaurito entro la Revisione di qualifica, GruppOne provvederà a discutere con il proponente un nuovo preventivo più basso.

% sub:fase_di_progettazione_architetturale (end)
\subsection{Fase di Incremento 1}%
\label{sub:preventivo_a_finire/fase_di_incremento_1}

Grazie al credito accumulato dalla riduzione nelle ore Verificatore, il costo in eccesso risultato in questa fase non influenza il preventivo a finire.
D'altra parte, il ritardo dello sviluppo rispetto la pianificazione comporta un carico di lavoro maggiore per la prossima fase, durante la quale il gruppo conta di completare gli obiettivi dell'Incremento 1 e svolgere quelli dell'Incremento 2, oltre a svolgere le attività di Preparazione in entrata alla RP\@.
Di conseguenza, per la prossima fase GruppOne prevede 5 ore Progettista e 4 ore Programmatore aggiuntive, per un costo di 170,00€ in più.
Nel complesso, il preventivo a finire rimane pari a quanto dichiarato nella fase precedente, con un credito di 101,00€.

% sub:fase_di_incremento_1 (end)
\subsection{Fase di Incremento 2 e preparazione RP}%
\label{sub:preventivo_a_finire/fase_di_incremento_2 e preparazione RP}

Il gruppo ha deciso di ripartire una parte del lavoro da svolgere nell'incremento 2 nella settimana successiva per migliorare la qualità della consegna dei documenti che era prevista alla fine dell'incremento 2.
Durante questo periodo infatti, anche grazie a questa nuova ripartizione il gruppo è riuscito a recuperare buona parte del debito accumulato nel precedente incremento e questo lo rende facilmente recuperabile nei prossimi incrementi.
Nella fase precedente il gruppo ha inoltre previsto di ridurre le ore da verificatore per ogni fase.
Durante questo incremento infatti le ore da verificatore preventivate sono state rispettate.
Abbiamo riscontrato una eccedenza di 4 ore nel ruolo programmatore e una carenza di 3 ore progettista e 2 ore di responsabile.
Vista la correttezza nella nuova pianificazione delle ore verificatore il gruppo ha deciso di mantenere per il prossimo incremento la stessa ripartizione delle ore.
Nel complesso, il preventivo a finire rimane pari a quanto dichiarato in precedenza seppur con un debito di 16,00€.

% sub:fase_di_incremento_2_preparazione_RP (end)

\subsection{Fase di Incremento 3}%
\label{sub:preventivo_a_finire/fase_di_incremento_3}

In seguito al costante difetto delle ore amministratore e responsabile effettivamente utilizzate il gruppo ha deciso di diminuire le ore previste per questi ruoli di 2, portando quindi il numero di ore preventivate a 4 per ognuno dei ruoli.
La motivazione di questa decisione è che la pianificazione sistematica e dettagliata a inizio incremento lascia poco spazio per i ruolo di responsabile e amministratore.
Inoltre si è rivelato sempre meno necessario in quanto la Continuous Integration messa a disposizione del gruppo è risultata solida e non necessita per ora di lavoro aggiuntivo.
Il credito così ottenuto rispetto al preventivo viene convertito in ore programmatore che si stanno rivelando sempre più predominanti negli incrementi.
Viene confermata ulteriormente la modifica alle ore verificatore, che resteranno 12 per il prossimo incremento.
Complessivamente il preventivo a finire rimane pari quanto dichiarato in precedenza con un credito di 29,00€.
Notando che la differenza tra le ore preventivate e quelle effettive è sempre lieve non ridiscuteremo la pianificazione con proponente e committente ma ci impegniamo a rispettare il preventivo presentato in sede di revisione dei requisiti.


% sub:fase_di_incremento_3 (end)

\subsection{Fase di Incremento 4}%
\label{sub:preventivo_a_finire/fase_di_incremento_4}

Questa nuova suddivisione delle ore ha permesso al gruppo di lavorare molto bene e di restare in pari con gli obiettivi stabiliti.
In questo incremento inoltre non si sono verificati discostamenti significativi rispetto alla pianificazione.
Al netto di queste considerazioni GruppOne conferma le precedenti modifiche anche per il prossimo incremento.
Nel prossimo incremento, come per l'incremento 2, il gruppo ha deciso di ripartire gli obiettivi nella settimana successiva di preparazione alla revisione in quanto il tempo a disposizione durante quest'ultima ci permetterebbe di completare al meglio gli obiettivi dell'incremento.
Nel complesso, il preventivo a finire rimane invariato, con un debito di 1,00€.

\subsection{Fase di Incremento 5 e Preparazione in entrata alla RQ}%
\label{sub:preventivo_a_finire/fase_di_incremento_5}

In questo incremento il gruppo ha implementato una parte cospicua dei requisiti e si è contemporaneamente preparato per la revisione di qualifica.
La suddivisione del lavoro preventivata non ha previsto abbastanza ore da amministratore e programmatore, che si sono rivelate insufficienti in questo periodo.
Nel complesso, il preventivo a finire rimane invariato, con un debito di 36,00€.

\subsection{Fase di Incremento 6}%
\label{sub:preventivo_a_finire/fase_di_incremento_6}

Durante l'incremento 6 il gruppo si è concentrato sulla pianificazione delle parti rimanenti e in parte sulla loro implementazione.
Non si sono verificati discostamenti orari evidenti in questo periodo e quindi il gruppo ha deciso di non modificare ulteriormente la pianificazione futura.
Nel complesso, il preventivo a finire rimane invariato, con un debito di 15,00€.


\subsection{Fase di Incremento 7}%
\label{sub:preventivo_a_finire/fase_di_incremento_7}

in quest'ultimo incremento i membri del gruppo hanno concentrato i loro sforzi nel terminare le feature del prodotto ancora non complete.
Di conseguenza le ore programmatore sono risultate insufficienti rispetto a quelle preventivate, tuttavia questo debito viene compensato dalle ore progettista non utilizzate.
Questa differenza rispetto alla pianificazione è dovuta alla mancanza di elementi da progettare, già terminati negli incrementi precedenti, mentre mancavano ancora componenti da programmare.
Nel complesso, il preventivo a finire rimane invariato, con un credito di 42,00€.

\subsection{Fase di verifica e collaudo}%
\label{sub:preventivo_a_finire/fase_di_incremento_4}

In questo periodo il lavoro del gruppo è proseguito senza intoppi e discrepanze rispetto alla pianificazione.
Nel complesso, il preventivo a finire rimane invariato, con un credito di 27,00€.

\subsection{Fase di preparazione in entrata alla RA}%
\label{sub:preventivo_a_finire/fase_di_incremento_4}

Durante quest'ultimo periodo il gruppo ha preparato la presentazione e la dimostrazione del prodotto finora realizzato ponendo particolare attenzione alla qualità e alla completezza di quest'ultima.
Il lavoro non ha avuto problemi e non ha quindi generato discrepanze rispetto alla pianificazione.
Nel complesso, il preventivo a finire rimane invariato, con un credito di 20,00€.


% sub:fase_di_incremento_4 (end)

\end{document}
