\documentclass[../piano-di-progetto.tex]{subfiles}

\begin{document}
Dopo le considerazioni in sezione §~\ref{sec:consuntivo_di_periodo} il gruppo presenta il preventivo a finire alla conclusione di ciascuna fase.
\subsection{Fase di Analisi preliminare}%
\label{sub:fase_di_analisi_preliminare_2}
% il _2 nel label serve per distinguere questa subsection da quella con
% lo stesso nome nel consuntivo di periodo. Provvisorio, finché non
% prendiamo una decisione su queste situazioni.
Viste le cause nelle differenze individuate in sottosezione §~\ref{sub:fase_di_analisi_preliminare}, il preventivo a finire rimane coincidente con quanto dichiarato in sezione §~\ref{sec:preventivo}. In particolare, la nostra inesperienza rimane un fattore significativo: anche l'esperienza acquisita nel corso di questa fase non ci permetterebbe di migliorare le stime presentate a preventivo, considerati i diversi ambiti su cui vertono le fasi successive.
Infatti, mentre in questa fase le attività riguardano quasi interamente gli ambiti decisionale ed organizzativo, nelle prossime riguarderanno maggiormente gli ambiti progettuale e produttivo.
Il costo delle ore rendicontate rimane quindi pari a quanto dichiarato a preventivo, mentre il costo totale previsto delle ore è 21~909€.
% sub:fase_di_analisi_preliminare (end)
\subsection{Fase di Preparazione in entrata alla RR}%
\label{sub:preventivo_a_finire/fase_di_preparazione_in_entrata_alla_rr}
Data l'assenza di discostamenti del consuntivo di periodo dal preventivo, il preventivo a finire rimane come indicato per la fase precedente.
% sub:fase_di_preparazione_in_entrata_alla_rr (end)
\end{document}
