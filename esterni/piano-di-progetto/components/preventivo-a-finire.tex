\documentclass[../piano-di-progetto.tex]{subfiles}

\begin{document}
Dopo le considerazioni in sezione §~\ref{sec:consuntivo_di_periodo} il gruppo presenta il preventivo a finire alla conclusione di ciascuna fase.
\subsection{Fase di Analisi preliminare}%
\label{sub:fase_di_analisi_preliminare_2}
% il _2 nel label serve per distinguere questa subsection da quella con
% lo stesso nome nel consuntivo di periodo. Provvisorio, finché non
% prendiamo una decisione su queste situazioni.
Viste le cause nelle differenze individuate in sottosezione §~\ref{sub:fase_di_analisi_preliminare}, il preventivo a finire rimane coincidente con quanto dichiarato in sezione §~\ref{sec:preventivo}. In particolare, la nostra inesperienza rimane un fattore significativo: anche l'esperienza acquisita nel corso di questa fase non ci permetterebbe di migliorare le stime presentate a preventivo, considerati i diversi ambiti su cui vertono le fasi successive.
Infatti, mentre in questa fase le attività riguardano quasi interamente gli ambiti decisionale ed organizzativo, nelle prossime riguarderanno maggiormente gli ambiti progettuale e produttivo.
Il costo delle ore rendicontate rimane quindi pari a quanto dichiarato a preventivo, mentre il costo totale previsto delle ore è 21~909€.
% sub:fase_di_analisi_preliminare (end)
\subsection{Fase di Preparazione in entrata alla RR}%
\label{sub:preventivo_a_finire/fase_di_preparazione_in_entrata_alla_rr}
Data l'assenza di discostamenti del consuntivo di periodo dal preventivo, il preventivo a finire rimane come indicato per la fase precedente.
% sub:fase_di_preparazione_in_entrata_alla_rr (end)
\subsection{Fase di Progettazione architetturale}%
\label{sub:preventivo_a_finire/fase_di_progettazione_architetturale}

Viste le considerazioni riportate in §\ref{subs:progettazione_architetturale_e_poc}, GruppOne ha deciso di modificare la ripartizione del lavoro fra gli incrementi.
In particolare, è ridotto il carico di lavoro richiesto per gli incrementi 3 e 6.
Per quanto riguarda le ore da verificatore, il gruppo ha deciso di ridurre il carico di ciascun componente di 0,5 ore per incremento, portando ad una riduzione di 60,00€ per incremento, per un risparmio totale di 420,00€.
Considerati i problemi insorti nella fase corrente, GruppOne ha deciso di mantenere comunque il preventivo a finire dichiarato per la fase precedente, e mantenere questo credito, risultante in 277,00€, per mitigare simili evenienze in futuro.
Ferma restante l'inesperienza dei membri del gruppo, se questo credito non sarà esaurito entro la Revisione di qualifica, GruppOne provvederà a discutere con il proponente un nuovo preventivo più basso.

% sub:fase_di_progettazione_architetturale (end)
\subsection{Fase di Incremento 1}%
\label{sub:preventivo_a_finire/fase_di_incremento_1}

Grazie al credito accumulato dalla riduzione nelle ore Verificatore, il costo in eccesso risultato in questa fase non influenza il preventivo a finire.
D'altra parte, il ritardo dello sviluppo rispetto la pianificazione comporta un carico di lavoro maggiore per la prossima fase, durante la quale il gruppo conta di completare gli obiettivi dell'Incremento 1 e svolgere quelli dell'Incremento 2, oltre a svolgere le attività di Preparazione in entrata alla RP\@.
Di conseguenza, per la prossima fase GruppOne prevede 5 ore Progettista e 4 ore Programmatore aggiuntive, per un costo di 170,00€ in più.
Nel complesso, il preventivo a finire rimane pari a quanto dichiarato nella fase precedente, con un credito di 101,00€.

% sub:fase_di_incremento_1 (end)
% \subsection{Fase di Incremento 2}%
% \label{sub:preventivo_a_finire/fase_di_incremento_2}

% % sub:fase_di_incremento_2 (end)
\end{document}
