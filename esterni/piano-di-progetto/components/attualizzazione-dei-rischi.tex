\documentclass[../piano-di-progetto]{subfiles}

\begin{document}
Per ogni fase riportiamo i rischi incontrati nella fase di analisi, le modalità utilizzate per affrontarli e come proponiamo di gestire i suddetti rischi in futuro.

\subsection{Attualizzazione per la fase di analisi preliminare}%
\label{sub:attualizzazione_fase_analisi_preliminare}
% Nella tabella sottostante riportiamo i rischi incontrati nella fase di analisi, le modalità utilizzate per affrontarli e come proponiamo di gestire i suddetti rischi in futuro:
% \renewcommand{\arraystretch}{2}
% \begin{longtable}[H]{m{10em} m{17em} m{17em}}
\begin{longtable}[H]{|p{10em}|p{17em}|p{17em}|}
  \caption{Attualizzazione dei rischi per la fase di analisi preliminare}%
  \label{tab:attualizzazione_fase_analisi_preliminare}                                                                                                                                                                                                                                                                                                                                                                                                                         \\

  \rowcolor{darkgray!90!}
  \multicolumn{1}{c}{\color{white}{\textbf{Rischio}}} & \multicolumn{1}{c}{\color{white}{\textbf{Gestione}}}                                                                                                                                                                    & \multicolumn{1}{c}{\color{white}{\textbf{Monitoraggio}}}                                                                                                                                     \\
  \textbf{RTEC001}                                    & Abbiamo incontrato problemi nell'utilizzo della strumentazione per gestire i documenti, questi problemi sono stati risolti grazie ad alcuni membri del gruppo che hanno trasmesso le loro conoscenze agli altri membri. & In futuro metteremo in cima alle priorità l'apprendimento delle nuove tecnologie delegandolo ad alcuni componenti che poi istruiranno gli altri sul funzionamento delle suddette tecnologie. \\
  \textbf{RPER001}                                    & Avendo riscontrato problemi nel gestire gli impegni accademici dei membri del gruppo abbiamo deciso di suddividerci i compiti.                                                                                          & Prossimamente cercheremo di pianificare e organizzare al meglio il carico di lavoro affidato ai singoli membri in modo da evitare ritardi nella consegna.                                    \\
  \textbf{RPER002}                                    & Abbiamo riscontrato problemi nel gestire gli impegni personali dei membri del gruppo, abbiamo quindi suddiviso i compiti in modo da limitare le sovrapposizioni.                                                        & In futuro cercheremo di comunicare il prima possibile eventuali impegni e di gestire il lavoro dei singoli membri in modo da evitare che si causino dei ritardi.                             \\
  \textbf{RPER003}                                    & Ci sono stati dei problemi nel comunicare con il proponente, in particolare gli impegni del gruppo e del proponente hanno reso difficile fissare delle chiamate in tempi brevi.                                         & In futuro il gruppo pianificherà con maggiore anticipo le chiamate per venire incontro alle esigenze del proponente.                                                                         \\
  \textbf{RPRG002}                                    & Abbiamo dovuto ridefinire alcuni casi d'uso e di conseguenza requisiti dopo aver dialogato con il proponente, abbiamo cercato di ottimizzare al massimo i tempi per correggere i documenti.                             & In futuro ci accorderemo con il proponente appena possibile per discutere di altri dubbi sui requisiti in modo da avere più tempo per apportare delle modifiche.                             \\
  \rowcolor{white}
  \caption{Attualizzazione dei rischi per la fase di analisi preliminare}%
  \label{tab:attualizzazione_fase_analisi_preliminare}
\end{longtable}

% sub:attualizzazione_fase_analisi_preliminare (end)

\subsection{Attualizzazione per la fase di preparazione alla RR}%
\label{sub:attualizzazione_fase_prep_RR}
In questa fase non si sono verificati eventi da segnalare.
% sub:attualizzazione_fase_prep_RR (end)

\subsection{Attualizzazione per la fase di progettazione architetturale}%
\label{sub:attualizzazione_per_la_fase_di_progettazione_architetturale}

\begin{longtable}[H]{|p{10em}|p{17em}|p{17em}|}
  \caption{Attualizzazione dei rischi per la fase di progettazione architetturale}%
  \label{tab:attualizzazione_per_la_fase_di_progettazione_architetturale} \\
  \rowcolor{darkgray!90!}
  \color{white}{\textbf{Rischio}} & \color{white}{\textbf{Gestione}} & \color{white}{\textbf{Monitoraggio}} \\
  \textbf{RPER001} & La sovrapposizione della fase corrente con la sessione invernale di esame ha distratto l'attenzione di gran parte del gruppo fino alla conclusione degli esami, causando un ritardo di circa tre settimane. & In futuro questa evenienza non dovrebbe ripresentarsi, in quanto le successive sessioni d'esame iniziano dopo la data di consegna prevista per il progetto. Inoltre, durante il secondo semestre non sono previsti insegnamenti obbligatori, quindi solo alcuni membri devono seguire alpiù un esame opzionale. Manteniamo comunque in atto il monitoraggio descritto in §\ref{tab:elenco_dei_rischi}. \\
  \textbf{RPRG001} & Legato al precedente, di conseguenza al ritardo accumulato durante la sessione d'esame il gruppo ha deciso di posporre la data di consegna, saltando la revisione di avanzamento di marzo, e portando ad un ritardo di circa un mese sulla data di consegna del progetto. Per un'analisi dettagliata delle conseguenze di questo ritardo sulla pianificazione, ci si riferisca a §\ref{sec:pianificazione} e §\ref{sec:preventivo_a_finire} & In futuro seguiremo la pianificazione più attentamente. \\
  \textbf{RPER004} & La diffusione della COVID-19, e le conseguenti misure di distanziamento sociale, hanno portato all'impossibilità di svolgere incontri di persona. Questo ha causato incomprensioni sulle priorità del gruppo, le quali hanno avuto un effetto negativo sulla produttività. & In futuro ci manterremo in contatto più stretto, accordandoci esplicitamente sul lavoro da svolgere. \\
\end{longtable}

% sub:attualizzazione_per_la_fase_di_progettazione_architetturale (end)
\subsection{Attualizzazione per la fase di incremento 1}%
\label{sub:attualizzazione_per_la_fase_di_incremento_1}

\begin{longtable}[H]{|p{10em}|p{17em}|p{17em}|}
  \caption{Attualizzazione dei rischi per la fase di progettazione architetturale}%
  \label{tab:attualizzazione_per_la_fase_di_progettazione_architetturale} \\
  \rowcolor{darkgray!90!}
  \color{white}{\textbf{Rischio}} & \color{white}{\textbf{Gestione}} & \color{white}{\textbf{Monitoraggio}} \\
  \textbf{RTEC001} & Gli strumenti necessari ad implementare l'autenticazione degli utenti si sono dimostrati più complessi da utilizzare di quanto previsto, causando un ritardo rispetto la pianificazione & Per le tecnologie che utilizzeremo in futuro impegneremo più tempo nell'autoapprendimento. \\
\end{longtable}

% sub:attualizzazione_per_la_fase_di_incremento_1 (end)
\end{document}
