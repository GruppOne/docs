\documentclass[../piano-di-progetto]{subfiles}

\begin{document}
Per ogni fase il gruppo riporta i rischi incontrati nella fase di analisi, le modalità utilizzate per affrontarli e come propone di gestire i suddetti rischi in futuro.

\subsection{Attualizzazione per la fase di analisi preliminare}%
\label{sub:attualizzazione_fase_analisi_preliminare}
% Nella tabella sottostante riportiamo i rischi incontrati nella fase di analisi, le modalità utilizzate per affrontarli e come proponiamo di gestire i suddetti rischi in futuro:
% \renewcommand{\arraystretch}{2}
\begin{longtable}[H]{|p{10em}|p{17em}|p{17em}|}
  \rowcolor{darkgray!90!}
  \multicolumn{1}{c}{\color{white}{\textbf{Rischio}}} & \multicolumn{1}{c}{\color{white}{\textbf{Gestione}}}                                                                                                                                                          & \multicolumn{1}{c}{\color{white}{\textbf{Monitoraggio}}}                                                                                                                        \\
  \textbf{RTEC001}                                    & Il gruppo ha incontrato problemi nell'utilizzo della strumentazione per gestire i documenti, questi problemi sono stati risolti grazie ad alcuni membri che hanno trasmesso le proprie conoscenze agli altri. & In futuro il gruppo darà una più alta priorità all'apprendimento delle nuove tecnologie, delegandolo ad alcuni componenti che poi istruiranno gli altri sul loro funzionamento. \\
  \textbf{RPER001}                                    & I membri del gruppo hanno riscontrato problemi nel gestire gli impegni accademici.                                                                                                                            & In futuro il gruppo attuerà una maggiore parallelizzazione del lavoro, prestando più attenzione ad una sua equa suddivisione.                                                   \\
  \textbf{RPER002}                                    & I membri del gruppo hanno riscontrato problemi nel gestire gli impegni personali.                                                                                                                             & In futuro i membri presteranno più attenzione a comunicare eventuali indisponibilità con anticipo sufficiente ad adeguare la pianificazione.                                    \\
  \textbf{RPER003}                                    & Ci sono stati dei problemi nel comunicare con il proponente, in particolare gli impegni del gruppo e del proponente hanno reso difficile fissare delle chiamate in tempi brevi.                               & In futuro il gruppo pianificherà con maggiore anticipo le chiamate per venire incontro alle esigenze del proponente.                                                            \\
  \textbf{RPRG002}                                    & Successivamente ad un dialogo con il proponente, il gruppo ha dovuto ridefinire alcuni casi d'uso e requisiti, e di conseguenza aggiornare i relativi documenti, a ridosso della revisione dei requisiti.     & In futuro il gruppo contatterà il proponente con più anticipo rispetto alle revisioni, in modo da avere più tempo per eventuali modifiche.                                      \\
  \rowcolor{white}
  \caption{Attualizzazione dei rischi per la fase di analisi preliminare}%
  \label{tab:attualizzazione_fase_analisi_preliminare}
\end{longtable}

% sub:attualizzazione_fase_analisi_preliminare (end)

\subsection{Attualizzazione per la fase di preparazione alla RR}%
\label{sub:attualizzazione_fase_prep_RR}
In questa fase non si sono verificati eventi da segnalare.
% sub:attualizzazione_fase_prep_RR (end)

\subsection{Attualizzazione per la fase di progettazione architetturale}%
\label{sub:attualizzazione_per_la_fase_di_progettazione_architetturale}

\begin{longtable}[H]{|p{10em}|p{17em}|p{17em}|}
  \rowcolor{darkgray!90!}
  \color{white}{\textbf{Rischio}} & \color{white}{\textbf{Gestione}}                                                                                                                                                                                                                                                                                                                                                                                                            & \color{white}{\textbf{Monitoraggio}}                                                                                                                                                                                                                                                                                                                                                         \\
  \textbf{RPER001}                & La sovrapposizione della fase corrente con la sessione invernale di esame ha distratto l'attenzione di gran parte del gruppo fino alla conclusione degli esami, causando un ritardo di circa tre settimane.                                                                                                                                                                                                                                 & In futuro questa evenienza non si ripresenterà, in quanto le successive sessioni d'esame iniziano dopo la data di consegna prevista per il progetto. Inoltre, durante il secondo semestre non sono previsti insegnamenti obbligatori, quindi solo alcuni membri devono seguire al più un esame opzionale. Rimane comunque in atto il monitoraggio descritto in §\ref{tab:elenco_dei_rischi}. \\
  \textbf{RPRG001}                & Legato al precedente, di conseguenza al ritardo accumulato durante la sessione d'esame il gruppo ha deciso di posporre la data di consegna, saltando la revisione di avanzamento di marzo, e portando ad un ritardo di circa un mese sulla data di consegna del progetto. Per un'analisi dettagliata delle conseguenze di questo ritardo sulla pianificazione, ci si riferisca a §\ref{sec:pianificazione} e §\ref{sec:preventivo_a_finire} & In futuro il gruppo seguirà la pianificazione più attentamente.                                                                                                                                                                                                                                                                                                                             \\
  \textbf{RPER004}                & La diffusione della COVID-19, e le conseguenti misure di distanziamento sociale, hanno portato all'impossibilità di svolgere incontri di persona. Questo ha causato incomprensioni sulle priorità del gruppo, le quali hanno avuto un effetto negativo sulla produttività.                                                                                                                                                                  & In futuro i membri del gruppo si manterranno in contatto più stretto, accordandosi esplicitamente sul lavoro da svolgere.                                                                                                                                                                                                                                                                                         \\
  \rowcolor{white}
  \caption{Attualizzazione dei rischi per la fase di progettazione architetturale}%
  \label{tab:attualizzazione_per_la_fase_di_progettazione_architetturale}
\end{longtable}

% sub:attualizzazione_per_la_fase_di_progettazione_architetturale (end)
\subsection{Attualizzazione per la fase di incremento 1}%
\label{sub:attualizzazione_per_la_fase_di_incremento_1}

\begin{longtable}[H]{|p{10em}|p{17em}|p{17em}|}
  \rowcolor{darkgray!90!}
  \color{white}{\textbf{Rischio}} & \color{white}{\textbf{Gestione}}                                                                                                                                                        & \color{white}{\textbf{Monitoraggio}}                                                       \\
  \textbf{RTEC001}                & Gli strumenti necessari ad implementare l'autenticazione degli utenti si sono dimostrati più complessi da utilizzare di quanto previsto, causando un ritardo rispetto la pianificazione & In futuro il gruppo impegnerà più tempo nell'autoapprendimento delle tecnologie che intende utilizzare. \\
  \rowcolor{white}
  \caption{Attualizzazione dei rischi per la fase di incremento 1}%
  \label{tab:attualizzazione_per_la_fase_di_incremento_1}
\end{longtable}

% sub:attualizzazione_per_la_fase_di_incremento_1 (end)
\subsection{Attualizzazione per la fase di incremento 2 e preparazione alla RP}%
\label{sub:attualizzazione_per_la_fase_di_incremento_2 e preparazione alla RP}

\begin{longtable}[H]{|p{10em}|p{17em}|p{17em}|}
  \rowcolor{darkgray!90!}
  \color{white}{\textbf{Rischio}} & \color{white}{\textbf{Gestione}}                                                                                                                                                        & \color{white}{\textbf{Monitoraggio}}                                                       \\
  \textbf{RTEC001}                & Lo sviluppo di un server asincrono si è rivelato più complicato del previsto. &  Pertanto abbiamo deciso di redistribuire le ore lavorative a favore di un maggior impegno nello sviluppo del server. \\
  \rowcolor{white}
  \textbf{RPER001}                & La preparazione per l'esame scritto di Ingegneria del Software ha tenuto impegnati alcuni membri del gruppo che hanno quindi potuto dedicare meno tempo al progetto. & Abbiamo redistribuito i compiti per la settimana corrente per coprire le ore non svolte dai nostri colleghi che verranno recuperate in seguito.\\
  \caption{Attualizzazione dei rischi per la fase di incremento 2}%
  \label{tab:attualizzazione_per_la_fase_di_incremento_2}
\end{longtable}

% sub:attualizzazione_per_la_fase_di_incremento_2_preparazione_RP (end)
\subsection{Attualizzazione per la fase di incremento 3}%
\label{sub:attualizzazione_per_la_fase_di_incremento_3}
In questa fase non si sono verificati eventi da segnalare.
% sub:attualizzazione_per_la_fase_di_incremento_3 (end)
\subsection{Attualizzazione per la fase di incremento 4}%
\label{sub:attualizzazione_per_la_fase_di_incremento_4}

\begin{longtable}[H]{|p{10em}|p{17em}|p{17em}|}
  \rowcolor{darkgray!90!}
  \color{white}{\textbf{Rischio}} & \color{white}{\textbf{Gestione}}                                                                                                                                                        & \color{white}{\textbf{Monitoraggio}}                                                       \\
  \textbf{RPER001}                & Un membro del gruppo ha iniziato lo stage interno per il conseguimento della laurea. & Il gruppo ha redistribuito parte dei compiti assegnati al nostro collega. \\
  \rowcolor{white}
  \caption{Attualizzazione dei rischi per la fase di incremento 4}%
  \label{tab:attualizzazione_per_la_fase_di_incremento_4}
\end{longtable}

% sub:attualizzazione_per_la_fase_di_incremento_4 (end)
\end{document}
