\documentclass[../piano-di-progetto.tex]{subfiles}

\begin{document}
Per lo sviluppo del prodotto richiesto dal proponente, il gruppo ha scelto il \glossario{modello incrementale}, il quale è sufficientemente rigido da adattarsi alla struttura sequenziale delle revisioni di avanzamento che ci sono richieste, ma al contempo evita la cosiddetta \textit{big-bang integration}, prevedendo rilasci multipli vicini nel tempo.
Inoltre, questo modello lascia la possibilità di rivalutare i requisiti in accordo con il proponente, ad esclusione per quelli coinvolti nel ciclo di incremento corrente.
L'imposizione di una gerarchia nei requisiti permette al gruppo di sviluppare prima le componenti più importanti: in questo modo il proponente può dare un feedback molto veloce ed eventualmente richiedere delle correzioni.
Un altro vantaggio decisivo del modello incrementale è la localizzazione degli errori al codice coinvolto nel ciclo di incremento corrente: ciascun ciclo si conclude con la verifica delle componenti aggiunte e con la loro integrazione nel sistema, quindi eventuali errori che insorgono sono necessariamente stati introdotti dopo la conclusione del ciclo precedente.
Allo stesso modo, la verifica e i test sono facilitati, perché più mirati.
\end{document}
