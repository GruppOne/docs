\documentclass[../piano-di-progetto.tex]{subfiles}


\begin{document}
	\subsection{Gestione dei rischi}%
  \label{sub:gestione_dei_rischi}
  Nel corso di un progetto complesso non è così raro incontrare problematiche che possono rallentare o addirittura annullare la distribuzione del software finale.
  Allo scopo di evitare per quanto possibile queste problematiche in questa sezione si riportata l'analisi svolta sui possibili rischi del progetto; I rischi analizzati sono stati classificati secondo le seguenti caratteristiche:
  \begin{description}
      \item[Identificazione dei rischi:] Stiliamo un elenco dei rischi in base alla categoria a cui appartengono, ossia alla causa generale del rischio.
      \item[Analisi dei rischi:] analizziamo i rischi individuati nel punto sopra e li cataloghiamo in base alla probabilità di incorrere in questi rischi e alla gravità delle conseguenze che essi comporterebbero per il progetto.
      \item[Pianificazione dei rischi:] elaboriamo dei piani per ridurre la probabilità che questi eventi accadano o ridurne le conseguenze.
      \item[Monitoraggio dei rischi:] Controlliamo regolarmente i rischi e li modifichiamo, rimuoviamo o aggiungiamo qualora le esigenze cambino nel corso del progetto.
    \end{description}
    Il gruppo ha classificato i rischi nelle seguenti tipologie:
    \begin{itemize}
      \item Rischi per il prodotto
      \item Rischio per il progetto
      \item Rischi del personale
      \item Rischi dei requisiti
      \item Rischi tecnologici
      \item Rischi degli strumenti.
    \end{itemize}
  % sub:gestione_dei_rischi (end)

  \subsection{Elenco dei rischi}%
  \label{sub:elenco_dei_rischi}
  Di seguito saranno indicati i principali rischi individuati, una loro descrizione, la pianificazione messa in atto dal gruppo per monitorare o evitare i rischi, la probabilità che il rischio si verifichi e il suo impatto sul progetto:
  I rischi saranno identificati da un codice univoco:
  \begin{center}
    \textbf{R[tipo][numero]}
  \end{center}
  Dove [tipo] correla il rischio alla sua tipologia mentre [numero] identifica in maniera univoca il rischio all'interno della tipologia a cui appartiene.
  \rowcolors{2}{lightgray}{white!80!lightgray!100}
      \renewcommand{\arraystretch}{2} % allarga le righe con dello spazio sotto e sopra
      \begin{longtable}[H]{|p{10em}|p{13em}|p{13em}|p{10em}|}
      \rowcolor{darkgray!90!}
      \color{white}
      {\textbf{Rischio}} & \color{white}{\textbf{Descrizione}} & \color{white}{\textbf{Pianificazione}} & \color{white}{\textbf{Indicatori}} \\
      \endhead{}
      Rischio & Descrizione & Pianificazione & Indicatori\\
      \hline % chktex 44
  \endhead{}
  Scarse conoscenze tecnologiche \textbf{RTEC001} & Inesperienza del gruppo riguardo alle tecnologie utilizzate, per la maggior parte nuove. & L'amministratore ha redatto le norme di progetto con velocità e ponendo particolare attenzione alle tecnologie necessarie per il progetto così da consentire ai membri del gruppo un apprendimento più veloce e completo dei nuovi strumenti. & \textbf{Probabilità:} Alta \textbf{Impatto:} Tollerabile \\
  Monitoraggio: & \multicolumn{3}{p{38.5em}|}{In caso vengano aggiunte nuove tecnologie i membri del gruppo verranno invitati ad aggiornarsi sulle nuove tecnologie.}\\
  \hline % chktex 44
  Ritardo nella consegna del progetto \textbf{RPRG001} & La nostra inesperienza nella gestione di un progetto software unita alle conoscenze limitate sulle tecnologie coinvolte potrebbero portare ad un ritardo nella consegna del prodotto. & Il \textit{Piano di Progetto} terrà conto dell'inesperienza del gruppo nella valutazione delle tempistiche e i membri del gruppo sono invitati a aggiornarsi preventivamente sulle aree in cui hanno minor competenza. & \textbf{Probabilità:}   Bassa \textbf{Impatto: }  Significativo  \\
  Monitoraggio: & \multicolumn{3}{p{38.5em}|}{Il responsabile monitorerà spesso lo stato del progetto per eventualmente riassegnare le risorse disponibili e limitare il ritardo.}\\
  \hline % chktex 44
  Problemi Accademici \textbf{RPER001} & La quasi totalità dei membri del gruppo porta avanti altre attività accademiche oltre a questo progetto e dunque possono sovrapporsi gli impegni dei vari membri. & Il team ha approvato sin dal primo periodo un calendario condiviso di impegni per cui era richiesta la partecipazione di tutti i componenti. & \textbf{Probabilità: }   Alta \textbf{Impatto: }    Basso\\
  Monitoraggio: & \multicolumn{3}{p{38.5em}|}{I membri del gruppo sono sempre invitati a comunicare preventivamente gli impegni accademici così da poter gestire al meglio gli impegni del gruppo.}\\
  \hline % chktex 44
  Problemi Personali \textbf{RPER002} & Possono intercorrere nel corso del progetto impegni personali che costringono alcuni membri del gruppo a sospendere le attività e risultare irreperibili. & Questi eventi, se non saranno compatibili con gli impegni già determinati per il progetto, andranno comunicati appena possibile al responsabile che se necessario provvederà a ridistribuire le risorse disponibili. & \textbf{Probabilità:}  Alta \textbf{Impatto: }    Molto Basso\\
  Monitoraggio: & \multicolumn{3}{p{38.5em}|}{In caso sopraggiungano impegni personali tutti i membri sono tenuti a comunicarli al responsabile il prima possibile.}\\
  \hline % chktex 44
  Comunicazioni esterne \textbf{RPER003} & Il proponente ha la sede lontana da Padova e dunque le comunicazioni faccia a faccia risultano complicate & Se sono necessari incontri col proponente in tempi brevi vengono preferite chiamate \glossario{Hangouts} a incontri fisici. & \textbf{Probabilità: }   Alta \textbf{Impatto: }   Basso\\
  Monitoraggio: & \multicolumn{3}{p{38.5em}|}{Il responsabile se possibile organizzerà con largo anticipo incontri e chiamate con il proponente.}\\
  \hline % chktex 44
  Calcolo costi del progetto \textbf{RPRG002} & La poca esperienza del gruppo potrebbe portare ad un calcolo dei costi differente da quello che sarà invece necessario per il progetto. & Le stime sui costi sono state discusse con tutti i membri del gruppo. Se la stima si rivelerà essere al ribasso si cercherà di ridurre al minimo le modifiche ai costi che verranno comunicate tempestivamente al committente. & \textbf{Probabilità: }  Media \textbf{Impatto: } Tollerabile\\
  Monitoraggio: & \multicolumn{3}{p{38.5em}|}{Per gestire al meglio eventuali problemi legati ai costi i membri del gruppo aggiornano costantemente le proprie ore di lavoro in modo che il responsabile e gli altri membri del gruppo possano controllare la situazione attuale in relazione alle previsioni fatte.}\\
  \hline % chktex 44
  Ridefinizione dei requisiti \textbf{RREQ001} & I requisiti decisi in fase di analisi potrebbero essere modificati in seguito ad esigenze del proponente o del gruppo. & I membri del gruppo hanno discusso le stime sui costi e le hanno proposte al proponente per trovare soluzioni comuni, nel caso sia necessario ridefinire i requisiti si parlerà approfonditamente con proponente e committente e si cercherà di gravare il meno possibile sul prospetto orario del progetto. & \textbf{Probabilità: }  Bassa \textbf{Impatto: }  Significativo\\
  Monitoraggio: & \multicolumn{3}{p{38.5em}|}{I requisiti verranno discussi attentamente e se ne parlerà con il proponente prima di consegnarli in revisione.}\\
  \hline % chktex 44
  Problema al \glossario{server} di \glossario{Imola Informatica} \textbf{RSTR001} & Il proponente mette a disposizione del gruppo un server dell'azienda che tuttavia potrebbe subire problematiche o essere spento per un certo periodo di tempo per motivazioni interne. & Il gruppo provvederà a produrre più copie dell'applicazione e in particolare verranno fatte diverse copie del \glossario{software} definitivo così da non dipendere necessariamente dall'hardware fornito da Imola Informatica. & \textbf{Probabilità: }  Bassa \textbf{Impatto: } Tollerabile\\
  Monitoraggio: & \multicolumn{3}{p{38.5em}|}{Il proponente notificherà il gruppo in caso di problemi se possibile e in ogni caso la copia dell'applicazione sul server non sarà l'unica copia disponibile.}\\
  \hline % chktex 44
  \hline % chktex 44
  \end{longtable}
  % sub:elenco_dei_rischi

  \end{document}
