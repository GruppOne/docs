\documentclass[../piano-di-progetto.tex]{subfiles}


\begin{document}
	\subsection{Gestione dei rischi}%
  \label{sub:gestione_dei_rischi}
  Nel corso di un progetto così complesso non è cosa rara incontrare problematiche che possono rallentare o addirittura annullare la distribuzione del software finale.
  Allo sopo di evitare per quanto possibile queste problematiche in questa sezione viene riportata l'analisi svolta sui possibili rischi del progetto; I rischi analizzati sono stati classificati secondo le seguenti caratteristiche:
  \begin{itemize}
      \item Identificazione dei rischi: Stiliamo un elenco dei rischi in base alla categoria a cui appartengono, ossia alla causa generale del rischio.
      \item Analisi dei rischi: analizziamo i rischi individuati nel punto sopra e li cataloghiamo in base alla probabilità di incorrere in questi rischi e alla gravità delle conseguenze che questi rischi comporterebbero per il progetto.
      \item Pianificazione dei rischi: elaboriamo dei piani per ridurre la probabilità che questi eventi accadano o ridurne le conseguenze.
      \item Monitoraggio dei rischi: Controlliamo regolarmente i rischi e li modifichiamo, rimuoviamo o aggiungiamo qual'ora le esigenze cambino nel corso del progetto.
    \end{itemize}
    I rischi sono stati classificati dal gruppo nei seguenti tipi:
    \begin{itemize}
      \item Rischi del Personale
      \item Rischi dei requisiti
      \item Rischi tecnologici
      \item Rischi degli strumenti
    \end{itemize}
  % sub:gestione_dei_rischi (end)

  \subsection{Elenco dei rischi}%
  \label{sub:elenco_dei_rischi}
  % sub:elenco_dei_rischi

  \end{document}