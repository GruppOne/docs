\documentclass[../piano-di-progetto.tex]{subfiles}

\begin{document}
Per ogni fase, il gruppo presenta un prospetto orario che raccoglie le ore da svolgere per ruolo per persona e i rispettivi totali, seguito da un prospetto economico che riassume i costi per ruolo e il totale. Per concludere, presenta il totale delle ore di lavoro pianificate e di quelle rendicontate, assieme ai rispettivi costi.

Per aumentare la leggibilità delle tabelle, le celle che indicano un numero nullo di ore sono indicate con ``-'' invece di ``0'' e i nomi dei ruoli sono sostituiti dalle abbreviazioni seguenti:
\begin{description}
  \item[Re] Responsabile
  \item[Am] Amministratore
  \item[An] Analista
  \item[Pt] Progettista
  \item[Pr] Programmatore
  \item[Ve] Verificatore
\end{description}
\subsection{Fase di analisi preliminare}%
\label{sub:fase_di_analisi_preliminare}

% sub:fase_di_analisi_preliminare (end)
\subsection{Fase di preparazione in entrata alla RR}%
\label{sub:fase_di_preparazione_in_entrata_alla_rr}

% sub:fase_di_preparazione_in_entrata_alla_rr (end)
\subsection{Fase di progettazione architetturale}%
\label{sub:fase_di_progettazione_architetturale}

% sub:fase_di_progettazione_architetturale (end)
\subsection{Fase di preparazione in entrata alla RP}%
\label{sub:fase_di_preparazione_in_entrata_alla_rp}

% sub:fase_di_preparazione_in_entrata_alla_rp (end)
\subsection{Fasi di incremento}%
\label{sub:fasi_di_incremento}
\plchold{preventivo complessivo degli incrementi? va bene?}
% sub:fasi_di_incremento (end)
\subsection{Fase di preparazione in entrata alla RQ}%
\label{sub:fase_di_preparazione_in_entrata_alla_rq}

% sub:fase_di_preparazione_in_entrata_alla_rq (end)
\subsection{Fase di verifica e collaudo finali}%
\label{sub:fase_di_verifica_e_collaudo_finali}

% sub:fase_di_verifica_e_collaudo_finali (end)
\subsection{Fase di preparazione in entrata alla Ra}%
\label{sub:fase_di_preparazione_in_entrata_alla_rr}

% sub:fase_di_preparazione_in_entrata_alla_ra (end)
\end{document}
