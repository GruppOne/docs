\documentclass[../piano-di-progetto.tex]{subfiles}

\begin{document}
Considerate le scadenze scelte in sottosezione~\ref{sub:scadenze_fissate} ed il modello di sviluppo scelto in sezione~\ref{sec:modello_di_sviluppo}, GruppOne ha individuato cinque \glossario{fasi} di sviluppo:
\begin{itemize}
  \item Analisi preliminare
  \item Preparazione in entrata
  \item Progettazione architetturale
  \item Sviluppo e verifica incrementi
  \item Verifica e collaudo finali
\end{itemize}
Ciascuna fase è caratterizzata da delle \glossario{precondizioni} e \glossario{postcondizioni} e divisa in \glossario{periodi}.
Per ciascun periodo sono fissate le \glossario{attività} da svolgervisi, e sono evidenziate (in grassetto) quelle che risultano essere completate al termine di esso \plchold{frase da riformulare}.
\subsection[Analisi preliminare]{Analisi preliminare {\normalsize\normalfont\itshape(14/11/2019 \symbol{8594} 14/01/2020)}}%
\label{sub:analisi_preliminare}
Nella fase di analisi preliminare il gruppo si forma (scegliendo, ad esempio, nome e logo) e si prepara alla Revisione dei Requisiti studiando i capitolati proposti e svolgendo l'analisi preliminare dei requisiti e la pianificazione globale del lavoro da svolgere.
I ruoli coinvolti sono:
\begin{itemize}
  \item Responsabile
  \item Amministratore
  \item Analista
  \item Verificatore
\end{itemize}
Le precondizioni sono:
\begin{itemize}
  \item I componenti del gruppo sono stati individuati
  \item I capitolati sono stati presentati
\end{itemize}
Le postcondizioni sono:
\begin{itemize}
  \item Il gruppo si è definito, decidendo nome e logo, ed ha creato la propria identità digitale. Nello specifico, il gruppo ha ottenuto un indirizzo e-mail ed un account presso un source code manager.
  \item Il gruppo ha scelto il capitolato per cui concorrere, raccogliendo le considerazioni su cui si è basato nel documento \textit{Studio di fattibilità}
  \item Il gruppo ha deciso quali standard seguire e quali strumenti utilizzare per la comunicazione e per la produzione dei documenti, e ha raccolto queste nozioni nel documento \textit{Norme di progetto}
  \item Il gruppo ha definito i requisiti del sistema in via preliminare, e li ha raccolti nel documento \textit{Analisi dei requisiti}
  \item Il gruppo ha pianificato il lavoro da svolgere, ed ha esposto la pianificazione nel documento \textit{Piano di progetto} assieme al preventivo delle spese ed al consuntivo delle ore investite.
  \item \plchold{piano di qualifica}
  \item Il gruppo ha raccolto eventuali termini dal significato ambiguo o poco chiaro nel documento \textit{Glossario}, accompagnati dalla rispettiva definizione
\end{itemize}
Questa fase è divisa in quattro periodi.
Per ciascuno si elencano le attività coinvolte ed i rispettivi compiti che devono essere svolti.
\subsubsection[Formazione del gruppo]{Formazione del gruppo {\normalsize\normalfont\itshape(14/11/2019 \symbol{8594} 06/12/2019)}}%
\label{subs:formazione_del_gruppo}
\begin{itemize}
  \item \textbf{Caratterizzazione del gruppo}
  \begin{itemize}
    \item Scelta del nome del gruppo
    \item Scelta del logo del gruppo
    \item Definizione dell'identità digitale del gruppo
    \item Distribuzione dei ruoli
  \end{itemize}
  \item Pianificazione di progetto
  \begin{itemize}
    \item Pianificazione delle attività preliminari
  \end{itemize}
  \item \textbf{Stesura studio di fattibilità}
  \begin{itemize}
    \item Descrizione del problema (per ciascun capitolato)
    \item Definizione del prodotto richiesto (per ciascun capitolato)
    \item Elencazione delle tecnologie interessate (per ciascun capitolato)
    \item Considerazioni sugli aspetti positivi e negativi (per ciascun capitolato)
  \end{itemize}
  \item Normazione
  \begin{itemize}
    \item Decisione dei canali di comunicazione principali
    \item Decisione degli strumenti utilizzati per scrivere i documenti
    \item Decisione degli strumenti di versionamento
  \end{itemize}
  \item Terminologia e nomenclatura
  \begin{itemize}
    \item Raccolta dei termini ambigui
    \item Definizione dei termini ambigui
  \end{itemize}
\end{itemize}
% subs:formazione_del_gruppo (end)
\subsubsection[Impostazione strumenti e template]{Impostazione strumenti e template {\normalsize\normalfont\itshape(07/12/2019 \symbol{8594} 20/12/2019)}}%
\label{subs:impostazione_strumenti_e_template}
\begin{itemize}
  \item Normazione
  \begin{itemize}
    \item Definizione delle impostazioni condivise per gli strumenti di scrittura
    \item Definizione dei template per i documenti
    \item Definizione delle impostazioni condivise per gli strumenti di versionamento
  \end{itemize}
  \item Analisi dei requisiti
  \begin{itemize}
    \item Descrizione puntuale del problema e della soluzione proposta
    \item Raccolta iniziale dei requisiti
  \end{itemize}
  \item Incontri con il proponente
  \begin{itemize}
    \item Confronto sull'analisi preliminare
  \end{itemize}
  \item Terminologia e nomenclatura
  \begin{itemize}
    \item Raccolta dei termini ambigui
    \item Definizione dei termini ambigui
  \end{itemize}
\end{itemize}
% subs:impostazione_strumenti_e_template (end)
\subsubsection[Analisi e pianificazione]{Stesura documenti {\normalsize\normalfont\itshape(21/12/2019 \symbol{8594} 07/01/2020)}}%
\label{subs:analisi_e_pianificazione}
\begin{itemize}
  \item Normazione
  \begin{itemize}
    \item Scelta degli standard internazionali da seguire
    \item Istanziazione ed adattamento degli standard scelti
  \end{itemize}
  \item Analisi dei requisiti
  \begin{itemize}
    \item Definizione dei casi d'uso
    \item Raccolta dei requisiti e loro classificazione
  \end{itemize}
  \item Pianificazione di progetto
  \begin{itemize}
    \item Individuazione dei rischi e loro classificazione
    \item Definizione scadenze di progetto
    \item Scelta del modello di sviluppo da adottare
    \item Suddivisione del periodo di lavoro in fasi
    \item Pianificazione delle attività
    \item Calcolo del preventivo dei costi, del consuntivo di periodo e del preventivo a finire
  \end{itemize}
  \item Pianificazione di qualifica
  \begin{itemize}
    \item \plchold{qualifica}
  \end{itemize}
  \item Terminologia e nomenclatura
  \begin{itemize}
    \item Raccolta dei termini ambigui
    \item Definizione dei termini ambigui
  \end{itemize}
  \item Produzione documentazione
  \begin{itemize}
    \item Stesura delle \textit{Norme di progetto}
    \item Stesura dell'\textit{Analisi dei requisiti}
    \item Stesura del \textit{Piano di progetto}
    \item Stesura del \textit{Piano di qualifica}
    \item Stesura del \textit{Glossario}
  \end{itemize}
  \item Verifica documentazione
  \begin{itemize}
    \item Verifica dei documenti prodotti
  \end{itemize}
\end{itemize}
% subs:analisi_e_pianificazione (end)
\subsubsection[Dettagli finali]{Dettagli finali {\normalsize\normalfont\itshape(21/12/2019 \symbol{8594} 07/01/2020)}}%
\label{subs:dettagli_finali}
\begin{itemize}
  \item Incontri con il proponente
  \begin{itemize}
    \item Confronto sull'analisi dei requisiti
    \item Confronto sulla pianificazione di progetto
    \item Confronto sulla pianificazione di qualifica
  \end{itemize}
  \item Verifica documentazione
  \begin{itemize}
    \item Verifica dei documenti prodotti
  \end{itemize}
\end{itemize}
% subs:dettagli_finali (end)
% sub:analisi_preliminare (end)
\subsection[Preparazione in entrata]{Preparazione in entrata {\normalsize\normalfont\itshape(15/01/2020 \symbol{8594} 21/01/2020)}}%
\label{sub:preparazione_in_entrata}

% sub:preparazione_in_entrata (end)
\subsection[Progettazione architetturale]{Progettazione architetturale {\normalsize\normalfont\itshape(22/01/2020 \symbol{8594} DD/MM/2020)}}%
\label{sub:progettazione_architetturale}

% sub:progettazione_architetturale (end)
\subsection[Sviluppo e verifica incrementi]{Sviluppo e verifica incrementi {\normalsize\normalfont\itshape(DD/MM/2020 \symbol{8594} DD/MM/2020)}}%
\label{sub:sviluppo_e_verifica_incrementi}

% sub:sviluppo_e_verifica_incrementi (end)
\subsection[Verifica e collaudo finali]{Verifica e collaudo finali {\normalsize\normalfont\itshape(DD/MM/2020 \symbol{8594} DD/MM/2020)}}%
\label{sub:verifica_e_collaudo_finali}

% sub:verifica_e_collaudo_finali (end)
\end{document}
