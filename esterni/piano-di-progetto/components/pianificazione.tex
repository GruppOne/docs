\documentclass[../piano-di-progetto.tex]{subfiles}

\begin{document}
Per lo sviluppo del prodotto richiesto dal proponente, il gruppo ha scelto il \glossario{modello incrementale}, il quale è sufficientemente rigido da adattarsi alla struttura sequenziale delle revisioni di avanzamento che ci sono richieste, ma al contempo evita la cosiddetta \textit{big-bang integration}, prevedendo rilasci multipli vicini nel tempo.
Considerate le scadenze scelte in sottosezione~\ref{sub:scadenze_fissate}, GruppOne ha individuato undici \glossario{fasi} di sviluppo:
\begin{itemize}
  \item Analisi preliminare
  \item Preparazione in entrata alla RR
  \item Progettazione architetturale
  \item Preparazione in entrata alla RP
  \item Sette \glossario{fasi di incremento}
  \item Verifica e collaudo finali
  \item Preparazione in entrata alla RA\@.
\end{itemize}
Ciascuna fase è caratterizzata da delle \glossario{precondizioni} e \glossario{postcondizioni} e divisa in \glossario{periodi}.
Per ciascun periodo sono fissate le \glossario{attività} da svolgervisi. Le fasi di incremento sono associate ai flussi individuati nell'\textit{Analisi dei requisiti}.
\subsection[Analisi preliminare]{Analisi preliminare {\normalsize\normalfont\itshape(14/11/2019 \symbol{8594} 14/01/2020)}}%
\label{sub:analisi_preliminare}
Nella fase di analisi preliminare il gruppo si forma (scegliendo, ad esempio, nome e logo) e si prepara alla Revisione dei Requisiti studiando i capitolati proposti e svolgendo l'analisi preliminare dei requisiti e la pianificazione globale del lavoro da svolgere.
I ruoli coinvolti sono:
\begin{itemize}
  \item Responsabile
  \item Amministratore
  \item Analista
  \item Progettista
  \item Verificatore.
\end{itemize}
Le precondizioni sono:
\begin{itemize}
  \item I componenti del gruppo sono stati individuati
  \item I capitolati sono stati presentati.
\end{itemize}
Le postcondizioni sono:
\begin{itemize}
  \item Il gruppo si è definito, decidendo nome e logo, ed ha creato la propria identità digitale. Nello specifico, il gruppo ha ottenuto un indirizzo email ed un account presso un source code manager.
  \item Il gruppo ha scelto il capitolato per cui concorrere, raccogliendo le considerazioni su cui si è basato nel documento \textit{Studio di fattibilità}.
  \item Il gruppo ha deciso quali standard seguire e quali strumenti utilizzare per la comunicazione e per la produzione dei documenti, e ha raccolto queste nozioni nel documento \textit{Norme di progetto}.
  \item Il gruppo ha definito i requisiti del sistema in via preliminare, e li ha raccolti nel documento \textit{Analisi dei requisiti}.
  \item Il gruppo ha pianificato il lavoro da svolgere, ed ha esposto la pianificazione nel documento \textit{Piano di progetto} assieme al preventivo delle spese ed al consuntivo delle ore investite.
  \item Il gruppo ha selezionato gli strumenti e le metriche utilizzati per garantire il rispetto delle \textit{Norme di progetto}, la qualità dei documenti e del codice e il soddisfacimento dei requisiti.
  \item Il gruppo ha raccolto eventuali termini dal significato ambiguo o poco chiaro nel documento \textit{Glossario}, accompagnati dalla rispettiva definizione.
\end{itemize}
Questa fase è divisa in quattro periodi.
Per ciascuno si elencano le attività coinvolte ed i rispettivi compiti che devono essere svolti.
\subsubsection[Formazione del gruppo]{Formazione del gruppo {\normalsize\normalfont\itshape(14/11/2019 \symbol{8594} 06/12/2019)}}%
\label{subs:formazione_del_gruppo}
\begin{itemize}
  \item Caratterizzazione del gruppo
  \begin{itemize}
    \item Scelta del nome del gruppo
    \item Scelta del logo del gruppo
    \item Definizione dell'identità digitale del gruppo
    \item Distribuzione dei ruoli.
  \end{itemize}
  \item Pianificazione di progetto
  \begin{itemize}
    \item Pianificazione delle attività preliminari.
  \end{itemize}
  \item Studio di fattibilità
  \begin{itemize}
    \item Descrizione del problema (per ciascun capitolato)
    \item Definizione del prodotto richiesto (per ciascun capitolato)
    \item Elencazione delle tecnologie interessate (per ciascun capitolato)
    \item Considerazioni sugli aspetti positivi e negativi (per ciascun capitolato).
  \end{itemize}
  \item Normazione
  \begin{itemize}
    \item Decisione dei canali di comunicazione principali
    \item Decisione degli strumenti utilizzati per scrivere i documenti
    \item Decisione degli strumenti di versionamento.
  \end{itemize}
  \item Produzione documentazione
  \begin{itemize}
    \item Stesura dello \textit{Studio di fattibilità}.
  \end{itemize}
  \item Terminologia e nomenclatura
  \begin{itemize}
    \item Raccolta dei termini ambigui
    \item Definizione dei termini ambigui.
  \end{itemize}
  \item Verifica documentazione.
  \begin{itemize}
    \item Verifica dello \textit{Studio di fattibilità}.
  \end{itemize}
\end{itemize}
% subs:formazione_del_gruppo (end)
\subsubsection[Impostazione strumenti e template]{Impostazione strumenti e template {\normalsize\normalfont\itshape(07/12/2019 \symbol{8594} 20/12/2019)}}%
\label{subs:impostazione_strumenti_e_template}
\begin{itemize}
  \item Normazione
  \begin{itemize}
    \item Definizione delle impostazioni condivise per gli strumenti di scrittura
    \item Definizione dei template per i documenti
    \item Definizione delle impostazioni condivise per gli strumenti di versionamento.
  \end{itemize}
  \item Analisi dei requisiti
  \begin{itemize}
    \item Descrizione puntuale del problema e della soluzione proposta
    \item Raccolta iniziale dei requisiti.
  \end{itemize}
  \item Incontri con il proponente
  \begin{itemize}
    \item Confronto sull'analisi preliminare.
  \end{itemize}
  \item Terminologia e nomenclatura
  \begin{itemize}
    \item Raccolta dei termini ambigui
    \item Definizione dei termini ambigui.
  \end{itemize}
\end{itemize}
% subs:impostazione_strumenti_e_template (end)
\subsubsection[Analisi e pianificazione]{Stesura documenti {\normalsize\normalfont\itshape(21/12/2019 \symbol{8594} 07/01/2020)}}%
\label{subs:analisi_e_pianificazione}
\begin{itemize}
  \item Normazione
  \begin{itemize}
    \item Scelta degli standard internazionali da seguire
    \item Istanziazione ed adattamento degli standard scelti.
  \end{itemize}
  \item Analisi dei requisiti
  \begin{itemize}
    \item Definizione dei casi d'uso
    \item Raccolta dei requisiti e loro classificazione.
  \end{itemize}
  \item Pianificazione di progetto
  \begin{itemize}
    \item Individuazione dei rischi e loro classificazione
    \item Definizione scadenze di progetto
    \item Scelta del modello di sviluppo da adottare
    \item Suddivisione del periodo di lavoro in fasi
    \item Pianificazione delle attività
    \item Calcolo del preventivo dei costi, del consuntivo di periodo e del preventivo a finire
    \item Stesura dell'organigramma.
  \end{itemize}
  \item Pianificazione di qualifica
  \begin{itemize}
    \item Selezione degli strumenti e dei criteri per assicurare il rispetto delle \textit{Norme di progetto}
    \item Selezione degli strumenti e delle metriche per la misura della qualità dei documenti
    \item Selezione degli strumenti e delle metriche per la misura della qualità del codice
    \item Selezione degli strumenti per il tracking dei requisiti.
  \end{itemize}
  \item Terminologia e nomenclatura
  \begin{itemize}
    \item Raccolta dei termini ambigui
    \item Definizione dei termini ambigui.
  \end{itemize}
  \item Produzione documentazione.
  \begin{itemize}
    \item Stesura delle \textit{Norme di progetto}
    \item Stesura dell'\textit{Analisi dei requisiti}
    \item Stesura del \textit{Piano di progetto}
    \item Stesura del \textit{Piano di qualifica}
    \item Stesura del \textit{Glossario}.
  \end{itemize}
\end{itemize}
% subs:analisi_e_pianificazione (end)
\subsubsection[Dettagli finali]{Dettagli finali {\normalsize\normalfont\itshape(08/01/2020 \symbol{8594} 14/01/2020)}}%
\label{subs:dettagli_finali}
\begin{itemize}
  \item Incontri con il proponente
  \begin{itemize}
    \item Confronto sull'analisi dei requisiti
    \item Confronto sulla pianificazione di progetto
    \item Confronto sulla pianificazione di qualifica.
  \end{itemize}
  \item Produzione documentazione
  \begin{itemize}
    \item Stesura della lettera di presentazione.
  \end{itemize}
  \item Verifica documentazione.
  \begin{itemize}
    \item Verifica dei documenti prodotti.
  \end{itemize}
\end{itemize}
% subs:dettagli_finali (end)
\begin{figure}[H]
  \centering
  \subfile{diagrams/gantt-analisi-preliminare}
  \caption{diagramma di Gantt per l'Analisi preliminare}%
~~\label{fig:gantt_analisi_preliminare}
\end{figure}
% sub:analisi_preliminare (end)
\subsection[Preparazione in entrata alla RR]{Preparazione in entrata alla RR {\normalsize\normalfont\itshape(15/01/2020 \symbol{8594} 21/01/2020)}}%
\label{sub:preparazione_in_entrata_alla_rr}
In questa fase il gruppo si prepara alla propria presentazione in sede di Revisione dei Requisiti.
I componenti si dividono gli argomenti da esporre e descrivono il lavoro svolto nella fase di Analisi preliminare attraverso delle diapositive.
Inoltre, i componenti del gruppo si occupano di approfondire individualmente la propria conoscenza delle tecnologie che verranno utilizzate nelle fasi successive.
I ruoli coinvolti sono:
\begin{itemize}
  \item Amministratore
  \item Verificatore.
\end{itemize}
Le precondizioni sono:
\begin{itemize}
  \item Tutte le postcondizioni della fase precedente sono state soddisfatte
  \item Il gruppo ha consegnato al committente i documenti richiesti per la partecipazione alla Revisione dei requisiti.
\end{itemize}
Le postcondizioni sono:
\begin{itemize}
  \item Il gruppo ha preparato la propria presentazione da esporre in sede di Revisione dei requisiti
  \item I componenti del gruppo hanno studiato le tecnologie necessarie al progredire del progetto.
\end{itemize}
Questa fase è composta di un unico periodo, che coincide con la fase stessa.
Le attività coinvolte ed i rispettivi compiti che devono essere svolti sono:
\begin{itemize}
  \item Preparazione presentazione.
  \begin{itemize}
    \item Ripartizione delle parti da esporre
    \item Preparazione delle diapositive.
  \end{itemize}
  \item Studio personale.
  \begin{itemize}
    \renewcommand{\labelitemii}{\(\times\)} % chktex 21
    \item Nota: questa attività ricade interamente nelle responsabilità personali di ciascun componente, quindi non verrà pianificata né documentata nel presente documento.
  \end{itemize}
\end{itemize}
\begin{figure}[H]
  \centering
  \subfile{diagrams/gantt-preparazione-rr}
  \caption{diagramma di Gantt per la Preparazione in entrata alla RR}%
~~\label{fig:gantt_preparazione_rr}
\end{figure}
% sub:preparazione_in_entrata_alla_rr (end)
\subsection[Progettazione architetturale]{Progettazione architetturale {\normalsize\normalfont\itshape(22/01/2020 \symbol{8594} 30/03/2020)}}%
\label{sub:progettazione_architetturale}
In questa fase il gruppo trova una soluzione architetturale che soddisfi tutti i requisiti individuati nell'\textit{Analisi dei requisiti}.
Prima di procedere alla progettazione, il gruppo completa le attività di analisi e pianificazione secondo i commenti ricevuti in sede di Revisione dei requisiti.
I ruoli coinvolti sono:
\begin{itemize}
  \item Responsabile
  \item Amministratore
  \item Progettista
  \item Programmatore
  \item Analista
  \item Verificatore.
\end{itemize}
Le precondizioni sono:
\begin{itemize}
  \item Tutte le postcondizioni della fase precedente sono state soddisfatte
  \item In fase di Revisione dei requisiti il gruppo ha aggiudicato la fornitura del prodotto ``Stalker''.
\end{itemize}
Le postcondizioni sono:
\begin{itemize}
  \item Il gruppo ha definito la propria \textit{Technology baseline}.
  \item Il gruppo ha prodotto un \textit{Proof of concept} che presenti le tecnologie che intende utilizzare.
\end{itemize}
Questa fase è divisa in tre periodi.
Per ciascuno si elencano le attività coinvolte ed i rispettivi compiti che devono essere svolti.
\subsubsection[Integrazione in uscita dalla RR]{Integrazione in uscita dalla RR {\normalsize\normalfont\itshape(22/01/2020 \symbol{8594} 02/03/2020)}}%
\label{subs:integrazione_in_uscita_dalla_rr}
\begin{itemize}
  \item Normazione
  \begin{itemize}
    \item Modifica (eventuale) basata sull'esito della Revisione dei requisiti.
  \end{itemize}
  \item Analisi dei requisiti
  \begin{itemize}
    \item Raffinamento dell'analisi effettuata basato sull'esito della Revisione dei requisiti.
  \end{itemize}
  \item Pianificazione di progetto
  \begin{itemize}
    \item Modifica (eventuale) basata sull'esito della Revisione dei requisiti.
  \end{itemize}
  \item Pianificazione di qualifica
  \begin{itemize}
    \item Modifica (eventuale) basata sull'esito della Revisione dei requisiti.
  \end{itemize}
  \item Terminologia e nomenclatura
  \begin{itemize}
    \item Modifica (eventuale) basata sull'esito della Revisione dei requisiti.
  \end{itemize}
  \item Produzione documentazione
  \begin{itemize}
    \item Integrazione delle modifiche nelle \textit{Norme di progetto}
    \item Integrazione delle modifiche nell'\textit{Analisi dei requisiti}
    \item Integrazione delle modifiche nel \textit{Piano di progetto}
    \item Integrazione delle modifiche nel \textit{Piano di qualifica}
    \item Integrazione delle modifiche nel \textit{Glossario}.
  \end{itemize}
  \item Verifica documentazione.
  \begin{itemize}
    \item Verifica dei documenti prodotti.
  \end{itemize}
\end{itemize}
% subs:integrazione_in_uscita_dalla_rr (end)
\subsubsection[Prima versione del PoC]{Prima versione del PoC {\normalsize\normalfont\itshape(03/03/2020 \symbol{8594} 16/03/2020)}}%
\label{subs:prima_versione_del_poc}
\begin{itemize}
  \item Soluzione architetturale
  \begin{itemize}
    \item Scelta delle tecnologie
    \item Progettazione delle classi
    \item Definizione della prima versione del \textit{Proof of concept}.
  \end{itemize}
  \item Codifica
  \begin{itemize}
    \item Codifica della prima versione del \textit{Proof of concept}.
  \end{itemize}
  \item Produzione documentazione
  \begin{itemize}
    \item Raccolta dei diagrammi che riassumono la \textit{Technology baseline}.
  \end{itemize}
  \item Verifica documentazione
  \begin{itemize}
    \item Verifica dei documenti prodotti.
  \end{itemize}
  \item Verifica codice.
  \begin{itemize}
    \item Verifica del \textit{Proof of concept}.
  \end{itemize}
  \item Incontri con il proponente.
  \begin{itemize}
    \item Confronto sulle tecnologie utilizzate
    \item Presentazione della prima versione del \textit{Proof of concept}.
  \end{itemize}
\end{itemize}
% subs:prima_versione_del_poc (end)
\subsubsection[Raffinamento del PoC]{Raffinamento del PoC {\normalsize\normalfont\itshape(17/03/2020 \symbol{8594} 30/03/2020)}}%
\label{subs:raffinamento_del_poc}
\begin{itemize}
  \item Soluzione architetturale
  \begin{itemize}
    \item Definizione dei margini di miglioramento
    \item Definizione delle versioni successive del \textit{Proof of concept}.
  \end{itemize}
  \item Codifica
  \begin{itemize}
    \item Codifica delle versioni successive del \textit{Proof of concept}.
  \end{itemize}
  \item Produzione documentazione
  \begin{itemize}
    \item Aggiornamento dei diagrammi che riassumono la \textit{Technology baseline}.
  \end{itemize}
  \item Verifica documentazione
  \begin{itemize}
    \item Verifica dei documenti prodotti.
  \end{itemize}
  \item Verifica codice
  \begin{itemize}
    \item Verifica del \textit{Proof of concept}.
  \end{itemize}
  \item Incontri con il proponente.
  \begin{itemize}
    \item Presentazione delle versioni successive del \textit{Proof of concept}.
  \end{itemize}
\end{itemize}
% subs:raffinamento_del_poc (end)
\begin{figure}[H]
  \centering
  \subfile{diagrams/gantt-progettazione-architetturale}
  \caption{diagramma di Gantt per la Progettazione architetturale}%
~~\label{fig:gantt_progettazione_architetturale}
\end{figure}
% sub:progettazione_architetturale (end)
\subsection[Incremento 1]{Incremento 1 {\normalsize\normalfont\itshape(31/03/2020 \symbol{8594} 06/04/2020)}}%
\label{sub:incremento_1}
Durante questa fase il gruppo progetta, implementa e verifica le componenti del sistema relative a \textit{registrazione}, \textit{login} e \textit{setup di root}.
I ruoli coinvolti sono:
\begin{itemize}
  \item Responsabile
  \item Amministratore
  \item Progettista
  \item Programmatore
  \item Verificatore.
\end{itemize}
Le precondizioni sono:
\begin{itemize}
  \item Tutte le postcondizioni della fase precedente sono state soddisfatte.
\end{itemize}
Le postcondizioni sono:
\begin{itemize}
  \item Il prodotto soddisfa tutti i requisiti associati con i flussi indicati
  \item Il gruppo ha raccolto le considerazioni della progettazione nella \textit{Product baseline}
  \item Il gruppo ha raccolto le istruzioni sull'utilizzo del sistema in \textit{Manuale utente}
  \item Il gruppo ha raccolto le informazioni rilevanti alla manutenzione del codice nel \textit{Manuale dello sviluppatore}.
\end{itemize}
Questa fase è composta di un unico periodo, che coincide con la fase stessa.
Le attività coinvolte ed i rispettivi compiti che devono essere svolti sono:
\begin{itemize}
  \item Progettazione
  \begin{itemize}
    \item Definizione delle classi e delle loro interrelazioni.
  \end{itemize}
  \item Produzione documentazione
  \begin{itemize}
    \item Stesura \textit{Product baseline}
    \item Stesura \textit{Manuale utente}
    \item Stesura \textit{Manuale dello sviluppatore}.
  \end{itemize}
  \item Implementazione
  \begin{itemize}
    \item Codifica delle classi.
  \end{itemize}
  \item Verifica codice
  \begin{itemize}
    \item Test del codice
    \item Collaudo delle funzionalità.
  \end{itemize}
  \item Verifica documentazione
  \begin{itemize}
    \item Verifica della \textit{Product baseline}
    \item Verifica del \textit{Manuale utente}
    \item Verifica del \textit{Manuale dello sviluppatore}.
  \end{itemize}
\end{itemize}
\begin{figure}[H]
  \centering
  \subfile{diagrams/gantt-incremento-1}
  \caption{diagramma di Gantt per l'Incremento 1}%
~~\label{fig:gantt_incremento_1}
\end{figure}
% sub:incremento_1 (end)
\subsection[Incremento 2]{Incremento 2 {\normalsize\normalfont\itshape(07/04/2020 \symbol{8594} 13/04/2020)}}%
\label{sub:incremento_2}
Durante questa fase il gruppo progetta, implementa e verifica le componenti del sistema relative alla \textit{creazione degli amministratori}.
I ruoli coinvolti sono:
\begin{itemize}
  \item Responsabile
  \item Amministratore
  \item Progettista
  \item Programmatore
  \item Verificatore.
\end{itemize}
Le precondizioni sono:
\begin{itemize}
  \item Tutte le postcondizioni della fase precedente sono state soddisfatte.
\end{itemize}
Le postcondizioni sono:
\begin{itemize}
  \item Il prodotto soddisfa tutti i requisiti associati con i flussi indicati
  \item Il gruppo ha raccolto le considerazioni della progettazione nella \textit{Product baseline}
  \item Il gruppo ha raccolto le istruzioni sull'utilizzo del sistema in \textit{Manuale utente}
  \item Il gruppo ha raccolto le informazioni rilevanti alla manutenzione del codice nel \textit{Manuale dello sviluppatore}.
\end{itemize}
Questa fase è composta di un unico periodo, che coincide con la fase stessa.
Le attività coinvolte ed i rispettivi compiti che devono essere svolti sono:
\begin{itemize}
  \item Progettazione
  \begin{itemize}
    \item Definizione delle classi e delle loro interrelazioni.
  \end{itemize}
  \item Produzione documentazione
  \begin{itemize}
    \item Stesura \textit{Product baseline}
    \item Stesura \textit{Manuale utente}
    \item Stesura \textit{Manuale dello sviluppatore}.
  \end{itemize}
  \item Implementazione
  \begin{itemize}
    \item Codifica delle classi.
  \end{itemize}
  \item Verifica codice
  \begin{itemize}
    \item Test del codice
    \item Collaudo delle funzionalità.
  \end{itemize}
  \item Verifica documentazione.
  \begin{itemize}
    \item Verifica della \textit{Product baseline}
    \item Verifica del \textit{Manuale utente}
    \item Verifica del \textit{Manuale dello sviluppatore}.
  \end{itemize}
\end{itemize}
\begin{figure}[H]
  \centering
  \subfile{diagrams/gantt-incremento-2}
  \caption{diagramma di Gantt per l'Incremento 2}%
~~\label{fig:gantt_incremento_2}
\end{figure}
% sub:incremento_2 (end)
\subsection[Preparazione in entrata alla RP]{Preparazione in entrata alla RP {\normalsize\normalfont\itshape(14/04/2020 \symbol{8594} 20/04/2020)}}%
\label{sub:preparazione_in_entrata_alla_rp}
In questa fase il gruppo si prepara alla propria presentazione in sede di Revisione di progettazione.
I componenti si dividono gli argomenti da esporre e descrivono il lavoro svolto nella fase di Progettazione architetturale attraverso delle diapositive.
I ruoli coinvolti sono:
\begin{itemize}
  \item Amministratore
  \item Verificatore.
\end{itemize}
Le precondizioni sono:
\begin{itemize}
  \item Tutte le postcondizioni della fase precedente sono state soddisfatte
  \item Il gruppo ha consegnato al committente i documenti richiesti per la partecipazione alla Revisione di progettazione.
\end{itemize}
Le postcondizioni sono:
\begin{itemize}
  \item Il gruppo ha preparato la propria presentazione da esporre in sede di Revisione di progettazione.
\end{itemize}
Questa fase è composta di un unico periodo, che coincide con la fase stessa.
Le attività coinvolte ed i rispettivi compiti che devono essere svolti sono:
\begin{itemize}
  \item Preparazione presentazione.
  \begin{itemize}
    \item Ripartizione delle parti da esporre
    \item Preparazione delle diapositive.
  \end{itemize}
\end{itemize}
\begin{figure}[H]
  \centering
  \subfile{diagrams/gantt-preparazione-rp}
  \caption{diagramma di Gantt per la Preparazione in entrata alla RP}%
~~\label{fig:gantt_preparazione_rp}
\end{figure}
% sub:preparazione_in_entrata_alla_rp (end)
\subsection[Incremento 3]{Incremento 3 {\normalsize\normalfont\itshape(21/04/2020 \symbol{8594} 27/04/2020)}}%
\label{sub:incremento_3}
Durante questa fase il gruppo progetta, implementa e verifica le componenti del sistema relative alla \textit{gestione delle organizzazioni}.
I ruoli coinvolti sono:
\begin{itemize}
  \item Responsabile
  \item Amministratore
  \item Progettista
  \item Programmatore
  \item Verificatore.
\end{itemize}
Le precondizioni sono:
\begin{itemize}
  \item Tutte le postcondizioni della fase precedente sono state soddisfatte.
\end{itemize}
Le postcondizioni sono:
\begin{itemize}
  \item Il prodotto soddisfa tutti i requisiti associati con i flussi indicati
  \item Il gruppo ha raccolto le considerazioni della progettazione nella \textit{Product baseline}
  \item Il gruppo ha raccolto le istruzioni sull'utilizzo del sistema in \textit{Manuale utente}
  \item Il gruppo ha raccolto le informazioni rilevanti alla manutenzione del codice nel \textit{Manuale dello sviluppatore}.
\end{itemize}
Questa fase è composta di un unico periodo, che coincide con la fase stessa.
Le attività coinvolte ed i rispettivi compiti che devono essere svolti sono:
\begin{itemize}
  \item Progettazione
  \begin{itemize}
    \item Definizione delle classi e delle loro interrelazioni.
  \end{itemize}
  \item Produzione documentazione
  \begin{itemize}
    \item Stesura \textit{Product baseline}
    \item Stesura \textit{Manuale utente}
    \item Stesura \textit{Manuale dello sviluppatore}.
  \end{itemize}
  \item Implementazione
  \begin{itemize}
    \item Codifica delle classi.
  \end{itemize}
  \item Verifica codice
  \begin{itemize}
    \item Test del codice
    \item Collaudo delle funzionalità.
  \end{itemize}
  \item Verifica documentazione.
  \begin{itemize}
    \item Verifica della \textit{Product baseline}
    \item Verifica del \textit{Manuale utente}
    \item Verifica del \textit{Manuale dello sviluppatore}.
  \end{itemize}
\end{itemize}
\begin{figure}[H]
  \centering
  \subfile{diagrams/gantt-incremento-3}
  \caption{diagramma di Gantt per l'Incremento 3}%
~~\label{fig:gantt_incremento_3}
\end{figure}
% sub:incremento_3 (end)
\subsection[Incremento 4]{Incremento 4 {\normalsize\normalfont\itshape(28/04/2020 \symbol{8594} 04/05/2020)}}%
\label{sub:incremento_4}
Durante questa fase il gruppo progetta, implementa e verifica le componenti del sistema relative all'\textit{interazione degli utenti con le organizzazioni}.
I ruoli coinvolti sono:
\begin{itemize}
  \item Responsabile
  \item Amministratore
  \item Progettista
  \item Programmatore
  \item Verificatore.
\end{itemize}
Le precondizioni sono:
\begin{itemize}
  \item Tutte le postcondizioni della fase precedente sono state soddisfatte.
\end{itemize}
Le postcondizioni sono:
\begin{itemize}
  \item Il prodotto soddisfa tutti i requisiti associati con i flussi indicati
  \item Il gruppo ha raccolto le considerazioni della progettazione nella \textit{Product baseline}
  \item Il gruppo ha raccolto le istruzioni sull'utilizzo del sistema in \textit{Manuale utente}
  \item Il gruppo ha raccolto le informazioni rilevanti alla manutenzione del codice nel \textit{Manuale dello sviluppatore}.
\end{itemize}
Questa fase è composta di un unico periodo, che coincide con la fase stessa.
Le attività coinvolte ed i rispettivi compiti che devono essere svolti sono:
\begin{itemize}
  \item Progettazione.
  \begin{itemize}
    \item Definizione delle classi e delle loro interrelazioni.
  \end{itemize}
  \item Produzione documentazione.
  \begin{itemize}
    \item Stesura \textit{Product baseline}
    \item Stesura \textit{Manuale utente}
    \item Stesura \textit{Manuale dello sviluppatore}.
  \end{itemize}
  \item Implementazione
  \begin{itemize}
    \item Codifica delle classi.
  \end{itemize}
  \item Verifica codice
  \begin{itemize}
    \item Test del codice
    \item Collaudo delle funzionalità.
  \end{itemize}
  \item Verifica documentazione.
  \begin{itemize}
    \item Verifica della \textit{Product baseline}
    \item Verifica del \textit{Manuale utente}
    \item Verifica del \textit{Manuale dello sviluppatore}.
  \end{itemize}
\end{itemize}
\begin{figure}[H]
  \centering
  \subfile{diagrams/gantt-incremento-4}
  \caption{diagramma di Gantt per l'Incremento 4}%
~~\label{fig:gantt_incremento_4}
\end{figure}
% sub:incremento_4 (end)
\subsection[Incremento 5]{Incremento 5 {\normalsize\normalfont\itshape(05/05/2020 \symbol{8594} 11/05/2020)}}%
\label{sub:incremento_5}
Durante questa fase il gruppo progetta, implementa e verifica le componenti del sistema relative a \textit{creazione dei gestori} e \textit{creazione dei visualizzatori}. Inoltre, si prepara alla propria presentazione in sede di Revisione di qualifica.
I ruoli coinvolti sono:
\begin{itemize}
  \item Responsabile
  \item Amministratore
  \item Progettista
  \item Programmatore
  \item Verificatore.
\end{itemize}
Le precondizioni sono:
\begin{itemize}
  \item Tutte le postcondizioni della fase precedente sono state soddisfatte
  \item Il gruppo ha consegnato al committente i documenti richiesti per la partecipazione alla Revisione dei qualifica.
\end{itemize}
Le postcondizioni sono:
\begin{itemize}
  \item Il prodotto soddisfa tutti i requisiti associati con i flussi indicati
  \item Il gruppo ha raccolto le considerazioni della progettazione nella \textit{Product baseline}
  \item Il gruppo ha raccolto le istruzioni sull'utilizzo del sistema in \textit{Manuale utente}
  \item Il gruppo ha preparato la propria presentazione da esporre in sede di Revisione di qualifica.
\end{itemize}
Questa fase è composta di un unico periodo, che coincide con la fase stessa.
Le attività coinvolte ed i rispettivi compiti che devono essere svolti sono:
\begin{itemize}
  \item Progettazione
  \begin{itemize}
    \item Definizione delle classi e delle loro interrelazioni.
  \end{itemize}
  \item Produzione documentazione
  \begin{itemize}
    \item Stesura \textit{Product baseline}
    \item Stesura \textit{Manuale utente}
    \item Stesura \textit{Manuale dello sviluppatore}.
  \end{itemize}
  \item Implementazione
  \begin{itemize}
    \item Codifica delle classi.
  \end{itemize}
  \item Verifica codice
  \begin{itemize}
    \item Test del codice
    \item Collaudo delle funzionalità.
  \end{itemize}
  \item Verifica documentazione
  \begin{itemize}
    \item Verifica della \textit{Product baseline}
    \item Verifica del \textit{Manuale utente}
    \item Verifica del \textit{Manuale dello sviluppatore}.
  \end{itemize}
  \item Preparazione presentazione.
  \begin{itemize}
    \item Ripartizione delle parti da esporre
    \item Preparazione delle diapositive.
  \end{itemize}
\end{itemize}
\begin{figure}[H]
  \centering
  \subfile{diagrams/gantt-incremento-5}
  \caption{diagramma di Gantt per l'Incremento 5}%
~~\label{fig:gantt_incremento_5}
\end{figure}
% sub:incremento_5 (end)
% TODO PREPARAZIONE IN ENTRATA ALLA RQ 12/05 -> 18/05
\subsection[Preparazione in entrata alla RQ]{Preparazione in entrata alla RQ {\normalsize\normalfont\itshape(12/05/2020 \symbol{8594} 18/05/2020)}}%
\label{sub:preparazione_in_entrata_alla_rq}
In questa fase il gruppo si prepara alla propria presentazione in sede di Revisione di qualifica.
I ruoli coinvolti sono:
\begin{itemize}
  \item Amministratore
  \item Verificatore.
\end{itemize}
Le precondizioni sono:
\begin{itemize}
  \item Tutte le postcondizioni della fase precedente sono state soddisfatte
  \item Il gruppo ha consegnato al committente i documenti richiesti per la candidatura alla Revisione di qualifica.
\end{itemize}
Le postcondizioni sono:
\begin{itemize}
  \item Il gruppo ha preparato la propria presentazione da esporre in sede di Revisione di qualifica.
\end{itemize}
Questa fase è composta di un unico periodo, che coincide con la fase stessa.
Le attività coinvolte ed i rispettivi compiti che devono essere svolti sono:
\begin{itemize}
  \item Preparazione presentazione.
  \begin{itemize}
    \item Ripartizione delle parti da esporre
    \item Preparazione delle diapositive.
  \end{itemize}
\end{itemize}
\begin{figure}[H]
  \centering
  \subfile{diagrams/gantt-preparazione-rq}
  \caption{diagramma di Gantt per la Preparazione in entrata alla RQ}%
~~\label{fig:gantt_preparazione_rq}
\end{figure}
% sub:preparazione_in_entrata_alla_rq (end)
\subsection[Incremento 6]{Incremento 6 {\normalsize\normalfont\itshape(19/05/2020 \symbol{8594} 25/05/2020)}}%
\label{sub:incremento_6}
Durante questa fase il gruppo progetta, implementa e verifica le componenti del sistema relative alla \textit{gestione dei luoghi}.
I ruoli coinvolti sono:
\begin{itemize}
  \item Responsabile
  \item Amministratore
  \item Progettista
  \item Programmatore
  \item Verificatore.
\end{itemize}
Le precondizioni sono:
\begin{itemize}
  \item Tutte le postcondizioni della fase precedente sono state soddisfatte.
\end{itemize}
Le postcondizioni sono:
\begin{itemize}
  \item Il prodotto soddisfa tutti i requisiti associati con i flussi indicati
  \item Il gruppo ha raccolto le considerazioni della progettazione nella \textit{Product baseline}
  \item Il gruppo ha raccolto le istruzioni sull'utilizzo del sistema in \textit{Manuale utente}
  \item Il gruppo ha raccolto le informazioni rilevanti alla manutenzione del codice nel \textit{Manuale dello sviluppatore}.
\end{itemize}
Questa fase è composta di un unico periodo, che coincide con la fase stessa.
Le attività coinvolte ed i rispettivi compiti che devono essere svolti sono:
\begin{itemize}
  \item Progettazione
  \begin{itemize}
    \item Definizione delle classi e delle loro interrelazioni.
  \end{itemize}
  \item Produzione documentazione
  \begin{itemize}
    \item Stesura \textit{Product baseline}
    \item Stesura \textit{Manuale utente}
    \item Stesura \textit{Manuale dello sviluppatore}.
  \end{itemize}
  \item Implementazione
  \begin{itemize}
    \item Codifica delle classi.
  \end{itemize}
  \item Verifica codice
  \begin{itemize}
    \item Test del codice
    \item Collaudo delle funzionalità.
  \end{itemize}
  \item Verifica documentazione.
  \begin{itemize}
    \item Verifica della \textit{Product baseline}
    \item Verifica del \textit{Manuale utente}
    \item Verifica del \textit{Manuale dello sviluppatore}.
  \end{itemize}
\end{itemize}
\begin{figure}[H]
  \centering
  \subfile{diagrams/gantt-incremento-6}
  \caption{diagramma di Gantt per l'Incremento 6}%
~~\label{fig:gantt_incremento_6}
\end{figure}
% sub:incremento_6 (end)
\subsection[Incremento 7]{Incremento 7 {\normalsize\normalfont\itshape(26/05/2020 \symbol{8594} 01/06/2020)}}%
\label{sub:incremento_7}
Durante questa fase il gruppo progetta, implementa e verifica le componenti del sistema relative a \textit{recupero credenziali}, \textit{storico degli accessi utente} e \textit{trasferimento proprietà}.
I ruoli coinvolti sono:
\begin{itemize}
  \item Responsabile
  \item Amministratore
  \item Progettista
  \item Programmatore
  \item Verificatore.
\end{itemize}
Le precondizioni sono:
\begin{itemize}
  \item Tutte le postcondizioni della fase precedente sono state soddisfatte.
\end{itemize}
Le postcondizioni sono:
\begin{itemize}
  \item Il prodotto soddisfa tutti i requisiti associati con i flussi indicati
  \item Il gruppo ha raccolto le considerazioni della progettazione nella \textit{Product baseline}
  \item Il gruppo ha raccolto le istruzioni sull'utilizzo del sistema in \textit{Manuale utente}
  \item Il gruppo ha raccolto le informazioni rilevanti alla manutenzione del codice nel \textit{Manuale dello sviluppatore}.
\end{itemize}
Questa fase è composta di un unico periodo, che coincide con la fase stessa.
Le attività coinvolte ed i rispettivi compiti che devono essere svolti sono:
\begin{itemize}
  \item Progettazione
  \begin{itemize}
    \item Definizione delle classi e delle loro interrelazioni.
  \end{itemize}
  \item Produzione documentazione
  \begin{itemize}
    \item Stesura \textit{Product baseline}
    \item Stesura \textit{Manuale utente}
    \item Stesura \textit{Manuale dello sviluppatore}.
  \end{itemize}
  \item Implementazione
  \begin{itemize}
    \item Codifica delle classi.
  \end{itemize}
  \item Verifica codice
  \begin{itemize}
    \item Test del codice
    \item Collaudo delle funzionalità.
  \end{itemize}
  \item Verifica documentazione
  \begin{itemize}
    \item Verifica della \textit{Product baseline}
    \item Verifica del \textit{Manuale utente}
    \item Verifica del \textit{Manuale dello sviluppatore}.
  \end{itemize}
\end{itemize}
\begin{figure}[H]
  \centering
  \subfile{diagrams/gantt-incremento-7}
  \caption{diagramma di Gantt per l'Incremento 7}%
~~\label{fig:gantt_incremento_7}
\end{figure}
% sub:incremento_7 (end)
\subsection[Verifica e collaudo finali]{Verifica e collaudo finali {\normalsize\normalfont\itshape(02/06/2020 \symbol{8594} 11/06/2020)}}%
\label{sub:verifica_e_collaudo_finali}
In questa fase il gruppo verifica l'integrità del sistema, si assicura che tutti i requisiti siano effettivamente rispettati, e risolve eventuali problemi insorti nel corso delle fasi di incremento.
I ruoli coinvolti sono:
\begin{itemize}
  \item Responsabile
  \item Amministratore
  \item Progettista
  \item Programmatore
  \item Verificatore.
\end{itemize}
Le precondizioni sono:
\begin{itemize}
  \item Tutte le postcondizioni della fase precedente sono state soddisfatte.
\end{itemize}
Le postcondizioni sono:
\begin{itemize}
  \item Il prodotto è pronto per il collaudo.
\end{itemize}
Questa fase è composta di un unico periodo, che coincide con la fase stessa.
Le attività coinvolte ed i rispettivi compiti che devono essere svolti sono:
\begin{itemize}
  \item Verifica codice
  \begin{itemize}
    \item Collaudo delle funzionalità.
  \end{itemize}
  \item Verifica documentazione
  \begin{itemize}
    \item Verifica del \textit{Manuale utente}
    \item Verifica del \textit{Manuale dello sviluppatore}.
  \end{itemize}
\end{itemize}
\begin{figure}[H]
  \centering
  \subfile{diagrams/gantt-verifica-e-collaudo-finali}
  \caption{diagramma di Gantt per Verifica e collaudo finali}%
~~\label{fig:gantt_verifica_e_collaudo_finali}
\end{figure}
% sub:verifica_e_collaudo_finali (end)
\subsection[Preparazione in entrata alla RA]{Preparazione in entrata alla RA {\normalsize\normalfont\itshape(12/06/2020 \symbol{8594} 18/06/2020)}}%
\label{sub:preparazione_in_entrata_alla_ra}
In questa fase il gruppo si prepara alla propria presentazione in sede di Revisione di accettazione.
I ruoli coinvolti sono:
\begin{itemize}
  \item Amministratore
  \item Verificatore.
\end{itemize}
Le precondizioni sono:
\begin{itemize}
  \item Tutte le postcondizioni della fase precedente sono state soddisfatte
  \item Il gruppo ha consegnato al committente i documenti richiesti per la partecipazione alla Revisione di accettazione.
\end{itemize}
Le postcondizioni sono:
\begin{itemize}
  \item Il gruppo ha preparato la propria presentazione da esporre in sede di Revisione di accettazione.
\end{itemize}
Questa fase è composta di un unico periodo, che coincide con la fase stessa.
Le attività coinvolte ed i rispettivi compiti che devono essere svolti sono:
\begin{itemize}
  \item Preparazione presentazione.
  \begin{itemize}
    \item Ripartizione delle parti da esporre
    \item Preparazione delle diapositive.
  \end{itemize}
\end{itemize}
\begin{figure}[H]
  \centering
  \subfile{diagrams/gantt-preparazione-ra}
  \caption{diagramma di Gantt per la Preparazione in entrata alla RA}%
~~\label{fig:gantt_preparazione_ra}
\end{figure}
% sub:preparazione_in_entrata_alla_ra (end)
\end{document}
