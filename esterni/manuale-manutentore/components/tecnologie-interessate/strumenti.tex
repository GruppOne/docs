\documentclass[../../../manuale-manutentore.tex]{subfiles}

\begin{document}

\subsection{Strumenti}%
\label{sub:strumenti}
In questa sezione sono descritti tutti gli strumenti utilizzati per l'implementazione di Stalker.

\subsubsection{MySQL}%
\label{subs:mysql}
Lo strumento utilizzato per la memorizzazione dei dati persistenti è \glossarioLocale{MySQL}, un DBMS open source relazionale.
Gestisce la maggior parte degli endpoint associati ad operazioni della web application, e comunica con il server tramite la libreria \glossarioLocale{R2DBC} del framework Spring.

\subsubsection{InfluxDB}%
\label{subs:influxdb}

Per rendere più efficienti le operazioni di archiviazione e recupero dei dati ad alta disponibilità abbiamo utilizzato il Time Series Database \glossarioLocale{InfluxDB}, un database appositamente progettato ed ottimizzato per catalogare serie temporali di dati.
InfluxDB si occupa di salvare coppie timestamp-valore, permettendo di registrare l'istante in cui è avvenuto il salvataggio dei dati.
Gestisce una parte degli endpoint associati ad operazioni della mobile application.

\subsubsection{Docker}%
\label{subs:docker}

Abbiamo utilizzato \glossarioLocale{Docker} per la containerizzazione di servizi e la simulazione di componenti software in ambienti virtuali.
Questa piattaforma permette di creare, testare e distribuire applicazioni con la massima rapidità.
Docker raccoglie il software in ambienti isolati, chiamati \glossarioLocale{container}, che offrono tutto il necessario per la loro corretta esecuzione, incluse librerie, strumenti di sistema, codice e runtime.
I container vengono configurati tramite dei particolari file chiamati \glossarioLocale{Dockerfile}, che contengono operazioni specifiche e configurazioni che vengono eseguite all'avvio.
Con Docker è quindi possibile distribuire, replicare e ricalibrare le risorse per un'applicazione in qualsiasi ambiente, tenendo sempre sotto controllo il codice eseguito.
Il vantaggio che deriva dal suo utilizzo è la possibilità di simulare l'esecuzione del proprio ambiente di sviluppo su macchine diverse, senza occuparsi dell'installazione e della configurazione dei servizi necessari.
Ne consegue che il consumo di risorse avviene solo nel momento in cui è attivo l'ambiente di sviluppo.
Nell'ambito di questo progetto, le risorse che sono containerizzate da Docker sono MySQL e InfluxDB.

\subsubsection{Docker Compose}%
\label{subs:docker_compose}

Lo strumento utilizzato per la definizione e l'esecuzione di applicazioni Docker multi-contenitore è \glossarioLocale{Docker Compose}.
Viene utilizzato un unico file, chiamato Compose, per la configurazione di tutti i servizi dell'applicazione, e un singolo comando per creare e avviare i servizi contenuti nel Compose.

% \subsubsection{Kubernetes}%
% \label{subs:kubernetes}

% % già pronta la sottosezione, nel caso dovesse essere implementato

\subsubsection{Karma}%
\label{subs:karma}

Per eseguire i test della web application abbiamo utilizzato \glossarioLocale{Karma}, per il quale è disponibile un plugin installabile su vari browser.
Abbiamo scelto questo strumento perché è ben integrato con Angular e con Jasmine.

\subsubsection{Cucumber}%
\label{subs:cucumber}

Abbiamo utilizzato \glossarioLocale{Cucumber} per testare i comportamenti software previsti in un linguaggio logico che gli sviluppatori possano comprendere facilmente, e che quindi supporta il Behaviour-Driven Development (\glossarioLocale{BDD}).
La stesura dei test viene fatta seguendo la sintassi \glossarioLocale{Gherkin}, consentendo di avere dei test autoesplicativi suddivisi in step che vengono implementati con l'aiuto della libreria di asserzioni \glossarioLocale{Chai} ed eseguiti tramite \glossarioLocale{Protractor}.

\subsubsection{Gradle}%
\label{subs:gradle}

Per la \glossarioLocale{build automation} e per la gestione delle dipendenze di server e applicazione mobile abbiamo utilizzato \glossarioLocale{Gradle}.
Gradle è stato preferito a \glossarioLocale{Maven}, in quanto Maven utilizza una configurazione rigida basata sul formato \glossarioLocale{XML}, mentre Gradle utilizza un linguaggio dichiarativo di scripting \glossarioLocale{Groovy} dove il codice è compilato runtime ed assume la forma di bytecode per la Java Virtual Machine (\glossarioLocale{JVM}).
Inoltre, Gradle è lo standard de facto nella programmazione Android, e abbiamo scelto di utilizzarlo anche nel server per uniformità.

\end{document}
