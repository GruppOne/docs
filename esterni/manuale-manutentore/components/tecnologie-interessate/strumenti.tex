\documentclass[../../../manuale-manutentore.tex]{subfiles}

\begin{document}

\subsection{Strumenti}%
\label{sub:strumenti}
In questa sezione sono descritti tutti gli strumenti utilizzati per l'implementazione di Stalker.

\subsubsection{MySQL}%
\label{subs:mysql}
Lo strumento lato backend utilizzato per la memorizzazione di dati persistenti è \glossarioLocale{MySQL}, un sistema open source di gestione del database relazionale SQL\@.

Gestisce la maggior parte degli endpoint associati ad operazioni della web application, e comunica con il server tramite la libreria \glossarioLocale{R2DBC} del framework Spring.

E' possibile amministrare il database di Stalker tramite l'interfaccia grafica web fornita da \glossarioLocale{PhpMyAdmin}.

\subsubsection{PhpMyAdmin}%
\label{subs:phpmyadmin}

Lo strumento lato backend utilizzato per l'amministrazione del database MySQL tramite un qualsiasi browser è PhpMyAdmin.

Per poter usufruire dell'interfaccia grafica, è necessario disporre le credenziali da amministratore.

\subsubsection{InfluxDB}%
\label{subs:influxdb}
Back End

Lo strumento lato backend utilizzato per ottimizzare le operazioni di archiviazione e il recupero dei dati ad alta disponibilità è il Time Series Database (TSDB) \glossarioLocale{InfluxDB}, un database appositamente progettato ed ottimizzato per catalogare serie temporali di dati.

InfluxDB si occupa di salvare coppie tempo-valore, conoscendo in quale preciso punto del tempo è avvenuto il salvataggio dei dati non persistenti.

Gestisce una parte degli endpoint associati ad operazioni della mobile application.

\subsubsection{Docker}%
\label{subs:docker}

Lo strumento lato backend utilizzato per la containerizzazione di servizi e la simulazione di componenti software in ambienti virtuali è \glossarioLocale{Docker}. Questa piattaforma software permette di creare, testare e distribuire applicazioni con la massima rapidità.

Docker raccoglie il software in ambienti isolati leggeri, chiamati \glossarioLocale{container}, che offrono tutto il necessario per la loro corretta esecuzione, incluse librerie, strumenti di sistema, codice e runtime.

I container vengono configurati tramite dei particolari file chiamati \glossarioLocale{Dockerfile}, che contengono operazioni specifiche e configurazioni che vengono eseguite al suo avvio.

Con Docker, è quindi possibile distribuire, replicare e ricalibrare le risorse per un'applicazione in qualsiasi ambiente, tenendo sempre sotto controllo il codice eseguito.

Il vantaggio che ne deriva dal suo utilizzo è la possibilità di simulare l'esecuzione del proprio ambiente di sviluppo su macchine diverse, senza occuparsi dell'installazione e della configurazione dei servizi necessari. Ne deriva che il consumo di risorse avviene solo nel momento in cui è attivo l'ambiente di sviluppo, ovvero quando viene riprodotto e utilizzato per l'attività di sviluppo.

Nell'ambito di questo progetto, le risorse che sono containerizzate da Docker sono MySQL, PhpMyAdmin e InfluxDB, ovvero tutti gli strumenti che vengono utilizzati nello strato di persistenza.

\subsubsection{Docker Compose}%
\label{subs:docker_compose}

Lo strumento lato backend utilizzato per la definizione e l'esecuzione di applicazioni Docker multi-contenitore è \glossarioLocale{Docker Compose}.

Viene utilizzato un unico file, chiamato Compose, per la configurazione di tutti i servizi dell'applicazione, e un singolo comando per creare e avviare i servizi contenuti nel Compose.

% \subsubsection{Kubernetes}%
% \label{subs:kubernetes}

% % già pronta la sottosezione, nel caso dovesse essere implementato

\subsubsection{Karma}%
\label{subs:karma}

Lo strumento utilizzato per testare l'esecuzione della web application su più browser è \glossarioLocale{Karma}.

E' possibile installare il plugin per diversi browser, ma è consigliato di installarlo su Google Chrome, sapendo che la maggior parte degli utenti Android utilizza questo browser, e quindi rispecchia il target della web application di Stalker.

\subsubsection{Cucumber}%
\label{subs:cucumber}

Lo strumento utilizzato per testare i comportamenti software previsti in un linguaggio logico che gli sviluppatori possano comprendere facilmente, e che quindi supporta il Behaviour-Driven Development (\glossarioLocale{BDD}), è \glossarioLocale{Cucumber}.

La stesura dei test viene fatta seguendo la sintassi \glossarioLocale{Gherkin}, consentendo di avere dei test auto-esplicativi suddivisi in step che vengono implementati con l’aiuto della libreria di asserzioni \glossarioLocale{Chai} ed eseguiti tramite \glossarioLocale{Protractor}.

\subsubsection{Gradle}%
\label{subs:gradle}

Lo strumento utilizzato per la gestione del software e la \glossarioLocale{build automation} è \glossarioLocale{Gradle}.

In particolare, Gradle è utilizzato sia lato backend che lato frontend per quanto riguarda la mobile application.

Gradle è stato preferito a \glossarioLocale{Maven}, in quanto Maven utilizza una configurazione rigida basata sul formato \glossarioLocale{XML}, mentre Gradle utilizza un linguaggio dichiarativo di scripting \glossarioLocale{Groovy} dove il codice è compilato runtime ed assume la forma di bytecode per la Java Virtual Machine (\glossarioLocale{JVM}).

Inoltre, Gradle è integrato da tempo nella programmazione Android, quindi per coerenza anche la parte server utilizza dipendenze Gradle.

Il file che serve a generare la build prende il nome di \textit{build.gradle}.

\end{document}
