\documentclass[../../../manuale-manutentore.tex]{subfiles}

\begin{document}

\subsection{Librerie}%
\label{sub:librerie}
In questa sezione sono descritte tutte le librerie utilizzate per l'implementazione di Stalker.


\subsubsection{Angular Material}%
\label{subs:angular_material}

La libreria utilizzata per definire gli stili lato frontend, in particolare nella web application, è \glossarioLocale{Angular Material}, che è una implementazione della specifica \glossarioLocale{Material Design} di Google.

Questo progetto fornisce una serie di componenti grafici testati e riusabili per \glossarioLocale{AngularJS}\@.

Gli stili forniti da Angular Material sono completamente personalizzabili attraverso il linguaggio di stili a cascata \glossarioLocale{CSS}.

\subsubsection{Leaflet}%
\label{subs:leaflet}

La libreria utilizzata per gestire mappe interattive lato frontend, in particolare nella web application, è \glossarioLocale{Leaflet}, una libreria che si può integrare con Angular che consente di implementare mappe user-friendly basate su \glossarioLocale{OpenStreetMap}.

Leaflet permette di creare, modificare ed eliminare luoghi direttamente sulla mappa grazie al componente \textit{Leaflet Draw}, ed ottenere le informazioni necessarie grazie alla funzionalità di \glossarioLocale{reverse geocoding}, svincolando l'utente dall'onere di inserire manualmente l'insieme di coordinate, l'indirizzo e il nome associato al luogo selezionato.

\subsubsection{Chai}%
\label{subs:chai}

\glossarioLocale{Chai} è una libreria di asserzioni BDD/TDD utilizzata per il lato frontend, in particolare nella web application, che offre diverse interfacce utilizzabili dallo sviluppatore e può essere abbinata a qualsiasi framework di test JavaScript.

Chai utilizza \glossarioLocale{Node.js}, infatti è disponibile l'installazione della libreria tramite \glossarioLocale{npm}. Si può utilizzare anche all'interno del browser web, includendo il file JavaScript Chai all'interno di una pagina.

\subsubsection{Volley}%
\label{subs:volley}

\glossarioLocale{Volley} è una libreria consigliata da Google, utilizzata per il lato frontend in particolare la mobile application, che forniscono strumenti wrapper per le funzionalità di connessione a Internet di basso livello.

Volley si integra facilmente con qualsiasi protocollo e supporta stringhe, immagini e JSON non elaborati.

Inoltre, svincola il programmatore dal scrivere codice boilerplate per concentrarsi unicamente sulla logica della mobile application.

Volley non è adatto per operazioni di download o streaming di grandi dimensioni, poiché conserva tutte le risposte in memoria durante l'analisi.

\subsubsection{Lombok}%
\label{subs:lombok}

\glossarioLocale{Lombok} è una libreria Java utilizza sia lato backend che lato frontend, in particolare per la mobile application, che genera automaticamente attraverso delle annotazioni Java buona parte del codice boilerplate.

Lombok si occupa di utilizzare una serie di annotazioni da inserire precedentemente alla dichiarazione di una classe o di un metodo, in modo da generare automaticamente i costruttori, i metodi \textit{getter}, \textit{setter}, \textit{toString ()}, \textit{hashCode ()} o i builder.

Per maggiori informazioni sulle possibili annotazioni da implementare, visitare \href{https://projectlombok.org/features/all}.

\subsubsection{JJWT}%
\label{subs:jjwt}
Mobile Application

La libreria che consente la creazione di \glossarioLocale{token di accesso} basati su \glossarioLocale{JSON} è \glossarioLocale{JJWT}.

JJWT coinvolge tutte le componenti di Stalker, e nello specifico serve a creare, codificare e ottenere informazioni dai token.

Inoltre, i token di accesso sono firmati utilizzando una chiave privata o una coppia chiave pubblica/privata, e quindi utilizzati nei casi di autenticazione ed autorizzazione.

Un file JSON viene inviato ad ogni richiesta di login da web application e mobile application.

Per maggiori informazioni, visitare \href{https://github.com/jwtk/jjwt}.

\end{document}
