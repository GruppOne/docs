\documentclass[../../manuale-manutentore.tex]{subfiles}

\begin{document}

\subsection{Librerie}%
\label{sub:librerie}
In questa sezione sono descritte tutte le librerie utilizzate per l'implementazione di Stalker.


\subsubsection{Angular Material}%
\label{subs:angular_material}

La libreria utilizzata per definire gli stili per la web application è \glossarioLocale{Angular Material}, una implementazione della specifica \glossarioLocale{Material Design} di Google.
Questo progetto fornisce una serie di componenti grafici testati e riusabili per \glossarioLocale{AngularJS}\@.
Gli stili forniti da Angular Material sono completamente personalizzabili attraverso il linguaggio \glossarioLocale{CSS}.

\subsubsection{Leaflet}%
\label{subs:leaflet}

La libreria utilizzata per gestire mappe interattive nella web application è \glossarioLocale{Leaflet}, una libreria integrabile con Angular che consente di implementare mappe user-friendly basate su \glossarioLocale{OpenStreetMap}.
Leaflet permette di creare, modificare ed eliminare luoghi direttamente sulla mappa grazie al componente \textit{Leaflet Draw}, ed ottenere le informazioni necessarie grazie alla funzionalità di \glossarioLocale{reverse geocoding}, svincolando l'utente dall'onere di inserire manualmente l'insieme di coordinate, l'indirizzo e il nome associato al luogo selezionato.

\subsubsection{Chai}%
\label{subs:chai}

\glossarioLocale{Chai} è una libreria di asserzioni BDD/TDD utilizzata per i test della web application.
Offre diverse interfacce utilizzabili dallo sviluppatore e può essere abbinata a qualsiasi framework di test JavaScript.
Chai utilizza \glossarioLocale{Node.js}, quindi è disponibile la sua installazione tramite \glossarioLocale{npm}.
Si può utilizzare anche all'interno del browser web, includendo il file JavaScript Chai all'interno di una pagina.

\subsubsection{Volley}%
\label{subs:volley}

\glossarioLocale{Volley} è una libreria consigliata da Google, utilizzata nella mobile application, che fornisce astrazioni per le funzionalità di connessione a Internet di basso livello.
Volley si integra facilmente con qualsiasi protocollo ed offre un supporto approfondito per lo scambio di stringhe, immagini e oggetti JSON non elaborati.
Inoltre, svincola il programmatore dallo scrivere codice boilerplate per concentrarsi unicamente sulla logica della mobile application.

\subsubsection{Lombok}%
\label{subs:lombok}

\glossarioLocale{Lombok} è una libreria Java utilizzata sia lato backend che lato frontend, in particolare per la mobile application, che genera automaticamente attraverso delle annotazioni Java buona parte del codice boilerplate.
Lombok sfrutta una serie di annotazioni da inserire alla dichiarazione di una classe o di un metodo, in modo da generare automaticamente, ad esempio, costruttori, metodi \textit{getter}, \textit{setter}, \java{toString ()}, \java{hashCode ()}, o i builder.
Per maggiori informazioni sulle annotazioni, visitare \href{https://projectlombok.org/features/all}{https://projectlombok.org/features/all}.

\subsubsection{JJWT}%
\label{subs:jjwt}

Per la creazione di \glossarioLocale{token di accesso} basati su \glossarioLocale{JSON} abbiamo utilizzato \glossarioLocale{JJWT}.
JJWT coinvolge tutte le componenti di Stalker, e nello specifico serve a creare, codificare e ottenere informazioni dai token.
Per maggiori informazioni, visitare \href{https://github.com/jwtk/jjwt}{https://github.com/jwtk/jjwt}.

\end{document}
