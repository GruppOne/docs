\documentclass[../../../manuale-manutentore.tex]{subfiles}

\begin{document}

\subsection{Linguaggi}%
\label{sub:linguaggi}
In questa sezione sono descritti tutti i linguaggi utilizzati per l'implementazione di Stalker.

\subsubsection{Java}%
\label{subs:java}

Il linguaggio di programmazione utilizzato per lo sviluppo del backend e della mobile-app è Java, in quanto nel primo caso ci permette di utilizzare \glossarioLocale{Spring}, la nostra prima scelta tra i framework per lo sviluppo del server, e nel secondo è il linguaggio meglio documentato e di più facile apprendimento per lo sviluppo di app Android.
Si specifica che nella parte backend la versione utilizzata è la 11, mentre nella mobile application la versione utilizzata è la 8.

\subsubsection{TypeScript}%
\label{subs:typescript}

Il linguaggio di programmazione utilizzato per lo sviluppo della web application è \glossarioLocale{TypeScript}, un linguaggio moderno che estende la sintassi di \glossarioLocale{JavaScript} introducendo un sistema di tipi.
Il codice Typescript viene tradotto a JavaScript attraverso un transpiler.
Inoltre, TypeScript è il linguaggio consigliato per lo sviluppo di applicazioni \textit{Angular}, che abbiamo scelto come framework per lo sviluppo della web application.

\subsubsection{OpenAPI}%
\label{subs:openapi}

Per descrivere i servizi che sono forniti dal server backend di Stalker abbiamo utilizzato la specifica \glossarioLocale{OpenAPI}, che definisce quali sono le interfacce per la comunicazione tra le componenti del sistema.
La specifica OpenAPI definisce quali sono gli endpoint per la comunicazione tra le varie componenti e, per ciascun endpoint, quali metodi HTTP sono utilizzabili e come\@.

\end{document}
