\documentclass[../../../manuale-manutentore.tex]{subfiles}

\begin{document}

\subsection{Linguaggi}%
\label{sub:linguaggi}
In questa sezione sono descritti tutti i linguaggi utilizzati per l'implementazione di Stalker.

\subsubsection{Java}%
\label{subs:java}

Il linguaggio di programmazione utilizzato per lo sviluppo del server backend è \glossarioLocale{Java}, in quanto si interfaccia facilmente con il framework \glossarioLocale{Spring}.

Inoltre, Java è utilizzato come linguaggio per lo sviluppo della mobile application, in quanto è il linguaggio di programmazione più diffuso per lo sviluppo di piattaforme Android, nonostante da qualche anno il linguaggio ufficiale sia \glossarioLocale{Kotlin}.

Questi fattori e le conoscenze pregresse dei membri del gruppo hanno indirizzato all'unanimità la scelta di questo linguaggio, che permette anche di scrivere ed eseguire test tramite la libreria \glossarioLocale{JUnit}.

Si specifica che nella parte backend la versione utilizzata è la 11, mentre nella mobile application la versione utilizzata è la 8.

\subsubsection{TypeScript}%
\label{subs:typescript}

Il linguaggio di programmazione utilizzato per lo sviluppo backend della web application è \glossarioLocale{TypeScript}, un linguaggio moderno che estende la sintassi di \glossarioLocale{JavaScript} in modo che il codice scritto in JavaScript sia in grado di funzionare con TypeScript senza nessuna modifica.

La particolarità di questo linguaggio è che deve essere convertito in linguaggio JavaScript prima di poter essere eseguito, grazie all'utilizzo di un tool chiamato \textit{transpiler} che converte il codice sorgente da un certo linguaggio ad un altro.

Inoltre, TypeScript è il linguaggio consigliato per lo sviluppo di applicazioni \textit{Angular}, che è stato utilizzato come framework per la web application nell'ambito del progetto Stalker.

\subsubsection{OpenAPI}%
\label{subs:openapi}

Il linguaggio utilizzato per descrivere i servizi che sono forniti dal server backend di Stalker è la specifica \glossarioLocale{OpenAPI}, che definisce quali sono le interfacce per la comunicazione tra le componenti del sistema, ovvero tra server-web application e server-mobile application.

La specifica OpenAPI semplifica la sincronizzazione che ci deve essere tra documentazione e codice sorgente, e definisce in particolare quali sono gli endpoint per la comunicazione tra le varie componenti e le sue operazioni tramite l'utilizzo del protocollo di rete HTTP\@.

Open API è in grado inoltre di generare codice e test case, ma in questo progetto viene utilizzato solo per la documentazione.

\end{document}
