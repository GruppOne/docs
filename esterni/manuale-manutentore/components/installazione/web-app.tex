\documentclass[../manuale-manutentore.tex]{subfiles}

\begin{document}

\subsection{Web application}%
\label{sub:web_application}

\subsubsection{Installazione delle dipendenze}%
\label{subs:installazione_delle_dipendenze}

Prima di effettuare qualsiasi operazione, l'utente deve posizionarsi nella root della cartella che contiene il codice della web application ed eseguire:

\begin{minted}{bash}
  npm install
\end{minted}

In questo modo saranno installate tutte le dipendenze necessarie per l'esecuzione della web application.
La procedura potrebbe richiedere qualche minuto.

\subsubsection{Avvio del server}%
\label{subs:avvio_del_server}

Per ottenere un server di sviluppo, utilizzare il seguente comando:

\begin{minted}{bash}
  ng serve
\end{minted}

Se la procedura è andata a buon fine, sarà possibile interagire con la web application all'indirizzo \texttt{http://localhost:4200/} e verificare che si visualizzi la pagina home del sito.
Altrimenti, la compilazione del progetto mostrerà un report contenente eventuali errori rilevati.
La procedura impiegherà qualche minuto.
Non serve che questo comando sia digitato più di una volta, in quanto le pagine web si aggiornano automaticamente al cambiamento dei file sorgente.

\subsubsection{Creazione di nuovi componenti}%
\label{subs:creazione_dei_nuovi_componenti}

Per generare un nuovo componente, eseguire il seguente comando:

\begin{minted}{bash}
  ng generate component component-name
\end{minted}

La sintassi completa di questo comando è:

\begin{minted}{bash}
  ng generate directive|pipe|service|class|guard|interface|enum|module
\end{minted}

\subsubsection{Build}%
\label{subs:build}

Per effettuare la build di questo progetto, eseguire il seguente comando:

\begin{minted}{bash}
  ng build
\end{minted}

Per effettuare una build di produzione, eseguire il seguente comando:

\begin{minted}{bash}
  ng build --prod
\end{minted}

\subsubsection{Configurazione}%
\label{subs:configurazione}

Per eseguire comandi con una configurazione specifica, eseguire il seguente comando:

\begin{minted}{bash}
  ng build|serve|test --configuration=your-configuration
\end{minted}

\bash{your-configuration} corrisponde alla configurazione desiderata.

Le configurazioni possibili sono:
\begin{itemize}
  \item \bash{localhost}
  \item \bash{imola}
  \item \bash{production}.
\end{itemize}

\subsubsection{Aiuto}%
\label{subs:aiuto}

Per ricevere informazioni sui comandi di Angular CLI, eseguire:
\begin{itemize}
  \item \textit{ng help}.
  \item oppure visionare il README di Angular CLI al seguente indirizzo \href{https://github.com/angular/angular-cli/blob/master/README.md}{https://github.com/angular/angular-cli/blob/master/README.md}.
\end{itemize}

\end{document}
