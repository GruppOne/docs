\documentclass[../../manuale-manutentore.tex]{subfiles}

\begin{document}

\subsection{Mobile application}%
\label{sub:mobile_application}

\subsubsection{Installazione IDE}%
\label{subs:installazione_ide}

Per configurare l'ambiente di lavoro dell'applicazione mobile di Stalker, è necessario installare l'\glossarioLocale{IDE} \glossarioLocale{IntelliJ IDEA}.
Il team ha deciso di utilizzare IntelliJ IDEA Community versione 2019.3.x come IDE di riferimento.
L'utente deve installare JetBrains Toolbox (\href{https://www.jetbrains.com/toolbox-app/}{https://www.jetbrains.com/toolbox-app/}), e poi selezionare per l'installazione IntelliJ IDEA\@.

\subsubsection{Debug ed esecuzione applicazione}%
\label{subs:debug_ed_esecuzione_applicazione}

Per il \glossarioLocale{debug} dell'applicazione mobile consigliamo di utilizzare gli strumenti integrati nell'IDE\@. In particolare, eseguendo l'applicazione con il comando \textit{Run} è possibile monitorare il suo stato attraverso la scheda omonima in basso a sinistra. Lo stesso si può fare da linea di comando, eseguendo però due comandi separati:

\begin{minted}{bash}
    ./gradlew installDebug
\end{minted}

per l'installazione, seguito da

\begin{minted}{bash}
    adb shell am start
    >> -n "tech.gruppone.stalker.app/tech.gruppone.stalker.app.view.SplashScreenActivity"
    >> -a android.intent.action.MAIN
    >> -c android.intent.category.LAUNCHER
\end{minted}

per l'esecuzione. Il monitoraggio del log eve essere eseguito attraverso il tool \bash{logcat}.

\end{document}
