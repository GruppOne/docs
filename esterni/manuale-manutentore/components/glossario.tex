\documentclass[../manuale-manutentore.tex]{subfiles}

\begin{document}

\phantomsection{}
\subsection{A}

\begin{description}
    \item[ADB] acronimo di Android Debug Bridge, permette la comunicazione tramite shell con il dispositivo Android.
    \item[Android SDK] pacchetto di sviluppo software che offre un kit di sviluppo per ottenere il codice sorgente di una particolare versione di Android.
    \item[Android Studio] IDE di Google per la realizzazione di applicazioni Android.
    \item[Angular] framework open source, sviluppato principalmente da Google, per lo sviluppo di applicazioni web nel linguaggio TypeScript.
    \item[Angular Material] libreria utilizzata per la definizione di stili per le pagine web Angular.
    \item[AngularJS] framework open source per lo sviluppo di applicazioni web nel linguaggio JavaScript.
    \item[API] acronimo di Application Programming Interface, si intende un insieme di procedure e funzioni offerte ai programmatori, ottenendo un'astrazione ad alto livello per facilitare lo sviluppo. Le API espongono blocchi di codice delle librerie di cui fanno parte, permettendo il riuso del codice.
\end{description}

\phantomsection{}
\subsection{B}
\begin{description}
    \item[BDD]: acronimo di Behaviour-Driven Development, è una metodologia di sviluppo basata sul Test-Driven Development (TDD).
    \item[Build Automation] processo per automatizzare un'ampia varietà di compiti che gli sviluppatori software svolgono nelle loro attività quotidiane, svincolandoli da esse.
\end{description}

\phantomsection{}
\subsection{C}

\begin{description}
    \item[Callback] porzione di codice che può essere passata come parametro in una chiamata di funzione.
    \item[Chai] libreria di asserzioni BDD/TDD\%.
    \item[Chrome] detto anche Google Chrome, è un browser web sviluppato da Google. Al momento della scrittura, Chrome è alla versione 81.0.
    \item[CLI] acronimo di Command Line Interface, è un tipo di interfaccia utente caratterizzata da un'interazione testuale tra utente ed elaboratore.
    \item[Container] file immagine che contiene un istanza da avviare in Docker.
    \item[Cucumber] strumento software che supporta il BDD\%.
\end{description}

\phantomsection{}
\subsection{D}

\begin{description}
    \item[Debug] attività che consiste nell'individuazione di uno o più errori, eseguendo un'analisi dinamica.
    \item[Docker] tecnologia di containerizzazione che consente la creazione e l'utilizzo dei container Linux, per isolare i processi in modo da poterli eseguire in maniera indipendente.
    \item[Docker Compose] comando che può essere utilizzato per comporre tutta l'infrastruttura in base ai servizi che si è scelto di integrare, facendo uso dei Dockerfile.
    \item[Dockerfile] file di configurazione che illustra i passaggi che devono essere realizzati per compilare un'immagine di un sistema operativo come base per eseguire un certo applicativo.
\end{description}

\phantomsection{}
\subsection{E}

\begin{description}
  \item[Edge] detto anche Microsoft Edge, è un browser web sviluppato da Microsoft, incluso come browser predefinito in Windows 10 rimpiazzando Internet Explorer. Al momento della scrittura, Edge è alla versione 81.0.416.12.
  \item[Espresso] Framework di test open source per le interfacce utente Android.
\end{description}

\phantomsection{}
\subsection{F}

\begin{description}
  \item[Firefox] detto anche Mozilla Firefox, è un web browser libero e multipiattaforma, mantenuto da Mozilla Foundation. Al momento della scrittura, Firefox è alla versione 76.
\end{description}

\phantomsection{}
\subsection{G}

\begin{description}
    \item[Gradle] kit di automazione di sviluppo che può essere integrato in diversi ambienti, attraverso plugins.
    \item[Groovy] linguaggio di programmazione ad oggetti per la Piattaforma Java alternativo al linguaggio Java.
\end{description}

\phantomsection{}
\subsection{H}

\begin{description}
  \item[HDD] acronimo di Hard Disk Drive, tradotto in italiano come disco rigido, è un dispositivo di massa di tipo magnetico per l'archiviazione di dati e applicazioni.
  \item[Hyper-V] conosciuto anche come Windows Server Virtualization, è la tecnologia per la virtualizzazione di Microsoft che consente di eseguire più sistemi operativi come macchine virtuali in Windows.
\end{description}

\phantomsection{}
\subsection{I}

\begin{description}
    \item[IDE] acronimo di Integrated Development Environment, è un ambiente di sviluppo con l'obiettivo di supportare il programmatore durante la codifica.
    \item[InfluxDB] database di serie temporali open source sviluppato da InfluxData.
    \item[IntelliJ IDEA] ambiente di sviluppo integrato per il linguaggio di programmazione Java.
    \item[IP] acronimo di Internet Protocol address, è un indirizzo che identifica univocamente un dispositivo, detto host, collegato a una rete informatica che utilizza l'Internet Protocol come protocollo di rete per l'instradamento delle informazioni nella rete.
\end{description}

\phantomsection{}
\subsection{J}

\begin{description}
    \item[Jasmine] framework BDD di test open source per JavaScript.
    \item[Java] linguaggio di programmazione ad alto livello, orientato agli oggetti e a tipizzazione statica, progettato per essere il più possibile indipendente dalla piattaforma di esecuzione.
    \item[JavaScript] linguaggio di scripting orientato agli oggetti e agli eventi, comunemente utilizzato nella programmazione web lato client per gestire gli effetti dinamici interattivi.
    \item[JetBrains Toolbox] suite di strumenti di sviluppo professionale per una vasta gamma di linguaggi di programmazione e tecnologie.
    \item[JJWT] acronimo di Java JSON Web Token, è una libreria open source per trasmettere informazioni tra due componenti in formato JSON\%.
    \item[JSON] acronimo di JavaScript Object Notation, è un formato adatto all'interscambio di dati fra applicazioni client/server.
    \item[JUnit] framework di unit testing per il linguaggio di programmazione Java.
    \item[JVM] acronimo di Java Virtual Machine, è il componente della piattaforma Java che esegue i programmi tradotti in bytecode dopo una prima fase di compilazione in bytecode.
\end{description}

\phantomsection{}
\subsection{K}

\begin{description}
    \item[Karma] strumento utilizzato per testare l'esecuzione della web application su più browser.
    \item[Kotlin] linguaggio di programmazione general purpose, multi-paradigma, open source sviluppato dall'azienda di software JetBrains.
\end{description}

\phantomsection{}
\subsection{L}

\begin{description}
    \item[Layout] nell'ambito Android, file XML che descrive in modo dichiarativo la struttura di un'interfaccia grafica.
    \item[Leaflet] libreria scritta in JavaScript per gestire mappe interattive su pagine web.
    \item[Lifetime] nell'ambito di Android, sistema su cui si basa l'implementazione del pattern MVVM del framework; la lifetime di un'activity corrisponde a tutto l'intervallo di tempo dalla sua prima costruzione alla sua ultima rimozione.
    \item[LiveReload] tecnica per la compilazione automatica ad ogni modifica del codice sorgente.
    \item[Lombok] libreria java che si collega automaticamente al tuo editor e crea strumenti, grazie all'utilizzo di annotazioni specifiche.
\end{description}

\phantomsection{}
\subsection{M}

\begin{description}
    \item[Maven] strumento di gestione di progetti software basati su Java e build automation.
    \item[Mock object] oggetti simulati, oppure finti, che riproducono il comportamento degli oggetti reali, ma inaccessibili o non implementati, in modo controllato.
    \item[Mockito] framework di test open source per Java rilasciato sotto licenza MIT\@.
    \item[MySQL] relational database management system composto da un client a riga di comando e un server.
\end{description}

\phantomsection{}
\subsection{N}

\begin{description}
  \item[Node.js] ambiente di runtime JavaScript open source multipiattaforma orientato agli eventi per l'esecuzione di codice JavaScript.
  \item[Npm] gestore di pacchetti per il linguaggio di programmazione JavaScript.
\end{description}

\phantomsection{}
\subsection{O}

\begin{description}
    \item[OpenAPI] standard open source frequentemente utilizzato per la descrizione delle API\@.
    \item[OpenStreetMap] progetto collaborativo finalizzato a creare mappe del mondo a contenuto libero. Il progetto punta ad una raccolta mondiale di dati geografici.
\end{description}

\phantomsection{}
\subsection{P}

\begin{description}
    \item[PhpMyAdmin] applicazione web che consente di amministrare un database MySQL o MariaDB\@.
    \item[Port Forwarding] operazione che permette il trasferimento dei dati da un computer ad un altro tramite una specifica porta di comunicazione.
    \item[Protractor] framework di test E2E che consente di testare applicazioni frontend
\end{description}

\phantomsection{}
\subsection{Q}

\phantomsection{}
\subsection{R}

\begin{description}
    \item[R2DBC] standard API per il reactive programming con un database SQL\@.
    \item[Relational Database] database digitale basato sul modello relazionale di dati.
    \item[Reverse geocoding] processo di decodifica di un punto in un indirizzo leggibile o nome di luogo.
\end{description}

\phantomsection{}
\subsection{S}

\begin{description}
    \item[Spring] Framework Java e contenitore IoC per lo sviluppo di applicazioni web basate su Java EE\@.
    \item[SSD] acronimo di Solid State Disk, tradotto in italiano come disco a stato solido, è un dispositivo di memoria di massa basato su semiconduttore, che utilizza memoria flash per l'archiviazione dei dati. In questi ultimi anni sta rimpiazzando gli HDD per le sue elevate prestazioni.
\end{description}

\phantomsection{}
\subsection{T}

\begin{description}
    \item[Test E2E] tecnica di testing sull'intero prodotto software dall'inizio alla fine, per garantire che il flusso dell'applicazione si comporti come previsto, convalidando il sistema.
    \item[Token di accesso] oggetto che incapsula le credenziali di sicurezza per una sessione di accesso e identifica l'utente, i gruppi dell'utente, i privilegi dell'utente e, in alcuni casi, una particolare applicazione.
    \item[TypeScript] linguaggio di programmazione open source sviluppato da Microsoft, estende il JavaScript rendendolo orientato agli oggetti.
\end{description}

\phantomsection{}
\subsection{U}

\phantomsection{}
\subsection{V}

\begin{description}
    \item[VM] acronimo di Virtual Machine, indica un software che, attraverso un processo di virtualizzazione, crea un ambiente virtuale che emula tipicamente il comportamento di una macchina fisica.
    \item[Volley] libreria open source HTTP sviluppata per le applicazioni Android che consente di eseguire, ricevere e gestire chiamate di rete.
\end{description}

\phantomsection{}
\subsection{W}

\phantomsection{}
\subsection{X}
\begin{description}
    \item[XML] acronimo di eXtensible Markup Language, è un metalinguaggio per la definizione di linguaggi di markup.
\end{description}

\phantomsection{}
\subsection{Y}

\phantomsection{}
\subsection{Z}

\end{document}
