\documentclass[../manuale-manutentore.tex]{subfiles}

\begin{document}

Di seguito vengono descritti i comandi da eseguire nella CLI per il test di ogni componente del sistema di Stalker.

% \subsection{Server}%
% \label{sub:}
%TODO

\subsection{Web application}%
\label{sub:}

\subsubsection{Esecuzione test di unità}%
\label{subs:esecuzione_test_di_unita}

Per eseguire i test di unità tramite Karma, eseguire il comando: \par\bigskip

\begin{center}
  \textit{ng test}
\end{center}
\par\bigskip

\subsubsection{estensione test di unità}%
\label{subs:estensione_test_unita_webapp}

I test di unità vengono definiti utilizzando la sintassi di Jasmine, dunque per aggiungere un test ad un componente è sufficiente creare un blocco come il seguente:

\typescript{  it('should create', () => { %chktex 36
  expect(component).toBeTruthy(); %chktex 36
});} %chktex 36

Poi definire un altro test.
I test in seguito verranno eseguiti con il framework Karma, che esegue i test sul browser Chrome.

% subs:estensione_test_unita_webapp (end)
\subsubsection{Esecuzione test end-to-end}%
\label{subs:esecuzione_test_end_to_end}

Per eseguire i test end-to-end tramite Protractor, eseguire il comando: \par\bigskip

\begin{center}
  \textit{ng e2e}
\end{center}
\par\bigskip

\subsubsection{Estensione test end-to-end}%
\label{subs:estensione_test_end_to_end}
I test end-to-end vengono definiti utilizzando la sintassi gherkin all'interno del framework cucumber, quindi la prima cosa da fare è definire testualmente il test.
In seguito vanno definiti gli step in codice a cui gli step testuali corrispondono.
Infine i test vanno eseguiti tramite Protractor, che simula l'interazione dell'utente con il browser.
% subs:estensione_test_end_to_end (end)
% \subsection{Mobile application}%
% \label{sub:}
%TODO


\end{document}
