\documentclass[../analisi-dei-requisiti]{subfiles}

\renewcommand{\commons}{../../../commons}

\begin{document}

\paragraph{Amministratore}
\label{par:amministratore}
Di seguito sono riportati tutti i casi d'uso che coinvolgono il \glossario{super utente} \glossario{amministratore}.

%inserire UML generale amministratore

\subsubsection{UC1 - Sistema di autenticazione amministratore}
\label{subsub:UC1}


\begin{plantuml}
@startuml UC1
!include ../../../commons/style/use-cases.pu
left to right direction

title Login SuperUtente

actor :SuperUtente non autenticato: as A1.1

rectangle UC1{
  together {
  usecase (UC1.1) as "UC1.1\nAutenticazione"
  usecase (UC1.2) as "UC1.2\nVerifica credenziali"
  usecase (UC1.3) as "UC1.3\nVisualizzazione credenziali errate"
  note "Condition: credenziali errate" as N1
  }
}
  
UC1.1 .> UC1.2 : <<include>>
UC1.3 . N1
N1 .> UC1.1 : <<extends>>

A1.1 -- UC1.1


@enduml
\end{plantuml}


\begin{itemize}
\item \textbf{Attori primari:} \glossario{super utente non autenticato};
\item \textbf{Descrizione:} il \glossario{super utente non autenticato} tenta di autenticarsi al sistema;
\item \textbf{Scenario principale:} il \glossario{super utente non autenticato} non è ancora autenticato e vuole eseguire il login;
\item \textbf{Precondizione:} il \glossario{super utente non autenticato} non è autenticato alla piattaforma;
\item \textbf{Postcondizione:} il \glossario{super utente} ha effettuato correttamente il login nel sistema.

\end{itemize}
% subsub:UC1 (end)
\subsubsection{UC1.1 - Autenticazione}
\label{subsub:UC1.1}

\begin{itemize}
\item \textbf{Attori primari:} \glossario{super utente non autenticato}
\item \textbf{Descrizione:} il \glossario{super utente non autenticato} visualizza la pagina di login, dove poter inserire le proprie credenziali. 
\item \textbf{Scenario principale:} il \glossario{super utente non autenticato} accede alla pagina di login, e visualizza tutti i campi da compilare;
\item Estensioni: \\\emph{UC1.3} Visualizzazione messaggio di credenziali errate;
\item Inclusioni: \\\emph{UC1.2} Verifica credenziali;
\item \textbf{Precondizione:} il sistema è raggiungibile e funzionante, il \glossario{super utente non autenticato} deve poter visualizzare la pagina di login;
\item \textbf{Postcondizione:} il \glossario{super utente non autenticato} ha inserito le possibili credenziali e sta tentando di effettuare il login. Ogni volta che cercherà di effettuare
login sarà verificato che le credenziali inserite siano corrette. In caso contrario verrà visualizzato un messaggio di errore, e l'accesso sarà negato;

\end{itemize}

% subsub:UC1,1 (end)
\subsubsection{UC1.2 - Verifica Credenziali}
\label{subsub:UC1.2}

\begin{itemize}
\item \textbf{Attori primari:} \glossario{super utente non autenticato};
\item \textbf{Descrizione:} vengono verificate le credenziali immesse dal \glossario{super utente non autenticato};
\item \textbf{Scenario principale:} il \glossario{super utente non autenticato} sta tentando di effettuare l'accesso e sta attendendo la verifica delle credenziali immesse;
\item \textbf{Precondizione:} il \glossario{super utente non autenticato} ha inviato al server le sue credenziali per tentare il login;
\item \textbf{Postcondizione:} il \glossario{super utente non autenticato} deve poter accedere alla sua area riservata, nel caso in cui le credenziali siano corrette. In caso
contrario deve essere visualizzato un messaggio di credenziali sbagliate. (\emph{UC1.3}).

\end{itemize}

% subsub:UC1,2 (end)
\subsubsection{UC1.3 - Visualizzazione credenziali errate}
\label{subsub:UC1.3}

\begin{itemize}
\item \textbf{Attori primari:} \glossario{super utente non autenticato};
\item \textbf{Descrizione:} viene visualizzato un errore di login;
\item \textbf{Scenario principale:} il \glossario{super utente non autenticato} cerca di effettuare il login con delle credenziali sbagliate;
\item \textbf{Precondizione:} il \glossario{super utente non autenticato} ha inviato al server le sue credenziali per tentare il login, e le credenziali sono state verificate;
\item \textbf{Postcondizione:} il \glossario{super utente non autenticato} visualizza un messaggio di credenziali sbagliate. (\emph{UC1.3}).

\end{itemize}

% subsub:UC1.3 (end)
\subsubsection{UC2 - Approvazione owner}
\label{subsub:UC2}

\begin{itemize}
\item \textbf{Attori primari:} \glossario{amministratore};
\item \textbf{Descrizione:} l'\glossario{amministratore} approva il compito di \glossario{owner} ad un gestore di una \glossario{organizzazione};
\item \textbf{Scenario principale:} deve essere assegnato ad un'\glossario{organizzazione} un \glossario{owner};
\item \textbf{Precondizione:} l'\glossario{organizzazione} scelta non deve avere nessun \glossario{owner} approvato;
\item \textbf{Postcondizione:} viene approvata la richiesta, e l'\glossario{organizzazione} avrà un nuovo \glossario{owner}.

\end{itemize}
% subsub:UC2 (end)
\subsubsection{UC3 - Approvazione creazione organizzazione}
\label{subsub:UC3}

\begin{itemize}
\item \textbf{Attori primari:} \glossario{amministratore};
\item \textbf{Descrizione:} l'\glossario{amministratore} approva la richiesta di creazione di una nuova \glossario{organizzazione};
\item \textbf{Scenario principale:} viene richiesta la creazione di una nuova \glossario{organizzazione}, che vuole utilizzare \glossario{Stalker};
\item \textbf{Precondizione:} l'\glossario{organizzazione} non deve essere già creata dal sistema, devono essere stati specificati i dati necessari per la creazione;
\item \textbf{Postcondizione:} viene approvata la richiesta, e l'\glossario{organizzazione} viene creata.

\end{itemize}
% subsub:UC3 (end)
\subsubsection{UC4 - Approvazione modifica organizzazione}
\label{subsub:UC4}

\begin{itemize}
\item \textbf{Attori primari:} \glossario{amministratore};
\item \textbf{Descrizione:} l'\glossario{amministratore} approva la richiesta di modifica di una \glossario{organizzazione};
\item \textbf{Scenario principale:} viene richiesta la modifica di una \glossario{organizzazione}, presente in \glossario{Stalker};
\item \textbf{Precondizione:} l'\glossario{organizzazione} deve essere già stata creata dal sistema, devono essere state apportate modifiche;
\item \textbf{Postcondizione:} viene approvata la richiesta, e l'\glossario{organizzazione} viene modificata.

\end{itemize}
% subsub:UC4 (end)
\subsubsection{UC5 - Approvazione aggiunta luogo}
\label{subsub:UC5}

\begin{itemize}
\item \textbf{Attori primari:} \glossario{amministratore};
\item \textbf{Descrizione:} l'\glossario{amministratore} approva la richiesta di aggiunta di un luogo all'interno di un'\glossario{organizzazione};
\item \textbf{Scenario principale:} viene richiesta l'aggiunta di un luogo all'interno di un'\glossario{organizzazione};
\item \textbf{Precondizione:} l'\glossario{organizzazione} deve essere già stata creata dal sistema, e deve essere stata fatta la richiesta di aggiunta luogo;
\item \textbf{Postcondizione:} viene approvata la richiesta, e viene aggiunto il luogo all'\glossario{organizzazione} interessata.

\end{itemize}
% subsub:UC5 (end)
\subsubsection{UC6 - Approvazione modifica luogo}
\label{subsub:UC6}

\begin{itemize}
\item \textbf{Attori primari:} \glossario{amministratore};
\item \textbf{Descrizione:} l'\glossario{amministratore} approva la richiesta di una modifica di un luogo all'interno di un'\glossario{organizzazione};
\item \textbf{Scenario principale:} viene richiesta una modifica di un luogo all'interno di un'\glossario{organizzazione};
\item \textbf{Precondizione:} l'\glossario{organizzazione} e il luogo devono essere già stati creati dal sistema, e deve essere stata fatta la richiesta di modifica luogo;
\item \textbf{Postcondizione:} viene approvata la richiesta, e viene modificato il luogo dell'\glossario{organizzazione} interessata.

\end{itemize}
% subsub:UC6 (end)
\subsubsection{UC7 - Approvazione trasferimento di proprietà organizzazione}
\label{subsub:UC7}

\begin{itemize}
\item \textbf{Attori primari:} \glossario{amministratore};
\item \textbf{Descrizione:} l'\glossario{amministratore} approva la richiesta di un trasferimento di proprietà di un'\glossario{organizzazione};
\item \textbf{Scenario principale:} viene richiesto un traferimento di proprietà di un'\glossario{organizzazione};
\item \textbf{Precondizione:} l'\glossario{organizzazione} deve essere già stata creata dal sistema, e deve essere stata fatta la richiesta trasferimento di proprietà;
\item \textbf{Postcondizione:} viene approvata la richiesta, e viene trasferita la proprietà dell'\glossario{organizzazione} interessata.

\end{itemize}
% subsub:UC7 (end)
% par:Amministratore (end)
\paragraph{Root}
Di seguito sono riportati tutti i casi d'uso che coinvolgono il \glossario{super utente} \glossario{root}.

\subsubsection{UC8 - Creazione amministratore}
\label{subsub:UC8}

\begin{itemize}
\item \textbf{Attori primari:} \glossario{root};
\item \textbf{Descrizione:} il \glossario{root} crea un \glossario{super utente} \glossario{amministratore};
\item \textbf{Scenario principale:} sorge la necessità di creare un nuovo \glossario{amministratore} per gestire \glossario{Stalker};
\item \textbf{Precondizione:} devono essere specificate le credenziali del nuovo \glossario{amministratore}, che devono essere univoche;
\item \textbf{Postcondizione:} l'\glossario{amministratore} viene creato.

\end{itemize}
% subsub:UC8 (end)

\subsubsection{UC9 - Eliminazione organizzazione}
\label{subsub:UC9}

\begin{itemize}
\item \textbf{Attori primari:} \glossario{root};
\item \textbf{Descrizione:} il \glossario{root} elimina un'\glossario{organizzazione};
\item \textbf{Scenario principale:} sorge la necessità di eliminare un'\glossario{organizzazione}, senza interagire con il suo \glossario{owner}; 
\item \textbf{Precondizione:} deve essere stata selezionata l'\glossario{organizzazione} da eliminare, presente nella lista di \glossario{Stalker};
\item \textbf{Postcondizione:} l'\glossario{organizzazione} viene eliminata.

\end{itemize}
% subsub:UC9 (end)


\subsubsection{UC10 - Modifica organizzazione}
\label{subsub:UC10}

\begin{itemize}
\item \textbf{Attori primari:} \glossario{root};
\item \textbf{Descrizione:} il \glossario{root} modifica un'\glossario{organizzazione};
\item \textbf{Scenario principale:} sorge la necessità di modificare un'\glossario{organizzazione}, senza interagire con il suo \glossario{owner}; 
\item \textbf{Precondizione:} deve essere stata selezionata l'\glossario{organizzazione} da modificare, presente nella lista di \glossario{Stalker}, effettivamente modificata;
\item \textbf{Postcondizione:} l'\glossario{organizzazione} viene modificata.

\end{itemize}
% subsub:UC10 (end)


\subsubsection{UC11 - Creazione organizzazione}
\label{subsub:UC11}

\begin{itemize}
\item \textbf{Attori primari:} \glossario{root};
\item \textbf{Descrizione:} il \glossario{root} crea un'\glossario{organizzazione};
\item \textbf{Scenario principale:} sorge la necessità di creare un'\glossario{organizzazione}, senza essere effettivamente richiesta; 
\item \textbf{Precondizione:} l'\glossario{organizzazione} non deve esistere nella lista di \glossario{Stalker}, deve essere specificato il suo nome;
\item \textbf{Postcondizione:} l'\glossario{organizzazione} viene creata.

\end{itemize}
% subsub:UC11 (end)

\subsubsection{UC12 - Eliminazione luogo}
\label{subsub:UC12}

\begin{itemize}
\item \textbf{Attori primari:} \glossario{root};
\item \textbf{Descrizione:} il \glossario{root} elimina un \glossario{luogo} di un'\glossario{organizzazione};
\item \textbf{Scenario principale:} sorge la necessità di elimanare un \glossario{luogo} di un'\glossario{organizzazione}, senza interagire con il suo \glossario{owner}; 
\item \textbf{Precondizione:} il \glossario{luogo} dell'\glossario{organizzazione} deve essere presente in \glossario{Stalker};
\item \textbf{Postcondizione:} il \glossario{luogo} dell'\glossario{organizzazione} viene eliminato.

\end{itemize}
% subsub:UC12 (end)

\subsubsection{UC13 - Modifica luogo}
\label{subsub:UC13}

\begin{itemize}
\item \textbf{Attori primari:} \glossario{root};
\item \textbf{Descrizione:} il \glossario{root} modifica un \glossario{luogo} di un'\glossario{organizzazione};
\item \textbf{Scenario principale:} sorge la necessità di modificare un \glossario{luogo} di un'\glossario{organizzazione}, senza interagire con il suo \glossario{owner}; 
\item \textbf{Precondizione:} il \glossario{luogo} dell'\glossario{organizzazione} deve essere presente in \glossario{Stalker};
\item \textbf{Postcondizione:} il \glossario{luogo} dell'\glossario{organizzazione} viene modificato.

\end{itemize}
% subsub:UC13 (end)

\subsubsection{UC14 - Eliminazione account}
\label{subsub:UC14}

\begin{itemize}
\item \textbf{Attori primari:} \glossario{root};
\item \textbf{Descrizione:} il \glossario{root} elimina un qualsiasi \glossario{account} registrato in \glossario{Stalker}. Quest'ultimo può essere un \glossario{utente} oppure un \glossario{super utente};
\item \textbf{Scenario principale:} sorge la necessità di eliminare un \glossario{account} per sconosciuti motivi; 
\item \textbf{Precondizione:} deve essere stato selezionato l'\glossario{account} da eliminare, che deve esistere in \glossario{Stalker};
\item \textbf{Postcondizione:} l'\glossario{account} selezionato è stato eliminato.

\end{itemize}
% subsub:UC14 (end)


% par:Root (end)

\paragraph{Visualizzatore}
Di seguito sono riportati tutti i casi d'uso che coinvolgono il \glossario{super utente} \glossario{Visualizzatore}.


\subsubsection{UC15 - Query sull'organizzazione}
\label{subsub:UC15}

\begin{itemize}
\item \textbf{Attori primari:} \glossario{visualizzatore};
\item \textbf{Descrizione:} il \glossario{visualizzatore} effettua delle interrogazioni per trarre informazioni relative all'\glossario{organizzazione} su cui opera;
\item \textbf{Scenario principale:} il \glossario{visualizzatore} vuole avere delle informazioni riguardanti l'\glossario{organizzazione} su cui opera;
\item \textbf{Precondizione:} il sistema risponde correttamente alle interrogazioni;
\item \textbf{Postcondizione:} il \glossario{visualizzatore} ottiene le informazioni di cui ha bisogno.

\end{itemize}
% subsub:UC15 (end)


\subsubsection{UC16 - Query sul dipendente}
\label{subsub:UC16}

\begin{itemize}
\item \textbf{Attori primari:} \glossario{visualizzatore};
\item \textbf{Descrizione:} il \glossario{visualizzatore} effettua delle interrogazioni per trarre informazioni relative al dipendente dell'\glossario{organizzazione} su cui opera;
\item \textbf{Scenario principale:} il \glossario{visualizzatore} vuole avere delle informazioni riguardanti il\glossario{dipendente} dell'\glossario{organizzazione} su cui opera;
\item \textbf{Precondizione:} il sistema risponde correttamente alle interrogazioni;
\item \textbf{Postcondizione:} il \glossario{visualizzatore} ottiene le informazioni di cui ha bisogno.

\end{itemize}
% subsub:UC16 (end)
% par:Visualizzatore (end)

\paragraph{Gestore}
Di seguito sono riportati tutti i casi d'uso che coinvolgono il \glossario{super utente} \glossario{Gestore}.

\subsubsection{UC17 - Richiesta aggiunta luogo}
\label{subsub:UC17}

\begin{itemize}
\item \textbf{Attori primari:} \glossario{gestore};
\item \textbf{Descrizione:} il \glossario{gestore} dell'\glossario{organizzazione} richiede di aggiungere un nuovo \glossario{luogo};
\item \textbf{Scenario principale:} il \glossario{gestore} vuole aggiungere un nuovo \glossario{luogo} all'\glossario{organizzazione} su cui opera;
\item \textbf{Precondizione:} il \glossario{luogo} da aggiungere non deve già esistere;
\item \textbf{Postcondizione:} la richiesta di aggiunta di un nuovo \glossario{luogo} viene posta.

\end{itemize}
% subsub:UC17 (end)

\subsubsection{UC18 - Richiesta rimozione luogo}
\label{subsub:UC18}

\begin{itemize}
\item \textbf{Attori primari:} \glossario{gestore};
\item \textbf{Descrizione:} il \glossario{gestore} dell'\glossario{organizzazione} richiede di eliminare un \glossario{luogo};
\item \textbf{Scenario principale:} il \glossario{gestore} vuole eliminare un \glossario{luogo} all'\glossario{organizzazione} su cui opera;
\item \textbf{Precondizione:} il \glossario{luogo} da eliminare deve esistere;
\item \textbf{Postcondizione:} la richiesta di eliminazione di un \glossario{luogo} viene posta.

\end{itemize}
% subsub:UC18 (end)

\subsubsection{UC19 - Richiesta modifica luogo}
\label{subsub:UC19}

\begin{itemize}
\item \textbf{Attori primari:} \glossario{gestore};
\item \textbf{Descrizione:} il \glossario{gestore} dell'\glossario{organizzazione} richiede di modificare un \glossario{luogo};
\item \textbf{Scenario principale:} il \glossario{gestore} vuole modificare un \glossario{luogo} dell'\glossario{organizzazione} su cui opera;
\item \textbf{Precondizione:} il \glossario{luogo} da modificare deve esistere;
\item \textbf{Postcondizione:} la richiesta di modifica di un \glossario{luogo} viene posta.

\end{itemize}
% subsub:UC19 (end)


\subsubsection{UC20 - Richiesta modifica parametri organizzazione}
\label{subsub:UC20}

\begin{itemize}
\item \textbf{Attori primari:} \glossario{gestore};
\item \textbf{Descrizione:} il \glossario{gestore} dell'\glossario{organizzazione} richiede di modificare i parametri di essa;
\item \textbf{Scenario principale:} il \glossario{gestore} vuole modificare i parametri dell'\glossario{organizzazione} su cui opera;
\item \textbf{Precondizione:} i parametri devono essere effettivamente modificati;
\item \textbf{Postcondizione:} la richiesta di modifica parametri viene posta.

\end{itemize}
% subsub:UC20 (end)
% par:Gestore (end)

\paragraph{Owner}
Di seguito sono riportati tutti i casi d'uso che coinvolgono il \glossario{super utente} \glossario{Owner}.

\subsubsection{UC21 - Nomina visualizzatore}
\label{subsub:UC21}

\begin{itemize}
\item \textbf{Attori primari:} \glossario{owner};
\item \textbf{Descrizione:} l' \glossario{owner} nomina un \glossario{visualizzatore} per la sua \glossario{organizzazione};
\item \textbf{Scenario principale:} l' \glossario{owner} vuole aggiungere un \glossario{visualizzatore} alla sua \glossario{organizzazione};
\item \textbf{Precondizione:} il nuovo \glossario{super utente} non deve già esistere come \glossario{visualizzatore};
\item \textbf{Postcondizione:} il nuovo \glossario{visualizzatore} è stato aggiunto.

\end{itemize}
% subsub:UC21 (end)


\subsubsection{UC22 - Nomina gestore}
\label{subsub:UC22}

\begin{itemize}
\item \textbf{Attori primari:} \glossario{owner};
\item \textbf{Descrizione:} l' \glossario{owner} nomina un \glossario{gestore} per la sua \glossario{organizzazione};
\item \textbf{Scenario principale:} l' \glossario{owner} vuole aggiungere un \glossario{gestore} alla sua \glossario{organizzazione};
\item \textbf{Precondizione:} il nuovo \glossario{super utente} non deve già esistere come \glossario{gestore};
\item \textbf{Postcondizione:} il nuovo \glossario{gestore} è stato aggiunto.

\end{itemize}
% subsub:UC22 (end)


\subsubsection{UC23 - Nomina gestore}
\label{subsub:UC23}

\begin{itemize}
\item \textbf{Attori primari:} \glossario{owner};
\item \textbf{Descrizione:} l' \glossario{owner} nomina un \glossario{gestore} per la sua \glossario{organizzazione};
\item \textbf{Scenario principale:} l' \glossario{owner} vuole aggiungere un \glossario{gestore} alla sua \glossario{organizzazione};
\item \textbf{Precondizione:} il nuovo \glossario{super utente} non deve già esistere come \glossario{gestore};
\item \textbf{Postcondizione:} il nuovo \glossario{gestore} è stato aggiunto.

\end{itemize}
% subsub:UC23 (end)


\subsubsection{UC24 - Richiesta creazione organizzazione}
\label{subsub:UC24}

\begin{itemize}
\item \textbf{Attori primari:} \glossario{owner};
\item \textbf{Descrizione:} l' \glossario{owner} richiede di creare una nuova \glossario{organizzazione};
\item \textbf{Scenario principale:} l' \glossario{owner} vuole creare una nuova \glossario{organizzazione};
\item \textbf{Precondizione:} l' \glossario{organizzazione} non deve già esistere;
\item \textbf{Postcondizione:} la richiesta di creare una nuova \glossario{organizzazione} è stata posta.

\end{itemize}
% subsub:UC24 (end)

\subsubsection{UC25 - Richiesta cedimento proprietà organizzazione}
\label{subsub:UC25}

\begin{itemize}
\item \textbf{Attori primari:} \glossario{owner};
\item \textbf{\textbf{Descrizione:}} l' \glossario{owner} richiede che l'\glossario{organizzazione} venga ceduta ad un altro futuro \glossario{owner};
\item \textbf{\textbf{Scenario principale:}} l' \glossario{owner} vuole cedere l'\glossario{organizzazione} ad un'altro \glossario{owner};
\item \textbf{Precondizione:} il futuro \glossario{owner} deve esistere;
\item \textbf{Postcondizione:} la richiesta di cedere l'\glossario{organizzazione} è stata posta.

\end{itemize}
% subsub:UC25 (end)

% par:Owner (end)

\end{document}