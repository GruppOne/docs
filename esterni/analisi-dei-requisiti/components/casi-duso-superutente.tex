\documentclass[../analisi-dei-requisiti]{subfiles}

\renewcommand{\commons}{../../../commons}

\begin{document}

\paragraph{Amministratore}
\label{par:amministratore}
Di seguito sono riportati tutti i casi d'uso che coinvolgono il \glossario{super utente} \glossario{amministratore}.

%inserire UML generale amministratore

\subsubsection{UC1 - Sistema di autenticazione amministratore}
\label{subsub:UC1}

\begin{itemize}
\item Attori primari: \glossario{amministratore};
\item Descrizione: il possibile \glossario{amministratore} tenta di autenticarsi al sistema;
\item Scenario principale: il possibile \glossario{amministratore} non è ancora autenticato e vuole eseguire il login;
\item Precondizione: il possibile \glossario{amministratore} non è autenticato alla piattaforma;
\item Postcondizione: l'\glossario{amministratore} ha effettuato correttamente il login nel sistema.

\end{itemize}
% subsub:UC1 (end)
\subsubsection{UC1.1 - Autenticazione}
\label{subsub:UC1.1}

\begin{itemize}
\item Attori primari: amministratore;
\item Descrizione: l'\glossario{amministratore} visualizza la pagina di login, dove poter inserire le proprie credenziali. 
\item Scenario principale: ll possibile \glossario{amministratore} accede alla pagina di login, e visualizza tutti i campi da compilare;
\item Estensioni: \\\emph{UC1.3} Visualizzazione messaggio di credenziali errate;
\item Inclusioni: \\\emph{UC1.2} Verifica credenziali;
\item Precondizione: il sistema è raggiungibile e funzionante, il possibile \glossario{amministratore} deve poter visualizzare la pagina di login;
\item Postcondizione: il possibile \glossario{amministratore} ha inserito le possibili credenziali e sta tentando di effettuare il login. Ogni volta che cercherà di effettuare
login sarà verificato che le credenziali inserite siano corrette. In caso contrario verrà visualizzato un messaggio di errore, e l'accesso sarà negato;

\end{itemize}

% subsub:UC1,1 (end)
\subsubsection{UC1.2 - Verifica Credenziali}
\label{subsub:UC1.2}

\begin{itemize}
\item Attori primari: \glossario{amministratore};
\item Descrizione: vengono verificate le credenziali immesse dal possibile \glossario{amministratore};
\item Scenario principale: il possibile \glossario{amministratore} sta tentando di effettuare l'accesso e sta attendendo la verifica delle credenziali immesse;
\item Precondizione: il possibile \glossario{amministratore} ha inviato al server le sue credenziali per tentare il login;
\item Postcondizione: il possibile \glossario{amministratore} deve poter accedere alla sua area riservata, nel caso in cui le credenziali siano corrette. In caso
contrario deve essere visualizzato un messaggio di credenziali sbagliate. (\emph{UC1.3}).

\end{itemize}

% subsub:UC1,2 (end)
\subsubsection{UC1.3 - Visualizzazione credenziali errate}
\label{subsub:UC1.3}

\begin{itemize}
\item Attori primari: \glossario{amministratore};
\item Descrizione: viene visualizzato un errore di login;
\item Scenario principale: il possibile \glossario{amministratore} cerca di effettuare il login con delle credenziali sbagliate;
\item Precondizione: il possibile \glossario{amministratore} ha inviato al server le sue credenziali per tentare il login, e le credenziali sono state verificate;
\item Postcondizione: il possibile amministraotre visualizza un messaggio di credenziali sbagliate. (\emph{UC1.3}).

\end{itemize}

% subsub:UC1.3 (end)
\subsubsection{UC2 - Approvazione owner}
\label{subsub:UC2}

\begin{itemize}
\item Attori primari: \glossario{amministratore};
\item Descrizione: l'\glossario{amministratore} approva il compito di \glossario{owner} ad un gestore di una \glossario{organizzazione};
\item Scenario principale: deve essere assegnato ad un'\glossario{organizzazione} un \glossario{owner};
\item Precondizione: l'\glossario{organizzazione} scelta non deve avere nessun \glossario{owner} approvato;
\item Postcondizione: viene approvata la richiesta, e l'\glossario{organizzazione} avrà un nuovo \glossario{owner}.

\end{itemize}
% subsub:UC2 (end)
\subsubsection{UC3 - Approvazione creazione organizzazione}
\label{subsub:UC3}

\begin{itemize}
\item Attori primari: \glossario{amministratore};
\item Descrizione: l'\glossario{amministratore} approva la richiesta di creazione di una nuova \glossario{organizzazione};
\item Scenario principale: viene richiesta la creazione di una nuova \glossario{organizzazione}, che vuole utilizzare \glossario{Stalker};
\item Precondizione: l'\glossario{organizzazione} non deve essere già creata dal sistema, devono essere stati specificati i dati necessari per la creazione;
\item Postcondizione: viene approvata la richiesta, e l'\glossario{organizzazione} viene creata.

\end{itemize}
% subsub:UC3 (end)
\subsubsection{UC4 - Approvazione modifica organizzazione}
\label{subsub:UC4}

\begin{itemize}
\item Attori primari: \glossario{amministratore};
\item Descrizione: l'\glossario{amministratore} approva la richiesta di modifica di una \glossario{organizzazione};
\item Scenario principale: viene richiesta la modifica di una \glossario{organizzazione}, presente in \glossario{Stalker};
\item Precondizione: l'\glossario{organizzazione} deve essere già stata creata dal sistema, devono essere state apportate modifiche;
\item Postcondizione: viene approvata la richiesta, e l'\glossario{organizzazione} viene modificata.

\end{itemize}
% subsub:UC4 (end)
\subsubsection{UC5 - Approvazione aggiunta luogo}
\label{subsub:UC5}

\begin{itemize}
\item Attori primari: \glossario{amministratore};
\item Descrizione: l'\glossario{amministratore} approva la richiesta di aggiunta di un luogo all'interno di un'\glossario{organizzazione};
\item Scenario principale: viene richiesta l'aggiunta di un luogo all'interno di un'\glossario{organizzazione};
\item Precondizione: l'\glossario{organizzazione} deve essere già stata creata dal sistema, e deve essere stata fatta la richiesta di aggiunta luogo;
\item Postcondizione: viene approvata la richiesta, e viene aggiunto il luogo all'\glossario{organizzazione} interessata.

\end{itemize}
% subsub:UC5 (end)
\subsubsection{UC6 - Approvazione modifica luogo}
\label{subsub:UC6}

\begin{itemize}
\item Attori primari: \glossario{amministratore};
\item Descrizione: l'\glossario{amministratore} approva la richiesta di una modifica di un luogo all'interno di un'\glossario{organizzazione};
\item Scenario principale: viene richiesta una modifica di un luogo all'interno di un'\glossario{organizzazione};
\item Precondizione: l'\glossario{organizzazione} e il luogo devono essere già stati creati dal sistema, e deve essere stata fatta la richiesta di modifica luogo;
\item Postcondizione: viene approvata la richiesta, e viene modificato il luogo dell'\glossario{organizzazione} interessata.

\end{itemize}
% subsub:UC6 (end)
\subsubsection{UC7 - Approvazione trasferimento di proprietà organizzazione}
\label{subsub:UC7}

\begin{itemize}
\item Attori primari: \glossario{amministratore};
\item Descrizione: l'\glossario{amministratore} approva la richiesta di un trasferimento di proprietà di un'\glossario{organizzazione};
\item Scenario principale: viene richiesto un traferimento di proprietà di un'\glossario{organizzazione};
\item Precondizione: l'\glossario{organizzazione} deve essere già stata creata dal sistema, e deve essere stata fatta la richiesta trasferimento di proprietà;
\item Postcondizione: viene approvata la richiesta, e viene trasferita la proprietà dell'\glossario{organizzazione} interessata.

\end{itemize}
% subsub:UC7 (end)
% par:Amministratore (end)
\paragraph{Root}
Di seguito sono riportati tutti i casi d'uso che coinvolgono il \glossario{super utente} \glossario{root}.

\subsubsection{UC8 - Creazione amministratore}
\label{subsub:UC8}

\begin{itemize}
\item Attori primari: \glossario{root};
\item Descrizione: il \glossario{root} crea un \glossario{super utente} \glossario{amministratore};
\item Scenario principale: sorge la necessità di creare un nuovo \glossario{amministratore} per gestire \glossario{Stalker};
\item Precondizione: devono essere specificate le credenziali del nuovo \glossario{amministratore}, che devono essere univoche;
\item Postcondizione: l'\glossario{amministratore} viene creato.

\end{itemize}
% subsub:UC8 (end)

\subsubsection{UC9 - Eliminazione organizzazione}
\label{subsub:UC9}

\begin{itemize}
\item Attori primari: \glossario{root};
\item Descrizione: il \glossario{root} elimina un'\glossario{organizzazione};
\item Scenario principale: sorge la necessità di eliminare un'\glossario{organizzazione}, senza interagire con il suo \glossario{owner}; 
\item Precondizione: deve essere stata selezionata l'\glossario{organizzazione} da eliminare, presente nella lista di \glossario{Stalker};
\item Postcondizione: l'\glossario{organizzazione} viene eliminata.

\end{itemize}
% subsub:UC9 (end)


\subsubsection{UC10 - Modifica organizzazione}
\label{subsub:UC10}

\begin{itemize}
\item Attori primari: \glossario{root};
\item Descrizione: il \glossario{root} modifica un'\glossario{organizzazione};
\item Scenario principale: sorge la necessità di modificare un'\glossario{organizzazione}, senza interagire con il suo \glossario{owner}; 
\item Precondizione: deve essere stata selezionata l'\glossario{organizzazione} da modificare, presente nella lista di \glossario{Stalker}, effettivamente modificata;
\item Postcondizione: l'\glossario{organizzazione} viene modificata.

\end{itemize}
% subsub:UC10 (end)


\subsubsection{UC11 - Creazione organizzazione}
\label{subsub:UC11}

\begin{itemize}
\item Attori primari: \glossario{root};
\item Descrizione: il \glossario{root} crea un'\glossario{organizzazione};
\item Scenario principale: sorge la necessità di creare un'\glossario{organizzazione}, senza essere effettivamente richiesta; 
\item Precondizione: l'\glossario{organizzazione} non deve esistere nella lista di \glossario{Stalker}, deve essere specificato il suo nome;
\item Postcondizione: l'\glossario{organizzazione} viene creata.

\end{itemize}
% subsub:UC11 (end)

\subsubsection{UC12 - Eliminazione luogo}
\label{subsub:UC12}

\begin{itemize}
\item Attori primari: \glossario{root};
\item Descrizione: il \glossario{root} elimina un \glossario{luogo} di un'\glossario{organizzazione};
\item Scenario principale: sorge la necessità di elimanare un \glossario{luogo} di un'\glossario{organizzazione}, senza interagire con il suo \glossario{owner}; 
\item Precondizione: il \glossario{luogo} dell'\glossario{organizzazione} deve essere presente in \glossario{Stalker};
\item Postcondizione: il \glossario{luogo} dell'\glossario{organizzazione} viene eliminato.

\end{itemize}
% subsub:UC12 (end)


% par:Root (end)
\end{document}