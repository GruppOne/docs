\documentclass[casi-duso]{subfiles}

\renewcommand{\commons}{../../../commons}

\begin{document}

\paragraph{Utente non autenticato}
\label{par:utente-non-autenticato}
Di seguito sono riportati tutti i casi d'uso che coinvolgono l'\glossario{utente non autenticato}.

\subsubsection{UC1 - Sistema di autenticazione}
\label{subsub:UC1}
%inserire diagramma UC1
%\begin{plantuml}
%  @startuml UC1
%  !include ../../../commons/style/use-cases.pu
%  left to right direction

%  title Login SuperUtente

%  actor :Utente non autenticato: as A1.1

%  rectangle UC1{
%      together {
%          usecase (UC1.1) as "UC1.1\nAutenticazione"
%          usecase (UC1.2) as "UC1.2\nVerifica credenziali"
%          usecase (UC1.3) as "UC1.3\nVisualizzazione credenziali errate"
%          note "Condition: credenziali errate" as N1
%        }
%    }

%  UC1.1 .> UC1.2 : <<include>>
%  UC1.3 . N1
%  N1 .> UC1.1 : <<extends>>

%  A1.1 -- UC1.1

%  @enduml
%\end{plantuml}
\begin{itemize}
  \item \textbf{Codice:} UC1
  \item \textbf{Titolo:} sistema di autenticazione
  \item \textbf{Attori primari:} utente non autenticato
  \item \textbf{Precondizione:} l'\glossario{utente non autenticato} non è autenticato alla piattaforma.
  \item \textbf{Postcondizione:} l'\glossario{utente} ha effettuato correttamente il login nel sistema.
  \item \textbf{Scenario principale:}
  \begin{enumerate}
    \item l'\glossario{utente non autenticato} non è ancora autenticato e vuole eseguire il login.
  \end{enumerate}
\end{itemize}
% subsub:UC1 (end)

\subsubsection{UC1.1 - Autenticazione}
\label{subsub:UC1.1}
%inserire diagramma UC1.1
%\begin{plantuml}
%@startuml UC1.1
%!include style.pu
%left to right direction

%title Autenticazione Utente

%actor :Utente non autenticato: as A1.1

%rectangle UC1.1{
%  together {
%  usecase (UC1.1.1) as "UC1.1.1\nInserimento e-mail"
%  usecase (UC1.1.2) as "UC1.1.2\nInserimento password"
%  }
%}
%A1.1 -- UC1.1.1
%A1.1 -- UC1.1.2

%@enduml
%\end{plantuml}
\begin{itemize}
  \item \textbf{Codice:} UC1.1
  \item \textbf{Titolo:} autenticazione
  \item \textbf{Attori primari:} utente non autenticato
  \item \textbf{Precondizione:} il sistema è raggiungibile e funzionante, l'\glossario{utente non autenticato} deve poter visualizzare la pagina di login.
  \item \textbf{Postcondizione:} l'\glossario{utente non autenticato} ha inserito le possibili credenziali e sta tentando di effettuare il login. Ogni volta che cercherà di effettuare
        login sarà verificato che le credenziali inserite siano corrette. In caso contrario verrà visualizzato un messaggio di errore, e l'accesso sarà negato.
  \item \textbf{Scenario principale:} 
  \begin{enumerate}
    \item  l'\glossario{utente non autenticato} accede alla pagina di login, e visualizza tutti i campi da compilare:
    \begin{enumerate}
      \item inserisce l’email associata all’account \emph{[UC1.1.1]};
      \item inserisce la password associata all’account \emph{[UC1.1.2]}.
    \end{enumerate}
  \end{enumerate}
  \item \textbf{Inclusioni:}
  \begin{enumerate}
    \item Verifica credenziali \emph{[UC1.2]}.
  \end{enumerate}
  \item \textbf{Estensioni:}
  \begin{enumerate}
    \item Visualizzazione messaggio di credenziali errate \emph{[UC1.3]}.
  \end{enumerate}
\end{itemize}
% subsub:UC1.1 (end)

\subsubsection{UC1.1.1 - Inserimento username}
\label{subsub:UC1.1.1}
\begin{itemize}
  \item \textbf{Codice:} UC1.1.1
  \item \textbf{Titolo:} inserimento username
  \item \textbf{Attori primari:} utente non autenticato
  \item \textbf{Precondizione:} il sistema ha reso disponibile il campo per l'inserimento del proprio username.
  \item \textbf{Postcondizione:} l'\glossario{utente non autenticato} ha compilato il campo relativo al proprio username.
  \item \textbf{Scenario principale:} 
  \begin{enumerate}
    \item l'\glossario{utente non autenticato} compila il campo relativo al proprio username di registrazione.  
  \end{enumerate}
\end{itemize}
% subsub:UC1.1.1 (end)

\subsubsection{UC1.1.2 - Inserimento password}
\label{subsub:UC1.1.2}
\begin{itemize}
  \item \textbf{Codice:} UC1.1.2
  \item \textbf{Titolo:} inserimento password
  \item \textbf{Attori primari:} utente non autenticato
  \item \textbf{Precondizione:} il sistema ha reso disponibile il campo per l'inserimento della password.
  \item \textbf{Postcondizione:} l'\glossario{utente non autenticato} ha compilato il campo relativo alla sua password.
  \item \textbf{Scenario principale:} 
  \begin{enumerate}
    \item l'\glossario{utente non autenticato} compila il campo relativo alla sua password di registrazione.
  \end{enumerate}
\end{itemize}
% subsub:UC1.1.2 (end)

\subsubsection{UC1.2 - Verifica Credenziali}
\label{subsub:UC1.2}
\begin{itemize}
  \item \textbf{Codice:} UC1.2
  \item \textbf{Titolo:} verifica credenziali
  \item \textbf{Attori primari:} utente non autenticato
  \item \textbf{Precondizione:} l'\glossario{utente non autenticato} ha inviato al server le sue credenziali per tentare il login
  \item \textbf{Postcondizione:} l'\glossario{utente non autenticato} deve poter accedere alla sua area riservata, nel caso in cui le credenziali siano corrette. In caso
  contrario deve essere visualizzato un messaggio di credenziali sbagliate \emph{[UC1.3]}.
  \item \textbf{Scenario principale:} 
  \begin{enumerate}
    \item l'\glossario{utente non autenticato} sta tentando di effettuare l'accesso e sta attendendo la verifica delle credenziali immesse.
  \end{enumerate}
\end{itemize}
% subsub:UC1.2 (end)

\subsubsection{UC1.3 - Visualizzazione credenziali errate}
\label{subsub:UC1.3}
\begin{itemize}
  \item \textbf{Codice:} UC1.3
  \item \textbf{Titolo:} visualizzazione credenziali errate
  \item \textbf{Attori primari:} utente non autenticato
  \item \textbf{Precondizione:} l'\glossario{utente non autenticato} ha inviato al server le sue credenziali per tentare il login, e le credenziali sono state verificate.
  \item \textbf{Postcondizione:} l'\glossario{utente non autenticato} visualizza un messaggio di credenziali sbagliate.
  \item \textbf{Scenario principale:} 
  \begin{enumerate}
    \item l'\glossario{utente non autenticato} cerca di effettuare il login con delle credenziali sbagliate. 
  \end{enumerate}
\end{itemize}
% subsub:UC1.3 (end)
% par:utente-non-autenticato (end)

\paragraph{Utente autenticato}
\label{par:utente-autenticato}
Di seguito sono riportati tutti i casi d'uso che coinvolgono l'\glossario{utente autenticato}.

\subsubsection{UC2 - Richiesta owner}
\label{subsub:UC2}
\begin{itemize}
  \item \textbf{Codice:} UC2
  \item \textbf{Titolo:} richiesta owner
  \item \textbf{Attori primari:} utente autenticato
  \item \textbf{Precondizione:} l'\glossario{utente autenticato} non deve essere un \glossario{owner}.
  \item \textbf{Postcondizione:} se viene approvata la richiesta, l'\glossario{utente autenticato} diventerà un \glossario{owner}.
  \item \textbf{Scenario principale:} 
  \begin{enumerate}
    \item l'\glossario{utente autenticato} vuole iniziare ad utilizzare \glossario{Stalker} creando una sua \glossario{organizzazione}.
  \end{enumerate}
  \item \textbf{Inclusioni}:
  \begin{enumerate}
    \item Verifica tipologia utente \emph{[UC2.1]};
    \item Invio richiesta owner \emph{[UC2.2]}.
  \end{enumerate}
  \item \textbf{Estensioni}:
  \begin{enumerate}
    \item Richiesta rifiutata \emph{[UC.2.3]}.
  \end{enumerate}
\end{itemize}
% subsub:UC2 (end)
% par:utente-autenticato (end)

\subsubsection{UC2.1 - Verifica tipologia utente}
\label{subsub:UC2.1}
\begin{itemize}
  \item \textbf{Codice:} UC2.1
  \item \textbf{Titolo:} verifica tipologia utente
  \item \textbf{Attori primari:} utente autenticato
  \item \textbf{Precondizione:} l'\glossario{utente autenticato} visualizza l'apposito form per inviare la richiesta per diventare \glossario{owner}.
  \item \textbf{Postcondizione:} il form relativo alla richiesta di \glossario{owner} viene abilitato o disabilitato di conseguenza.
  \item \textbf{Scenario principale:} 
  \begin{enumerate}
    \item l'\glossario{utente autenticato} può richiedere di diventare un \glossario{owner}, nel caso in cui sia:
    \begin{itemize}
      \item un \glossario{amministratore},
      \item un \glossario{visualizzatore},
      \item un \glossario{gestore} oppure un utente senza alcun privilegio;
    \end{itemize}
    \item ogni tentativo di richiesta da un già \glossario{owner} viene rifiutato;
    \item \glossario{root} è già un \glossario{owner};
    \item l'\glossario{amministratore} potrà accettare la propria richiesta inviata.
  \end{enumerate}
\end{itemize}
% subsub:UC2.1 (end)

\subsubsection{UC2.2 - Invio richiesta owner}
\label{subsub:UC2.2}
\begin{itemize}
  \item \textbf{Codice:} UC2.2
  \item \textbf{Titolo:} invio richiesta owner
  \item \textbf{Attori primari:} utente autenticato
  \item \textbf{Precondizione:} l'\glossario{utente autenticato} utilizza l'apposito form per inviare la richiesta per diventare \glossario{owner}.
  \item \textbf{Postcondizione:} viene inviata la richiesta, in attesa di essere approvata da un \glossario{amministratore}.
  \item \textbf{Scenario principale:} 
  \begin{enumerate}
    \item l'\glossario{utente autenticato} invia la richiesta per diventare \glossario{owner}.
  \end{enumerate}
\end{itemize}
% subsub:UC2.2 (end)

\subsubsection{UC2.3 - Richiesta rifiutata}
\label{subsub:UC2.3}
\begin{itemize}
  \item \textbf{Codice:} UC2.3
  \item \textbf{Titolo:} richiesta rifiutata
  \item \textbf{Attori primari:} utente autenticato
  \item \textbf{Precondizione:} l'\glossario{utente autenticato} ha inviato la propria richiesta per diventare \glossario{owner}, l'\glossario{amministratore} ha rifiutato la richiesta.
  \item \textbf{Postcondizione:} l'\glossario{utente autenticato} non ha ottenuto l'abilitazione per un \glossario{owner}, visualizza un errore.
  \item \textbf{Scenario principale:} 
  \begin{enumerate}
    \item l'\glossario{amministratore} non ha accettato la richiesta posta dall'\glossario{utente autenticato}.
  \end{enumerate}
\end{itemize}
% subsub:UC2.2 (end)

\subsubsection{UC3 - Logout}
\label{subsub:UC3}
\begin{itemize}
  \item \textbf{Codice:} UC3
  \item \textbf{Titolo:} logout
  \item \textbf{Attori primari:} utente autenticato
  \item \textbf{Precondizione:} il sistema ha reso disponibile la possibilità di effettuare il logout.
  \item \textbf{Postcondizione:} l'\glossario{utente autenticato} diventerà un \glossario{utente non autenticato}, effettuando il logout.
  \item \textbf{Scenario principale:} 
  \begin{enumerate}
    \item l'\glossario{utente autenticato} vuole effettuare il logout dalla piattaforma;
  \end{enumerate}
\end{itemize}
% subsub:UC3 (end)
% par:utente-autenticato (end)

\paragraph{Amministratore}
\label{par:amministratore}
Di seguito sono riportati tutti i casi d'uso che coinvolgono il \glossario{super utente} \glossario{amministratore}.

%inserire UML generale amministratore

\subsubsection{UC4 - Gestione richieste owner}
\label{subsub:UC4}
\begin{itemize}
  \item \textbf{Codice:} UC4
  \item \textbf{Titolo:} gestione richieste owner
  \item \textbf{Attori primari:} amministratore
  \item \textbf{Precondizione:} l'\glossario{utente autenticato} deve aver richiesto di diventare un \glossario{owner}.
  \item \textbf{Postcondizione:} viene approvata o rifiutata la richiesta, e l'\glossario{utente autenticato} sarà informato.
  \item \textbf{Scenario principale:} l'\glossario{amministratore} gestisce le richieste inviate dagli utenti autenticati di diventare \glossario{owner}.
  \item \textbf{Inclusioni:}
  \begin{enumerate}
    \item accetta richiesta owner\emph{[UC4.1]};
    \item rifiuta richiesta owner\emph{[UC4.2]}.
  \end{enumerate}
\end{itemize}
% subsub:UC4 (end)

\subsubsection{UC4.1 - Accetta richiesta owner}
\label{subsub:UC4.1}
\begin{itemize}
  \item \textbf{Codice:} UC4.1
  \item \textbf{Titolo:} accetta richiesta owner
  \item \textbf{Attori primari:} amministratore
  \item \textbf{Precondizione:} l'\glossario{utente autenticato} deve aver richiesto di diventare un \glossario{owner}.
  \item \textbf{Postcondizione:} viene approvata la richiesta da parte di un \glossario{amministratore}.
  \item \textbf{Scenario principale:} 
  \begin{enumerate}
    \item l'\glossario{amministratore} accetta la richiesta inviata dall'\glossario{utente autenticato} di diventare \glossario{owner};
  \end{enumerate}
\end{itemize}
% subsub:UC4.1 (end)

\subsubsection{UC4.2 - Rifiuta richiesta owner}
\label{subsub:UC4.2}
\begin{itemize}
  \item \textbf{Codice:} UC4.2
  \item \textbf{Titolo:} rifiuta richiesta owner
  \item \textbf{Attori primari:} amministratore
  \item \textbf{Precondizione:} l'\glossario{utente autenticato} deve aver richiesto di diventare un \glossario{owner}.
  \item \textbf{Postcondizione:} viene rifiutata la richiesta da parte di un \glossario{amministratore}.
  \item \textbf{Scenario principale:} 
  \begin{enumerate}
    \item l'\glossario{amministratore} rifiuta la richiesta inviata dall'\glossario{utente autenticato} di diventare \glossario{owner};
  \end{enumerate}
\end{itemize}
% subsub:UC4.2 (end)

\subsubsection{UC5 - Gestione richiesta creazione organizzazione}
\label{subsub:UC5}
\begin{itemize}
  \item \textbf{Codice:} UC5
  \item \textbf{Titolo:} gestione richiesta creazione organizzazione
  \item \textbf{Attori primari:} amministratore
  \item \textbf{Precondizione:} l'\glossario{owner} deve aver richiesto di creare un organizzazione.
  \item \textbf{Postcondizione:} viene approvata o rifiutata la richiesta, e l'\glossario{owner} viene informato.
  \item \textbf{Scenario principale:} 
  \begin{enumerate}
    \item viene richiesta la creazione di una nuova \glossario{organizzazione}, che vuole utilizzare \glossario{Stalker}. L'\glossario{amministratore} può accettare o rifiutare la richiesta.
  \end{enumerate}
  \item \textbf{Inclusioni}:
  \begin{enumerate}
    \item accetta richiesta creazione organizzazione\emph{[UC5.1]};
    \item rifiuta richiesta creazione organizzazione\emph{[UC5.2]}.
  \end{enumerate}
\end{itemize}
% subsub:UC5 (end)

\subsubsection{UC5.1 - Accetta richiesta creazione organizzazione}
\label{subsub:UC5.1}
\begin{itemize}
  \item \textbf{Codice:} UC5.1
  \item \textbf{Titolo:} accetta richiesta creazione organizzazione
  \item \textbf{Attori primari:} amministratore
  \item \textbf{Precondizione:} l'\glossario{owner} deve aver richiesto di creare un organizzazione.
  \item \textbf{Postcondizione:} viene approvata la richiesta, e l'\glossario{owner} viene informato.
  \item \textbf{Scenario principale:} 
  \begin{enumerate}
    \item  l'\glossario{amministratore} accetta la richiesta inviata dall'\glossario{owner} di creare una nuova \glossario{organizzazione};
  \end{enumerate}
\end{itemize}
% subsub:UC5.1 (end)

\subsubsection{UC5.2 - Rifiuta richiesta creazione organizzazione}
\label{subsub:UC5.2}
\begin{itemize}
  \item \textbf{Codice:} UC5.2
  \item \textbf{Titolo:} rifiuta richiesta creazione organizzazione
  \item \textbf{Attori primari:} amministratore
  \item \textbf{Precondizione:} l'\glossario{owner} deve aver richiesto di creare un organizzazione;
  \item \textbf{Postcondizione:} viene rifiutata la richiesta, e l'\glossario{owner} viene informato;
  \item \textbf{Scenario principale:} 
  \begin{enumerate}
    \item  l'\glossario{amministratore} rifiutala richiesta inviata dall'\glossario{owner} di creare una nuova \glossario{organizzazione};
  \end{enumerate}
\end{itemize}
% subsub:UC5.2 (end)

\subsubsection{UC6 - Gestione richiesta modifica organizzazione}
\label{subsub:UC6}
\begin{itemize}
  \item \textbf{Codice:} UC6
  \item \textbf{Titolo:} gestione richiesta modifica organizzazione
  \item \textbf{Attori primari:} amministratore
  \item \textbf{Precondizione:} un \glossario{gestore} deve aver richiesto di modificare un'organizzazione.
  \item \textbf{Postcondizione:} viene approvata o rifiutata la richiesta, e il \glossario{gestore} viene informato.
  \item \textbf{Scenario principale:} 
  \begin{enumerate}
    \item viene richiesta la modifica di una \glossario{organizzazione}. L'\glossario{amministratore} può accettare o rifiutare la richiesta.
  \end{enumerate}
  \item \textbf{Inclusioni}:
  \begin{enumerate}
    \item accetta richiesta modifica organizzazione\emph{[UC6.1]};
    \item rifiuta richiesta modifica organizzazione\emph{[UC6.2]}.
  \end{enumerate}
\end{itemize}
% subsub:UC6 (end)

\subsubsection{UC6.1 - Accetta richiesta modifica organizzazione}
\label{subsub:UC6.1}
\begin{itemize}
  \item \textbf{Codice:} UC6.1
  \item \textbf{Titolo:} accetta richiesta modifica organizzazione
  \item \textbf{Attori primari:} amministratore
  \item \textbf{Precondizione:} il \glossario{gestore} deve aver richiesto di modificare un'organizzazione.
  \item \textbf{Postcondizione:} viene approvata la richiesta, e il \glossario{gestore} viene informato.
  \item \textbf{Scenario principale:} 
  \begin{enumerate}
    \item  l'\glossario{amministratore} accetta la richiesta inviata dal \glossario{gestore} di modificare un'\glossario{organizzazione};
  \end{enumerate}
\end{itemize}
% subsub:UC6.1 (end)

\subsubsection{UC6.2 - Rifiuta richiesta modifica organizzazione}
\label{subsub:UC6.2}
\begin{itemize}
  \item \textbf{Codice:} UC6.2
  \item \textbf{Titolo:} rifiuta richiesta modifica organizzazione
  \item \textbf{Attori primari:} amministratore
  \item \textbf{Precondizione:} il \glossario{gestore} deve aver richiesto di modificare un'organizzazione.
  \item \textbf{Postcondizione:} viene rifiutata la richiesta, e il \glossario{gestore} viene informato.
  \item \textbf{Scenario principale:} 
  \begin{enumerate}
    \item l'\glossario{amministratore} rifiuta la richiesta inviata dal \glossario{gestore} di modificareun'\glossario{organizzazione};
  \end{enumerate}
\end{itemize}
% subsub:UC6.2 (end)

\subsubsection{UC7 - Gestione richiesta aggiunta luogo}
\label{subsub:UC7}
\begin{itemize}
  \item \textbf{Codice:} UC7
  \item \textbf{Titolo:} gestione richiesta aggiunta luogo
  \item \textbf{Attori primari:} amministratore
  \item \textbf{Precondizione:} un \glossario{gestore} deve aver richiesto di aggiungere un luogo.
  \item \textbf{Postcondizione:} viene approvata o rifiutata la richiesta, e il \glossario{gestore} viene informato.
  \item \textbf{Scenario principale:} 
  \begin{enumerate}
    \item viene richiesto di aggiungere un luogo. L'\glossario{amministratore} può accettare o rifiutare la richiesta.
  \end{enumerate}
  \item \textbf{Inclusioni}:
  \begin{enumerate}
    \item accetta richiesta aggiunta luogo\emph{[UC7.1]};
    \item rifiuta richiesta aggiunta luogo\emph{[UC7.2]}.
  \end{enumerate}
\end{itemize}
% subsub:UC7 (end)

\subsubsection{UC7.1 - Accetta richiesta aggiungi luogo}
\label{subsub:UC7.1}
\begin{itemize}
  \item \textbf{Codice:} UC7.1
  \item \textbf{Titolo:} accetta richiesta aggiungi luogo
  \item \textbf{Attori primari:} amministratore
  \item \textbf{Precondizione:} il \glossario{gestore} deve aver richiesto di aggiungere un luogo.
  \item \textbf{Postcondizione:} viene approvata la richiesta, e il \glossario{gestore} viene informato.
  \item \textbf{Scenario principale:} 
  \begin{enumerate}
    \item  l'\glossario{amministratore} accetta la richiesta inviata dal \glossario{gestore} di aggiungere un luogo.
  \end{enumerate}
\end{itemize}
% subsub:UC7.1 (end)

\subsubsection{UC7.2 - Rifiuta richiesta aggiungi luogo}
\label{subsub:UC7.2}
\begin{itemize}
  \item \textbf{Codice:} UC7.2
  \item \textbf{Titolo:} rifiuta richiesta aggiungi luogo
  \item \textbf{Attori primari:} amministratore
  \item \textbf{Precondizione:} il \glossario{gestore} deve aver richiesto di aggiungere un luogo.
  \item \textbf{Postcondizione:} viene rifiutata la richiesta, e il \glossario{gestore} viene informato.
  \item \textbf{Scenario principale:} 
  \begin{enumerate}
    \item l'\glossario{amministratore} rifiuta la richiesta inviata dal \glossario{gestore} di aggiungere un luogo.
  \end{enumerate}
\end{itemize}
% subsub:UC7.2 (end)

\subsubsection{UC8 - Gestione richiesta modifica luogo}
\label{subsub:UC8}
\begin{itemize}
  \item \textbf{Codice:} UC7
  \item \textbf{Titolo:} gestione richiesta modifica luogo
  \item \textbf{Attori primari:} amministratore
  \item \textbf{Precondizione:} un \glossario{gestore} deve aver richiesto di modificare un luogo.
  \item \textbf{Postcondizione:} viene approvata o rifiutata la richiesta, e il \glossario{gestore} viene informato.
  \item \textbf{Scenario principale:} 
  \begin{enumerate}
    \item viene richiesto di modificare un luogo. L'\glossario{amministratore} può accettare o rifiutare la richiesta
  \end{enumerate}
  \item \textbf{Inclusioni}:
  \begin{enumerate}
    \item accetta richiesta modifica luogo\emph{[UC8.1]};
    \item rifiuta richiesta modifica luogo\emph{[UC8.2]}.
  \end{enumerate}
\end{itemize}
% subsub:UC8 (end)

\subsubsection{UC8.1 - Accetta richiesta modifica luogo}
\label{subsub:UC8.1}
\begin{itemize}
  \item \textbf{Codice:} UC8.1
  \item \textbf{Titolo:} accetta richiesta modifica luogo
  \item \textbf{Attori primari:} amministratore
  \item \textbf{Precondizione:} il \glossario{gestore} deve aver richiesto di modificare un luogo.
  \item \textbf{Postcondizione:} viene approvata la richiesta, e il \glossario{gestore} viene informato.
  \item \textbf{Scenario principale:} 
  \begin{enumerate}
    \item  l'\glossario{amministratore} accetta la richiesta inviata dal \glossario{gestore} di modificare un luogo.
  \end{enumerate}
\end{itemize}
% subsub:UC8.1 (end)

\subsubsection{UC8.2 - Rifiuta richiesta modifica luogo}
\label{subsub:UC8.2}
\begin{itemize}
  \item \textbf{Codice:} UC8.2
  \item \textbf{Titolo:} rifiuta richiesta modifica luogo
  \item \textbf{Attori primari:} amministratore
  \item \textbf{Precondizione:} il \glossario{gestore} deve aver richiesto di modificare un luogo.
  \item \textbf{Postcondizione:} viene rifiutata la richiesta, e il \glossario{gestore} viene informato.
  \item \textbf{Scenario principale:}
  \begin{enumerate}
    \item l'\glossario{amministratore} rifiuta la richiesta inviata dal \glossario{gestore} di modificare un luogo.
  \end{enumerate}  
\end{itemize}
% subsub:UC8.2 (end)

\subsubsection{UC9 - Gestione richiesta trasferimento di proprietà organizzazione}
\label{subsub:UC9}
\begin{itemize}
  \item \textbf{Codice:} UC9
  \item \textbf{Titolo:} gestione richiesta trasferimento di proprietà organizzazione
  \item \textbf{Attori primari:} amministratore
  \item \textbf{Precondizione:} l'\glossario{organizzazione} deve essere già stata creata dal sistema, e deve essere stata fatta la richiesta trasferimento di proprietà.
  \item \textbf{Postcondizione:} viene approvata o rifiutata la richiesta, e l'\glossario{owner} viene informato.
  \item \textbf{Scenario principale:}
  \begin{enumerate}
    \item l'\glossario{amministratore} approva o rifiuta la richiesta di un trasferimento di proprietà di un'\glossario{organizzazione}.
  \end{enumerate}
  \item \textbf{Inclusioni}:
  \begin{enumerate}
    \item accetta richiesta trasferimento di proprietà organizzazione\emph{[UC9.1]};
    \item rifiuta richiesta trasferimento di proprietà organizzazione\emph{[UC9.2]}.
  \end{enumerate}
\end{itemize}
% subsub:UC9 (end)

\subsubsection{UC9.1 - Accetta richiesta trasferimento di proprietà organizzazione}
\label{subsub:UC9.1}
\begin{itemize}
  \item \textbf{Codice:} UC9.1
  \item \textbf{Titolo:} accetta richiesta trasferimento di proprietà organizzazione
  \item \textbf{Attori primari:} amministratore
  \item \textbf{Precondizione:} l'\glossario{owner} deve aver fatto richiesta di trasferimento di proprietà di organizzazione.
  \item \textbf{Postcondizione:} viene approvata la richiesta, e l'\glossario{owner} viene informato.
  \item \textbf{Scenario principale:} 
  \begin{enumerate}
    \item l'\glossario{amministratore} approva la richiesta di un trasferimento di proprietà di un'\glossario{organizzazione}.
  \end{enumerate}
\end{itemize}  
% subsub:UC9.1 (end)

\subsubsection{UC9.2 - Rifiuta richiesta trasferimento di proprietà organizzazione}
\label{subsub:UC9.2}
\begin{itemize}
  \item \textbf{Codice:} UC9.2
  \item \textbf{Titolo:} rifiuta richiesta trasferimento di proprietà organizzazione
  \item \textbf{Attori primari:} amministratore
  \item \textbf{Precondizione:} l'\glossario{owner} deve aver fatto richiesta di trasferimento di proprietà di organizzazione.
  \item \textbf{Postcondizione:} viene rifiutata la richiesta, e l'\glossario{owner} viene informato.
  \item \textbf{Scenario principale:} 
  \begin{enumerate}
    \item  l'\glossario{amministratore} rifiuta la richiesta di un trasferimento di proprietà di un'\glossario{organizzazione}.
  \end{enumerate}
\end{itemize}
% subsub:UC9.2 (end)
% par:Amministratore (end)

\paragraph{Root}
Di seguito sono riportati tutti i casi d'uso che coinvolgono il \glossario{super utente} \glossario{root}.

\subsubsection{UC10 - Creazione amministratore}
\label{subsub:UC10}
\begin{itemize}
  \item \textbf{Codice:} UC10
  \item \textbf{Titolo:} creazione amministratore
  \item \textbf{Attori primari:} root
  \item \textbf{Precondizione:} devono essere specificate le credenziali del nuovo \glossario{amministratore}, che devono essere univoche.
  \item \textbf{Postcondizione:} l'\glossario{amministratore} viene creato.
  \item \textbf{Scenario principale:} 
  \begin{enumerate}
    \item sorge la necessità di creare un nuovo \glossario{amministratore} per gestire \glossario{Stalker}.
  \end{enumerate}
  \item \textbf{Inclusioni}:
  \begin{enumerate}
    \item inserimento nuovo username\emph{[UC10.1]};
    \item inserimento nuova password\emph{[UC10.2]};
    \item verifica credenziali\emph{[UC10.3]}.
  \end{enumerate}
  \item \textbf{Estensioni}:
  \begin{enumerate}
    \item  visualizza creazione fallita\emph{[UC10.4]}.
  \end{enumerate}
\end{itemize}
% subsub:UC10 (end)

\subsubsection{UC10.1 - Inserimento nuovo username}
\label{subsub:UC10.1}
\begin{itemize}
  \item \textbf{Codice:} UC10.1
  \item \textbf{Titolo:} inserimento nuovo username
  \item \textbf{Attori primari:} root
  \item \textbf{Precondizione:} il sistema ha reso disponibile il campo per l'inserimento dell'username.
  \item \textbf{Postcondizione:} \glossario{Root} ha inserito l'username dell'\glossario{amministratore} che vuole creare.
  \item \textbf{Scenario principale:} 
  \begin{enumerate}
    \item \glossario{Root} inserisce l'username dell'\glossario{amministratore} che vuole creare.
  \end{enumerate}
\end{itemize}
% subsub:UC10.1 (end)

\subsubsection{UC10.2 - Inserimento nuova password}
\label{subsub:UC10.2}
\begin{itemize}
  \item \textbf{Codice:} UC10.2
  \item \textbf{Titolo:} inserimento nuova password
  \item \textbf{Attori primari:} root
  \item \textbf{Precondizione:} il sistema ha reso disponibile il campo per l'inserimento della password.
  \item \textbf{Postcondizione:} \glossario{Root} ha inserito la password dell'\glossario{amministratore} che vuole creare.
  \item \textbf{Scenario principale:} 
  \begin{enumerate}
    \item \glossario{Root} inserisce la password dell'\glossario{amministratore} che vuole creare.
  \end{enumerate}
\end{itemize}
% subsub:UC10.2 (end)

\subsubsection{UC10.3 - Verifica credenziali}
\label{subsub:UC10.3}
\begin{itemize}
  \item \textbf{Codice:} UC10.3
  \item \textbf{Titolo:} verifica credenziali
  \item \textbf{Attori primari:} root
  \item \textbf{Precondizione:} \glossario{Root} invia richiesta di creazione nuovo amministratore.
  \item \textbf{Postcondizione:} il nuovo \glossario{amministratore} viene creato solo se i requisiti sono stati rispettati.
  \item \textbf{Scenario principale:} 
  \begin{enumerate}
    \item il sistema verifica se le credenziali immesse rispettano i requisiti:
    \begin{enumerate}
      \item username e password non siano vuoti;
      \item username e password contengano solo i caratteri consentiti;
      \item lo username non esista.
    \end{enumerate}
  \end{enumerate}
\end{itemize}
% subsub:UC10.3 (end)

\subsubsection{UC10.4 - Visualizza creazione fallita}
\label{subsub:UC10.4}
\begin{itemize}
  \item \textbf{Codice:} UC10.4
  \item \textbf{Titolo:} visualizza creazione fallita
  \item \textbf{Attori primari:} root
  \item \textbf{Precondizione:} la verifica delle credenziali è fallita.
  \item \textbf{Postcondizione:} \glossario{root} visualizza un messaggio di creazione fallita.
  \item \textbf{Scenario principale:} 
  \begin{enumerate}
    \item \glossario{root} cerca di creare un nuovo \glossario{amministratore} che non rispetta i requisiti.
  \end{enumerate}
\end{itemize}
% subsub:UC10.4 (end)

\subsubsection{UC11 - Eliminazione organizzazione}
\label{subsub:UC11}
\begin{itemize}
  \item \textbf{Codice:} UC11
  \item \textbf{Titolo:} eliminazione organizzazione
  \item \textbf{Attori primari:} root
  \item \textbf{Precondizione:} deve essere stata selezionata l'\glossario{organizzazione} da eliminare, presente nella lista di \glossario{Stalker}.
  \item \textbf{Postcondizione:} l'\glossario{organizzazione} viene eliminata.
  \item \textbf{Scenario principale:} 
  \begin{enumerate}
    \item sorge la necessità di eliminare un'\glossario{organizzazione}, senza interagire con il suo \glossario{owner};
  \end{enumerate}
\end{itemize}
% subsub:UC11 (end)

\subsubsection{UC12 - Modifica organizzazione}
\label{subsub:UC12}
\begin{itemize}
  \item \textbf{Codice:} UC12
  \item \textbf{Titolo:} modifica organizzazione
  \item \textbf{Attori primari:} root
  \item \textbf{Precondizione:} deve essere stata selezionata l'\glossario{organizzazione} da modificare, presente nella lista di \glossario{Stalker}.
  \item \textbf{Postcondizione:} l'\glossario{organizzazione} viene modificata.
  \item \textbf{Scenario principale:}
  \begin{enumerate}
    \item sorge la necessità di modificare un'\glossario{organizzazione}, senza interagire con il suo \glossario{owner};
  \end{enumerate}
  \item \textbf{Inclusioni:}
  \begin{enumerate}
    \item Modifica nome organizzazione\emph{[UC12.1]};
    \item Modifica indirizzo organizzazione\emph{[UC12.2]};
    \item Modifica descrizione organizzazione\emph{[UC12.3]}.
  \end{enumerate}
\end{itemize}
% subsub:UC12 (end)

\subsubsection{UC12.1 - Modifica nome organizzazione}
\label{subsub:UC12.1}
\begin{itemize}
  \item \textbf{Codice:} UC12.1
  \item \textbf{Titolo:} modifica nome organizzazione
  \item \textbf{Attori primari:} root
  \item \textbf{Precondizione:} il sistema fornisce il campo di modifica nome.
  \item \textbf{Postcondizione:} il nome viene opportunamente modificato.
  \item \textbf{Scenario principale:} 
  \begin{enumerate}
    \item si vuole modificare il nome di un'\glossario{organizzazione}.
  \end{enumerate}
  
\end{itemize}
% subsub:UC12.1 (end)

\subsubsection{UC12.2 - Modifica indirizzo organizzazione}
\label{subsub:UC12.2}
\begin{itemize}
  \item \textbf{Codice:} UC12.2
  \item \textbf{Titolo:} Modifica indirizzo organizzazione
  \item \textbf{Attori primari:} root
  \item \textbf{Precondizione:} il sistema fornisce il campo di modifica indirizzo organizzazione.
  \item \textbf{Postcondizione:} l'indirizzo viene opportunamente modificato.
  \item \textbf{Scenario principale:}
  \begin{enumerate}
    \item si vuole modificare l'indirizzo di un'\glossario{organizzazione}.
  \end{enumerate}
\end{itemize}
% subsub:UC12.2 (end)

\subsubsection{UC12.3 - Modifica descrizione organizzazione}
\label{subsub:UC12.3}
\begin{itemize}
  \item \textbf{Codice:} UC12.3
  \item \textbf{Titolo:} Modifica descrizione organizzazione
  \item \textbf{Attori primari:} root
  \item \textbf{Precondizione:} il sistema fornisce il campo di modifica descrizione organizzazione.
  \item \textbf{Postcondizione:} la descrizione viene opportunamente modificata.
  \item \textbf{Scenario principale:}
  \begin{enumerate}
    \item si vuole modificare la descrizione di un'\glossario{organizzazione}.
  \end{enumerate}
\end{itemize}
% subsub:UC12.3 (end)


\subsubsection{UC13 - Creazione organizzazione}
\label{subsub:UC13}
\begin{itemize}
  \item \textbf{Attori primari:} \glossario{root};
  \item \textbf{Precondizione:} l'\glossario{organizzazione} non deve esistere nella lista di \glossario{Stalker}, deve essere specificato il suo nome.
  \item \textbf{Postcondizione:} l'\glossario{organizzazione} viene creata.
  \item \textbf{Scenario principale:} sorge la necessità di creare un'\glossario{organizzazione}, senza essere effettivamente richiesta;
  \item \textbf{Inclusioni:}
  \begin{enumerate}
    \item Inserisci nome organizzazione\emph{[UC12.1]};
    \item Inserisci indirizzo organizzazione\emph{[UC12.2]};
    \item Inserisci descrizione organizzazione\emph{[UC12.3]}.
  \end{enumerate}


  \subsubsection{UC13.1 - Inserisci nome organizzazione}
  \label{subsub:UC13.1}
  \begin{itemize}
    \item \textbf{Codice:} UC13.1
    \item \textbf{Titolo:} Inserisci nome organizzazione
    \item \textbf{Attori primari:} root
    \item \textbf{Precondizione:} il sistema fornisce il campo di inserimento nome.
    \item \textbf{Postcondizione:} il nome viene opportunamente inserito.
    \item \textbf{Scenario principale:} 
    \begin{enumerate}
      \item si vuole inserire il nome di un'\glossario{organizzazione}.
    \end{enumerate}
    
  \end{itemize}
  % subsub:UC13.1 (end)
  
  \subsubsection{UC13.2 - Inserisci indirizzo organizzazione}
  \label{subsub:UC13.2}
  \begin{itemize}
    \item \textbf{Codice:} UC13.2
    \item \textbf{Titolo:} Inserisci indirizzo organizzazione
    \item \textbf{Attori primari:} root
    \item \textbf{Precondizione:} il sistema fornisce il campo di inserimento indirizzo organizzazione.
    \item \textbf{Postcondizione:} l'indirizzo viene opportunamente inserito.
    \item \textbf{Scenario principale:}
    \begin{enumerate}
      \item si vuole inserire l'indirizzo di un'\glossario{organizzazione}.
    \end{enumerate}
  \end{itemize}
  % subsub:UC13.2 (end)
  
  \subsubsection{UC13.3 - Inserisci descrizione organizzazione}
  \label{subsub:UC13.3}
  \begin{itemize}
    \item \textbf{Codice:} UC13.3
    \item \textbf{Titolo:} Inserisci descrizione organizzazione
    \item \textbf{Attori primari:} root
    \item \textbf{Precondizione:} il sistema fornisce il campo di inserimento descrizione organizzazione.
    \item \textbf{Postcondizione:} la descrizione viene opportunamente inserito.
    \item \textbf{Scenario principale:}
    \begin{enumerate}
      \item si vuole inserire la descrizione di un'\glossario{organizzazione}.
    \end{enumerate}
  \end{itemize}
  % subsub:UC13.3 (end)



\end{itemize}
% subsub:UC13 (end)

\subsubsection{UC14 - Eliminazione luogo}
\label{subsub:UC14}
\begin{itemize}
  \item \textbf{Attori primari:} \glossario{root};
  \item \textbf{Scenario principale:} sorge la necessità di elimanare un \glossario{luogo} di un'\glossario{organizzazione}, senza interagire con il suo \glossario{owner};
  \item \textbf{Precondizione:} il \glossario{luogo} dell'\glossario{organizzazione} deve essere presente in \glossario{Stalker};
  \item \textbf{Postcondizione:} il \glossario{luogo} dell'\glossario{organizzazione} viene eliminato.

\end{itemize}
% subsub:UC14 (end)

\subsubsection{UC15 - Modifica luogo}
\label{subsub:UC15}
\begin{itemize}
  \item \textbf{Attori primari:} \glossario{root};
  \item \textbf{Descrizione:} il \glossario{root} modifica un \glossario{luogo} di un'\glossario{organizzazione};
  \item \textbf{Scenario principale:} sorge la necessità di modificare un \glossario{luogo} di un'\glossario{organizzazione}, senza interagire con il suo \glossario{owner};
  \item \textbf{Precondizione:} il \glossario{luogo} dell'\glossario{organizzazione} deve essere presente in \glossario{Stalker};
  \item \textbf{Postcondizione:} il \glossario{luogo} dell'\glossario{organizzazione} viene modificato.

\end{itemize}
% subsub:UC15 (end)

\subsubsection{UC16 - Eliminazione account}
\label{subsub:UC16}
\begin{itemize}
  \item \textbf{Attori primari:} \glossario{root};
  \item \textbf{Descrizione:} il \glossario{root} elimina un qualsiasi \glossario{account} registrato in \glossario{Stalker}. Quest'ultimo può essere un \glossario{utente} oppure un \glossario{super utente};
  \item \textbf{Scenario principale:} sorge la necessità di eliminare un \glossario{account} per sconosciuti motivi;
  \item \textbf{Precondizione:} deve essere stato selezionato l'\glossario{account} da eliminare, che deve esistere in \glossario{Stalker};
  \item \textbf{Postcondizione:} l'\glossario{account} selezionato è stato eliminato.

\end{itemize}
% subsub:UC16 (end)


% par:Root (end)

\paragraph{Visualizzatore}
Di seguito sono riportati tutti i casi d'uso che coinvolgono il \glossario{super utente} \glossario{Visualizzatore}.

\subsubsection{UC17 - Query sull'organizzazione}
\label{subsub:UC17}
\begin{itemize}
  \item \textbf{Attori primari:} \glossario{visualizzatore};
  \item \textbf{Descrizione:} il \glossario{visualizzatore} effettua delle interrogazioni per trarre informazioni relative all'\glossario{organizzazione} su cui opera;
  \item \textbf{Scenario principale:} il \glossario{visualizzatore} vuole avere delle informazioni riguardanti l'\glossario{organizzazione} su cui opera;
  \item \textbf{Precondizione:} il sistema risponde correttamente alle interrogazioni;
  \item \textbf{Postcondizione:} il \glossario{visualizzatore} ottiene le informazioni di cui ha bisogno.
\end{itemize}
% subsub:UC17 (end)

\subsubsection{UC18 - Query sul dipendente}
\label{subsub:UC18}
\begin{itemize}
  \item \textbf{Attori primari:} \glossario{visualizzatore};
  \item \textbf{Descrizione:} il \glossario{visualizzatore} effettua delle interrogazioni per trarre informazioni relative al dipendente dell'\glossario{organizzazione} su cui opera;
  \item \textbf{Scenario principale:} il \glossario{visualizzatore} vuole avere delle informazioni riguardanti il\glossario{dipendente} dell'\glossario{organizzazione} su cui opera;
  \item \textbf{Precondizione:} il sistema risponde correttamente alle interrogazioni;
  \item \textbf{Postcondizione:} il \glossario{visualizzatore} ottiene le informazioni di cui ha bisogno.
\end{itemize}
% subsub:UC18 (end)
% par:Visualizzatore (end)

\paragraph{Gestore}
Di seguito sono riportati tutti i casi d'uso che coinvolgono il \glossario{super utente} \glossario{Gestore}.

\subsubsection{UC19 - Richiesta aggiunta luogo}
\label{subsub:UC19}
\begin{itemize}
  \item \textbf{Attori primari:} \glossario{gestore};
  \item \textbf{Descrizione:} il \glossario{gestore} dell'\glossario{organizzazione} richiede di aggiungere un nuovo \glossario{luogo};
  \item \textbf{Scenario principale:} il \glossario{gestore} vuole aggiungere un nuovo \glossario{luogo} all'\glossario{organizzazione} su cui opera;
  \item \textbf{Precondizione:} il \glossario{luogo} da aggiungere non deve già esistere;
  \item \textbf{Postcondizione:} la richiesta di aggiunta di un nuovo \glossario{luogo} viene posta.
\end{itemize}
% subsub:UC19 (end)

\subsubsection{UC20 - Richiesta rimozione luogo}
\label{subsub:UC20}
\begin{itemize}
  \item \textbf{Attori primari:} \glossario{gestore};
  \item \textbf{Descrizione:} il \glossario{gestore} dell'\glossario{organizzazione} richiede di eliminare un \glossario{luogo};
  \item \textbf{Scenario principale:} il \glossario{gestore} vuole eliminare un \glossario{luogo} all'\glossario{organizzazione} su cui opera;
  \item \textbf{Precondizione:} il \glossario{luogo} da eliminare deve esistere;
  \item \textbf{Postcondizione:} la richiesta di eliminazione di un \glossario{luogo} viene posta.
\end{itemize}
% subsub:UC20 (end)

\subsubsection{UC21 - Richiesta modifica luogo}
\label{subsub:UC21}
\begin{itemize}
  \item \textbf{Attori primari:} \glossario{gestore};
  \item \textbf{Descrizione:} il \glossario{gestore} dell'\glossario{organizzazione} richiede di modificare un \glossario{luogo};
  \item \textbf{Scenario principale:} il \glossario{gestore} vuole modificare un \glossario{luogo} dell'\glossario{organizzazione} su cui opera;
  \item \textbf{Precondizione:} il \glossario{luogo} da modificare deve esistere;
  \item \textbf{Postcondizione:} la richiesta di modifica di un \glossario{luogo} viene posta.
\end{itemize}
% subsub:UC21 (end)


\subsubsection{UC22 - Richiesta modifica parametri organizzazione}
\label{subsub:UC22}
\begin{itemize}
  \item \textbf{Attori primari:} \glossario{gestore};
  \item \textbf{Descrizione:} il \glossario{gestore} dell'\glossario{organizzazione} richiede di modificare i parametri di essa;
  \item \textbf{Scenario principale:} il \glossario{gestore} vuole modificare i parametri dell'\glossario{organizzazione} su cui opera;
  \item \textbf{Precondizione:} i parametri devono essere effettivamente modificati;
  \item \textbf{Postcondizione:} la richiesta di modifica parametri viene posta.
\end{itemize}
% subsub:UC22 (end)
% par:Gestore (end)

\paragraph{Owner}
Di seguito sono riportati tutti i casi d'uso che coinvolgono il \glossario{super utente} \glossario{Owner}.

\subsubsection{UC23 - Nomina visualizzatore}
\label{subsub:UC23}
\begin{itemize}
  \item \textbf{Attori primari:} \glossario{owner};
  \item \textbf{Descrizione:} l' \glossario{owner} nomina un \glossario{visualizzatore} per la sua \glossario{organizzazione};
  \item \textbf{Scenario principale:} l' \glossario{owner} vuole aggiungere un \glossario{visualizzatore} alla sua \glossario{organizzazione};
  \item \textbf{Precondizione:} il nuovo \glossario{super utente} non deve già esistere come \glossario{visualizzatore};
  \item \textbf{Postcondizione:} il nuovo \glossario{visualizzatore} è stato aggiunto.
\end{itemize}
% subsub:UC23 (end)

\subsubsection{UC24 - Nomina gestore}
\label{subsub:UC24}
\begin{itemize}
  \item \textbf{Attori primari:} \glossario{owner};
  \item \textbf{Descrizione:} l' \glossario{owner} nomina un \glossario{gestore} per la sua \glossario{organizzazione};
  \item \textbf{Scenario principale:} l' \glossario{owner} vuole aggiungere un \glossario{gestore} alla sua \glossario{organizzazione};
  \item \textbf{Precondizione:} il nuovo \glossario{super utente} non deve già esistere come \glossario{gestore};
  \item \textbf{Postcondizione:} il nuovo \glossario{gestore} è stato aggiunto.
\end{itemize}
% subsub:UC24 (end)

\subsubsection{UC25 - Nomina gestore}
\label{subsub:UC25}
\begin{itemize}
  \item \textbf{Attori primari:} \glossario{owner};
  \item \textbf{Descrizione:} l' \glossario{owner} nomina un \glossario{gestore} per la sua \glossario{organizzazione};
  \item \textbf{Scenario principale:} l' \glossario{owner} vuole aggiungere un \glossario{gestore} alla sua \glossario{organizzazione};
  \item \textbf{Precondizione:} il nuovo \glossario{super utente} non deve già esistere come \glossario{gestore};
  \item \textbf{Postcondizione:} il nuovo \glossario{gestore} è stato aggiunto.
\end{itemize}
% subsub:UC25 (end)

\subsubsection{UC26 - Richiesta creazione organizzazione}
\label{subsub:UC26}
\begin{itemize}
  \item \textbf{Attori primari:} \glossario{owner};
  \item \textbf{Descrizione:} l' \glossario{owner} richiede di creare una nuova \glossario{organizzazione};
  \item \textbf{Scenario principale:} l' \glossario{owner} vuole creare una nuova \glossario{organizzazione};
  \item \textbf{Precondizione:} l' \glossario{organizzazione} non deve già esistere;
  \item \textbf{Postcondizione:} la richiesta di creare una nuova \glossario{organizzazione} è stata posta.
\end{itemize}
% subsub:UC26 (end)

\subsubsection{UC27 - Richiesta cedimento proprietà organizzazione}
\label{subsub:UC27}
\begin{itemize}
  \item \textbf{Attori primari:} \glossario{owner};
  \item \textbf{\textbf{Descrizione:}} l' \glossario{owner} richiede che l'\glossario{organizzazione} venga ceduta ad un altro futuro \glossario{owner};
  \item \textbf{\textbf{Scenario principale:}} l' \glossario{owner} vuole cedere l'\glossario{organizzazione} ad un'altro \glossario{owner};
  \item \textbf{Precondizione:} il futuro \glossario{owner} deve esistere;
  \item \textbf{Postcondizione:} la richiesta di cedere l'\glossario{organizzazione} è stata posta.
\end{itemize}
% subsub:UC27 (end)

% par:Owner (end)

\end{document}
