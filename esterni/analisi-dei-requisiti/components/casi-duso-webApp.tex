\documentclass[casi-duso]{subfiles}

%\renewcommand{\commons}{../../../commons}

\begin{document}

\paragraph{Utente non autenticato}
\label{par:utente-non-autenticato}
Di seguito sono riportati tutti i casi d'uso che coinvolgono l'\glossario{utente non autenticato}.

\subsubsection{UC1 - Sistema di autenticazione}
\label{subsub:UC1}


\begin{plantuml}
  @startuml UC1
  !include ../../../commons/style/use-cases.pu
  left to right direction

  title Login SuperUtente

  actor :Utente non autenticato: as A1.1

  rectangle UC1{
      together {
          usecase (UC1.1) as "UC1.1\nAutenticazione"
          usecase (UC1.2) as "UC1.2\nVerifica credenziali"
          usecase (UC1.3) as "UC1.3\nVisualizzazione credenziali errate"
          note "Condition: credenziali errate" as N1
        }
    }

  UC1.1 .> UC1.2 : <<include>>
  UC1.3 . N1
  N1 .> UC1.1 : <<extends>>

  A1.1 -- UC1.1


  @enduml
\end{plantuml}


\begin{itemize}
  \item \textbf{Codice:} UC1;
  \item \textbf{Titolo:} Sistema di autenticazione;
  \item \textbf{Attori primari:} \glossario{utente non autenticato};
  \item \textbf{Precondizione:} l'\glossario{utente non autenticato} non è autenticato alla piattaforma;
  \item \textbf{Postcondizione:} l'\glossario{utente} ha effettuato correttamente il login nel sistema;
  \item \textbf{Scenario principale:} l'\glossario{utente non autenticato} non è ancora autenticato e vuole eseguire il login.
\end{itemize}
% subsub:UC1 (end)
\subsubsection{UC1.1 - Autenticazione}
\label{subsub:UC1.1}

\begin{plantuml}
@startuml UC1.1.1
!include style.pu
left to right direction

title Autenticazione Utente

actor :Utente non autenticato: as A1.1

rectangle UC1.1{
  together {
  usecase (UC1.1.1) as "UC1.1.1\nInserimento e-mail"
  usecase (UC1.1.2) as "UC1.1.2\nInserimento password"
  }
}
A1.1 -- UC1.1.1
A1.1 -- UC1.1.2

@enduml
\end{plantuml}


\begin{itemize}
  \item \textbf{Codice:} UC1.1;
  \item \textbf{Titolo:} Autenticazione;
  \item \textbf{Attori primari:} \glossario{utente non autenticato};
  \item \textbf{Precondizione:} il sistema è raggiungibile e funzionante, l'\glossario{utente non autenticato} deve poter visualizzare la pagina di login;
  \item \textbf{Postcondizione:} l'\glossario{utente non autenticato} ha inserito le possibili credenziali e sta tentando di effettuare il login. Ogni volta che cercherà di effettuare
        login sarà verificato che le credenziali inserite siano corrette. In caso contrario verrà visualizzato un messaggio di errore, e l'accesso sarà negato;
  \item \textbf{Scenario principale:} l'\glossario{utente non autenticato} accede alla pagina di login, e visualizza tutti i campi da compilare:
        \\a. Inserisce l’email associata all’account \emph{[UC1.1.1]};
        \\b. Inserisce la password associata all’account \emph{[UC1.1.2]}.
  \item Estensioni:
        \\a. Visualizzazione messaggio di credenziali errate \emph{[UC1.3]}.
  \item Inclusioni:
        \\a. Verifica credenziali \emph{[UC1.2]}.
  
\end{itemize}
% subsub:UC1.1 (end)

\subsubsection{UC1.1.1 - Inserimento username}
\label{subsub:UC1.1.1}

\begin{itemize}
  \item \textbf{Codice:} UC1.1.1;
  \item \textbf{Titolo:} Inserimento username;
  \item \textbf{Attori primari:} \glossario{utente non autenticato};
  \item \textbf{Descrizione:} al fine di portare a termine il processo di login di un \glossario{utente}, esso deve inserire il proprio username, campo obbligatorio;
  \item \textbf{Precondizione:} il sistema ha reso disponibile il campo per l'inserimento del proprio username;
  \item \textbf{Postcondizione:} l'\glossario{utente non autenticato} ha compilato il campo relativo al proprio username;
  \item \textbf{Scenario principale:} l'\glossario{utente non autenticato} compila il campo relativo al proprio username di registrazione.

\end{itemize}
% subsub:UC1.1.1 (end)

\subsubsection{UC1.1.2 - Inserimento password}
\label{subsub:UC1.1.2}

\begin{itemize}
  \item \textbf{Codice:} UC1.1.2;
  \item \textbf{Titolo:} Inserimento password;
  \item \textbf{Attori primari:} \glossario{utente non autenticato};
  \item \textbf{Descrizione:} al fine di portare a termine il processo di login di un \glossario{utente}, esso deve inserire la propria password, campo obbligatorio;
  \item \textbf{Precondizione:} il sistema ha reso disponibile il campo per l'inserimento della password;
  \item \textbf{Postcondizione:} l'\glossario{utente non autenticato} ha compilato il campo relativo alla sua password;
  \item \textbf{Scenario principale:} l'\glossario{utente non autenticato} compila il campo relativo alla sua password di registrazione.
  
\end{itemize}
% subsub:UC1.1.2 (end)


\subsubsection{UC1.2 - Verifica Credenziali}
\label{subsub:UC1.2}

\begin{itemize}
  \item \textbf{Codice:} UC1.2;
  \item \textbf{Titolo:} Verifica credenziali;
  \item \textbf{Attori primari:} \glossario{utente non autenticato};
  \item \textbf{Descrizione:} vengono verificate le credenziali immesse dall'\glossario{utente non autenticato};
  \item \textbf{Precondizione:} l'\glossario{utente non autenticato} ha inviato al server le sue credenziali per tentare il login;
  \item \textbf{Postcondizione:} l'\glossario{utente non autenticato} deve poter accedere alla sua area riservata, nel caso in cui le credenziali siano corrette. In caso
        contrario deve essere visualizzato un messaggio di credenziali sbagliate \emph{[UC1.3]};
  \item \textbf{Scenario principale:} l'\glossario{utente non autenticato} sta tentando di effettuare l'accesso e sta attendendo la verifica delle credenziali immesse.
\end{itemize}

% subsub:UC1.2 (end)
\subsubsection{UC1.3 - Visualizzazione credenziali errate}
\label{subsub:UC1.3}

\begin{itemize}
  \item \textbf{Codice:} UC1.3;
  \item \textbf{Titolo:} Visualizzazione credenziali errate;
  \item \textbf{Attori primari:} \glossario{utente non autenticato};
  \item \textbf{Descrizione:} viene visualizzato un errore di login;
  \item \textbf{Precondizione:} l'\glossario{utente non autenticato} ha inviato al server le sue credenziali per tentare il login, e le credenziali sono state verificate;
  \item \textbf{Postcondizione:} l'\glossario{utente non autenticato} visualizza un messaggio di credenziali sbagliate;
  \item \textbf{Scenario principale:} l'\glossario{utente non autenticato} cerca di effettuare il login con delle credenziali sbagliate.
  

\end{itemize}

% subsub:UC1.3 (end)
% par:utente-non-autenticato (end)


\paragraph{Utente autenticato}
\label{par:utente-autenticato}
Di seguito sono riportati tutti i casi d'uso che coinvolgono l'\glossario{utente autenticato}.

\subsubsection{UC2 - Richiesta owner}
\label{subsub:UC2}

\begin{itemize}
  \item \textbf{Codice:} UC2;
  \item \textbf{Titolo:} Richiesta owner;
  \item \textbf{Attori primari:} \glossario{utente autenticato};
  \item \textbf{Descrizione:} l'\glossario{utente autenticato} richiede il compito di \glossario{owner}, per poi richiedere di creare un'\glossario{organizzazione};
  \item \textbf{Precondizione:} l'\glossario{utente autenticato} non deve essere un \glossario{owner};
  \item \textbf{Postcondizione:} se viene approvata la richiesta, l'\glossario{utente autenticato} diventerà un \glossario{owner};
  \item \textbf{Scenario principale:} l'\glossario{utente autenticato} vuole iniziare ad utilizzare \glossario{Stalker} creando una sua \glossario{organizzazione};
  \item \textbf{Inclusioni}:
        \\a. Verifica tipologia utente \emph{[UC2.1]};
        \\b. Invio richiesta owner \emph{[UC2.2]}.
  \item \textbf{Estensioni}:
  \item \\a. Richiesta rifiutata \emph{[UC.2.3]}.

\end{itemize}
% subsub:UC2 (end)


\subsubsection{UC2.1 - Verifica tipologia utente}
\label{subsub:UC2.1}

\begin{itemize}
  \item \textbf{Codice:} UC2.1;
  \item \textbf{Titolo:} Verifica tipologia utente;
  \item \textbf{Attori primari:} \glossario{utente autenticato};
  \item \textbf{Precondizione:} l'\glossario{utente autenticato} visualizza l'apposito form per inviare la richiesta per diventare \glossario{owner};
  \item \textbf{Postcondizione:} il form relativo alla richiesta di \glossario{owner} viene abilitato o disabilitato di conseguenza;
  \item \textbf{Scenario principale:} l'\glossario{utente autenticato} può richiedere di diventare un \glossario{owner}, nel caso in cui sia:
  un \glossario{amministratore}, un \glossario{visualizzatore}, un \glossario{gestore} oppure un utente senza alcun privilegio. \\
  Ogni tentativo di richiesta da un già \glossario{owner} viene rifiutato.
  \glossario{Root} è già un \glossario{owner}.
  L'\glossario{amministratore} potrà accettare la propria richiesta inviata. 
  

\end{itemize}
% subsub:UC2.1 (end)

\subsubsection{UC2.2 - Invio richiesta owner}
\label{subsub:UC2.2}

\begin{itemize}
  \item \textbf{Codice:} UC2.2;
  \item \textbf{Titolo:} Invio richiesta owner;
  \item \textbf{Attori primari:} \glossario{utente autenticato};
  \item \textbf{Precondizione:} l'\glossario{utente autenticato} utilizza l'apposito form per inviare la richiesta per diventare \glossario{owner};
  \item \textbf{Postcondizione:} viene inviata la richiesta, in attesa di essere approvata da un \glossario{amministratore};
  \item \textbf{Scenario principale:} l'\glossario{utente autenticato} invia la richiesta per diventare \glossario{owner}.
\end{itemize}
% subsub:UC2.2 (end)


\subsubsection{UC2.3 - Richiesta rifiutata}
\label{subsub:UC2.3}

\begin{itemize}
  \item \textbf{Codice:} UC2.3;
  \item \textbf{Titolo:} Richiesta rifiutata;
  \item \textbf{Attori primari:} \glossario{utente autenticato};
  \item \textbf{Precondizione:} l'\glossario{utente autenticato} ha inviato la propria richiesta per diventare \glossario{owner}, l'\glossario{amministratore} ha rifiutato la richiesta;
  \item \textbf{Postcondizione:} l'\glossario{utente autenticato} non ha ottenuto l'abilitazione per un \glossario{owner}, visualizza un errore.
  \item \textbf{Scenario principale:} l'\glossario{amministratore} non ha accettato la richiesta posta dall'\glossario{utente autenticato}.
\end{itemize}
% subsub:UC2.2 (end)


\subsubsection{UC3 - Logout}
\label{subsub:UC3}

\begin{itemize}
  \item \textbf{Codice:} UC3;
  \item \textbf{Titolo:} Logout;
  \item \textbf{Attori primari:} \glossario{utente autenticato};
  \item \textbf{Precondizione:} il sistema ha reso disponibile la possibilità di effettuare il logout;
  \item \textbf{Postcondizione:} l'\glossario{utente autenticato} diventerà un \glossario{utente non autenticato}, effettuando il logout.
  \item \textbf{Scenario principale:} l'\glossario{utente autenticato} vuole effettuare il logout dalla piattaforma;
\end{itemize}
% subsub:UC3 (end)


% par:utente-autenticato (end)

\paragraph{Amministratore}
\label{par:amministratore}
Di seguito sono riportati tutti i casi d'uso che coinvolgono il \glossario{super utente} \glossario{amministratore}.

%inserire UML generale amministratore

\subsubsection{UC4 - Gestione richieste owner}
\label{subsub:UC4}

\begin{itemize}
  \item \textbf{Codice:} UC4;
  \item \textbf{Titolo:} Gestione richieste owner;
  \item \textbf{Attori primari:} \glossario{amministratore};
  \item \textbf{Precondizione:} l'\glossario{utente autenticato} deve aver richiesto di diventare un \glossario{owner};
  \item \textbf{Postcondizione:} viene approvata o rifiutata la richiesta, e l'\glossario{utente autenticato} sarà informato;
  \item \textbf{Scenario principale:} l'\glossario{amministratore} gestisce le richieste inviate dagli utenti autenticati di diventare \glossario{owner};
  \item \textbf{Inclusioni}:
        \\a. Accetta richiesta owner\emph{[UC4.1]};
        \\b. Rifiuta richiesta owner\emph{[UC4.2]}.
\end{itemize}
% subsub:UC4 (end)

\subsubsection{UC4.1 - Accetta richiesta owner}
\label{subsub:UC4.1}

\begin{itemize}
  \item \textbf{Codice:} UC4.1;
  \item \textbf{Titolo:} Accetta richiesta owner;
  \item \textbf{Attori primari:} \glossario{amministratore};
  \item \textbf{Precondizione:} l'\glossario{utente autenticato} deve aver richiesto di diventare un \glossario{owner};
  \item \textbf{Postcondizione:} viene approvata la richiesta da parte di un \glossario{amministratore};
  \item \textbf{Scenario principale:} l'\glossario{amministratore} accetta la richiesta inviata dall'\glossario{utente autenticato} di diventare \glossario{owner};
\end{itemize}
% subsub:UC4.1 (end)

\subsubsection{UC4.2 - Rifiuta richiesta owner}
\label{subsub:UC4.2}

\begin{itemize}
  \item \textbf{Codice:} UC4.2;
  \item \textbf{Titolo:} Accetta richiesta owner;
  \item \textbf{Attori primari:} \glossario{amministratore};
  \item \textbf{Precondizione:} l'\glossario{utente autenticato} deve aver richiesto di diventare un \glossario{owner};
  \item \textbf{Postcondizione:} viene rifiutata la richiesta da parte di un \glossario{amministratore};
  \item \textbf{Scenario principale:} l'\glossario{amministratore} rifiuta la richiesta inviata dall'\glossario{utente autenticato} di diventare \glossario{owner};
\end{itemize}
% subsub:UC4.2 (end)

\subsubsection{UC5 - Gestione richiesta creazione organizzazione}
\label{subsub:UC5}
\begin{itemize}
  \item \textbf{Codice:} UC5;
  \item \textbf{Titolo:} Gestione richiesta creazione organizzazione;
  \item \textbf{Attori primari:} \glossario{amministratore};
  \item \textbf{Precondizione:} l'\glossario{owner} deve aver richiesto di creare un organizzazione;
  \item \textbf{Postcondizione:} viene approvata o rifiutata la richiesta, e l'\glossario{owner} viene informato;
  \item \textbf{Scenario principale:} viene richiesta la creazione di una nuova \glossario{organizzazione}, che vuole utilizzare \glossario{Stalker}. 
  L'\glossario{amministratore} può accettare o rifiutare la richiesta.
  \item \textbf{Inclusioni}:
        \\a. Accetta richiesta creazione organizzazione\emph{[UC5.1]};
        \\b. Rifiuta richiesta creazione organizzazione\emph{[UC5.2]}.
\end{itemize}
% subsub:UC5 (end)

\subsubsection{UC5.1 - Accetta richiesta creazione organizzazione}
\label{subsub:UC5.1}
\begin{itemize}
  \item \textbf{Codice:} UC5.1;
  \item \textbf{Titolo:} Accetta richiesta creazione organizzazione;
  \item \textbf{Attori primari:} \glossario{amministratore};
  \item \textbf{Precondizione:} l'\glossario{owner} deve aver richiesto di creare un organizzazione;
  \item \textbf{Postcondizione:} viene approvata la richiesta, e l'\glossario{owner} viene informato;
  \item \textbf{Scenario principale:} l'\glossario{amministratore} accetta la richiesta inviata dall'\glossario{owner} di creare
  una nuova \glossario{organizzazione};
\end{itemize}
% subsub:UC5.1 (end)

\subsubsection{UC5.2 - Rifiuta richiesta creazione organizzazione}
\label{subsub:UC5.2}
\begin{itemize}
  \item \textbf{Codice:} UC5.2;
  \item \textbf{Titolo:} Rifiuta richiesta creazione organizzazione;
  \item \textbf{Attori primari:} \glossario{amministratore};
  \item \textbf{Precondizione:} l'\glossario{owner} deve aver richiesto di creare un organizzazione;
  \item \textbf{Postcondizione:} viene rifiutata la richiesta, e l'\glossario{owner} viene informato;
  \item \textbf{Scenario principale:} l'\glossario{amministratore} rifiuta la richiesta inviata dall'\glossario{owner} di creare
  una nuova \glossario{organizzazione};
\end{itemize}
% subsub:UC5.2 (end)

\subsubsection{UC6 - Gestione richiesta modifica organizzazione}
\label{subsub:UC6}

\begin{itemize}
  \item \textbf{Codice:} UC6;
  \item \textbf{Titolo:} Gestione richiesta modifica organizzazione;
  \item \textbf{Attori primari:} \glossario{amministratore};
  \item \textbf{Precondizione:} un \glossario{gestore} deve aver richiesto di modificare un'organizzazione;
  \item \textbf{Postcondizione:} viene approvata o rifiutata la richiesta, e il \glossario{gestore} viene informato;
  \item \textbf{Scenario principale:} viene richiesta la modifica di una \glossario{organizzazione}. L'\glossario{amministratore} può accettare o rifiutare la richiesta.
  \item \textbf{Inclusioni}:
        \\a. Accetta richiesta modifica organizzazione\emph{[UC6.1]};
        \\b. Rifiuta richiesta modifica organizzazione\emph{[UC6.2]}.
\end{itemize}
% subsub:UC6 (end)


\subsubsection{UC6.1 - Accetta richiesta modifica organizzazione}
\label{subsub:UC6.1}
\begin{itemize}
  \item \textbf{Codice:} UC6.1;
  \item \textbf{Titolo:} Accetta richiesta modifica organizzazione;
  \item \textbf{Attori primari:} \glossario{amministratore};
  \item \textbf{Precondizione:} il \glossario{gestore} deve aver richiesto di modificare un'organizzazione;
  \item \textbf{Postcondizione:} viene approvata la richiesta, e il \glossario{gestore} viene informato;
  \item \textbf{Scenario principale:} l'\glossario{amministratore} accetta la richiesta inviata dal \glossario{gestore} di modificare
  un'\glossario{organizzazione};
\end{itemize}
% subsub:UC6.1 (end)

\subsubsection{UC6.2 - Rifiuta richiesta modifica organizzazione}
\label{subsub:UC6.2}
\begin{itemize}
  \item \textbf{Codice:} UC6.2;
  \item \textbf{Titolo:} Rifiuta richiesta modifica organizzazione;
  \item \textbf{Attori primari:} \glossario{amministratore};
  \item \textbf{Precondizione:} il \glossario{gestore} deve aver richiesto di modificare un'organizzazione;
  \item \textbf{Postcondizione:} viene rifiutata la richiesta, e il \glossario{gestore} viene informato;
  \item \textbf{Scenario principale:} l'\glossario{amministratore} rifiuta la richiesta inviata dal \glossario{gestore} di modificare
  un'\glossario{organizzazione};
\end{itemize}
% subsub:UC6.2 (end)


\subsubsection{UC7 - Gestione richiesta aggiunta luogo}
\label{subsub:UC7}

\begin{itemize}
  \item \textbf{Codice:} UC7;
  \item \textbf{Titolo:} Gestione richiesta aggiunta luogo;
  \item \textbf{Attori primari:} \glossario{amministratore};
  \item \textbf{Precondizione:} un \glossario{gestore} deve aver richiesto di aggiungere un luogo;
  \item \textbf{Postcondizione:} viene approvata o rifiutata la richiesta, e il \glossario{gestore} viene informato;
  \item \textbf{Scenario principale:} viene richiesto di aggiungere un luogo. L'\glossario{amministratore} può accettare o rifiutare la richiesta
  \item \textbf{Inclusioni}:
        \\a. Accetta richiesta aggiunta luogo\emph{[UC7.1]};
        \\b. Rifiuta richiesta aggiunta luogo\emph{[UC7.2]}.

\end{itemize}
% subsub:UC7 (end)


\subsubsection{UC7.1 - Accetta richiesta aggiungi luogo}
\label{subsub:UC7.1}
\begin{itemize}
  \item \textbf{Codice:} UC7.1;
  \item \textbf{Titolo:} Accetta richiesta aggiungi luogo;
  \item \textbf{Attori primari:} \glossario{amministratore};
  \item \textbf{Precondizione:} il \glossario{gestore} deve aver richiesto di aggiungere un luogo;
  \item \textbf{Postcondizione:} viene approvata la richiesta, e il \glossario{gestore} viene informato;
  \item \textbf{Scenario principale:} l'\glossario{amministratore} accetta la richiesta inviata dal \glossario{gestore} di aggiungere un luogo.
\end{itemize}
% subsub:UC7.1 (end)

\subsubsection{UC7.2 - Rifiuta richiesta aggiungi luogo}
\label{subsub:UC7.2}
\begin{itemize}
  \item \textbf{Codice:} UC7.2;
  \item \textbf{Titolo:} Rifiuta richiesta aggiungi luogo;
  \item \textbf{Attori primari:} \glossario{amministratore};
  \item \textbf{Precondizione:} il \glossario{gestore} deve aver richiesto di aggiungere un luogo;
  \item \textbf{Postcondizione:} viene rifiutata la richiesta, e il \glossario{gestore} viene informato;
  \item \textbf{Scenario principale:} l'\glossario{amministratore} rifiuta la richiesta inviata dal \glossario{gestore} di aggiungere un luogo.
\end{itemize}
% subsub:UC7.2 (end)


\subsubsection{UC6 - Approvazione modifica luogo}
\label{subsub:UC6}

\begin{itemize}
  \item \textbf{Attori primari:} \glossario{amministratore};
  \item \textbf{Descrizione:} l'\glossario{amministratore} approva la richiesta di una modifica di un luogo all'interno di un'\glossario{organizzazione};
  \item \textbf{Scenario principale:} viene richiesta una modifica di un luogo all'interno di un'\glossario{organizzazione};
  \item \textbf{Precondizione:} l'\glossario{organizzazione} e il luogo devono essere già stati creati dal sistema, e deve essere stata fatta la richiesta di modifica luogo;
  \item \textbf{Postcondizione:} viene approvata la richiesta, e viene modificato il luogo dell'\glossario{organizzazione} interessata.

\end{itemize}
% subsub:UC6 (end)
\subsubsection{UC7 - Approvazione trasferimento di proprietà organizzazione}
\label{subsub:UC7}

\begin{itemize}
  \item \textbf{Attori primari:} \glossario{amministratore};
  \item \textbf{Descrizione:} l'\glossario{amministratore} approva la richiesta di un trasferimento di proprietà di un'\glossario{organizzazione};
  \item \textbf{Scenario principale:} viene richiesto un traferimento di proprietà di un'\glossario{organizzazione};
  \item \textbf{Precondizione:} l'\glossario{organizzazione} deve essere già stata creata dal sistema, e deve essere stata fatta la richiesta trasferimento di proprietà;
  \item \textbf{Postcondizione:} viene approvata la richiesta, e viene trasferita la proprietà dell'\glossario{organizzazione} interessata.

\end{itemize}
% subsub:UC7 (end)





% par:Amministratore (end)
\paragraph{Root}
Di seguito sono riportati tutti i casi d'uso che coinvolgono il \glossario{super utente} \glossario{root}.

\subsubsection{UC8 - Creazione amministratore}
\label{subsub:UC8}

\begin{itemize}
  \item \textbf{Attori primari:} \glossario{root};
  \item \textbf{Descrizione:} il \glossario{root} crea un \glossario{super utente} \glossario{amministratore};
  \item \textbf{Scenario principale:} sorge la necessità di creare un nuovo \glossario{amministratore} per gestire \glossario{Stalker};
  \item \textbf{Precondizione:} devono essere specificate le credenziali del nuovo \glossario{amministratore}, che devono essere univoche;
  \item \textbf{Postcondizione:} l'\glossario{amministratore} viene creato.

\end{itemize}
% subsub:UC8 (end)

\subsubsection{UC9 - Eliminazione organizzazione}
\label{subsub:UC9}

\begin{itemize}
  \item \textbf{Attori primari:} \glossario{root};
  \item \textbf{Descrizione:} il \glossario{root} elimina un'\glossario{organizzazione};
  \item \textbf{Scenario principale:} sorge la necessità di eliminare un'\glossario{organizzazione}, senza interagire con il suo \glossario{owner};
  \item \textbf{Precondizione:} deve essere stata selezionata l'\glossario{organizzazione} da eliminare, presente nella lista di \glossario{Stalker};
  \item \textbf{Postcondizione:} l'\glossario{organizzazione} viene eliminata.

\end{itemize}
% subsub:UC9 (end)


\subsubsection{UC10 - Modifica organizzazione}
\label{subsub:UC10}

\begin{itemize}
  \item \textbf{Attori primari:} \glossario{root};
  \item \textbf{Descrizione:} il \glossario{root} modifica un'\glossario{organizzazione};
  \item \textbf{Scenario principale:} sorge la necessità di modificare un'\glossario{organizzazione}, senza interagire con il suo \glossario{owner};
  \item \textbf{Precondizione:} deve essere stata selezionata l'\glossario{organizzazione} da modificare, presente nella lista di \glossario{Stalker}, effettivamente modificata;
  \item \textbf{Postcondizione:} l'\glossario{organizzazione} viene modificata.

\end{itemize}
% subsub:UC10 (end)


\subsubsection{UC11 - Creazione organizzazione}
\label{subsub:UC11}

\begin{itemize}
  \item \textbf{Attori primari:} \glossario{root};
  \item \textbf{Descrizione:} il \glossario{root} crea un'\glossario{organizzazione};
  \item \textbf{Scenario principale:} sorge la necessità di creare un'\glossario{organizzazione}, senza essere effettivamente richiesta;
  \item \textbf{Precondizione:} l'\glossario{organizzazione} non deve esistere nella lista di \glossario{Stalker}, deve essere specificato il suo nome;
  \item \textbf{Postcondizione:} l'\glossario{organizzazione} viene creata.

\end{itemize}
% subsub:UC11 (end)

\subsubsection{UC12 - Eliminazione luogo}
\label{subsub:UC12}

\begin{itemize}
  \item \textbf{Attori primari:} \glossario{root};
  \item \textbf{Descrizione:} il \glossario{root} elimina un \glossario{luogo} di un'\glossario{organizzazione};
  \item \textbf{Scenario principale:} sorge la necessità di elimanare un \glossario{luogo} di un'\glossario{organizzazione}, senza interagire con il suo \glossario{owner};
  \item \textbf{Precondizione:} il \glossario{luogo} dell'\glossario{organizzazione} deve essere presente in \glossario{Stalker};
  \item \textbf{Postcondizione:} il \glossario{luogo} dell'\glossario{organizzazione} viene eliminato.

\end{itemize}
% subsub:UC12 (end)

\subsubsection{UC13 - Modifica luogo}
\label{subsub:UC13}

\begin{itemize}
  \item \textbf{Attori primari:} \glossario{root};
  \item \textbf{Descrizione:} il \glossario{root} modifica un \glossario{luogo} di un'\glossario{organizzazione};
  \item \textbf{Scenario principale:} sorge la necessità di modificare un \glossario{luogo} di un'\glossario{organizzazione}, senza interagire con il suo \glossario{owner};
  \item \textbf{Precondizione:} il \glossario{luogo} dell'\glossario{organizzazione} deve essere presente in \glossario{Stalker};
  \item \textbf{Postcondizione:} il \glossario{luogo} dell'\glossario{organizzazione} viene modificato.

\end{itemize}
% subsub:UC13 (end)

\subsubsection{UC14 - Eliminazione account}
\label{subsub:UC14}

\begin{itemize}
  \item \textbf{Attori primari:} \glossario{root};
  \item \textbf{Descrizione:} il \glossario{root} elimina un qualsiasi \glossario{account} registrato in \glossario{Stalker}. Quest'ultimo può essere un \glossario{utente} oppure un \glossario{super utente};
  \item \textbf{Scenario principale:} sorge la necessità di eliminare un \glossario{account} per sconosciuti motivi;
  \item \textbf{Precondizione:} deve essere stato selezionato l'\glossario{account} da eliminare, che deve esistere in \glossario{Stalker};
  \item \textbf{Postcondizione:} l'\glossario{account} selezionato è stato eliminato.

\end{itemize}
% subsub:UC14 (end)


% par:Root (end)

\paragraph{Visualizzatore}
Di seguito sono riportati tutti i casi d'uso che coinvolgono il \glossario{super utente} \glossario{Visualizzatore}.


\subsubsection{UC15 - Query sull'organizzazione}
\label{subsub:UC15}

\begin{itemize}
  \item \textbf{Attori primari:} \glossario{visualizzatore};
  \item \textbf{Descrizione:} il \glossario{visualizzatore} effettua delle interrogazioni per trarre informazioni relative all'\glossario{organizzazione} su cui opera;
  \item \textbf{Scenario principale:} il \glossario{visualizzatore} vuole avere delle informazioni riguardanti l'\glossario{organizzazione} su cui opera;
  \item \textbf{Precondizione:} il sistema risponde correttamente alle interrogazioni;
  \item \textbf{Postcondizione:} il \glossario{visualizzatore} ottiene le informazioni di cui ha bisogno.

\end{itemize}
% subsub:UC15 (end)


\subsubsection{UC16 - Query sul dipendente}
\label{subsub:UC16}

\begin{itemize}
  \item \textbf{Attori primari:} \glossario{visualizzatore};
  \item \textbf{Descrizione:} il \glossario{visualizzatore} effettua delle interrogazioni per trarre informazioni relative al dipendente dell'\glossario{organizzazione} su cui opera;
  \item \textbf{Scenario principale:} il \glossario{visualizzatore} vuole avere delle informazioni riguardanti il\glossario{dipendente} dell'\glossario{organizzazione} su cui opera;
  \item \textbf{Precondizione:} il sistema risponde correttamente alle interrogazioni;
  \item \textbf{Postcondizione:} il \glossario{visualizzatore} ottiene le informazioni di cui ha bisogno.

\end{itemize}
% subsub:UC16 (end)
% par:Visualizzatore (end)

\paragraph{Gestore}
Di seguito sono riportati tutti i casi d'uso che coinvolgono il \glossario{super utente} \glossario{Gestore}.

\subsubsection{UC17 - Richiesta aggiunta luogo}
\label{subsub:UC17}

\begin{itemize}
  \item \textbf{Attori primari:} \glossario{gestore};
  \item \textbf{Descrizione:} il \glossario{gestore} dell'\glossario{organizzazione} richiede di aggiungere un nuovo \glossario{luogo};
  \item \textbf{Scenario principale:} il \glossario{gestore} vuole aggiungere un nuovo \glossario{luogo} all'\glossario{organizzazione} su cui opera;
  \item \textbf{Precondizione:} il \glossario{luogo} da aggiungere non deve già esistere;
  \item \textbf{Postcondizione:} la richiesta di aggiunta di un nuovo \glossario{luogo} viene posta.

\end{itemize}
% subsub:UC17 (end)

\subsubsection{UC18 - Richiesta rimozione luogo}
\label{subsub:UC18}

\begin{itemize}
  \item \textbf{Attori primari:} \glossario{gestore};
  \item \textbf{Descrizione:} il \glossario{gestore} dell'\glossario{organizzazione} richiede di eliminare un \glossario{luogo};
  \item \textbf{Scenario principale:} il \glossario{gestore} vuole eliminare un \glossario{luogo} all'\glossario{organizzazione} su cui opera;
  \item \textbf{Precondizione:} il \glossario{luogo} da eliminare deve esistere;
  \item \textbf{Postcondizione:} la richiesta di eliminazione di un \glossario{luogo} viene posta.

\end{itemize}
% subsub:UC18 (end)

\subsubsection{UC19 - Richiesta modifica luogo}
\label{subsub:UC19}

\begin{itemize}
  \item \textbf{Attori primari:} \glossario{gestore};
  \item \textbf{Descrizione:} il \glossario{gestore} dell'\glossario{organizzazione} richiede di modificare un \glossario{luogo};
  \item \textbf{Scenario principale:} il \glossario{gestore} vuole modificare un \glossario{luogo} dell'\glossario{organizzazione} su cui opera;
  \item \textbf{Precondizione:} il \glossario{luogo} da modificare deve esistere;
  \item \textbf{Postcondizione:} la richiesta di modifica di un \glossario{luogo} viene posta.

\end{itemize}
% subsub:UC19 (end)


\subsubsection{UC20 - Richiesta modifica parametri organizzazione}
\label{subsub:UC20}

\begin{itemize}
  \item \textbf{Attori primari:} \glossario{gestore};
  \item \textbf{Descrizione:} il \glossario{gestore} dell'\glossario{organizzazione} richiede di modificare i parametri di essa;
  \item \textbf{Scenario principale:} il \glossario{gestore} vuole modificare i parametri dell'\glossario{organizzazione} su cui opera;
  \item \textbf{Precondizione:} i parametri devono essere effettivamente modificati;
  \item \textbf{Postcondizione:} la richiesta di modifica parametri viene posta.

\end{itemize}
% subsub:UC20 (end)
% par:Gestore (end)

\paragraph{Owner}
Di seguito sono riportati tutti i casi d'uso che coinvolgono il \glossario{super utente} \glossario{Owner}.

\subsubsection{UC21 - Nomina visualizzatore}
\label{subsub:UC21}

\begin{itemize}
  \item \textbf{Attori primari:} \glossario{owner};
  \item \textbf{Descrizione:} l' \glossario{owner} nomina un \glossario{visualizzatore} per la sua \glossario{organizzazione};
  \item \textbf{Scenario principale:} l' \glossario{owner} vuole aggiungere un \glossario{visualizzatore} alla sua \glossario{organizzazione};
  \item \textbf{Precondizione:} il nuovo \glossario{super utente} non deve già esistere come \glossario{visualizzatore};
  \item \textbf{Postcondizione:} il nuovo \glossario{visualizzatore} è stato aggiunto.

\end{itemize}
% subsub:UC21 (end)


\subsubsection{UC22 - Nomina gestore}
\label{subsub:UC22}

\begin{itemize}
  \item \textbf{Attori primari:} \glossario{owner};
  \item \textbf{Descrizione:} l' \glossario{owner} nomina un \glossario{gestore} per la sua \glossario{organizzazione};
  \item \textbf{Scenario principale:} l' \glossario{owner} vuole aggiungere un \glossario{gestore} alla sua \glossario{organizzazione};
  \item \textbf{Precondizione:} il nuovo \glossario{super utente} non deve già esistere come \glossario{gestore};
  \item \textbf{Postcondizione:} il nuovo \glossario{gestore} è stato aggiunto.

\end{itemize}
% subsub:UC22 (end)


\subsubsection{UC23 - Nomina gestore}
\label{subsub:UC23}

\begin{itemize}
  \item \textbf{Attori primari:} \glossario{owner};
  \item \textbf{Descrizione:} l' \glossario{owner} nomina un \glossario{gestore} per la sua \glossario{organizzazione};
  \item \textbf{Scenario principale:} l' \glossario{owner} vuole aggiungere un \glossario{gestore} alla sua \glossario{organizzazione};
  \item \textbf{Precondizione:} il nuovo \glossario{super utente} non deve già esistere come \glossario{gestore};
  \item \textbf{Postcondizione:} il nuovo \glossario{gestore} è stato aggiunto.

\end{itemize}
% subsub:UC23 (end)


\subsubsection{UC24 - Richiesta creazione organizzazione}
\label{subsub:UC24}

\begin{itemize}
  \item \textbf{Attori primari:} \glossario{owner};
  \item \textbf{Descrizione:} l' \glossario{owner} richiede di creare una nuova \glossario{organizzazione};
  \item \textbf{Scenario principale:} l' \glossario{owner} vuole creare una nuova \glossario{organizzazione};
  \item \textbf{Precondizione:} l' \glossario{organizzazione} non deve già esistere;
  \item \textbf{Postcondizione:} la richiesta di creare una nuova \glossario{organizzazione} è stata posta.

\end{itemize}
% subsub:UC24 (end)

\subsubsection{UC25 - Richiesta cedimento proprietà organizzazione}
\label{subsub:UC25}

\begin{itemize}
  \item \textbf{Attori primari:} \glossario{owner};
  \item \textbf{\textbf{Descrizione:}} l' \glossario{owner} richiede che l'\glossario{organizzazione} venga ceduta ad un altro futuro \glossario{owner};
  \item \textbf{\textbf{Scenario principale:}} l' \glossario{owner} vuole cedere l'\glossario{organizzazione} ad un'altro \glossario{owner};
  \item \textbf{Precondizione:} il futuro \glossario{owner} deve esistere;
  \item \textbf{Postcondizione:} la richiesta di cedere l'\glossario{organizzazione} è stata posta.

\end{itemize}
% subsub:UC25 (end)

% par:Owner (end)

\end{document}
