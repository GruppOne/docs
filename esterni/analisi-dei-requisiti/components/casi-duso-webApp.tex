\documentclass[casi-duso]{subfiles}

\renewcommand{\commons}{../../../commons}

\begin{document}

\paragraph{Utente non autenticato}
\label{par:utente-non-autenticato}
Di seguito sono riportati tutti i casi d'uso che coinvolgono l'\glossario{utente non autenticato}.

\subsubsection{AUC1 - Sistema di autenticazione}
\label{subsub:AUC1}
%inserire diagramma AUC1
%\begin{plantuml}
%  @startuml AUC1
%  !include ../../../commons/style/use-cases.pu
%  left to right direction

%  title Login SuperUtente

%  actor :Utente non autenticato: as A1.1

%  rectangle AUC1{
%      together {
%          usecase (AUC1.1) as "AUC1.1\nAutenticazione"
%          usecase (AUC1.2) as "AUC1.2\nVerifica credenziali"
%          usecase (AUC1.3) as "AUC1.3\nVisualizzazione credenziali errate"
%          note "Condition: credenziali errate" as N1
%        }
%    }

%  AUC1.1 .> AUC1.2 : <<include>>
%  AUC1.3 . N1
%  N1 .> AUC1.1 : <<extends>>

%  A1.1 -- AUC1.1

%  @enduml
%\end{plantuml}
\begin{itemize}
  \item \textbf{Codice:} AUC1
  \item \textbf{Titolo:} sistema di autenticazione
  \item \textbf{Attori primari:} utente non autenticato
  \item \textbf{Precondizione:} l'\glossario{utente non autenticato} non è autenticato alla piattaforma.
  \item \textbf{Postcondizione:} l'\glossario{utente} ha effettuato correttamente il login nel sistema.
  \item \textbf{Scenario principale:}
  \begin{enumerate}
    \item l'\glossario{utente non autenticato} non è ancora autenticato e vuole eseguire il login.
  \end{enumerate}
\end{itemize}
% subsub:AUC1 (end)

\subsubsection{AUC1.1 - Autenticazione}
\label{subsub:AUC1.1}
%inserire diagramma AUC1.1
%\begin{plantuml}
%@startuml AUC1.1
%!include style.pu
%left to right direction

%title Autenticazione Utente

%actor :Utente non autenticato: as A1.1

%rectangle AUC1.1{
%  together {
%  usecase (AUC1.1.1) as "AUC1.1.1\nInserimento e-mail"
%  usecase (AUC1.1.2) as "AUC1.1.2\nInserimento password"
%  }
%}
%A1.1 -- AUC1.1.1
%A1.1 -- AUC1.1.2

%@enduml
%\end{plantuml}
\begin{itemize}
  \item \textbf{Codice:} AUC1.1
  \item \textbf{Titolo:} autenticazione
  \item \textbf{Attori primari:} utente non autenticato
  \item \textbf{Precondizione:} il sistema è raggiungibile e funzionante, l'\glossario{utente non autenticato} deve poter visualizzare la pagina di login.
  \item \textbf{Postcondizione:} l'\glossario{utente non autenticato} ha inserito le possibili credenziali e sta tentando di effettuare l'autenticazione. Ogni volta che cercherà di effettuare
        login sarà verificato che le credenziali inserite siano corrette. In caso contrario verrà visualizzato un messaggio di errore, e l'accesso sarà negato.
  \item \textbf{Scenario principale:} 
  \begin{enumerate}
    \item  l'\glossario{utente non autenticato} accede alla pagina di login, e visualizza tutti i campi da compilare:
    \begin{enumerate}
      \item inserisce l’email associata all’account \emph{[AUC1.1.1]};
      \item inserisce la password associata all’account \emph{[AUC1.1.2]}.
    \end{enumerate}
  \end{enumerate}
  \item \textbf{Inclusioni:}
  \begin{enumerate}
    \item Verifica credenziali \emph{[AUC1.2]}.
  \end{enumerate}
  \item \textbf{Estensioni:}
  \begin{enumerate}
    \item Visualizzazione messaggio di credenziali errate \emph{[AUC1.3]}.
  \end{enumerate}
\end{itemize}
% subsub:AUC1.1 (end)

\subsubsection{AUC1.1.1 - Inserimento e-mail}
\label{subsub:AUC1.1.1}
\begin{itemize}
  \item \textbf{Codice:} AUC1.1.1
  \item \textbf{Titolo:} inserimento e-mail
  \item \textbf{Attori primari:} utente non autenticato
  \item \textbf{Precondizione:} il sistema ha reso disponibile il campo per l'inserimento della propria e-mail.
  \item \textbf{Postcondizione:} l'\glossario{utente non autenticato} ha compilato il campo relativo alla propria e-mail.
  \item \textbf{Scenario principale:} 
  \begin{enumerate}
    \item l'\glossario{utente non autenticato} compila il campo relativo al proprio username di registrazione.  
  \end{enumerate}
\end{itemize}
% subsub:AUC1.1.1 (end)

\subsubsection{AUC1.1.2 - Inserimento password}
\label{subsub:AUC1.1.2}
\begin{itemize}
  \item \textbf{Codice:} AUC1.1.2
  \item \textbf{Titolo:} inserimento password
  \item \textbf{Attori primari:} utente non autenticato
  \item \textbf{Precondizione:} il sistema ha reso disponibile il campo per l'inserimento della password.
  \item \textbf{Postcondizione:} l'\glossario{utente non autenticato} ha compilato il campo relativo alla sua password.
  \item \textbf{Scenario principale:} 
  \begin{enumerate}
    \item l'\glossario{utente non autenticato} compila il campo relativo alla sua password di registrazione.
  \end{enumerate}
\end{itemize}
% subsub:AUC1.1.2 (end)

\subsubsection{AUC1.2 - Verifica Credenziali}
\label{subsub:AUC1.2}
\begin{itemize}
  \item \textbf{Codice:} AUC1.2
  \item \textbf{Titolo:} verifica credenziali
  \item \textbf{Attori primari:} utente non autenticato
  \item \textbf{Precondizione:} l'\glossario{utente non autenticato} ha inviato al server le sue credenziali per tentare il login
  \item \textbf{Postcondizione:} l'\glossario{utente non autenticato} deve poter accedere alla sua area riservata, nel caso in cui le credenziali siano corrette. In caso
  contrario deve essere visualizzato un messaggio di credenziali sbagliate \emph{[AUC1.3]}.
  \item \textbf{Scenario principale:} 
  \begin{enumerate}
    \item l'\glossario{utente non autenticato} sta tentando di effettuare l'accesso e sta attendendo la verifica delle credenziali immesse.
  \end{enumerate}
\end{itemize}
% subsub:AUC1.2 (end)

\subsubsection{AUC1.3 - Visualizzazione credenziali errate}
\label{subsub:AUC1.3}
\begin{itemize}
  \item \textbf{Codice:} AUC1.3
  \item \textbf{Titolo:} visualizzazione credenziali errate
  \item \textbf{Attori primari:} utente non autenticato
  \item \textbf{Precondizione:} l'\glossario{utente non autenticato} ha inviato al server le sue credenziali per tentare il login, e le credenziali sono state verificate.
  \item \textbf{Postcondizione:} l'\glossario{utente non autenticato} visualizza un messaggio di credenziali sbagliate.
  \item \textbf{Scenario principale:} 
  \begin{enumerate}
    \item l'\glossario{utente non autenticato} cerca di effettuare il login con delle credenziali sbagliate. 
  \end{enumerate}
\end{itemize}
% subsub:AUC1.3 (end)
% par:utente-non-autenticato (end)

\paragraph{Utente autenticato}
\label{par:utente-autenticato}
Di seguito sono riportati tutti i casi d'uso che coinvolgono l'\glossario{utente autenticato}.

\subsubsection{AUC2 - Richiesta owner}
\label{subsub:AUC2}
\begin{itemize}
  \item \textbf{Codice:} AUC2
  \item \textbf{Titolo:} richiesta owner
  \item \textbf{Attori primari:} utente autenticato
  \item \textbf{Precondizione:} l'\glossario{utente autenticato} non deve essere un \glossario{owner}.
  \item \textbf{Postcondizione:} se viene approvata la richiesta, l'\glossario{utente autenticato} diventerà un \glossario{owner}.
  \item \textbf{Scenario principale:} 
  \begin{enumerate}
    \item l'\glossario{utente autenticato} vuole iniziare ad utilizzare \glossario{Stalker} creando una sua \glossario{organizzazione}.
  \end{enumerate}
\end{itemize}
% subsub:AUC2 (end)
% par:utente-autenticato (end)

\subsubsection{AUC2.1 - Verifica tipologia utente}
\label{subsub:AUC2.1}
\begin{itemize}
  \item \textbf{Codice:} AUC2.1
  \item \textbf{Titolo:} verifica tipologia utente
  \item \textbf{Attori primari:} utente autenticato
  \item \textbf{Precondizione:} l'\glossario{utente autenticato} visualizza l'apposito form per inviare la richiesta per diventare \glossario{owner}.
  \item \textbf{Postcondizione:} il form relativo alla richiesta di \glossario{owner} viene abilitato o disabilitato di conseguenza.
  \item \textbf{Scenario principale:} 
  \begin{enumerate}
    \item l'\glossario{utente autenticato} può richiedere di diventare un \glossario{owner}, nel caso in cui sia:
    \begin{itemize}
      \item un \glossario{amministratore},
      \item un \glossario{visualizzatore},
      \item un \glossario{gestore} oppure un utente senza alcun privilegio;
    \end{itemize}
    \item ogni tentativo di richiesta da un già \glossario{owner} viene rifiutato;
    \item \glossario{root} è già un \glossario{owner};
    \item l'\glossario{amministratore} potrà accettare la propria richiesta inviata.
  \end{enumerate}
\end{itemize}
% subsub:AUC2.1 (end)

\subsubsection{AUC2.2 - Invio richiesta owner}
\label{subsub:AUC2.2}
\begin{itemize}
  \item \textbf{Codice:} AUC2.2
  \item \textbf{Titolo:} invio richiesta owner
  \item \textbf{Attori primari:} utente autenticato
  \item \textbf{Precondizione:} l'\glossario{utente autenticato} utilizza l'apposito form per inviare la richiesta per diventare \glossario{owner}.
  \item \textbf{Postcondizione:} viene inviata la richiesta, in attesa di essere approvata da un \glossario{amministratore}.
  \item \textbf{Scenario principale:} 
  \begin{enumerate}
    \item l'\glossario{utente autenticato} invia la richiesta per diventare \glossario{owner}.
  \end{enumerate}
  \item \textbf{Inclusioni}:
  \begin{enumerate}
    \item Verifica tipologia utente \emph{[AUC2.1]};
  \end{enumerate}
  \item \textbf{Estensioni}:
  \begin{enumerate}
    \item Richiesta rifiutata \emph{[AUC.2.3]}.
  \end{enumerate}
\end{itemize}
% subsub:AUC2.2 (end)

\subsubsection{AUC2.3 - Richiesta rifiutata}
\label{subsub:AUC2.3}
\begin{itemize}
  \item \textbf{Codice:} AUC2.3
  \item \textbf{Titolo:} richiesta rifiutata
  \item \textbf{Attori primari:} utente autenticato
  \item \textbf{Precondizione:} l'\glossario{utente autenticato} ha inviato la propria richiesta per diventare \glossario{owner}, l'\glossario{amministratore} ha rifiutato la richiesta.
  \item \textbf{Postcondizione:} l'\glossario{utente autenticato} non ha ottenuto l'abilitazione per un \glossario{owner}, visualizza un errore.
  \item \textbf{Scenario principale:} 
  \begin{enumerate}
    \item l'\glossario{amministratore} non ha accettato la richiesta posta dall'\glossario{utente autenticato}.
  \end{enumerate}
\end{itemize}
% subsub:AUC2.2 (end)

\subsubsection{AUC3 - Disconnessione}
\label{subsub:AUC3}
\begin{itemize}
  \item \textbf{Codice:} AUC3
  \item \textbf{Titolo:} disconnessione
  \item \textbf{Attori primari:} utente autenticato
  \item \textbf{Precondizione:} il sistema ha reso disponibile la possibilità di effettuare la disconnessione.
  \item \textbf{Postcondizione:} l'\glossario{utente autenticato} diventerà un \glossario{utente non autenticato}, effettuando la disconnessione.
  \item \textbf{Scenario principale:} 
  \begin{enumerate}
    \item l'\glossario{utente autenticato} vuole effettuare il logout dalla piattaforma;
  \end{enumerate}
\end{itemize}
% subsub:AUC3 (end)
% par:utente-autenticato (end)

\paragraph{Amministratore}
\label{par:amministratore}
Di seguito sono riportati tutti i casi d'uso che coinvolgono il \glossario{super utente} \glossario{amministratore}.

%inserire UML generale amministratore


\subsubsection{AUC4 - Gestione richieste}
\label{subsub:AUC4}
\begin{itemize}
  \item \textbf{Codice:} AUC4
  \item \textbf{Titolo:} gestione richieste
  \item \textbf{Attori primari:} amministratore
  \item \textbf{Precondizione:} il sistema deve rendere disponibili le richieste effettuate dai \glossario{super utenti}.
  \item \textbf{Postcondizione:} vengono approvate o rifiutate le richieste.
  \item \textbf{Scenario principale:} 
  \begin{enumerate}
    \item l'\glossario{amministratore} gestisce le richieste.
    \item le richieste vengono accettate o rifiutate.
  \end{enumerate}
\end{itemize}

\subsubsection{AUC4.1 - Gestione richieste owner}
\label{subsub:AUC4.1}
\begin{itemize}
  \item \textbf{Codice:} AUC4.1
  \item \textbf{Titolo:} gestione richieste owner
  \item \textbf{Attori primari:} amministratore
  \item \textbf{Precondizione:} l'\glossario{utente autenticato} deve aver richiesto di diventare un \glossario{owner}.
  \item \textbf{Postcondizione:} viene approvata o rifiutata la richiesta, e l'\glossario{utente autenticato} sarà informato.
  \item \textbf{Scenario principale:} 
  \begin{enumerate}
    \item l'\glossario{amministratore} gestisce le richieste inviate dagli utenti autenticati di diventare \glossario{owner};
    \item le richieste vengono accettate o rifiutate.
  \end{enumerate}
\end{itemize}
% subsub:AUC4.1 (end)

\subsubsection{AUC4.1.1 - Accetta richiesta owner}
\label{subsub:AUC4.1.1}
\begin{itemize}
  \item \textbf{Codice:} AUC4.1.1
  \item \textbf{Titolo:} accetta richiesta owner
  \item \textbf{Attori primari:} amministratore
  \item \textbf{Precondizione:} l'\glossario{utente autenticato} deve aver richiesto di diventare un \glossario{owner}.
  \item \textbf{Postcondizione:} viene approvata la richiesta da parte di un \glossario{amministratore}.
  \item \textbf{Scenario principale:} 
  \begin{enumerate}
    \item l'\glossario{amministratore} accetta la richiesta inviata dall'\glossario{utente autenticato} di diventare \glossario{owner};
  \end{enumerate}
\end{itemize}
% subsub:AUC4.1.1 (end)

\subsubsection{AUC4.1.2 - Rifiuta richiesta owner}
\label{subsub:AUC4.1.2}
\begin{itemize}
  \item \textbf{Codice:} AUC4.1.2
  \item \textbf{Titolo:} rifiuta richiesta owner
  \item \textbf{Attori primari:} amministratore
  \item \textbf{Precondizione:} l'\glossario{utente autenticato} deve aver richiesto di diventare un \glossario{owner}.
  \item \textbf{Postcondizione:} viene rifiutata la richiesta da parte di un \glossario{amministratore}.
  \item \textbf{Scenario principale:} 
  \begin{enumerate}
    \item l'\glossario{amministratore} rifiuta la richiesta inviata dall'\glossario{utente autenticato} di diventare \glossario{owner};
  \end{enumerate}
\end{itemize}
% subsub:AUC4.1.2 (end)

\subsubsection{AUC4.2 - Gestione richiesta creazione organizzazione}
\label{subsub:AUC4.2}
\begin{itemize}
  \item \textbf{Codice:} AUC4.2
  \item \textbf{Titolo:} gestione richiesta creazione organizzazione
  \item \textbf{Attori primari:} amministratore
  \item \textbf{Precondizione:} l'\glossario{owner} deve aver richiesto di creare un organizzazione.
  \item \textbf{Postcondizione:} viene approvata o rifiutata la richiesta, e l'\glossario{owner} viene informato.
  \item \textbf{Scenario principale:} 
  \begin{enumerate}
    \item viene richiesta la creazione di una nuova \glossario{organizzazione}, che vuole utilizzare \glossario{Stalker};
    \item L'\glossario{amministratore} può accettare o rifiutare la richiesta.
  \end{enumerate}
\end{itemize}
% subsub:AUC4.2 (end)

\subsubsection{AUC4.2.1 - Accetta richiesta creazione organizzazione}
\label{subsub:AUC4.2.1}
\begin{itemize}
  \item \textbf{Codice:} AUC4.2.1
  \item \textbf{Titolo:} accetta richiesta creazione organizzazione
  \item \textbf{Attori primari:} amministratore
  \item \textbf{Precondizione:} l'\glossario{owner} deve aver richiesto di creare un organizzazione.
  \item \textbf{Postcondizione:} viene approvata la richiesta, e l'\glossario{owner} viene informato.
  \item \textbf{Scenario principale:} 
  \begin{enumerate}
    \item  l'\glossario{amministratore} accetta la richiesta inviata dall'\glossario{owner} di creare una nuova \glossario{organizzazione};
  \end{enumerate}
\end{itemize}
% subsub:AUC4.2.1 (end)

\subsubsection{AUC4.2.2 - Rifiuta richiesta creazione organizzazione}
\label{subsub:AUC4.2.2}
\begin{itemize}
  \item \textbf{Codice:} AUC4.2.2
  \item \textbf{Titolo:} rifiuta richiesta creazione organizzazione
  \item \textbf{Attori primari:} amministratore
  \item \textbf{Precondizione:} l'\glossario{owner} deve aver richiesto di creare un organizzazione;
  \item \textbf{Postcondizione:} viene rifiutata la richiesta, e l'\glossario{owner} viene informato;
  \item \textbf{Scenario principale:} 
  \begin{enumerate}
    \item  l'\glossario{amministratore} rifiutala richiesta inviata dall'\glossario{owner} di creare una nuova \glossario{organizzazione};
  \end{enumerate}
\end{itemize}
% subsub:AUC4.2.2 (end)

\subsubsection{AUC4.3 - Gestione richiesta modifica organizzazione}
\label{subsub:AUC4.3}
\begin{itemize}
  \item \textbf{Codice:} AUC4.3
  \item \textbf{Titolo:} gestione richiesta modifica organizzazione
  \item \textbf{Attori primari:} amministratore
  \item \textbf{Precondizione:} un \glossario{gestore} deve aver richiesto di modificare un'organizzazione.
  \item \textbf{Postcondizione:} viene approvata o rifiutata la richiesta, e il \glossario{gestore} viene informato.
  \item \textbf{Scenario principale:} 
  \begin{enumerate}
    \item viene richiesta la modifica di una \glossario{organizzazione};
    \item L'\glossario{amministratore} può accettare o rifiutare la richiesta.
  \end{enumerate}
\end{itemize}

\subsubsection{AUC4.3.1 - Accetta richiesta modifica organizzazione}
\label{subsub:AUC4.3.1}
\begin{itemize}
  \item \textbf{Codice:} AUC4.3.1
  \item \textbf{Titolo:} accetta richiesta modifica organizzazione
  \item \textbf{Attori primari:} amministratore
  \item \textbf{Precondizione:} il \glossario{gestore} deve aver richiesto di modificare un'organizzazione.
  \item \textbf{Postcondizione:} viene approvata la richiesta, e il \glossario{gestore} viene informato.
  \item \textbf{Scenario principale:} 
  \begin{enumerate}
    \item  l'\glossario{amministratore} accetta la richiesta inviata dal \glossario{gestore} di modificare un'\glossario{organizzazione};
  \end{enumerate}
\end{itemize}
% subsub:AUC4.3.1 (end)

\subsubsection{AUC4.3.2 - Rifiuta richiesta modifica organizzazione}
\label{subsub:AUC4.3.2}
\begin{itemize}
  \item \textbf{Codice:} AUC4.3.2
  \item \textbf{Titolo:} rifiuta richiesta modifica organizzazione
  \item \textbf{Attori primari:} amministratore
  \item \textbf{Precondizione:} il \glossario{gestore} deve aver richiesto di modificare un'organizzazione.
  \item \textbf{Postcondizione:} viene rifiutata la richiesta, e il \glossario{gestore} viene informato.
  \item \textbf{Scenario principale:} 
  \begin{enumerate}
    \item l'\glossario{amministratore} rifiuta la richiesta inviata dal \glossario{gestore} di modificareun'\glossario{organizzazione};
  \end{enumerate}
\end{itemize}
% subsub:AUC4.3.2 (end)
% subsub:AUC4.3 (end)

\subsubsection{AUC4.4 - Gestione richiesta aggiunta luogo}
\label{subsub:AUC4.4}
\begin{itemize}
  \item \textbf{Codice:} AUC4.4
  \item \textbf{Titolo:} gestione richiesta aggiunta luogo
  \item \textbf{Attori primari:} amministratore
  \item \textbf{Precondizione:} un \glossario{gestore} deve aver richiesto di aggiungere un luogo.
  \item \textbf{Postcondizione:} viene approvata o rifiutata la richiesta, e il \glossario{gestore} viene informato.
  \item \textbf{Scenario principale:} 
  \begin{enumerate}
    \item viene richiesto di aggiungere un luogo;
    \item L'\glossario{amministratore} può accettare o rifiutare la richiesta.
  \end{enumerate}
\end{itemize}
% subsub:AUC4.4 (end)

\subsubsection{AUC4.4.1 - Accetta richiesta aggiungi luogo}
\label{subsub:AUC4.4.1}
\begin{itemize}
  \item \textbf{Codice:} AUC4.4.1
  \item \textbf{Titolo:} accetta richiesta aggiungi luogo
  \item \textbf{Attori primari:} amministratore
  \item \textbf{Precondizione:} il \glossario{gestore} deve aver richiesto di aggiungere un luogo.
  \item \textbf{Postcondizione:} viene approvata la richiesta, e il \glossario{gestore} viene informato.
  \item \textbf{Scenario principale:} 
  \begin{enumerate}
    \item  l'\glossario{amministratore} accetta la richiesta inviata dal \glossario{gestore} di aggiungere un luogo.
  \end{enumerate}
\end{itemize}
% subsub:AUC4.4.1 (end)

\subsubsection{AUC4.4.2 - Rifiuta richiesta aggiungi luogo}
\label{subsub:AUC4.4.2}
\begin{itemize}
  \item \textbf{Codice:} AUC4.4.2
  \item \textbf{Titolo:} rifiuta richiesta aggiungi luogo
  \item \textbf{Attori primari:} amministratore
  \item \textbf{Precondizione:} il \glossario{gestore} deve aver richiesto di aggiungere un luogo.
  \item \textbf{Postcondizione:} viene rifiutata la richiesta, e il \glossario{gestore} viene informato.
  \item \textbf{Scenario principale:} 
  \begin{enumerate}
    \item l'\glossario{amministratore} rifiuta la richiesta inviata dal \glossario{gestore} di aggiungere un luogo.
  \end{enumerate}
\end{itemize}
% subsub:AUC4.4.2 (end)

\subsubsection{AUC4.5 - Gestione richiesta modifica luogo}
\label{subsub:AUC4.5}
\begin{itemize}
  \item \textbf{Codice:} AUC4.5
  \item \textbf{Titolo:} gestione richiesta modifica luogo
  \item \textbf{Attori primari:} amministratore
  \item \textbf{Precondizione:} un \glossario{gestore} deve aver richiesto di modificare un luogo.
  \item \textbf{Postcondizione:} viene approvata o rifiutata la richiesta, e il \glossario{gestore} viene informato.
  \item \textbf{Scenario principale:} 
  \begin{enumerate}
    \item viene richiesto di modificare un luogo;
    \item L'\glossario{amministratore} può accettare o rifiutare la richiesta.
  \end{enumerate}
\end{itemize}
% subsub:AUC4.5 (end)

\subsubsection{AUC4.5.1 - Accetta richiesta modifica luogo}
\label{subsub:AUC4.5.1}
\begin{itemize}
  \item \textbf{Codice:} AUC4.5.1
  \item \textbf{Titolo:} accetta richiesta modifica luogo
  \item \textbf{Attori primari:} amministratore
  \item \textbf{Precondizione:} il \glossario{gestore} deve aver richiesto di modificare un luogo.
  \item \textbf{Postcondizione:} viene approvata la richiesta, e il \glossario{gestore} viene informato.
  \item \textbf{Scenario principale:} 
  \begin{enumerate}
    \item  l'\glossario{amministratore} accetta la richiesta inviata dal \glossario{gestore} di modificare un luogo.
  \end{enumerate}
\end{itemize}
% subsub:AUC4.5.1 (end)

\subsubsection{AUC4.5.2 - Rifiuta richiesta modifica luogo}
\label{subsub:AUC4.5.2}
\begin{itemize}
  \item \textbf{Codice:} AUC4.5.2
  \item \textbf{Titolo:} rifiuta richiesta modifica luogo
  \item \textbf{Attori primari:} amministratore
  \item \textbf{Precondizione:} il \glossario{gestore} deve aver richiesto di modificare un luogo.
  \item \textbf{Postcondizione:} viene rifiutata la richiesta, e il \glossario{gestore} viene informato.
  \item \textbf{Scenario principale:}
  \begin{enumerate}
    \item l'\glossario{amministratore} rifiuta la richiesta inviata dal \glossario{gestore} di modificare un luogo.
  \end{enumerate}  
\end{itemize}
% subsub:AUC4.5.2 (end)

\subsubsection{AUC4.6 - Gestione richiesta trasferimento di proprietà organizzazione}
\label{subsub:AUC4.6}
\begin{itemize}
  \item \textbf{Codice:} AUC4.6
  \item \textbf{Titolo:} gestione richiesta trasferimento di proprietà organizzazione
  \item \textbf{Attori primari:} amministratore
  \item \textbf{Precondizione:} l'\glossario{organizzazione} deve essere già stata creata dal sistema, e deve essere stata fatta la richiesta trasferimento di proprietà.
  \item \textbf{Postcondizione:} viene approvata o rifiutata la richiesta, e l'\glossario{owner} viene informato.
  \item \textbf{Scenario principale:}
  \begin{enumerate}
    \item l'\glossario{amministratore} approva o rifiuta la richiesta di un trasferimento di proprietà di un'\glossario{organizzazione}.
  \end{enumerate}
\end{itemize}


\subsubsection{AUC4.6.1 - Accetta richiesta trasferimento di proprietà organizzazione}
\label{subsub:AUC4.6.1}
\begin{itemize}
  \item \textbf{Codice:} AUC4.6.1
  \item \textbf{Titolo:} accetta richiesta trasferimento di proprietà organizzazione
  \item \textbf{Attori primari:} amministratore
  \item \textbf{Precondizione:} l'\glossario{owner} deve aver fatto richiesta di trasferimento di proprietà di organizzazione.
  \item \textbf{Postcondizione:} viene approvata la richiesta, e l'\glossario{owner} viene informato.
  \item \textbf{Scenario principale:} 
  \begin{enumerate}
    \item l'\glossario{amministratore} approva la richiesta di un trasferimento di proprietà di un'\glossario{organizzazione}.
  \end{enumerate}
\end{itemize}  
% subsub:AUC4.6.1 (end)

\subsubsection{AUC4.6.2 - Rifiuta richiesta trasferimento di proprietà organizzazione}
\label{subsub:AUC4.6.2}
\begin{itemize}
  \item \textbf{Codice:} AUC4.6.2
  \item \textbf{Titolo:} rifiuta richiesta trasferimento di proprietà organizzazione
  \item \textbf{Attori primari:} amministratore
  \item \textbf{Precondizione:} l'\glossario{owner} deve aver fatto richiesta di trasferimento di proprietà di organizzazione.
  \item \textbf{Postcondizione:} viene rifiutata la richiesta, e l'\glossario{owner} viene informato.
  \item \textbf{Scenario principale:} 
  \begin{enumerate}
    \item  l'\glossario{amministratore} rifiuta la richiesta di un trasferimento di proprietà di un'\glossario{organizzazione}.
  \end{enumerate}
\end{itemize}
% subsub:AUC4.6.2 (end)
% subsub:AUC4.6 (end)
% par:Amministratore (end)

\paragraph{Root}
Di seguito sono riportati tutti i casi d'uso che coinvolgono il \glossario{super utente} \glossario{root}.

\subsubsection{AUC5 - Creazione amministratore}
\label{subsub:AUC5}
\begin{itemize}
  \item \textbf{Codice:} AUC5
  \item \textbf{Titolo:} creazione amministratore
  \item \textbf{Attori primari:} root
  \item \textbf{Precondizione:} devono essere specificate le credenziali del nuovo \glossario{amministratore}, che devono essere univoche.
  \item \textbf{Postcondizione:} l'\glossario{amministratore} viene creato.
  \item \textbf{Scenario principale:} 
  \begin{enumerate}
    \item sorge la necessità di creare un nuovo \glossario{amministratore} per gestire \glossario{Stalker}.
  \end{enumerate}
\end{itemize}


\subsubsection{AUC5.1 - Inserimento credenziali}
\label{subsub:AUC5.1}
\begin{itemize}
  \item \textbf{Codice:} AUC5.1
  \item \textbf{Titolo:} inserimento credenziali
  \item \textbf{Attori primari:} root
  \item \textbf{Precondizione:} il sistema deve rendere disponibile la pagina di creazione nuovo \glossario{amministratore}.
  \item \textbf{Postcondizione:} \glossario{Root} ha inserito le credenziali dell'\glossario{amministratore} che vuole creare.
  \item \textbf{Scenario principale:}
  \begin{enumerate}
    \item \glossario{Root} inserisce le credenziali dell'\glossario{amministratore} che vuole creare.
  \end{enumerate}
  \item \textbf{Inclusioni}:
  \begin{enumerate}
    \item verifica credenziali\emph{[AUC5.2]}.
  \end{enumerate}
  \item \textbf{Estensioni}:
  \begin{enumerate}
    \item  visualizza creazione fallita\emph{[AUC5.3]}.
  \end{enumerate}
\end{itemize}
% subsub:AUC5.1 (end)

\subsubsection{AUC5.1.1 - Inserimento nuova e-mail}
\label{subsub:AUC5.1.1}
\begin{itemize}
  \item \textbf{Codice:} AUC5.1.1
  \item \textbf{Titolo:} inserimento nuova e-mail
  \item \textbf{Attori primari:} root
  \item \textbf{Precondizione:} il sistema ha reso disponibile il campo per l'inserimento dell'e-mail.
  \item \textbf{Postcondizione:} \glossario{Root} ha inserito l'e-mail dell'\glossario{amministratore} che vuole creare.
  \item \textbf{Scenario principale:}
  \begin{enumerate}
    \item \glossario{Root} inserisce l'e-mail dell'\glossario{amministratore} che vuole creare.
  \end{enumerate}
\end{itemize}
% subsub:AUC5.1.1 (end)

\subsubsection{AUC5.1.2 - Inserimento nuova password}
\label{subsub:AUC5.1.2}
\begin{itemize}
  \item \textbf{Codice:} AUC5.1.2
  \item \textbf{Titolo:} inserimento nuova password
  \item \textbf{Attori primari:} root
  \item \textbf{Precondizione:} il sistema ha reso disponibile il campo per l'inserimento della password.
  \item \textbf{Postcondizione:} \glossario{Root} ha inserito la password dell'\glossario{amministratore} che vuole creare.
  \item \textbf{Scenario principale:}
  \begin{enumerate}
    \item \glossario{Root} inserisce la password dell'\glossario{amministratore} che vuole creare.
  \end{enumerate}
\end{itemize}
% subsub:AUC5.1.2 (end)
\subsubsection{AUC5.1.3 - Conferma password}
\label{subsub:AUC5.1.3}
\begin{itemize}
  \item \textbf{Codice:} AUC5.1.3
  \item \textbf{Titolo:} conferma password
  \item \textbf{Attori primari:} root
  \item \textbf{Precondizione:} il sistema ha reso disponibile il campo per l'inserimento del campo conferma password.
  \item \textbf{Postcondizione:} \glossario{Root} ha inserito la password nel campo di conferma dell'\glossario{amministratore} che vuole creare.
  \item \textbf{Scenario principale:}
  \begin{enumerate}
    \item \glossario{Root} inserisce la password nel campo di conferma dell'\glossario{amministratore} che vuole creare.
  \end{enumerate}
\end{itemize}
% subsub:AUC5.1.3 (end)

\subsubsection{AUC5.2 - Verifica credenziali}
\label{subsub:AUC5.2}
\begin{itemize}
  \item \textbf{Codice:} AUC5.2
  \item \textbf{Titolo:} verifica credenziali
  \item \textbf{Attori primari:} root
  \item \textbf{Precondizione:} \glossario{Root} invia richiesta di creazione nuovo amministratore.
  \item \textbf{Postcondizione:} il nuovo \glossario{amministratore} viene creato solo se i requisiti sono stati rispettati.
  \item \textbf{Scenario principale:} 
  \begin{enumerate}
    \item il sistema verifica se le credenziali immesse rispettano i requisiti:
    \begin{enumerate}
      \item e-mail e password non siano vuoti;
      \item e-mail e password contengano solo i caratteri consentiti;
      \item lo username non esista.
    \end{enumerate}
  \end{enumerate}
\end{itemize}
% subsub:AUC5.2 (end)

\subsubsection{AUC5.3 - Visualizza creazione fallita}
\label{subsub:AUC5.3}
\begin{itemize}
  \item \textbf{Codice:} AUC5.3
  \item \textbf{Titolo:} visualizza creazione fallita
  \item \textbf{Attori primari:} root
  \item \textbf{Precondizione:} la verifica delle credenziali è fallita.
  \item \textbf{Postcondizione:} \glossario{root} visualizza un messaggio di creazione fallita.
  \item \textbf{Scenario principale:} 
  \begin{enumerate}
    \item \glossario{root} cerca di creare un nuovo \glossario{amministratore} che non rispetta i requisiti.
  \end{enumerate}
\end{itemize}
% subsub:AUC5.3 (end)
% subsub:AUC5 (end)



\subsubsection{AUC6 - Gestione organizzazione}
\label{subsub:AUC6}
\begin{itemize}
  \item \textbf{Codice:} AUC6
  \item \textbf{Titolo:} gestione organizzazione
  \item \textbf{Attori primari:} root
  \item \textbf{Precondizione:} il sistema deve rendere disponibile la pagina di gestione \glossario{organizzazione}.
  \item \textbf{Postcondizione:} vengono gestite una o più \glossario{organizzazioni}.
  \item \textbf{Scenario principale:}
  \begin{enumerate}
    \item sorge la necessità di effettuare operazioni su una o più \glossario{organizzazioni};
  \end{enumerate}
\end{itemize}


\subsubsection{AUC6.1 - Creazione organizzazione}
\label{subsub:AUC6.1}
\begin{itemize}
  \item \textbf{Attori primari:} \glossario{root};
  \item \textbf{Precondizione:} l'\glossario{organizzazione} non deve esistere nella lista di \glossario{Stalker}, deve essere specificato il suo nome.
  \item \textbf{Postcondizione:} l'\glossario{organizzazione} viene creata.
  \item \textbf{Scenario principale:} sorge la necessità di creare un'\glossario{organizzazione}, senza essere effettivamente richiesta;
\end{itemize}

  \subsubsection{AUC6.1.1 - Inserisci nome organizzazione}
  \label{subsub:AUC6.1.1}
  \begin{itemize}
    \item \textbf{Codice:} AUC6.1.1
    \item \textbf{Titolo:} Inserisci nome organizzazione
    \item \textbf{Attori primari:} root
    \item \textbf{Precondizione:} il sistema fornisce il campo di inserimento nome.
    \item \textbf{Postcondizione:} il nome viene opportunamente inserito.
    \item \textbf{Scenario principale:} 
    \begin{enumerate}
      \item si vuole inserire il nome di un'\glossario{organizzazione}.
    \end{enumerate}
    
  \end{itemize}
  % subsub:AUC6.1.1 (end)
  
  \subsubsection{AUC6.1.2 - Inserisci indirizzo organizzazione}
  \label{subsub:AUC6.1.2}
  \begin{itemize}
    \item \textbf{Codice:} AUC6.1.2
    \item \textbf{Titolo:} Inserisci indirizzo organizzazione
    \item \textbf{Attori primari:} root
    \item \textbf{Precondizione:} il sistema fornisce il campo di inserimento indirizzo organizzazione.
    \item \textbf{Postcondizione:} l'indirizzo viene opportunamente inserito.
    \item \textbf{Scenario principale:}
    \begin{enumerate}
      \item si vuole inserire l'indirizzo di un'\glossario{organizzazione}.
    \end{enumerate}
  \end{itemize}
  % subsub:AUC6.1.2 (end)
  
  \subsubsection{AUC6.1.3 - Inserisci descrizione organizzazione}
  \label{subsub:AUC6.1.3}
  \begin{itemize}
    \item \textbf{Codice:} AUC6.1.3
    \item \textbf{Titolo:} Inserisci descrizione organizzazione
    \item \textbf{Attori primari:} root
    \item \textbf{Precondizione:} il sistema fornisce il campo di inserimento descrizione organizzazione.
    \item \textbf{Postcondizione:} la descrizione viene opportunamente inserito.
    \item \textbf{Scenario principale:}
    \begin{enumerate}
      \item si vuole inserire la descrizione di un'\glossario{organizzazione}.
    \end{enumerate}
  \end{itemize}
  % subsub:AUC6.1.3 (end)



\subsubsection{AUC6.2 - Eliminazione organizzazione}
\label{subsub:AUC6.2}
\begin{itemize}
  \item \textbf{Codice:} AUC6.2
  \item \textbf{Titolo:} eliminazione organizzazione
  \item \textbf{Attori primari:} root
  \item \textbf{Precondizione:} deve essere stata selezionata l'\glossario{organizzazione} da eliminare, presente nella lista di \glossario{Stalker}.
  \item \textbf{Postcondizione:} l'\glossario{organizzazione} viene eliminata.
  \item \textbf{Scenario principale:} 
  \begin{enumerate}
    \item sorge la necessità di eliminare un'\glossario{organizzazione}, senza interagire con il suo \glossario{owner};
  \end{enumerate}
  \item \textbf{Inclusioni:}
  \begin{enumerate}
    \item Seleziona organizzazione\emph{[AUC6.4]};
  \end{enumerate}
\end{itemize}
% subsub:AUC6.2 (end)


\subsubsection{AUC6.3 - Modifica organizzazione}
\label{subsub:AUC6.3}
\begin{itemize}
  \item \textbf{Codice:} AUC6.3
  \item \textbf{Titolo:} modifica organizzazione
  \item \textbf{Attori primari:} root
  \item \textbf{Precondizione:} deve essere stata selezionata l'\glossario{organizzazione} da modificare, presente nella lista di \glossario{Stalker}.
  \item \textbf{Postcondizione:} l'\glossario{organizzazione} viene modificata.
  \item \textbf{Scenario principale:}
  \begin{enumerate}
    \item sorge la necessità di modificare un'\glossario{organizzazione}, senza interagire con il suo \glossario{owner};
  \end{enumerate}
  \item \textbf{Inclusioni:}
  \begin{enumerate}
    \item Seleziona organizzazione\emph{[AUC6.4]};
  \end{enumerate}
\end{itemize}
% subsub:AUC6.3 (end)

\subsubsection{AUC6.3.1 - Modifica nome organizzazione}
\label{subsub:AUC6.3.1}
\begin{itemize}
  \item \textbf{Codice:} AUC6.3.1
  \item \textbf{Titolo:} modifica nome organizzazione
  \item \textbf{Attori primari:} root
  \item \textbf{Precondizione:} il sistema fornisce il campo di modifica nome.
  \item \textbf{Postcondizione:} il nome viene opportunamente modificato.
  \item \textbf{Scenario principale:} 
  \begin{enumerate}
    \item si vuole modificare il nome di un'\glossario{organizzazione}.
  \end{enumerate}
  
\end{itemize}
% subsub:AUC6.3.1 (end)

\subsubsection{AUC6.3.2 - Modifica indirizzo organizzazione}
\label{subsub:AUC6.3.2}
\begin{itemize}
  \item \textbf{Codice:} AUC6.3.2
  \item \textbf{Titolo:} Modifica indirizzo organizzazione
  \item \textbf{Attori primari:} root
  \item \textbf{Precondizione:} il sistema fornisce il campo di modifica indirizzo organizzazione.
  \item \textbf{Postcondizione:} l'indirizzo viene opportunamente modificato.
  \item \textbf{Scenario principale:}
  \begin{enumerate}
    \item si vuole modificare l'indirizzo di un'\glossario{organizzazione}.
  \end{enumerate}
\end{itemize}
% subsub:AUC6.3.2 (end)

\subsubsection{AUC6.3.3 - Modifica descrizione organizzazione}
\label{subsub:AUC6.3.3}
\begin{itemize}
  \item \textbf{Codice:} AUC6.3.3
  \item \textbf{Titolo:} Modifica descrizione organizzazione
  \item \textbf{Attori primari:} root
  \item \textbf{Precondizione:} il sistema fornisce il campo di modifica descrizione organizzazione.
  \item \textbf{Postcondizione:} la descrizione viene opportunamente modificata.
  \item \textbf{Scenario principale:}
  \begin{enumerate}
    \item si vuole modificare la descrizione di un'\glossario{organizzazione}.
  \end{enumerate}
\end{itemize}
% subsub:AUC6.3.3 (end)

\subsubsection{AUC6.3.4 - Gestione luoghi}
\label{subsub:AUC6.3.4}
\begin{itemize}
  \item \textbf{Codice:} AUC6.3.4
  \item \textbf{Titolo:} gestione luoghi
  \item \textbf{Attori primari:} root
  \item \textbf{Precondizione:} il sistema deve rendere disponibile la pagina di gestione dei luoghi di un'\glossario{organizzazione}.
  \item \textbf{Postcondizione:} vengono gestiti i luoghi di un'\glossario{organizzazione}.
  \item \textbf{Scenario principale:}
  \begin{enumerate}
    \item sorge la necessità di effettuare operazioni sul luogo di un'organizzazione, e viene offerta la possibilità di selezionarlo;
  \end{enumerate}
\end{itemize}
% subsub:AUC6.3.4 (end)

\subsubsection{AUC6.3.4.1 - Aggiungi luogo}
\label{subsub:AUC6.3.4.1}
\begin{itemize}
  \item \textbf{Codice:} AUC6.3.4.1
  \item \textbf{Titolo:} aggiungi luogo
  \item \textbf{Attori primari:} \glossario{root}
  \item \textbf{Precondizione:} il \glossario{luogo} dell'\glossario{organizzazione} deve non esistere.
  \item \textbf{Postcondizione:} il \glossario{luogo} dell'\glossario{organizzazione} viene aggiunto.
  \item \textbf{Scenario principale:}
  \begin{enumerate}
    \item sorge la necessità di aggiungere un \glossario{luogo} ad un'\glossario{organizzazione}, senza interagire con il suo \glossario{owner};
  \end{enumerate}
\end{itemize}
% subsub:AUC6.3.4.1 (end)


\subsubsection{AUC6.3.4.2 - Eliminazione luogo}
\label{subsub:AUC6.3.4.2}
\begin{itemize}
  \item \textbf{Codice:} AUC6.3.4.2
  \item \textbf{Titolo:} eliminazione luogo
  \item \textbf{Attori primari:} \glossario{root}
  \item \textbf{Precondizione:} il \glossario{luogo} dell'\glossario{organizzazione} deve essere presente in \glossario{Stalker}.
  \item \textbf{Postcondizione:} il \glossario{luogo} dell'\glossario{organizzazione} viene eliminato.
  \item \textbf{Scenario principale:}
  \begin{enumerate}
    \item sorge la necessità di eliminare un \glossario{luogo} di un'\glossario{organizzazione}, senza interagire con il suo \glossario{owner};
  \end{enumerate}
  \item \textbf{Inclusioni:}
  \begin{enumerate}
    \item Seleziona luogo\emph{[AUC6.3.4.4]};
  \end{enumerate}
\end{itemize}
% subsub:AUC6.3.4.2 (end)

\subsubsection{AUC6.3.4.3 - Modifica luogo}
\label{subsub:AUC6.3.4.3}
\begin{itemize}
  \item \textbf{Codice:} AUC6.3.4.3
  \item \textbf{Titolo:} modifica luogo
  \item \textbf{Attori primari:} \glossario{root};
  \item \textbf{Precondizione:} il \glossario{luogo} dell'\glossario{organizzazione} deve essere presente in \glossario{Stalker};
  \item \textbf{Postcondizione:} il \glossario{luogo} dell'\glossario{organizzazione} viene modificato.
  \item \textbf{Scenario principale:}
  \begin{enumerate}
    \item sorge la necessità di modificare un \glossario{luogo} di un'\glossario{organizzazione}, senza interagire con il suo \glossario{owner};
  \end{enumerate}
  \item \textbf{Inclusioni:}
  \begin{enumerate}
    \item Seleziona luogo\emph{[AUC6.3.4.4]};
  \end{enumerate}
\end{itemize}
% subsub:AUC6.3.4.3 (end)

\subsubsection{AUC6.3.4.4 - Seleziona luogo}
\label{subsub:AUC6.3.4.4}
\begin{itemize}
  \item \textbf{Codice:} AUC6.3.4.4
  \item \textbf{Titolo:} seleziona luogo
  \item \textbf{Attori primari:} root
  \item \textbf{Precondizione:} il sistema deve mostrare la lista dei luoghi all'interno di una \glossario{organizzazione}.
  \item \textbf{Postcondizione:} viene scelto il luogo desiderato.
  \item \textbf{Scenario principale:}
  \begin{enumerate}
    \item sorge la necessità di effettuare operazioni sul luogo di un'\glossario{organizzazione}, e viene offerta la possibilità di selezionarlo;
  \end{enumerate}
\end{itemize}
% subsub:AUC6.3.4.4 (end)
% subsub:AUC6.3.4 (end)

\subsubsection{AUC6.4 - Seleziona organizzazione}
\label{subsub:AUC6.4}
\begin{itemize}
  \item \textbf{Codice:} AUC6.4
  \item \textbf{Titolo:} seleziona organizzazione
  \item \textbf{Attori primari:} root
  \item \textbf{Precondizione:} il sistema deve mostrare la lista di organizzazioni in \glossario{Stalker}.
  \item \textbf{Postcondizione:} viene scelta l'\glossario{organizzazione} desiderata.
  \item \textbf{Scenario principale:}
  \begin{enumerate}
    \item sorge la necessità di effettuare operazioni su un'organizzazione, e viene offerta la possibilità di selezionarla;
  \end{enumerate}
\end{itemize}
% subsub:AUC6.4 (end)
% subsub:AUC6 (end)

\subsubsection{AUC15 - Eliminazione account}
\label{subsub:AUC15}
\begin{itemize}
  \item \textbf{Codice:} AUC15
  \item \textbf{Titolo:} eliminazione account
  \item \textbf{Attori primari:} \glossario{root};
  \item \textbf{Precondizione:} deve essere stato selezionato l'\glossario{account} da eliminare, che deve esistere in \glossario{Stalker};
  \item \textbf{Postcondizione:} l'\glossario{account} selezionato è stato eliminato.
  \item \textbf{Scenario principale:}
  \begin{enumerate}
    \item sorge la necessità di eliminare un \glossario{account} per sconosciuti motivi;
  \end{enumerate}
  \begin{enumerate}
    \item Seleziona account\emph{[AUC15.1]};
  \end{enumerate}
\end{itemize}
% subsub:AUC15 (end)

\subsubsection{AUC15.1 - Seleziona account}
\label{subsub:AUC15}
\begin{itemize}
  \item \textbf{Codice:} AUC15
  \item \textbf{Titolo:} seleziona account
  \item \textbf{Attori primari:} \glossario{root};
  \item \textbf{Precondizione:} il sistema deve rendere disponibile la lista degli account registrati a \glossario{Stalker}.
  \item \textbf{Postcondizione:} l'\glossario{account} è stato selezionato.
  \item \textbf{Scenario principale:}
  \begin{enumerate}
    \item \glossario{Root} seleziona l'account da eliminare;
  \end{enumerate}
\end{itemize}
% subsub:AUC15 (end)
% par:Root (end)

\paragraph{Visualizzatore}
Di seguito sono riportati tutti i casi d'uso che coinvolgono il \glossario{super utente} \glossario{Visualizzatore}.

\subsubsection{AUC16 - Query sull'organizzazione}
\label{subsub:AUC16}
\begin{itemize}
  \item \textbf{Codice:} AUC16
  \item \textbf{Titolo:} query sull'organizzazione
  \item \textbf{Attori primari:} \glossario{visualizzatore};
  \item \textbf{Precondizione:} il sistema risponde correttamente alle interrogazioni;
  \item \textbf{Postcondizione:} il \glossario{visualizzatore} ottiene le informazioni di cui ha bisogno.
  \item \textbf{Scenario principale:}
  \begin{enumerate}
    \item il \glossario{visualizzatore} vuole avere delle informazioni riguardanti l'\glossario{organizzazione} su cui opera;
  \end{enumerate}
\end{itemize}
% subsub:AUC16 (end)

\subsubsection{AUC17 - Query sul dipendente}
\label{subsub:AUC17}
\begin{itemize}
  \item \textbf{Codice:} AUC17
  \item \textbf{Titolo:} query sul dipendente
  \item \textbf{Attori primari:} \glossario{visualizzatore};
  \item \textbf{Precondizione:} il sistema risponde correttamente alle interrogazioni;
  \item \textbf{Postcondizione:} il \glossario{visualizzatore} ottiene le informazioni di cui ha bisogno.
  \item \textbf{Scenario principale:}
  \begin{enumerate}
    \item il \glossario{visualizzatore} vuole avere delle informazioni riguardanti il\glossario{dipendente} dell'\glossario{organizzazione} su cui opera;
  \end{enumerate}
\end{itemize}
% subsub:AUC17 (end)
% par:Visualizzatore (end)

\paragraph{Gestore}
Di seguito sono riportati tutti i casi d'uso che coinvolgono il \glossario{super utente} \glossario{Gestore}.

\subsubsection{AUC19 - Richiesta su luoghi}
\label{subsub:AUC19}
\begin{itemize}
  \item \textbf{Attori primari:} \glossario{gestore};
  \item \textbf{Descrizione:} il \glossario{gestore} dell'\glossario{organizzazione} richiede di aggiungere un nuovo \glossario{luogo};
  \item \textbf{Scenario principale:} il \glossario{gestore} vuole aggiungere un nuovo \glossario{luogo} all'\glossario{organizzazione} su cui opera;
  \item \textbf{Precondizione:} il \glossario{luogo} da aggiungere non deve già esistere;
  \item \textbf{Postcondizione:} la richiesta di aggiunta di un nuovo \glossario{luogo} viene posta.
\end{itemize}
% subsub:AUC19 (end)



\subsubsection{AUC19 - Richiesta aggiunta luogo}
\label{subsub:AUC19}
\begin{itemize}
  \item \textbf{Attori primari:} \glossario{gestore};
  \item \textbf{Descrizione:} il \glossario{gestore} dell'\glossario{organizzazione} richiede di aggiungere un nuovo \glossario{luogo};
  \item \textbf{Scenario principale:} il \glossario{gestore} vuole aggiungere un nuovo \glossario{luogo} all'\glossario{organizzazione} su cui opera;
  \item \textbf{Precondizione:} il \glossario{luogo} da aggiungere non deve già esistere;
  \item \textbf{Postcondizione:} la richiesta di aggiunta di un nuovo \glossario{luogo} viene posta.
\end{itemize}
% subsub:AUC19 (end)

\subsubsection{AUC20 - Richiesta rimozione luogo}
\label{subsub:AUC20}
\begin{itemize}
  \item \textbf{Attori primari:} \glossario{gestore};
  \item \textbf{Descrizione:} il \glossario{gestore} dell'\glossario{organizzazione} richiede di eliminare un \glossario{luogo};
  \item \textbf{Scenario principale:} il \glossario{gestore} vuole eliminare un \glossario{luogo} all'\glossario{organizzazione} su cui opera;
  \item \textbf{Precondizione:} il \glossario{luogo} da eliminare deve esistere;
  \item \textbf{Postcondizione:} la richiesta di eliminazione di un \glossario{luogo} viene posta.
\end{itemize}
% subsub:AUC20 (end)

\subsubsection{AUC21 - Richiesta modifica luogo}
\label{subsub:AUC21}
\begin{itemize}
  \item \textbf{Attori primari:} \glossario{gestore};
  \item \textbf{Descrizione:} il \glossario{gestore} dell'\glossario{organizzazione} richiede di modificare un \glossario{luogo};
  \item \textbf{Scenario principale:} il \glossario{gestore} vuole modificare un \glossario{luogo} dell'\glossario{organizzazione} su cui opera;
  \item \textbf{Precondizione:} il \glossario{luogo} da modificare deve esistere;
  \item \textbf{Postcondizione:} la richiesta di modifica di un \glossario{luogo} viene posta.
\end{itemize}
% subsub:AUC21 (end)


\subsubsection{AUC22 - Richiesta modifica parametri organizzazione}
\label{subsub:AUC22}
\begin{itemize}
  \item \textbf{Attori primari:} \glossario{gestore};
  \item \textbf{Descrizione:} il \glossario{gestore} dell'\glossario{organizzazione} richiede di modificare i parametri di essa;
  \item \textbf{Scenario principale:} il \glossario{gestore} vuole modificare i parametri dell'\glossario{organizzazione} su cui opera;
  \item \textbf{Precondizione:} i parametri devono essere effettivamente modificati;
  \item \textbf{Postcondizione:} la richiesta di modifica parametri viene posta.
\end{itemize}
% subsub:AUC22 (end)
% par:Gestore (end)

\paragraph{Owner}
Di seguito sono riportati tutti i casi d'uso che coinvolgono il \glossario{super utente} \glossario{Owner}.

\subsubsection{AUC23 - Nomina visualizzatore}
\label{subsub:AUC23}
\begin{itemize}
  \item \textbf{Attori primari:} \glossario{owner};
  \item \textbf{Descrizione:} l' \glossario{owner} nomina un \glossario{visualizzatore} per la sua \glossario{organizzazione};
  \item \textbf{Scenario principale:} l' \glossario{owner} vuole aggiungere un \glossario{visualizzatore} alla sua \glossario{organizzazione};
  \item \textbf{Precondizione:} il nuovo \glossario{super utente} non deve già esistere come \glossario{visualizzatore};
  \item \textbf{Postcondizione:} il nuovo \glossario{visualizzatore} è stato aggiunto.
\end{itemize}
% subsub:AUC23 (end)

\subsubsection{AUC24 - Nomina gestore}
\label{subsub:AUC24}
\begin{itemize}
  \item \textbf{Attori primari:} \glossario{owner};
  \item \textbf{Descrizione:} l' \glossario{owner} nomina un \glossario{gestore} per la sua \glossario{organizzazione};
  \item \textbf{Scenario principale:} l' \glossario{owner} vuole aggiungere un \glossario{gestore} alla sua \glossario{organizzazione};
  \item \textbf{Precondizione:} il nuovo \glossario{super utente} non deve già esistere come \glossario{gestore};
  \item \textbf{Postcondizione:} il nuovo \glossario{gestore} è stato aggiunto.
\end{itemize}
% subsub:AUC24 (end)

\subsubsection{AUC25 - Nomina gestore}
\label{subsub:AUC25}
\begin{itemize}
  \item \textbf{Attori primari:} \glossario{owner};
  \item \textbf{Descrizione:} l' \glossario{owner} nomina un \glossario{gestore} per la sua \glossario{organizzazione};
  \item \textbf{Scenario principale:} l' \glossario{owner} vuole aggiungere un \glossario{gestore} alla sua \glossario{organizzazione};
  \item \textbf{Precondizione:} il nuovo \glossario{super utente} non deve già esistere come \glossario{gestore};
  \item \textbf{Postcondizione:} il nuovo \glossario{gestore} è stato aggiunto.
\end{itemize}
% subsub:AUC25 (end)

\subsubsection{AUC26 - Richiesta creazione organizzazione}
\label{subsub:AUC26}
\begin{itemize}
  \item \textbf{Attori primari:} \glossario{owner};
  \item \textbf{Descrizione:} l' \glossario{owner} richiede di creare una nuova \glossario{organizzazione};
  \item \textbf{Scenario principale:} l' \glossario{owner} vuole creare una nuova \glossario{organizzazione};
  \item \textbf{Precondizione:} l' \glossario{organizzazione} non deve già esistere;
  \item \textbf{Postcondizione:} la richiesta di creare una nuova \glossario{organizzazione} è stata posta.
\end{itemize}
% subsub:AUC26 (end)

\subsubsection{AUC27 - Richiesta cedimento proprietà organizzazione}
\label{subsub:AUC27}
\begin{itemize}
  \item \textbf{Attori primari:} \glossario{owner};
  \item \textbf{\textbf{Descrizione:}} l' \glossario{owner} richiede che l'\glossario{organizzazione} venga ceduta ad un altro futuro \glossario{owner};
  \item \textbf{\textbf{Scenario principale:}} l' \glossario{owner} vuole cedere l'\glossario{organizzazione} ad un'altro \glossario{owner};
  \item \textbf{Precondizione:} il futuro \glossario{owner} deve esistere;
  \item \textbf{Postcondizione:} la richiesta di cedere l'\glossario{organizzazione} è stata posta.
\end{itemize}
% subsub:AUC27 (end)

% par:Owner (end)

\end{document}
