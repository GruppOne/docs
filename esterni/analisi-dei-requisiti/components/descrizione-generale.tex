\documentclass[../analisi-dei-requisiti]{subfiles}

\renewcommand{\commons}{../../../commons}

\begin{document}
\subsection{Obiettivo del prodotto}%
\label{sub:obiettivo_del_prodotto}
Il prodotto richiesto da capitolato si pone l'obiettivo di tracciare il numero di persone presenti all’interno di una data organizzazione.\\
Per soddisfare la richiesta, sarà implementato un server con annessa \glossario{UI} per la gestione delle organizzazioni e il monitoraggio degli utenti, e
di un'\glossario{applicazione mobile} per l'utente finale che gli permetta di essere monitorato dalla specifica organizzazione scelta.

% sub:obiettivo_del_prodotto (end)
\subsection{Funzionalità del prodotto}%
\label{sub:funzionalita_del_prodotto}
Le funzionalità che sono offerte dall'applicazione mobile sono:
\begin{itemize}
  \item collegamento ad un’organizzazione;
  \item recupero della lista delle organizzazioni;
  \item predisposizione di un pulsante “anonimo” che permetta, all’interno di una organizzazione, di risultare presente in maniera anonima;
  \item storico degli accessi;
  \item visualizzazione in tempo reale della propria presenza o meno all’interno di un luogo monitorato e cronometro del tempo trascorso al suo interno.
\end{itemize}

Le funzionalità che, invece, sono offerte dal server sono:
\begin{itemize}
  \item creare e gestire più organizzazioni;
  \item definire se prevedere una tracciatura nota oppure incognita
  \item solo in presenza di autorizzazione specifica, effettuare query di monitoraggio per singolo utente all’interno delle organizzazioni
\end{itemize}
Le comunicazioni tra applicazione mobile e server avvengono solo al momento d'ingresso ed uscita dai luoghi designati dalle organizzazioni, al fine di garantire
la privacy dell'utente.

% sub:funzionalita_del_prodotto (end)
\subsection{Caratteristiche degli utenti}%
\label{sub:caratteristiche_degli_utenti}
Nell'ambito di questo progetto, sono presenti due tipologie di utenti, con caratteristiche diverse dovute all'utilizzo del prodotto:
\begin{itemize}
  \item \textbf{Amministratore:} è l'utente che detiene i privilegi per gestire il server e monitorare gli utenti generici;
  \item \textbf{Utente generico:} è l'utente utilizzatore dell'applicazione mobile. Può essere un dipendente di un'azienda che monitora la sua presenza, oppure
  un visitatore di un evento pubblico.
\end{itemize}
Le due tipologie di utenti sono descritte nel dettaglio al capitolo 3, sezione \emph{Casi d'uso - Attori nei casi d'uso}

% sub:caratteristiche_degli_utenti (end)
\subsection{Macroarchitetture del progetto}%
\label{sub:macroarchitetture_del_progetto}
\paragraph{Back-end}%
\label{par:back-end}
E' richiesto lo sviluppo \glossario{back-end} del server, dove è richiesto l'utilizzo di protocolli asincroni per le comunicazioni tra applicazione mobile e server.\\
Sarà necessario lo sviluppo di un database che contenga tutti i dati d'interesse per il prodotto, dai dati appartenenti ai vari utenti fino ai dati che appartengono ai
luoghi delle organizzazioni.
% par:back-end (end)
\paragraph{Front-end}%
\label{par:front-end}
Lo sviluppo \glossario{front-end} sarà costituito:
\begin{itemize}
  \item dalla UI del server, raggiungibile solo dall'amministratore tramite autenticazione, per eseguire determinate operazioni in base al tipo di privilegi;
  \item dall'applicazione mobile, scaricabile da ogni utente interessato all'utilizzo del prodotto, che consente di visualizzare le informazioni d'interesse ed eseguire determinate
  operazioni in presenza di rete.
\end{itemize}

% par:front-end (end)
% sub:macroarchitetture_del_progetto (end)
\end{document}