\documentclass[../../../analisi-dei-requisiti.tex]{subfiles}

\begin{document}

\begin{figure}[H]
  \centering
  \begin{plantuml}
  @startuml
  !include ../../commons/style/use-cases.pu

  actor :amministratore: as A

  rectangle {
    together {
      usecase (AUC15.1) as "AUC15.1\nInserimento email owner"
    }
  }

  :A: -- AUC15.1

  @enduml
  \end{plantuml}
  \caption{AUC15: Eliminazione owner}
  \label{fig:auc15}
\end{figure}

\begin{description}
  \item[Codice:] AUC15
  \item[Titolo:] Eliminazione owner
  \item[Attori primari:] amministratore
  \item[Precondizione:] l'owner deve esistere.
  \item[Postcondizione:] l'utente non è più owner.
  \item[Scenario principale:]
  \begin{enumerate}
    \item l'amministratore vuole eliminare i privilegi ad un'utente owner. Le credenziali dell'utente rimangono.
  \end{enumerate}
\end{description}

\subsubsection{AUC15.1 - Inserimento email owner}%
\label{subs:AUC15.1}
\begin{description}
  \item[Codice:] AUC15.1
  \item[Titolo:] Inserimento email owner
  \item[Attori primari:] amministratore
  \item[Precondizione:] il sistema deve rendere disponibile il campo per l'inserimento della mail dell'owner da eliminare.
  \item[Postcondizione:] il campo relativo all'owner viene compilato.
  \item[Scenario principale:]
  \begin{enumerate}
    \item l'amministratore inserisce l'email dell'owner da eliminare.
  \end{enumerate}
\end{description}

\end{document}
