\documentclass[../../../analisi-dei-requisiti.tex]{subfiles}

\begin{document}

\begin{figure}[H]
  \centering
  \begin{plantuml}
  @startuml
  !include ../../commons/style/use-cases.pu

  actor :amministratore: as A

  rectangle {
    together {
      usecase (AUC6.1) as "AUC6.1\nSeleziona account"
    }
  }

  :A: -- AUC6.1

  @enduml
  \end{plantuml}
  \caption{AUC6: Eliminazione account}%
  \label{fig:auc6}
\end{figure}

\begin{description}
  \item[Codice:] AUC6;
  \item[Titolo:] Eliminazione account;
  \item[Attori primari:] amministratore;
  \item[Precondizione:] deve essere stato selezionato l'\glossario{account} da eliminare, che deve esistere in \emph{Stalker};
  \item[Postcondizione:] l'account selezionato è stato eliminato;
  \item[Scenario principale:]
  \begin{enumerate}
    \item sorge la necessità di eliminare un account.
  \end{enumerate}
\end{description}

\subsubsection{AUC6.1: Seleziona account}%
\label{subs:AUC6.1}
\begin{description}
  \item[Codice:] AUC6.1;
  \item[Titolo:] Seleziona account;
  \item[Attori primari:] amministratore;
  \item[Precondizione:] il sistema deve rendere disponibile la lista degli account registrati a \emph{Stalker};
  \item[Postcondizione:] l'account è stato selezionato;
  \item[Scenario principale:]
  \begin{enumerate}
    \item l'amministratore seleziona l'account da eliminare.
  \end{enumerate}
\end{description}


\end{document}
