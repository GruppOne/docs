\documentclass[../../../analisi-dei-requisiti.tex]{subfiles}

\begin{document}

\begin{figure}[H]
  \centering
  \begin{plantuml}
    @startuml
    !include ../../commons/style/use-cases.pu

    actor :amministratore: as A

    rectangle {
        together {
            usecase (AUC3.5) as "AUC3.5\nInvio richiesta aggiornamento\nlista organizzazioni"
            usecase (AUC3.4) as "AUC3.4\nSeleziona organizzazione"
            usecase (AUC3.3) as "AUC3.3\nModifica organizzazione"
            usecase (AUC3.2) as "AUC3.2\nEliminazione organizzazione"
            usecase (AUC3.1) as "AUC3.1\nCreazione organizzazione"
          }
      }

    :A: -- AUC3.1
    :A: -- AUC3.2
    :A: -- AUC3.3

    (AUC3.4) <.up. (AUC3.2) : <<include>>
    (AUC3.4) <.up. (AUC3.3) : <<include>>

    (AUC3.5) <.up. (AUC3.1) : <<include>>
    (AUC3.5) <.up. (AUC3.2) : <<include>>

    @enduml
  \end{plantuml}
  \caption{AUC3: Gestione organizzazione}%
  \label{fig:AUC3}
\end{figure}

\begin{description}
  \item[Codice:] AUC3;
  \item[Titolo:] Gestione organizzazione;
  \item[Attori primari:] amministratore, gestore;
  \item[Precondizione:] il sistema deve rendere disponibile la pagina di gestione dell'organizzazione;
  \item[Postcondizione:] viene gestita un'organizzazione;
  \item[Scenario principale:]
        \begin{enumerate}
          \item sorge la necessità di effettuare operazioni su un'organizzazione;
        \end{enumerate}
\end{description}

\subsubsection{AUC3.1: Creazione organizzazione}%
\label{subs:AUC3.1}

\begin{figure}[H]
  \centering
  \begin{plantuml}
    @startuml
    !include ../../commons/style/use-cases.pu

    actor :amministratore: as A1

    rectangle {
        together {
            usecase (AUC3.1.1) as "AUC3.1.1\nInserisci nome organizzazione"
            usecase (AUC3.1.2) as "AUC3.1.2\nInserisci descrizione organizzazione"
            usecase (AUC3.1.3) as "AUC3.1.3\nConfigurazione dettagli server LDAP"
          }
      }

    :A1: -- AUC3.1.1
    :A1: -- AUC3.1.2
    :A1: -- AUC3.1.3
    @enduml
  \end{plantuml}
  \caption{AUC3.1: Creazione organizzazione}%
  \label{fig:AUC3_1}
\end{figure}

\begin{description}
  \item[Codice:] AUC3.1;
  \item[Titolo:] Creazione organizzazione;
  \item[Attori primari:] amministratore, owner;
  \item[Precondizione:] l'organizzazione non deve esistere nella lista di \emph{Stalker};
  \item[Postcondizione:] l'organizzazione viene creata;
  \item[Scenario principale:]
        \begin{enumerate}
          \item sorge la necessità di creare un'organizzazione;
        \end{enumerate}
  \item[Inclusioni:]
        \begin{enumerate}
          \item alla fine della procedura di creazione dell'organizzazione, tutte le applicazioni mobile riceveranno una notifica di aggiornamento
                della lista di organizzazioni \emph{[AUC3.5]};
        \end{enumerate}
\end{description}

\subsubsection{AUC3.1.1: Inserisci nome organizzazione}%
\label{subs:AUC3.1.1}
\begin{description}
  \item[Codice:] AUC3.1.1;
  \item[Titolo:] Inserisci nome organizzazione;
  \item[Attori primari:] amministratore, owner;
  \item[Precondizione:] il sistema fornisce il campo di inserimento del nome;
  \item[Postcondizione:] il nome viene opportunamente inserito;
  \item[Scenario principale:]
        \begin{enumerate}
          \item si vuole inserire il nome di un'organizzazione.
        \end{enumerate}

\end{description}

\subsubsection{AUC3.1.2: Inserisci descrizione organizzazione}%
\label{subs:AUC3.1.2}
\begin{description}
  \item[Codice:] AUC3.1.2;
  \item[Titolo:] Inserisci descrizione organizzazione;
  \item[Attori primari:] amministratore, owner;
  \item[Precondizione:] il sistema fornisce il campo di inserimento della descrizione dell'organizzazione;
  \item[Postcondizione:] il campo relativo alla descrizione viene riempito;
  \item[Scenario principale:]
        \begin{enumerate}
          \item si vuole inserire la descrizione di un'organizzazione.
        \end{enumerate}
\end{description}

\subsubsection{AUC3.1.3: Configurazione dettagli server LDAP}%
\label{subs:AUC3.1.3}
\begin{description}
  \item[Codice:] AUC3.1.3;
  \item[Titolo:] Configurazione dettagli server LDAP;
  \item[Attori primari:] amministratore, owner;
  \item[Precondizione:] il sistema fornisce i campi per la configurazione del server LDAP;
  \item[Postcondizione:] il server LDAP è stato configurato;
  \item[Scenario principale:]
        \begin{enumerate}
          \item si vogliono configurare i dettagli del \glossario{server LDAP} che le applicazioni mobile dovranno utilizzare per registrarsi ad un'organizzazione;
          \item se l'organizzazione è segnata come pubblica, i parametri del server LDAP non verranno configurati.
        \end{enumerate}
\end{description}

\subsubsection{AUC3.2: Eliminazione organizzazione}%
\label{subs:AUC3.2}
\begin{description}
  \item[Codice:] AUC3.2;
  \item[Titolo:] Eliminazione organizzazione;
  \item[Attori primari:] amministratore, owner;
  \item[Precondizione:] deve essere stata selezionata l'organizzazione da eliminare, presente nella lista di \emph{Stalker};
  \item[Postcondizione:] l'organizzazione viene eliminata.
  \item[Scenario principale:]
        \begin{enumerate}
          \item sorge la necessità di eliminare un'organizzazione;
        \end{enumerate}
  \item[Inclusioni:]
        \begin{enumerate}
          \item viene selezionata un'organizzazione \emph{[AUC3.4]};
          \item alla fine della procedura di eliminazione dell'organizzazione, tutte le applicazioni mobile riceveranno una notifica di aggiornamento della lista di organizzazioni \emph{[AUC3.5]};
        \end{enumerate}
\end{description}

\subsubsection{AUC3.3: Modifica organizzazione}%
\label{subs:AUC3.3}

\begin{figure}[H]
  \centering
  \begin{plantuml}
    @startuml
    !include ../../commons/style/use-cases.pu

    actor :amministratore: as A

    rectangle {
        together {
            usecase (AUC3.3.1) as "AUC3.3.1\nModifica nome organizzazione"
            usecase (AUC3.3.2) as "AUC3.3.2\nModifica descrizione organizzazione"
            usecase (AUC3.3.3) as "AUC3.3.3\nGestione luoghi"
            usecase (AUC3.3.4) as "AUC3.3.4\nModifica configurazione dettagli server LDAP"
          }
      }

    :A: -- AUC3.3.1
    :A: -- AUC3.3.2
    :A: -- AUC3.3.3
    :A: -- AUC3.3.4

    @enduml
  \end{plantuml}
  \caption{AUC3.3: Modifica organizzazione}%
  \label{fig:AUC3_3}
\end{figure}

\begin{description}
  \item[Codice:] AUC3.3;
  \item[Titolo:] Modifica organizzazione;
  \item[Attori primari:] amministratore, gestore;
  \item[Precondizione:] l'amministratore seleziona l'organizzazione da modificare, presente nella lista di \emph{Stalker};
  \item[Postcondizione:] l'organizzazione viene modificata;
  \item[Scenario principale:]
        \begin{enumerate}
          \item sorge la necessità di modificare un'organizzazione, senza interagire con il suo owner.
        \end{enumerate}
  \item[Inclusioni:]
        \begin{enumerate}
          \item viene scelta un'organizzazione \emph{[AUC3.4]}.
        \end{enumerate}
\end{description}

\subsubsection{AUC3.3.1: Modifica nome organizzazione}%
\label{subs:AUC3.3.1}
\begin{description}
  \item[Codice:] AUC3.3.1;
  \item[Titolo:] Modifica nome organizzazione;
  \item[Attori primari:] amministratore, owner;
  \item[Precondizione:] il sistema fornisce il campo di modifica del nome;
  \item[Postcondizione:] il nome viene opportunamente modificato;
  \item[Scenario principale:]
        \begin{enumerate}
          \item si vuole modificare il nome di un'organizzazione.
        \end{enumerate}
\end{description}

\subsubsection{AUC3.3.2: Modifica descrizione organizzazione}%
\label{subs:AUC3.3.2}
\begin{description}
  \item[Codice:] AUC3.3.2;
  \item[Titolo:] Modifica descrizione organizzazione;
  \item[Attori primari:] amministratore, owner;
  \item[Precondizione:] il sistema fornisce il campo di modifica della descrizione dell'organizzazione;
  \item[Postcondizione:] la descrizione viene opportunamente modificata;
  \item[Scenario principale:]
        \begin{enumerate}
          \item si vuole modificare la descrizione di un'organizzazione.
        \end{enumerate}
\end{description}

%gestione luoghi (start)

\subsubsection{AUC3.3.3: Gestione luoghi}%
\label{subs:AUC3.3.3}

\begin{figure}[H]
  \centering
  \begin{plantuml}
  @startuml
  !include ../../commons/style/use-cases.pu

  actor :amministratore: as A

  rectangle {
    together {
      usecase (AUC3.3.3.1) as "AUC3.3.3.1\nAggiungi luogo"
      usecase (AUC3.3.3.2) as "AUC3.3.3.2\nEliminazione luogo"
      usecase (AUC3.3.3.3) as "AUC3.3.3.3\nModifica luogo"
    }
    usecase (AUC3.3.3.4) as "AUC3.3.3.4\nSeleziona luogo"
  }

  :A: -- AUC3.3.3.1
  :A: -- AUC3.3.3.2
  :A: -- AUC3.3.3.3

  (AUC3.3.3.4) .up.|> (AUC3.3.3.2) : <<include>>
  (AUC3.3.3.4) .up.|> (AUC3.3.3.3) : <<include>>

  @enduml
  \end{plantuml}
  \caption{AUC3.3.3: Gestione luoghi}%
  \label{fig:AUC3_3_3}
\end{figure}

\begin{description}
  \item[Codice:] AUC3.3.3;
  \item[Titolo:] Gestione luoghi;
  \item[Attori primari:] amministratore, gestore;
  \item[Precondizione:] il sistema deve rendere disponibile la pagina di gestione dei luoghi di un'organizzazione;
  \item[Postcondizione:] vengono gestiti i luoghi di un'organizzazione;
  \item[Scenario principale:]
  \begin{enumerate}
    \item sorge la necessità di effettuare operazioni sul luogo di un'organizzazione, e viene offerta la possibilità di selezionarlo.
  \end{enumerate}
\end{description}

\subsubsection{AUC3.3.3.1: Aggiungi luogo}%
\label{subs:AUC3.3.3.1}

\begin{figure}[H]
  \centering
  \begin{plantuml}
  @startuml
  !include ../../commons/style/use-cases.pu

  actor :amministratore: as A

  rectangle {
    together {
      usecase (AUC3.3.3.1.1) as "AUC3.3.3.1.1\nInserisci coordinate geografiche"
      usecase (AUC3.3.3.1.2) as "AUC3.3.3.1.2\nInserisci indirizzo luogo"
    }
  }

  :A: -- AUC3.3.3.1.1
  :A: -- AUC3.3.3.1.2

  @enduml
  \end{plantuml}
  \caption{AUC3.3.3.1: Aggiungi luogo}%
  \label{fig:AUC3_3_3_1}
\end{figure}

\begin{description}
  \item[Codice:] AUC3.3.3.1;
  \item[Titolo:] Aggiungi luogo;
  \item[Attori primari:] amministratore, gestore;
  \item[Precondizione:] il \glossario{luogo} da aggiungere nell'organizzazione non deve esistere;
  \item[Postcondizione:] il nuovo luogo viene aggiunto nell'organizzazione;
  \item[Scenario principale:]
  \begin{enumerate}
    \item sorge la necessità di aggiungere un luogo ad un'organizzazione.
  \end{enumerate}
\end{description}

\subsubsection{AUC3.3.3.1.1: Inserisci coordinate geografiche}%
\label{subs:AUC3.3.3.1.1}
\begin{description}
  \item[Codice:] AUC3.3.3.1.1;
  \item[Titolo:] Inserisci coordinate geografiche;
  \item[Attori primari:] amministratore, gestore;
  \item[Precondizione:] il sistema deve fornire i campi relativi alle coordinate geografiche;
  \item[Postcondizione:] le coordinate geografiche sono inserite correttamente nei loro campi;
  \item[Scenario principale:]
  \begin{enumerate}
    \item l'amministratore inserisce le coordinate geografiche di un nuovo luogo.
  \end{enumerate}
\end{description}

\subsubsection{AUC3.3.3.1.2: Inserisci indirizzo luogo}%
\label{subs:AUC3.3.3.1.2}
\begin{description}
  \item[Codice:] AUC3.3.3.1.2;
  \item[Titolo:] Inserisci indirizzo luogo;
  \item[Attori primari:] amministratore, gestore;
  \item[Precondizione:] il sistema deve fornire il campo relativo all'indirizzo del luogo;
  \item[Postcondizione:] l'indirizzo è inserito correttamente nei loro campi;
  \item[Scenario principale:]
  \begin{enumerate}
    \item l'amministratore inserisce l'indirizzo di un nuovo luogo.
  \end{enumerate}
\end{description}

\subsubsection{AUC3.3.3.2: Eliminazione luogo}%
\label{subs:AUC3.3.3.2}
\begin{description}
  \item[Codice:] AUC3.3.3.2;
  \item[Titolo:] Eliminazione luogo;
  \item[Attori primari:] amministratore, gestore;
  \item[Precondizione:] il luogo dell'organizzazione deve essere presente in \emph{Stalker};
  \item[Postcondizione:] il luogo dell'organizzazione viene eliminato;
  \item[Scenario principale:]
  \begin{enumerate}
    \item sorge la necessità di eliminare un luogo di un'organizzazione, senza interagire con il suo owner.
  \end{enumerate}
  \item[Inclusioni:]
  \begin{enumerate}
    \item Seleziona luogo\emph{[AUC3.3.3.4]};
  \end{enumerate}
\end{description}

\subsubsection{AUC3.3.3.3: Modifica luogo}%
\label{subs:AUC3.3.3.3}

\begin{figure}[H]
  \centering
  \begin{plantuml}
  @startuml
  !include ../../commons/style/use-cases.pu

  actor :amministratore: as A

  rectangle {
    together {
      usecase (AUC3.3.3.3.1) as "AUC3.3.3.3.1\nModifica coordinate geografiche"
      usecase (AUC3.3.3.3.2) as "AUC3.3.3.3.2\nModifica indirizzo luogo"
    }
  }

  :A: -- AUC3.3.3.3.1
  :A: -- AUC3.3.3.3.2

  @enduml
  \end{plantuml}
  \caption{AUC3.3.3.3: Modifica luogo}%
  \label{fig:AUC3_3_3_3}
\end{figure}

\begin{description}
  \item[Codice:] AUC3.3.3.3;
  \item[Titolo:] Modifica luogo;
  \item[Attori primari:] amministratore, gestore;
  \item[Precondizione:] il luogo dell'organizzazione deve essere presente in \emph{Stalker};
  \item[Postcondizione:] il luogo dell'organizzazione viene modificato;
  \item[Scenario principale:]
  \begin{enumerate}
    \item sorge la necessità di modificare un luogo di un'organizzazione, senza interagire con il suo owner.
  \end{enumerate}
  \item[Inclusioni:]
  \begin{enumerate}
    \item Seleziona luogo\emph{[AUC3.3.3.4]};
  \end{enumerate}
\end{description}

\subsubsection{AUC3.3.3.3.1: Modifica coordinate geografiche}%
\label{subs:AUC3.3.3.3.1}
\begin{description}
  \item[Codice:] AUC3.3.3.3.1;
  \item[Titolo:] Modifica coordinate geografiche;
  \item[Attori primari:] amministratore, gestore;
  \item[Precondizione:] il sistema deve fornire i campi relativi alle coordinate geografiche;
  \item[Postcondizione:] i campi relativi alle coordinate geografiche sono modificati;
  \item[Scenario principale:]
  \begin{enumerate}
    \item l'amministratore vuole modificare le coordinate geografiche di un luogo.
  \end{enumerate}
\end{description}

\subsubsection{AUC3.3.3.3.2: Modifica indirizzo luogo}%
\label{subs:AUC3.3.3.3.2}
\begin{description}
  \item[Codice:] AUC3.3.3.3.2;
  \item[Titolo:] Modifica indirizzo luogo;
  \item[Attori primari:] amministratore, gestore;
  \item[Precondizione:] il sistema deve fornire il campo relativo all'indirizzo del luogo;
  \item[Postcondizione:] il campo relativo all'indirizzo del luogo è modificato;
  \item[Scenario principale:]
  \begin{enumerate}
    \item l'amministratore vuole modificare l'indirizzo di un luogo.
  \end{enumerate}
\end{description}

\subsubsection{AUC3.3.3.4: Seleziona luogo}%
\label{subs:AUC3.3.3.4}
\begin{description}
  \item[Codice:] AUC3.3.3.4;
  \item[Titolo:] Seleziona luogo;
  \item[Attori primari:] amministratore, gestore;
  \item[Precondizione:] il sistema deve mostrare la lista dei luoghi all'interno di una organizzazione;
  \item[Postcondizione:] viene scelto il luogo desiderato;
  \item[Scenario principale:]
  \begin{enumerate}
    \item sorge la necessità di effettuare operazioni sul luogo di un'organizzazione, e viene offerta la possibilità di selezionarlo.
  \end{enumerate}
\end{description}

%gestione luoghi (end)

\subsubsection{AUC3.3.4: Modifica configurazione dettagli server LDAP}%
\label{subs:AUC3.3.4}
\begin{description}
  \item[Codice:] AUC3.3.4;
  \item[Titolo:] Configurazione dettagli server LDAP;
  \item[Attori primari:] amministratore, gestore;
  \item[Precondizione:] il sistema fornisce i campi per la configurazione del server LDAP;
  \item[Postcondizione:] il server LDAP è stato configurato;
  \item[Scenario principale:]
        \begin{enumerate}
          \item si vuole modificare la configurazione del server LDAP.
        \end{enumerate}
\end{description}

\subsubsection{AUC3.4: Seleziona organizzazione}%
\label{subs:AUC3.4}
\begin{description}
  \item[Codice:] AUC3.4;
  \item[Titolo:] Seleziona organizzazione;
  \item[Attori primari:] amministratore, gestore;
  \item[Precondizione:] il sistema deve mostrare la lista di organizzazioni in \emph{Stalker};
  \item[Postcondizione:] viene scelta l'organizzazione desiderata;
  \item[Scenario principale:]
        \begin{enumerate}
          \item sorge la necessità di effettuare operazioni su un'organizzazione, e viene offerta la possibilità di selezionarla.
        \end{enumerate}
\end{description}

\subsubsection{AUC3.5: Invio richiesta aggiornamento lista organizzazioni}%
\label{subs:AUC3.5}
\begin{description}
  \item[Codice:] AUC3.5;
  \item[Titolo:] Invio richiesta aggiornamento lista organizzazioni;
  \item[Attori primari:] amministratore, gestore;
  \item[Precondizione:] il sistema mostra la pagina di creazione o eliminazione di un'organizzazione;
  \item[Postcondizione:] la nuova lista delle organizzazioni viene inviata a tutte le applicazioni mobile;
  \item[Scenario principale:]
        \begin{enumerate}
          \item una volta creata o eliminata un'organizzazione, la lista delle organizzazioni viene aggiornata e inviata a tutti gli utenti che hanno installato l'applicazione mobile.
        \end{enumerate}
\end{description}

\end{document}
