\documentclass[../../../analisi-dei-requisiti.tex]{subfiles}

\begin{document}


\begin{figure}[H]
  \centering
  \begin{plantuml}
  @startuml
  !include ../../commons/style/use-cases.pu

  actor :gestore: as A

  rectangle {
    together {
      usecase (AUC9.1) as "AUC10.1\nAggiunta luogo"
      usecase (AUC9.2) as "AUC10.2\nRimozione luogo"
      usecase (AUC9.3) as "AUC10.3\nModifica luogo"
    }
    together {
      usecase (AUC5.3.3.3.1) as "AUC5.3.3.3.1\nModifica coordinate geografiche"
      usecase (AUC5.3.3.3.2) as "AUC5.3.3.3.2\nModifica indirizzo luogo"
      usecase (AUC5.3.3.1.1) as "AUC5.3.3.1.1\nInserisci coordinate geografiche"
      usecase (AUC5.3.3.1.2) as "AUC5.3.3.1.2\nInserisci indirizzo luogo"
    }
  }

  :A: -- AUC9.1
  :A: -- AUC9.2
  :A: -- AUC9.3

  (AUC5.3.3.3.1) .up.|> (AUC9.3) : <<include>>
  (AUC5.3.3.3.2) .up.|> (AUC9.3) : <<include>>
  (AUC5.3.3.1.1) .up.|> (AUC9.1) : <<include>>
  (AUC5.3.3.1.2) .up.|> (AUC9.1) : <<include>>

  @enduml
  \end{plantuml}
  \caption{AUC9: Gestione luoghi del gestore}%
  \label{fig:AUC9}
\end{figure}

\begin{description}
  \item[Codice:] AUC9
  \item[Titolo:] Gestione luoghi del gestore
  \item[Attori primari:] gestore
  \item[Precondizione:] il sistema deve rendere disponibile la pagina della gestione dei luoghi.
  \item[Postcondizione:] il gestore è all'interno della pagina di gestione.
  \item[Scenario principale:]
  \begin{enumerate}
    \item il gestore vuole gestire i luoghi dell'organizzazione su cui opera;
  \end{enumerate}
\end{description}

\subsubsection{AUC9.1: Aggiunta luogo}%
\label{subs:AUC9.1}
\begin{description}
  \item[Codice:] AUC9.1
  \item[Titolo:] Aggiunta luogo
  \item[Attori primari:] gestore
  \item[Precondizione:] il luogo da aggiungere non deve già esistere;
  \item[Postcondizione:] viene aggiunto un nuovo luogo all'organizzazione corrente.
  \item[Scenario principale:]
  \begin{enumerate}
    \item il gestore vuole aggiungere un nuovo luogo all'organizzazione su cui opera;
  \end{enumerate}
  \item[Inclusioni:]
  \begin{enumerate}
    \item il gestore inserisce le coordinate geografiche da aggiungere \emph{[AUC5.3.3.1.1]};
    \item il gestore inseriscce l'indirizzo del luogo da aggiungere \emph{[AUC5.3.3.1.2]};
  \end{enumerate}
\end{description}

\subsubsection{AUC9.2: Rimozione luogo}%
\label{subs:AUC9.2}
\begin{description}
  \item[Codice:] AUC9.2
  \item[Titolo:] Rimozione luogo
  \item[Attori primari:] gestore
  \item[Precondizione:] il luogo da eliminare deve esistere.
  \item[Postcondizione:] il luogo selezionato è eliminato.
  \item[Scenario principale:]
  \begin{enumerate}
    \item il gestore vuole eliminare un luogo all'organizzazione su cui opera;
  \end{enumerate}
\end{description}

\subsubsection{AUC9.3: Modifica luogo}%
\label{subs:AUC9.3}
\begin{description}
  \item[Codice:] AUC9.3
  \item[Titolo:] Modifica luogo
  \item[Attori primari:] gestore
  \item[Precondizione:] il luogo da modificare deve esistere.
  \item[Postcondizione:] il luogo selezionato è modificato.
  \item[Scenario principale:]
  \begin{enumerate}
    \item il gestore vuole modificare un luogo dell'organizzazione su cui opera;
  \end{enumerate}
  \item[Inclusioni:]
  \begin{enumerate}
    \item il gestore modifica le coordinate geografiche del luogo selezionato \emph{[AUC5.3.3.3.1]};
    \item il gestore modifica l'indirizzo del luogo selezionato \emph{[AUC5.3.3.3.2]};
  \end{enumerate}
\end{description}



\end{document}
