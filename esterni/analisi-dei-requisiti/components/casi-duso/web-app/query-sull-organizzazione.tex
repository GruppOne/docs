\documentclass[../../../analisi-dei-requisiti.tex]{subfiles}

\begin{document}

\begin{figure}[h!]
  \centering
  \begin{plantuml}
  @startuml
  !include ../../commons/style/use-cases.pu
  scale 3/4

  actor :visualizzatore: as A

  rectangle {
    together {
      usecase (AUC7) as "AUC7\nQuery sull'organizzazione"
    }
  }

  :A: -- AUC7

  @enduml
  \end{plantuml}
  \caption{AUC7: Query sull'organizzazione}
  \label{fig:AUC7}
\end{figure}

\begin{description}
  \item[Codice:] AUC7
  \item[Titolo:] Query sull'organizzazione
  \item[Attori primari:] visualizzatore
  \item[Precondizione:] il sistema risponde correttamente alle interrogazioni;
  \item[Postcondizione:] il visualizzatore ottiene le informazioni di cui ha bisogno.
  \item[Scenario principale:]
  \begin{enumerate}
    \item il visualizzatore vuole ottenere delle informazioni riguardo l'organizzazione sulla quale opera;
  \end{enumerate}
\end{description}
% subsub:AUC7 (end)

\end{document}
