\documentclass[../../../analisi-dei-requisiti.tex]{subfiles}

\begin{document}


\begin{figure}[h!]
  \centering
  \begin{plantuml}
  @startuml
  !include ../../../../../commons/style/use-cases.pu

  actor :owner: as A

  rectangle {
    together {
      usecase (AUC11) as "AUC11\nRichiesta creazione organizzazione"
    }
    together {
      usecase (AUC5.1.3) as "AUC5.1.3\nConfigurazione dettagli server LDAP"
      usecase (AUC5.1.2) as "AUC5.1.2\nInserisci descrizione organizzazione"
      usecase (AUC5.1.1) as "AUC5.1.1\nInserisci nome organizzazione"
    }
  }

  :A: -- AUC11

  (AUC5.1.1) .up.|> (AUC11) : <<include>>
  (AUC5.1.2) .up.|> (AUC11) : <<include>>
  (AUC5.1.3) .up.|> (AUC11) : <<include>>

  @enduml
  \end{plantuml}
  \caption{AUC11: Richiesta creazione organizzazione}
  \label{fig:auc11}
\end{figure}

\begin{description}
  \item[Codice:] AUC11
  \item[Titolo:] Richiesta creazione organizzazione
  \item[Attori primari:] owner
  \item[Precondizione:] l'organizzazione non deve già esistere.
  \item[Postcondizione:] la richiesta di creare una nuova organizzazione è stata posta.
  \item[Scenario principale:]
  \begin{enumerate}
    \item l'owner vuole creare una nuova organizzazione.
  \end{enumerate}
  \item[Inclusioni:]
  \begin{itemize}
    \item In caso di richiesta accettata, sono permesse tutte le operazioni di modifica:
    \begin{enumerate}
      \item Inserimento nome organizzazione \emph{[AUC5.1.1]}.
      \item Inserimento descrizione organizzazione \emph{[AUC5.1.2]}.
      \item Configurazione dettagli server LDAP \emph{[AUC5.1.3]}
    \end{enumerate}
  \end{itemize}
\end{description}


\end{document}
