\documentclass[../../../analisi-dei-requisiti.tex]{subfiles}

\begin{document}

\begin{figure}[H]
  \centering
  \begin{plantuml}
  @startuml
  !include ../../commons/style/use-cases.pu

  actor :owner: as A

  rectangle {
    together {
      usecase (AUC7.1) as "AUC7.1\nInserimento email gestore"
    }
  }

  :A: -- AUC7.1

  @enduml
  \end{plantuml}
  \caption{AUC7: Aggiunta gestore}%
  \label{fig:AUC7}
\end{figure}

\begin{description}
  \item[Codice:] AUC7;
  \item[Titolo:] Aggiunta gestore;
  \item[Attori primari:] amministratore, owner;
  \item[Precondizione:] il sistema deve rendere disponibile la pagina per l'aggiunta del gestore, l'utente deve avere i requisiti per diventare gestore:
  \begin{enumerate}
    \item l'utente deve essere collegato o essersi collegato all'organizzazione nella quale svolgere il ruolo di gestore;
    \item l'utente deve essere un utente senza privilegi o un visualizzatore.
  \end{enumerate}
  \item[Postcondizione:] viene aggiunto il gestore specificato dall'owner o dall'amministratore;
  \item[Scenario principale:]
  \begin{enumerate}
    \item l'owner vuole aggiungere un gestore ad una delle sue organizzazioni.
  \end{enumerate}
\end{description}

\subsubsection{AUC7.1: Inserimento email gestore}%
\label{subs:AUC7.1}
\begin{description}
  \item[Codice:] AUC7.1;
  \item[Titolo:] Inserimento email gestore;
  \item[Attori primari:] amministratore, owner;
  \item[Precondizione:] il sistema deve rendere disponibile il campo per l'inserimento della mail del nuovo gestore;
  \item[Postcondizione:] il campo relativo al gestore viene compilato;
  \item[Scenario principale:]
  \begin{enumerate}
    \item l'amministratore o l'owner inserisce l'email del nuovo gestore.
  \end{enumerate}
\end{description}

\end{document}
