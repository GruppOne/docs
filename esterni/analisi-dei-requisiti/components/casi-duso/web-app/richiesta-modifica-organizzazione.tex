\documentclass[../../../analisi-dei-requisiti.tex]{subfiles}

\begin{document}


\begin{figure}[H]
  \centering
  \begin{plantuml}
  @startuml
  !include ../../commons/style/use-cases.pu

  actor :owner: as A

  rectangle {
    together {
      usecase (AUC10) as "AUC10\nRichiesta modifica organizzazione"
    }
    together {
      usecase (AUC5.3.4) as "AUC5.3.4\nModifica configurazione dettagli server LDAP"
      usecase (AUC5.3.3) as "AUC5.3.3\nModifica descrizione organizzazione"
      usecase (AUC5.3.2) as "AUC5.3.2\nModifica indirizzo organizzazione"
      usecase (AUC5.3.1) as "AUC5.3.1\nModifica nome organizzazione"
    }
  }

  :A: -- AUC10

  (AUC5.3.1) .up.|> (AUC10) : <<include>>
  (AUC5.3.2) .up.|> (AUC10) : <<include>>
  (AUC5.3.3) .up.|> (AUC10) : <<include>>
  (AUC5.3.4) .up.|> (AUC10) : <<include>>

  @enduml
  \end{plantuml}
  \caption{AUC10: Richiesta modifica organizzazione}%
  \label{fig:AUC10}
\end{figure}

\begin{description}
  \item[Codice:] AUC10
  \item[Titolo:] Richiesta modifica organizzazione
  \item[Attori primari:] owner
  \item[Precondizione:] l'organizzazione deve esistere.
  \item[Postcondizione:] la richiesta di modificare un+organizzazione esistente è stata posta.
  \item[Scenario principale:]
  \begin{enumerate}
    \item l'owner vuole modificare un'organizzazione di sua gestione;
  \end{enumerate}
  \item[Inclusioni:]
  \begin{itemize}
    \item In caso di richiesta accettata, sono permesse tutte le operazioni di modifica:
    \begin{enumerate}
      \item inserimento nome organizzazione \emph{[AUC5.3.1]};
      \item inserimento indirizzo organizzazione \emph{[AUC5.3.2]};
      \item inserimento descrizione organizzazione \emph{[AUC5.3.3]};
      \item configurazione dettagli server LDAP \emph{[AUC5.3.4]}.
    \end{enumerate}
  \end{itemize}
\end{description}


\end{document}
