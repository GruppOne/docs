\documentclass[../../../analisi-dei-requisiti.tex]{subfiles}

\begin{document}

\begin{figure}[h!]
  \centering
  \begin{plantuml}
  @startuml
  !include ../../commons/style/use-cases.pu
  scale 3/4

  actor :utente non autenticato: as A

  rectangle {
    together {
      usecase (AUC1.1) as "AUC1.1\nAutenticazione\n--\nExtension points:\nVisualizzazione errore se\nle credenziali inserite\n non sono corrette"
      usecase (AUC1.2) as "AUC1.2\nVisualizzazione credenziali errate"
    }
  }

  :A: -- AUC1.1

  (AUC1.3) .up.|> (AUC1.1) : <<extends>>

  @enduml
  \end{plantuml}
  \caption{AUC1: Sistema di autenticazione}
  \label{fig:auc1}
\end{figure}

\begin{description}
  \item[Codice:] AUC1
  \item[Titolo:] Sistema di autenticazione
  \item[Attori primari:] utente non autenticato
  \item[Precondizione:] l'utente non è autenticato alla piattaforma.
  \item[Postcondizione:] l'\glossario{utente} ha effettuato correttamente l'autenticazione nel sistema.
  \item[Scenario principale:]
  \begin{enumerate}
    \item l'utente non è ancora autenticato e vuole eseguire l'autenticazione.
  \end{enumerate}
\end{description}

\subsubsection{AUC1.1 - Autenticazione}%
\label{subs:AUC1.1}

\begin{figure}[h!]
  \centering
  \begin{plantuml}
  @startuml
  !include ../../commons/style/use-cases.pu
  scale 3/4

  actor :utente non autenticato: as A

  rectangle {
    together {
    usecase (AUC1.1.1) as "AUC1.1.1\nInserimento email"
    usecase (AUC1.1.2) as "AUC1.1.2\nInserimento password"
    }
  }

  :A: -- AUC1.1.1
  :A: -- AUC1.1.2

  @enduml
  \end{plantuml}
  \caption{AUC1.1: Autenticazione}
  \label{fig:auc1_1}
\end{figure}

\begin{description}
  \item[Codice:] AUC1.1
  \item[Titolo:] Autenticazione
  \item[Attori primari:] utente non autenticato
  \item[Precondizione:] il sistema è raggiungibile e funzionante, l'utente non autenticato deve poter visualizzare la pagina di autenticazione.
  \item[Postcondizione:] l'autenticazione è andata a buon fine e l'utente è autenticato.
  \item[Scenario principale:]
  \begin{enumerate}
    \item  l'utente non autenticato accede alla pagina di autenticazione, e visualizza tutti i campi che deve compilare:
    \begin{enumerate}
      \item inserisce l’email associata all’account \emph{[AUC1.1.1]};
      \item inserisce la password associata all’account \emph{[AUC1.1.2]}.
    \end{enumerate}
    \item
  \end{enumerate}
  \item[Estensioni:]
  \begin{enumerate}
    \item se l'utente inserisce le credenziali in modo errato, verrà visualizzato un messaggio d'errore \emph{[AUC1.2]}.
  \end{enumerate}
\end{description}

\subsubsection{AUC1.1.1 - Inserimento email}%
\label{subs:AUC1.1.1}
\begin{description}
  \item[Codice:] AUC1.1.1
  \item[Titolo:] Inserimento email
  \item[Attori primari:] utente non autenticato
  \item[Precondizione:] il sistema ha reso disponibile il campo per l'inserimento della propria email.
  \item[Postcondizione:] l'utente ha compilato il campo relativo alla propria email.
  \item[Scenario principale:]
  \begin{enumerate}
    \item l'utente compila il campo relativo alla propria mail di registrazione.
  \end{enumerate}
\end{description}

\subsubsection{AUC1.1.2 - Inserimento password}%
\label{subs:AUC1.1.2}
\begin{description}
  \item[Codice:] AUC1.1.2
  \item[Titolo:] Inserimento password
  \item[Attori primari:] utente non autenticato
  \item[Precondizione:] il sistema ha reso disponibile il campo per l'inserimento della password.
  \item[Postcondizione:] l'utente ha compilato il campo relativo alla sua password.
  \item[Scenario principale:]
  \begin{enumerate}
    \item l'utente compila il campo relativo alla propria password di registrazione.
  \end{enumerate}
\end{description}

\subsubsection{AUC1.2 - Visualizzazione credenziali errate}%
\label{subs:AUC1.2}
\begin{description}
  \item[Codice:] AUC1.2
  \item[Titolo:] Visualizzazione credenziali errate
  \item[Attori primari:] utente non autenticato
  \item[Precondizione:] l'utente ha inviato al server le credenziali per effettuare l'autenticazione.
  \item[Postcondizione:] l'utente non autenticato visualizza un messaggio di credenziali sbagliate.
  \item[Scenario principale:]
  \begin{enumerate}
    \item l'utente cerca di effettuare l'autenticazione con credenziali errate.
  \end{enumerate}
\end{description}

\end{document}
