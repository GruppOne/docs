\documentclass[../../../analisi-dei-requisiti.tex]{subfiles}

\begin{document}

\begin{figure}[h!]
  \centering
  \begin{plantuml}
  @startuml
  !include ../../commons/style/use-cases.pu
  scale 3/4

  actor :utente autenticato: as A

  rectangle {
    together {
      usecase (AUC2.1) as "AUC2.1\nInvio richiesta owner\n--\nExtension points:\nVisualizzazione errore nel\ncaso in cui la richiesta venga\nrifiutata dall'amministratore"
      usecase (AUC2.2) as "AUC2.2\nVerifica tipologia utente\n--\nExtension points:\nVisualizzazione errore nel\ncaso in cui l'utente\nautenticato sia già owner"
    }
      usecase (AUC2.3) as "AUC2.3\nRichiesta rifiutata"
    }

  :A: -- AUC2.1

  (AUC2.2) .right.|> (AUC2.1) : <<include>>

  (AUC2.3) .up.|> (AUC2.1) : <<extends>>
  (AUC2.3) .up.|> (AUC2.2) : <<extends>>

  @enduml
  \end{plantuml}
  \caption{AUC2: Richiesta owner}
  \label{fig:auc2}
\end{figure}

\begin{description}
  \item[Codice:] AUC2
  \item[Titolo:] Richiesta owner
  \item[Attori primari:] utente autenticato
  \item[Precondizione:] l'utente autenticato non deve essere un \glossario{owner}.
  \item[Postcondizione:] la richiesta viene accettata e l'utente autenticato diventa un owner.
  \item[Scenario principale:]
  \begin{enumerate}
    \item l'utente autenticato vuole iniziare ad utilizzare \emph{Stalker} creando una sua organizzazione.
  \end{enumerate}
\end{description}

\subsubsection{AUC2.1 - Invio richiesta owner}%
\label{subs:AUC2.1}
\begin{description}
  \item[Codice:] AUC2.1
  \item[Titolo:] Invio richiesta owner
  \item[Attori primari:] utente autenticato
  \item[Precondizione:] l'utente autenticato utilizza l'apposito form per inviare la richiesta di diventare owner.
  \item[Postcondizione:] viene inviata la richiesta, in attesa di essere approvata da un amministratore.
  \item[Scenario principale:]
  \begin{enumerate}
    \item l'utente autenticato invia la richiesta per diventare owner.
  \end{enumerate}
  \item[Inclusioni:]
  \begin{enumerate}
    \item viene effettuato un controllo sulla tipologia di utente che vuole effettuare la richiesta \emph{[AUC2.2]};
  \end{enumerate}
  \item[Estensioni:]
  \begin{enumerate}
    \item la richiesta di diventare owner può essere rifiutata \emph{[AUC2.3]}.
  \end{enumerate}
\end{description}

\subsubsection{AUC2.2 - Verifica tipologia utente}%
\label{subs:AUC2.2}
\begin{description}
  \item[Codice:] AUC2.2
  \item[Titolo:] Verifica tipologia utente
  \item[Attori primari:] utente autenticato
  \item[Precondizione:] l'utente autenticato visualizza l'apposito form per inviare la richiesta di diventare owner.
  \item[Postcondizione:] il form relativo alla richiesta di owner viene abilitato o disabilitato di conseguenza.
  \item[Scenario principale:]
  \begin{enumerate}
    \item l'utente autenticato può richiedere di diventare un owner, nel caso in cui sia:
    \begin{description}
      \item un \glossario{amministratore},
      \item un \glossario{visualizzatore},
      \item un \glossario{gestore},
      \item oppure un utente senza alcun privilegio;
    \end{description}
    \item ogni tentativo di richiesta da parte di un utente che è già owner viene rifiutato;
    \item \glossario{root} è già un owner;
    \item l'amministratore può accettare la propria richiesta inviata.
  \end{enumerate}
\end{description}

\subsubsection{AUC2.3 - Richiesta rifiutata}%
\label{subs:AUC2.3}
\begin{description}
  \item[Codice:] AUC2.3
  \item[Titolo:] Richiesta rifiutata
  \item[Attori primari:] utente autenticato
  \item[Precondizione:] l'utente ha inviato la propria richiesta per diventare owner.
  \item[Postcondizione:] l'utente autenticato non ha ottenuto l'abilitazione a diventare un owner e visualizza un messaggio d'errore.
  \item[Scenario principale:]
  \begin{enumerate}
    \item l'amministratore non ha accettato la richiesta effettuata dall'utente autenticato.
  \end{enumerate}
\end{description}

\end{document}
