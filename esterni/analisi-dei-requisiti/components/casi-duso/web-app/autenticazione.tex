\documentclass[../../../analisi-dei-requisiti.tex]{subfiles}

\begin{document}

\begin{figure}[H]
  \centering
  \begin{plantuml}
  @startuml
  !include ../../commons/style/use-cases.pu

  actor :utente non autenticato: as A

  rectangle {
    together {
    usecase (AUC1.1) as "AUC1.1\nInserimento email"
    usecase (AUC1.2) as "AUC1.2\nInserimento password"
    }
    usecase (AUC1.3) as "AUC1.3\nVisualizzazione credenziali errate"
  }

  :A: -- AUC1.1
  :A: -- AUC1.2

  (AUC1.3) .left.|> (AUC1.1) : <<extends>>
  (AUC1.3) .up.|> (AUC1.2) : <<extends>>

  @enduml
  \end{plantuml}
  \caption{AUC1: Autenticazione}%
  \label{fig:auc1}
\end{figure}

\begin{description}
  \item[Codice:] AUC1;
  \item[Titolo:] Autenticazione;
  \item[Attori primari:] utente non autenticato;
  \item[Precondizione:] il sistema è raggiungibile e funzionante, l'utente non autenticato deve poter visualizzare la pagina di autenticazione;
  \item[Postcondizione:] l'autenticazione è andata a buon fine e l'utente è autenticato.
  \item[Scenario principale:]
  \begin{enumerate}
    \item l'utente non autenticato accede alla pagina di autenticazione, e visualizza tutti i campi che deve compilare:
    \begin{enumerate}
      \item inserisce l’email associata all’account \emph{[AUC1.1]};
      \item inserisce la password associata all’account \emph{[AUC1.2]}.
    \end{enumerate}
    \item
  \end{enumerate}
  \item[Estensioni:]
  \begin{enumerate}
    \item se l'utente inserisce le credenziali in modo errato, verrà visualizzato un messaggio d'errore \emph{[AUC1.3]}.
  \end{enumerate}
\end{description}

\subsubsection{AUC1.1: Inserimento email}%
\label{subs:AUC1.1}
\begin{description}
  \item[Codice:] AUC1.1;
  \item[Titolo:] Inserimento email;
  \item[Attori primari:] utente non autenticato;
  \item[Precondizione:] il sistema ha reso disponibile il campo per l'inserimento della propria email;
  \item[Postcondizione:] l'utente ha compilato il campo relativo alla propria email;
  \item[Scenario principale:]
  \begin{enumerate}
    \item l'utente compila il campo relativo alla propria mail di registrazione.
  \end{enumerate}
  \item[Estensioni:]
  \begin{enumerate}
    \item se l'utente inserisce l'email in modo errato, verrà visualizzato un messaggio d'errore \emph{[AUC1.3]}.
  \end{enumerate}
\end{description}

\subsubsection{AUC1.2: Inserimento password}%
\label{subs:AUC1.2}
\begin{description}
  \item[Codice:] AUC1.2;
  \item[Titolo:] Inserimento password;
  \item[Attori primari:] utente non autenticato;
  \item[Precondizione:] il sistema ha reso disponibile il campo per l'inserimento della password;
  \item[Postcondizione:] l'utente ha compilato il campo relativo alla sua password;
  \item[Scenario principale:]
  \begin{enumerate}
    \item l'utente compila il campo relativo alla propria password di registrazione.
  \end{enumerate}
  \item[Estensioni:]
  \begin{enumerate}
    \item se l'utente inserisce la password in modo errato, verrà visualizzato un messaggio d'errore \emph{[AUC1.3]}.
  \end{enumerate}
\end{description}

\subsubsection{AUC1.3: Visualizzazione credenziali errate}%
\label{subs:AUC1.3}
\begin{description}
  \item[Codice:] AUC1.3;
  \item[Titolo:] Visualizzazione credenziali errate;
  \item[Attori primari:] utente non autenticato;
  \item[Precondizione:] l'utente ha inviato al server le credenziali per effettuare l'autenticazione;
  \item[Postcondizione:] l'utente non autenticato visualizza un messaggio di credenziali sbagliate;
  \item[Scenario principale:]
  \begin{enumerate}
    \item l'utente cerca di effettuare l'autenticazione con credenziali errate.
  \end{enumerate}
\end{description}

\end{document}
