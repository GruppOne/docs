\documentclass[../../../analisi-dei-requisiti.tex]{subfiles}

\begin{document}

\begin{figure}[h!]
  \centering
  \begin{plantuml}
  @startuml
  !include ../../commons/style/use-cases.pu

  actor :utente autenticato: as A3
  actor :utente collegato: as A3.1
  :A3.1: -up-|> :A3:

  rectangle {
    together {
      usecase (UUC9.1) as "UUC9.1\nScelta noto o incognito\n--\nExtension points:\nVisualizzazione errore in caso di\noperazione di cambio contesto\n con rete mancante"
      usecase (UUC5) as "UUC5\nVisualizzazione errore\nrete mancante"
    }
  }

  :A3.1: -- (UUC9.1)

  (UUC5) .up.|> (UUC9.1) : <<extends>>

  @enduml
  \end{plantuml}
  \caption{UUC9: Passaggio noto/incognito}
  \label{fig:uuc9}
\end{figure}

\begin{description}
  \item[Caso d’uso:] UUC9
  \item[Titolo:] Passaggio noto/incognito
  \item[Attori primari:] utente autenticato, in particolare utente collegato
  \item[Precondizione:] l'utente si trova sulla schermata specifica dell'utente collegato ad un'organizzazione.
  \item[Postcondizione:] se l'utente era noto diventa incognito; vale anche il contrario.
  \item[Scenario principale:]
        \begin{enumerate}
          \item l'utente ha la possibilità di cambiare il tipo di monitoraggio che l'organizzazione è tenuta a conoscere.
        \end{enumerate}
  \item[Estensioni:]
        \begin{enumerate}
          \item se viene eseguita questa operazione in mancanza di rete, verrà visualizzato un'errore \emph{[UUC5]}.
        \end{enumerate}
\end{description}

\subsubsection{UUC9.1 - Scelta noto o incognito}%
\label{subs:UUC9.1}

\begin{figure}[h!]
  \centering
  \begin{plantuml}
  @startuml
  !include ../../commons/style/use-cases.pu

  actor :utente collegato: as A3

  actor :utente noto: as A3.1
  actor :utente incognito: as A3.2

  :A3.1: -up-|> :A3:
  :A3.2: -up-|> :A3:

  rectangle {
    together {
      usecase (UUC9.1.1) as "UUC9.1.1\nPassaggio a incognito"
      usecase (UUC9.1.2) as "UUC9.1.2\nPassaggio a noto"
    }
  }

  :A3.1: -- (UUC9.1.1)
  :A3.2: -- (UUC9.1.2)

  @enduml
  \end{plantuml}
  \caption{UUC9.1: Scelta noto o incognito}
  \label{fig:uuc9_1}
\end{figure}

\begin{description}
  \item[Caso d’uso:] UUC9.1
  \item[Titolo:] Scelta noto o incognito
  \item[Attori primari:] utente collegato, in particolare utente noto ed utente incognito
  \item[Precondizione:] l'utente si trova sulla schermata specifica dell'utente collegato ad un'organizzazione.
  \item[Postcondizione:] se l'utente era noto diventa incognito; vale anche il contrario.
  \item[Scenario principale:]
        \begin{enumerate}
          \item l'utente è collegato ad un'organizzazione in cui la sua presenza è nota; l'utente ha la possibilità di passare in modalità incognito.
          \item l'utente è collegato ad un'organizzazione in cui la sua presenza è incognita; l'utente ha la possibilità di passare in modalità nota.
        \end{enumerate}
  \item[Estensioni:]
        \begin{enumerate}
          \item se viene eseguita questa operazione in mancanza di rete, verrà visualizzato un'errore \emph{[UUC5]}.
        \end{enumerate}
\end{description}

\subsubsection{UUC9.1.1 - Passaggio a incognito}%
\label{subs:UUC9.1.1}
\begin{description}
  \item[Caso d’uso:] UUC9.1.1
  \item[Titolo:] Passaggio a incognito
  \item[Attori primari:] utente collegato, in particolare utente noto
  \item[Precondizione:] l'utente deve essere collegato e noto.
  \item[Postcondizione:] l'utente è collegato ed incognito.
  \item[Scenario Principale:]
        \begin{enumerate}
          \item l'utente è noto in un'organizzazione e vuole diventare incognito: in questo modo la sua presenza è nota, ma non lo è la sua identità.
        \end{enumerate}
\end{description}

\subsubsection{UUC9.1.2 - Passaggio a noto}%
\label{subs:UUC9.1.2}
\begin{description}
  \item[Caso d’uso:] UUC9.1.2
  \item[Titolo:] Passaggio a noto
  \item[Attori primari:] utente collegato, in particolare utente incognito
  \item[Precondizione:] l'utente deve essere collegato ed incognito.
  \item[Postcondizione:] l'utente è collegato e noto.
  \item[Scenario Principale:]
        \begin{enumerate}
          \item l'utente è incognito in un'organizzazione e vuole diventare noto: in questo modo, sia la sua presenza che la sua identità sono note.
        \end{enumerate}
\end{description}


\end{document}
