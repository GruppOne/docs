\documentclass[../../../analisi-dei-requisiti.tex]{subfiles}

\begin{document}


\begin{figure}[H]
  \centering
  \begin{plantuml}
  @startuml
  !include ../../commons/style/use-cases.pu

  actor :utente non collegato: as A3

  rectangle {
    together {
      usecase (UUC4) as "UUC4\nRecupero lista organizzazioni\n--\nExtension points:\nVisualizzazione errore in caso di\nselezionamento di un'organizzazione\n con rete mancante"
      usecase (UUC5.1) as "UUC5.1\nFiltra organizzazioni \nnon collegato"
      usecase (UUC5) as "UUC.nd\nVisualizzazione errore\nrete mancante"
    }
  }

  :A3: -- (UUC4)

  (UUC5.1) <.up. (UUC4) : <<include>>
  (UUC5) .up.> (UUC4) : <<extends>>

  @enduml
  \end{plantuml}
  \caption{UUC5: Collegamento organizzazione}%
  \label{fig:uuc5}
\end{figure}

\begin{description}
  \item[Caso d’uso:] UUC5;
  \item[Titolo:] Collegamento organizzazione;
  \item[Attori primari:] utente non collegato;
  \item[Precondizione:] l'utente si trova sulla lista delle organizzazioni;
  \item[Postcondizione:] l'utente si collega a una o più organizzazioni;
  \item[Scenario principale:]
        \begin{enumerate}
          \item l'utente sceglie una o più organizzazioni alla quale collegarsi.
        \end{enumerate}
  \item[Estensioni:]
        \begin{enumerate}
          \item in caso di rete mancante, non è possibile collegarsi ad un'organizzazione \emph{[UUC5]}.
        \end{enumerate}
\end{description}


\subsubsection{UUC5.1: Filtra organizzazioni non collegato}%
\begin{description}
  \item[Caso d’uso:] UUC5.1;
  \item[Titolo:] Filtra organizzazioni non collegato;
  \item[Attori primari:] utente non collegato;
  \item[Precondizione:] l'utente si trova sulla lista delle organizzazioni;
  \item[Postcondizione:] l'utente visualizza solo le organizzazioni su cui non è collegato;
  \item[Scenario principale:]
        \begin{enumerate}
          \item l'utente visualizza la lista delle organizzazioni su cui non è collegato.
        \end{enumerate}
\end{description}

\end{document}
