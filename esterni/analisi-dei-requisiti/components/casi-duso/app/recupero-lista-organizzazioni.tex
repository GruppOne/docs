\documentclass[../../../analisi-dei-requisiti.tex]{subfiles}

\begin{document}


\begin{figure}[h!]
  \centering
  \begin{plantuml}
  @startuml
  !include ../../commons/style/use-cases.pu
  scale 3/4

  actor :utente autenticato: as A

  rectangle {
    together {
      usecase (UUC4.1) as "UUC4.1\nVisualizzazione\nlista organizzazioni\n--\nExtension points:\nVisualizzazione errore causa\n-richiesta di visualizzazione della lista con rete mancante\n-aggiornamento della lista con rete mancante"
      usecase (UUC4.1.1) as "UUC4.1.1\nAggiornamento\nlista organizzazioni"
      usecase (UUC5) as "UUC5\nVisualizzazione errore\nrete mancante"
    }
  }

  :A: -- (UUC4.1)

  (UUC4.1.1) .up.|> (UUC4.1) : <<include>>
  (UUC5) .up.|> (UUC4.1) : <<extends>>

  @enduml
  \end{plantuml}
  \caption{UUC4: Recupero lista organizzazioni}
  \label{fig:uuc4}
\end{figure}

\begin{description}
  \item[Caso d’uso:] UUC4
  \item[Titolo:] Recupero lista organizzazioni
  \item[Attori primari:] utente autenticato
  \item[Precondizione:] l'utente si è appena autenticato.
  \item[Postcondizione:] l'utente visualizza una lista di tutte le organizzazioni.
  \item[Scenario principale:]
        \begin{enumerate}
          \item l'utente ha la possibilità di recuperare una lista di organizzazioni alla quale si può collegare;
        \end{enumerate}
  \item[Inclusioni:]
        \begin{enumerate}
          \item la lista delle organizzazioni si può aggiornare \emph{[UUC4.1.1]}.
        \end{enumerate}
  \item[Estensioni:]
        \begin{enumerate}
          \item in caso di rete mancante, non possono essere eseguite queste operazioni e quindi verrà notificato un errore \emph{[UUC5]}.
        \end{enumerate}
\end{description}

\subsubsection{UUC4.1 - Visualizzazione lista organizzazioni}%
\label{subs:UUC4.1}
\begin{description}
  \item[Caso d’uso:] UUC4.1
  \item[Titolo:] Visualizzazione lista organizzazioni
  \item[Attori primari:] utente autenticato
  \item[Precondizione:] l'utente si trova nella schermata post-autenticazione.
  \item[Postcondizione:] l'utente visualizza la lista di tutte le organizzazioni.
  \item[Scenario principale:]
        \begin{enumerate}
          \item l'utente visualizza, tramite un'apposita operazione, la lista di tutte le organizzazioni autorizzate a monitorare l'utente autenticato.
        \end{enumerate}
  \item[Inclusioni:]
        \begin{enumerate}
          \item la lista delle organizzazioni può essere aggiornata \emph{[UUC4.1]}.
        \end{enumerate}
  \item[Estensioni:]
        \begin{enumerate}
          \item in caso di rete mancante, si visualizzerà un errore in caso di richiesta di visualizzazione della lista delle organizzazioni
                oppure di aggiornamento della lista delle organizzazioni \emph{[UUC5]}.
        \end{enumerate}
\end{description}

\subsubsection{UUC4.1.1 - Aggiornamento lista organizzazioni}%
\label{subs:UUC4.1.1}
\begin{description}
  \item[Caso d’uso:] UUC4.1.1
  \item[Titolo:] Aggiornamento lista organizzazioni
  \item[Attori primari:] utente autenticato
  \item[Precondizione:] l'utente visualizza la lista delle organizzazioni.
  \item[Postcondizione:] l'utente ha aggiornato la lista delle organizzazioni.
  \item[Scenario principale:]
        \begin{enumerate}
          \item l'utente ha la possibilità di aggiornare la lista delle organizzazioni, e viene avvisato mediante \glossario{notifica} dell'applicazione mobile.
        \end{enumerate}
\end{description}


\end{document}
