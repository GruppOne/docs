\documentclass[../../../analisi-dei-requisiti.tex]{subfiles}

\begin{document}


\begin{figure}[H]
  \centering
  \scalegraphics{recupero-lista-organizzazioni.png}
  \caption{UUC4: Recupero lista organizzazioni}%
  \label{fig:uuc4}
\end{figure}

\begin{description}
  \item[Caso d’uso:] UUC4;
  \item[Titolo:] Recupero lista organizzazioni;
  \item[Attori primari:] utente autenticato;
  \item[Precondizione:] l'utente si è appena autenticato;
  \item[Postcondizione:] l'utente visualizza una lista di tutte le organizzazioni;
  \item[Scenario principale:]
        \begin{enumerate}
          \item l'utente ha la possibilità di recuperare una lista di organizzazioni alla quale si può collegare.
        \end{enumerate}
  \item[Estensioni:]
        \begin{enumerate}
          \item in caso di rete mancante, non possono essere eseguite queste operazioni e quindi verrà notificato un errore~\ref{subs:UUC12};
          \item in caso l'organizzazione sia privata, è richiesta l'autenticazione da parte dell'utente~\ref{subs:UUC4.3}.
        \end{enumerate}
\end{description}


\subsubsection{UUC4.1: Aggiornamento lista organizzazioni}%
\label{subs:UUC4.1}
\begin{description}
  \item[Caso d’uso:] UUC4.1;
  \item[Titolo:] Aggiornamento lista organizzazioni;
  \item[Attori primari:] utente autenticato;
  \item[Precondizione:] l'utente visualizza la lista delle organizzazioni;
  \item[Postcondizione:] l'utente ha aggiornato la lista delle organizzazioni;
  \item[Scenario principale:]
        \begin{enumerate}
          \item l'utente ha la possibilità di aggiornare la lista delle organizzazioni, e viene avvisato mediante \glossario{notifica} dell'applicazione mobile.
        \end{enumerate}
  \item[Estensioni:]
        \begin{enumerate}
          \item in caso di rete mancante, non può essere eseguito l'aggiornamento della lista di organizzazioni e quindi verrà notificato un errore~\ref{subs:UUC12}.
        \end{enumerate}
\end{description}


\subsubsection{UUC4.2: Seleziona organizzazioni}%
\label{subs:UUC4.2}
\begin{description}
  \item[Caso d’uso:] UUC4.2;
  \item[Titolo:] Seleziona organizzazioni;
  \item[Attori primari:] utente autenticato;
  \item[Precondizione:] l'utente visualizza la lista delle organizzazioni;
  \item[Postcondizione:] l'utente ha selezionato una o più organizzazioni;
  \item[Scenario principale:]
        \begin{enumerate}
          \item l'utente ha la possibilità di selezionare una o più organizzazioni.
        \end{enumerate}
  \item[Estensioni:]
        \begin{enumerate}
          \item in caso di rete mancante, non possono essere selezionate organizzazioni e quindi verrà notificato un errore~\ref{subs:UUC12};
          \item in caso l'organizzazione alla quale l'utente vuole autenticarsi sia privata, è richiesta l'autenticazione~\ref{subs:UUC4.3}.
        \end{enumerate}
\end{description}

\subsubsection{UUC4.3: Autenticazione organizzazione}%
\label{subs:UUC4.3}
\begin{description}
  \item[Caso d’uso:] UUC4.3;
  \item[Titolo:] Autenticazione organizzazione;
  \item[Attori primari:] utente autenticato;
  \item[Precondizione:] l'utente ha selezionato un'organizzazione privata;
  \item[Postcondizione:] l'utente è autenticato;
  \item[Scenario principale:]
        \begin{enumerate}
          \item l'utente ha selezionato un'organizzazione privata, in quanto la tracciatura prevista è nota.
          \item l'utente visualizza un form per l'autenticazione, con i relativi campi da inserire. Le credenziali non sono le stesse per l'autenticazione al servizio di Stalker, ma sono decise dall'owner dell'organizzazione.
        \end{enumerate}
\end{description}

\end{document}