\documentclass[../../../analisi-dei-requisiti.tex]{subfiles}

\begin{document}

\begin{figure}[h!]
  \centering
  \begin{plantuml}
  @startuml
  !include ../../commons/style/use-cases.pu

  actor :utente non autenticato: as A

  rectangle {
    together {
      usecase (UUC2.1) as "UUC2.1\nInserimento email\n--\nExtension points:\nVisualizzazione errore\ncredenziali errate"
      usecase (UUC2.2) as "UUC2.2\nInserimento password\n--\nExtension points:\nVisualizzazione errore\ncredenziali errate"
      usecase (UUC2.3) as "UUC2.3\nInformazioni autenticazione non valide"
    }
  }

  :A: -- (UUC2.1)
  :A: -- (UUC2.2)

  (UUC2.3) .up.|> (UUC2.2) : <<extends>>
  (UUC2.3) .up.|> (UUC2.1) : <<extends>>

  @enduml
  \end{plantuml}
  \caption{UUC2: Autenticazione}
  \label{fig:uuc2}
\end{figure}

\begin{description}
  \item[Caso d’uso:] UUC2
  \item[Titolo:] Autenticazione
  \item[Attori primari:] utente non autenticato
  \item[Precondizione:] l'utente è registrato al servizio.
  \item[Postcondizione:] l'utente è autenticato al servizio.
  \item[Scenario principale:]
  \begin{enumerate}
    \item l'utente visualizza la schermata per l'autenticazione e inserisce email e password sugli appositi campi.
  \end{enumerate}
  \item[Estensioni:]
  \begin{enumerate}
    \item se la coppia email-password inserite non sono presenti all'interno del database di Stalker, allora verrà visualizzato un'errore \emph{[UUC2.3]}.
  \end{enumerate}
\end{description}



\subsubsection{UUC2.1 - Inserimento email}%
\label{subs:UUC2.1}
\begin{description}
  \item[Caso d’uso:] UUC2.1
  \item[Titolo:] Inserimento email
  \item[Attori primari:] utente non autenticato
  \item[Precondizione:] l'utente si posiziona sul campo dell'email.
  \item[Postcondizione:] la email inserita è corretta.
  \item[Scenario principale:]
  \begin{enumerate}
    \item l'utente seleziona il campo dell'email e la inserisce.
  \end{enumerate}
  \item[Estensioni:]
  \begin{enumerate}
    \item se la email inserita non è presente nel database, si visualizzerà un errore di credenziali errate (per motivi di sicurezza non si indica qual è il campo errato) \emph{[UUC2.3]}.
  \end{enumerate}
\end{description}



\subsubsection{UUC2.2 - Inserimento password}%
\label{subs:UUC2.2}
\begin{description}
  \item[Caso d’uso:] UUC2.2
  \item[Titolo:] Inserimento password
  \item[Attori primari:] utente non autenticato
  \item[Precondizione:] l'utente si posiziona sul campo della password.
  \item[Postcondizione:] la password inserita è corretta.
  \item[Scenario principale:]
  \begin{enumerate}
    \item l'utente seleziona il campo della password e la inserisce.
  \end{enumerate}
  \item[Estensioni:]
  \begin{enumerate}
    \item se la password inserita non è presente nel database, si visualizzerà un errore di credenziali errate (per motivi di sicurezza non si indica qual è il campo errato) \emph{[UUC2.3]}.
  \end{enumerate}
\end{description}



\subsubsection{UUC2.3 - Informazioni autenticazione non valide}%
\label{subs:UUC2.3}
\begin{description}
  \item[Caso d’uso:] UUC2.3
  \item[Titolo:] Informazioni autenticazione non valide
  \item[Attori primari:] utente non autenticato
  \item[Precondizione:] l'utente si trova nella schermata di autenticazione e non rispetta i vincoli imposti sui campi presenti.
  \item[Postcondizione:] l'applicazione mobile comunica all'utente l'errore.
  \item[Scenario principale:]
        \begin{enumerate}
          \item l'utente cerca di effettuare l'autenticazione con credenziali errate.
        \end{enumerate}
\end{description}





\end{document}
