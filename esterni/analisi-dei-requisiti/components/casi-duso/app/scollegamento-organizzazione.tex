\documentclass[../../../analisi-dei-requisiti.tex]{subfiles}

\begin{document}

\subsubsection{UUC5: Scollegamento organizzazione}%
\label{subs:UUC5}

\begin{figure}[H]
  \centering
  \scalegraphics{scollegamento-organizzazione.png}
  \caption{UUC5: Scollegamento organizzazione}%
  \label{fig:UUC11}
\end{figure}

\begin{description}
  \item[Caso d’uso:] UUC5;
  \item[Titolo:] Scollegamento organizzazione;
  \item[Attori primari:] utente autenticato;
  \item[Precondizione:] l'utente si trova sulla lista delle organizzazioni;
  \item[Postcondizione:] l'utente si scollega da una o più organizzazioni;
  \item[Scenario principale:]
        \begin{enumerate}
          \item l'utente sceglie una o più organizzazioni dalle quali scollegarsi.
        \end{enumerate}
  \item[Estensioni:]
        \begin{enumerate}
          \item in caso di rete mancante, non è possibile collegarsi ad un'organizzazione~\ref{subs:UUC11}.
        \end{enumerate}
\end{description}


\subsubsection{UUC5.1: Filtra organizzazioni collegate}%
\begin{description}
  \item[Caso d’uso:] UUC5.1;
  \item[Titolo:] Filtra organizzazioni collegate;
  \item[Attori primari:] utente autenticato;
  \item[Precondizione:] l'utente si trova sulla lista delle organizzazioni;
  \item[Postcondizione:] l'utente visualizza solo le organizzazioni su cui è collegato;
  \item[Scenario principale:]
        \begin{enumerate}
          \item l'utente visualizza la lista delle organizzazioni su cui è collegato.
        \end{enumerate}
\end{description}

\end{document}
