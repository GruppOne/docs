\documentclass[components/casi-duso-app]{subfiles}

\begin{document}

\begin{figure}[H]
  \centering
  \begin{plantuml}
  @startuml
  !include ../../commons/style/use-cases.pu

  actor :utente non autenticato: as A

  rectangle {
    together {
      usecase (UUC1.1) as "UUC1.1\nVisualizzazione EULA"
      usecase (UUC1.2) as "UUC1.2\nInserimento dati registrazione"
    }
  }

  :A: -- (UUC1.1)
  :A: -- (UUC1.2)

  @enduml
  \end{plantuml}
  \caption{UUC1: Registrazione}%
  \label{fig:uuc1}
\end{figure}

\begin{description}
  \item[Caso d’uso:] UUC1;
  \item[Titolo:] Registrazione;
  \item[Attori primari:] utente non autenticato;
  \item[Precondizione:] l'utente non è registrato al servizio;
  \item[Postcondizione:] la registrazione va a buon fine e l'utente è registrato;
  \item[Scenario principale:]
        \begin{enumerate}
          \item prima di procedere con la registrazione, l'utente visualizza l'\glossario{EULA} e per continuare deve accettare i vincoli stabiliti da essa;
          \item una volta accettata l'EULA, l'utente provvede all'inserimento dei dati personali per la registrazione.
        \end{enumerate}
\end{description}


\subsubsection{UUC1.1: Visualizzazione EULA}%
\label{subs:UUC1.1}
\begin{description}
  \item[Caso d’uso:] UUC1.1;
  \item[Titolo:] Visualizzazione EULA;
  \item[Attori primari:] utente non autenticato;
  \item[Precondizione:] l'utente è all'interno della schermata di registrazione;
  \item[Postcondizione:] l'utente visualizza l'EULA;
  \item[Scenario principale:]
        \begin{enumerate}
          \item l'utente visualizza una schermata con tutti i vincoli da accettare presenti nell'EULA.
        \end{enumerate}
\end{description}



\subsubsection{UUC1.1.1: Conferma EULA}%
\label{subs:UUC1.1.1}

\begin{figure}[H]
  \centering
  \begin{plantuml}
  @startuml
  !include ../../commons/style/use-cases.pu

  actor :utente non autenticato: as A

  rectangle {
    together {
      usecase (UUC1.1.1) as "UUC1.1.1\nConferma EULA"
    }
  }

  :A: -- (UUC1.1.1)

  @enduml
  \end{plantuml}
  \caption{UUC1.1: Visualizza EULA}%
  \label{fig:uuc1.1}
\end{figure}


\begin{description}
  \item[Caso d’uso:] UUC1.1.1;
  \item[Titolo:] Conferma EULA;
  \item[Attori primari:] utente non autenticato;
  \item[Precondizione:] l'utente visualizza l'EULA;
  \item[Postcondizione:] l'utente continua la registrazione in quanto ha accettato l'EULA;
  \item[Scenario principale:]
        \begin{enumerate}
          \item E' presente un'opzione per accettare l'EULA: in caso affermativo, la registrazione può andare avanti.
        \end{enumerate}
\end{description}



\subsubsection{UUC1.2: Inserimento dati registrazione}%
\label{subs:UUC1.2}

\begin{figure}[H]
  \centering
  \begin{plantuml}
  @startuml
  !include ../../commons/style/use-cases.pu

  actor :utente non autenticato: as A

  rectangle {
    together {
      usecase (UUC1.2.1) as "UUC1.2.1\nRegistrazione email\n--\nExtension points:\nVisualizzazione errore se la mail\nnon è scritta in modo corretto"
      usecase (UUC1.2.2) as "UUC1.2.2\nRegistrazione password\n--\nExtension points:\nVisualizzazione errore se la\npassword non rispetta i criteri"
      usecase (UUC1.2.3) as "UUC1.2.3\nConferma password\n--\nExtension points:\nVisualizzazione errore se le\ndue password non coincidono"
      usecase (UUC1.2.4) as "UUC1.2.4\nInserimento dati anagrafici\n--\nExtension points:\nVisualizzazione errore\ndati mancanti"
      usecase (UUC1.2.5) as "UUC1.2.5\nInformazioni registrazione non valide"
    }
  }

  :A: -- (UUC1.2.1)
  :A: -- (UUC1.2.2)
  :A: -- (UUC1.2.3)
  :A: -- (UUC1.2.4)

  (UUC1.2.5) .up.|> (UUC1.2.4) : <<extends>>
  (UUC1.2.5) .up.|> (UUC1.2.3) : <<extends>>
  (UUC1.2.5) .up.|> (UUC1.2.2) : <<extends>>
  (UUC1.2.5) .up.|> (UUC1.2.1) : <<extends>>

  @enduml
  \end{plantuml}
  \caption{UUC1.2: Inserimento dati registrazione}%
  \label{fig:uuc1_2}
\end{figure}

\begin{description}
  \item[Caso d’uso:] UUC1.2;
  \item[Titolo:] Inserimento dati registrazione;
  \item[Attori primari:] utente non autenticato;
  \item[Precondizione:] l'utente ha accettato l'EULA;
  \item[Postcondizione:] l'utente ha inserito correttamente tutti i dati necessari alla registrazione e diventa un utente registrato;
  \item[Scenario principale:]
  \begin{enumerate}
    \item l'utente visualizza una serie di campi da compilare.
  \end{enumerate}
  \item[Estensioni:]
  \begin{enumerate}
    \item se i dati inseriti non sono validi, l'utente visualizzerà un errore \emph{[UUC1.2.5]}.
  \end{enumerate}
\end{description}



\subsubsection{UUC1.2.1: Registrazione email}%
\label{subs:UUC1.2.1}
\begin{description}
  \item[Caso d’uso:] UUC1.2.1;
  \item[Titolo:] Registrazione email;
  \item[Attori primari:] utente non autenticato;
  \item[Precondizione:] l'utente si trova nella schermata di registrazione nel campo dell'email;
  \item[Postcondizione:] l'utente ha inserito la email in modo corretto;
  \item[Scenario principale:]
  \begin{enumerate}
    \item l'utente seleziona il campo della email e la inserisce.
  \end{enumerate}
  \item[Estensioni:]
  \begin{enumerate}
    \item se la mail è scritta in una forma non corretta, verrà visualizzato un'errore di email non valida \emph{[UUC1.2.5]}.
  \end{enumerate}
\end{description}



\subsubsection{UUC1.2.2: Registrazione password}%
\label{subs:UUC1.2.2}
\begin{description}
  \item[Caso d’uso:] UUC1.2.2;
  \item[Titolo:] Registrazione password;
  \item[Attori primari:] utente non autenticato;
  \item[Precondizione:] l'utente si trova nella schermata di registrazione nel campo della password;
  \item[Postcondizione:] l'utente ha inserito la password in modo corretto;
  \item[Scenario principale:]
  \begin{enumerate}
    \item l'utente seleziona il campo della password e la crea.
  \end{enumerate}
  \item[Estensioni:]
  \begin{enumerate}
    \item se la password non rispetta determinati vincoli, verrà visualizzato un'errore di password non valida \emph{[UUC1.2.5]}.
  \end{enumerate}
\end{description}



\subsubsection{UUC1.2.3: Conferma password}%
\label{subs:UUC1.2.3}
\begin{description}
  \item[Caso d’uso:] UUC1.2.3;
  \item[Titolo:] Conferma password;
  \item[Attori primari:] utente non autenticato;
  \item[Precondizione:] l'utente si trova nella schermata di registrazione nel campo della conferma della password,
        e la password inserita in precedenza dall'utente rispetta tutti i vincoli;
  \item[Postcondizione:] l'utente ha inserito la conferma della password in modo corretto;
  \item[Scenario principale:]
  \begin{enumerate}
    \item l'utente seleziona il campo della conferma della password e la inserisce.
  \end{enumerate}
  \item[Estensioni:]
  \begin{enumerate}
    \item se la password inserita non è uguale a quella precedente, verrà visualizzato un'errore di conferma password non valida \emph{[UUC1.2.5]}.
  \end{enumerate}
\end{description}



\subsubsection{UUC1.2.4: Inserimento dati anagrafici}%
\label{subs:UUC1.2.4}
\begin{description}
  \item[Caso d’uso:] UUC1.2.4;
  \item[Titolo:] Inserimento dati anagrafici;
  \item[Attori primari:] utente non autenticato;
  \item[Precondizione:] l'utente si trova nella schermata di registrazione nei campi dei dati anagrafici;
  \item[Postcondizione:] l'utente ha inserito i dati anagrafici in modo corretto;
  \item[Scenario principale:]
  \begin{enumerate}
    \item l'utente seleziona i vari campi inerenti ai dati anagrafici e gli inserisce.
  \end{enumerate}
  \item[Estensioni:]
  \begin{enumerate}
    \item se uno o più dati anagrafici inseriti non rispettano determinati vincoli, verrà visualizzato un'errore sui campi in cui è segnalato l'errore \emph{[UUC1.2.5]}.
  \end{enumerate}
\end{description}



\subsubsection{UUC1.2.5: Informazioni registrazione non valide}%
\label{subs:UUC1.2.5}
\begin{description}
  \item[Caso d’uso:] UUC1.2.5;
  \item[Titolo:] informazioni registrazione non valide;
  \item[Attori primari:] utente non autenticato;
  \item[Precondizione:] l'utente si trova nella schermata di registrazione e non rispetta i vincoli imposti sui campi presenti;
  \item[Postcondizione:] l'applicazione mobile comunica all'utente l'errore;
  \item[Scenario Principale:]
        \begin{enumerate}
          \item l'utente visualizza il messaggio d'errore, in quanto le informazioni inserite non sono valide.
        \end{enumerate}
\end{description}

\end{document}