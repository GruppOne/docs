\documentclass[../../../analisi-dei-requisiti.tex]{subfiles}

\begin{document}

\begin{figure}[H]
  \centering
  \begin{plantuml}
  @startuml
  !include ../../commons/style/use-cases.pu

  actor :utente autenticato: as A

  rectangle {
    together {
      usecase (UUC10) as "UUC10\nVisualizzazione storico accessi\n--\nExtension points:\nVisualizzazione errore in\ncaso di richiesta storico\nin mancanza di rete"
      usecase (UUC5) as "UUC5\nVisualizzazione errore\nrete mancante"
      note right of (UUC10): implementazione opzionale
    }
  }

  :A: -- (UUC10)

  (UUC5) .up.|> (UUC10) : <<extends>>

  @enduml
  \end{plantuml}
  \caption{UUC10: Storico utente}%
  \label{fig:uuc10}
\end{figure}

\begin{description}
  \item[Caso d’uso:] UUC10;
  \item[Titolo:] Storico utente;
  \item[Attori primari:] utente autenticato;
  \item[Precondizione:]  l'utente deve poter accedere al pulsante dello storico;
  \item[Postcondizione:] l'utente visualizza lo storico degli accessi;
  \item[Scenario principale:]
        \begin{enumerate}
          \item l'utente accede ad una schermata per la visualizzazione dello storico personale degli accessi.
        \end{enumerate}
  \item[Estensioni:]
        \begin{enumerate}
          \item se l'utente cerca di visualizzare lo storico degli accessi in assenza di rete, si visualizzerà un errore \emph{[UUC5]}.
        \end{enumerate}
\end{description}


\end{document}
