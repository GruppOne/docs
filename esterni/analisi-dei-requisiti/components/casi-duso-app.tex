\documentclass[casi-duso]{subfiles}

%\renewcommand{\commons}{../../../commons}

\begin{document}

%UC1 (start)
\subsubsection{UC1 - Apertura applicazione mobile}
\label{subsub:uc1utente}
%inserire diagramma UC1
\begin{itemize}
  \item \textbf{Caso d’uso:} UC1 
  \item \textbf{Titolo:} Apertura applicazione mobile
  \item \textbf{Attori primari:} utente non registrato, utente non autenticato
  \item \textbf{Precondizione:} l'utente deve installare l'applicazione mobile sul proprio dispositivo.
  \item \textbf{Postcondizione:} l'utente ha installato l'applicazione mobile e la apre.
  \item \textbf{Scenario principale:} 
  \begin{enumerate}
    \item Al momento dell'apertura dell'applicazione mobile, l'utente visualizza una schermata con tre opzioni:
    \begin{enumerate}
      \item registrazione, se l'utente accede la prima volta e non è registrato;
      \item autenticazione, se l'utente è già registrato ma non è ancora autenticato;
      \item recupero credenziali, nel caso in cui l'utente voglia autenticarsi ma ha smarrito le sue credenziali.
    \end{enumerate}
  \end{enumerate}   
\end{itemize}
% subsub:uc1utente (end)

%UC2 (start)
\subsubsection{UC2 - Registrazione}
\label{subsub:uc2utente}
%inserire diagramma UC2
\begin{itemize}
  \item \textbf{Caso d’uso:} UC1 
  \item \textbf{Titolo:} Registrazione
  \item \textbf{Attori primari:} utente non registrato
  \item \textbf{Precondizione:} l'utente non è registrato al servizio.
  \item \textbf{Postcondizione:} la registrazione va a buon fine e l'utente è registrato.
  \item \textbf{Scenario principale:} 
  \begin{enumerate}
    \item prima di procedere con la registrazione, l'utente visualizza l'\glossario{EULA} e per continuare deve accettare i vincoli stabiliti da essa;
    \item una volta accettata l'EULA, l'utente provvede all'inserimento dei dati personali per la registrazione.
  \end{enumerate}  
  \item \textbf{Inclusioni:} 
  \begin{enumerate}
    \item EULA deve essere confermata, altrimenti la registrazione non può continuare.
  \end{enumerate}  
\end{itemize}
% subsub:uc2utente (end)

%UC2.1 (start)
\subsubsection{UC2.1 - Visualizzazione EULA}
\label{subsub:uc2.1utente}
\begin{itemize}
  \item \textbf{Caso d’uso:} UC2.1 
  \item \textbf{Titolo:} Visualizzazione EULA
  \item \textbf{Attori primari:} utente non registrato
  \item \textbf{Precondizione:} l'utente è all'interno della schermata di registrazione
  \item \textbf{Postcondizione:} l'utente visualizza l'EULA
  \item \textbf{Scenario principale:} 
  \begin{enumerate}
    \item l'utente visualizza una schermata con tutti i vincoli da accettare presenti nell'EULA.
  \end{enumerate}  
  \item \textbf{Inclusioni:} 
  \begin{enumerate}
    \item l'utente deve confermare l'EULA per continuare la registrazione.
  \end{enumerate}     
\end{itemize}
% subsub:uc2.1utente (end)

%UC2.2 (start)
\subsubsection{UC2.1.1 - Conferma EULA}
\label{subsub:uc2.1.1utente}
\begin{itemize}
  \item \textbf{Caso d’uso:} UC2.1.1 
  \item \textbf{Titolo:} Conferma EULA
  \item \textbf{Attori primari:} utente non registrato
  \item \textbf{Precondizione:} l'utente visualizza l'EULA.
  \item \textbf{Postcondizione:} l'utente continua la registrazione in quanto ha accettato l'EULA. 
  \item \textbf{Scenario principale:} 
  \begin{enumerate}
    \item E' presente un'opzione per accettare l'EULA: in caso affermativo, la registrazione può andare avanti. 
  \end{enumerate}    
\end{itemize}
% subsub:uc2.1.1utente (end)

%UC2.2 (start)
\subsubsection{UC2.2 - Inserimento dati registrazione}
\label{subsub:uc2.2utente}
%inserire diagramma UC2.2
\begin{itemize}
  \item \textbf{Caso d’uso:} UC2.1 
  \item \textbf{Titolo:} Visualizzazione EULA
  \item \textbf{Attori primari:} utente non registrato
  \item \textbf{Precondizione:} l'utente ha accettato l'EULA.
  \item \textbf{Postcondizione:} l'utente ha inserito correttamente tutti i dati necessari alla registrazione e diventa un utente registrato. 
  \item \textbf{Scenario principale:} 
  \begin{enumerate}
    \item l'utente visualizza una serie di campi da compilare.
  \end{enumerate}
  \item \textbf{Estensioni:} 
  \begin{enumerate}
    \item se i dati inseriti non sono validi, l'utente visualizzerà un errore.
  \end{enumerate}     
\end{itemize}
% subsub:uc2.2utente (end)

%UC2.2.1 (start)
\subsubsection{UC2.2.1 - Registrazione email}
\label{subsub:uc2.2.1utente}
\begin{itemize}
  \item \textbf{Caso d’uso:} UC2.2.1 
  \item \textbf{Titolo:} Registrazione email
  \item \textbf{Attori primari:} utente non registrato
  \item \textbf{Precondizione:} l'utente si trova nella schermata di registrazione nel campo dell'email.
  \item \textbf{Postcondizione:} l'utente ha inserito la email in modo corretto. 
  \item \textbf{Scenario principale:} 
  \begin{enumerate}
    \item l'utente seleziona il campo della email e la inserisce.
  \end{enumerate}
  \item \textbf{Estensioni:} 
  \begin{enumerate}
    \item se la mail è scritta in una forma non corretta, verrà visualizzato un'errore di mail non valida.
  \end{enumerate}     
\end{itemize}
% subsub:uc2.2.1utente (end)

%UC2.2.2 (start)
\subsubsection{UC2.2.2 - Registrazione password}
\label{subsub:uc2.2.2utente}
\begin{itemize}
  \item \textbf{Caso d’uso:} UC2.2.2 
  \item \textbf{Titolo:} Registrazione password
  \item \textbf{Attori primari:} utente non registrato
  \item \textbf{Precondizione:} l'utente si trova nella schermata di registrazione nel campo della password.
  \item \textbf{Postcondizione:} l'utente ha inserito la password in modo corretto.  
  \item \textbf{Scenario principale:} 
  \begin{enumerate}
    \item l'utente seleziona il campo della password e la crea.
  \end{enumerate}
  \item \textbf{Estensioni:} 
  \begin{enumerate}
    \item se la password non rispetta determinati vincoli, verrà visualizzato un'errore di password non valida.
  \end{enumerate}     
\end{itemize}
% subsub:uc2.2.2utente (end)

%UC2.2.3 (start)
\subsubsection{UC2.2.3 - Conferma password}
\label{subsub:uc2.2.3utente}
\begin{itemize}
  \item \textbf{Caso d’uso:} UC2.2.1 
  \item \textbf{Titolo:} Conferma password
  \item \textbf{Attori primari:} utente non registrato
  \item \textbf{Precondizione:} l'utente si trova nella schermata di registrazione nel campo della conferma della password, 
  e la password inserita in precedenza dall'utente rispetta tutti i vincoli.
  \item \textbf{Postcondizione:} l'utente ha inserito la conferma della password in modo corretto. 
  \item \textbf{Scenario principale:} 
  \begin{enumerate}
    \item l'utente seleziona il campo della conferma della password e la inserisce.
  \end{enumerate}
  \item \textbf{Estensioni:} 
  \begin{enumerate}
    \item se la password inserita non è uguale a quella precedente, verrà visualizzato un'errore di conferma password non valida.
  \end{enumerate}     
\end{itemize}
% subsub:uc2.2.3utente (end)

%UC2.2.4 (start)
\subsubsection{UC2.2.4 - Inserimento dati anagrafici}
\label{subsub:uc2.2.4utente}
\begin{itemize}
  \item \textbf{Caso d’uso:} UC2.2.4 
  \item \textbf{Titolo:} Inserimento dati anagrafici
  \item \textbf{Attori primari:} utente non registrato
  \item \textbf{Precondizione:} l'utente si trova nella schermata di registrazione nei campi dei dati anagrafici.
  \item \textbf{Postcondizione:} l'utente ha inserito i dati anagrafici in modo corretto. 
  \item \textbf{Scenario principale:} 
  \begin{enumerate}
    \item l'utente seleziona i vari campi inerenti ai dati anagrafici e gli inserisce.
  \end{enumerate}
  \item \textbf{Estensioni:} 
  \begin{enumerate}
    \item se uno o più dati anagrafici inseriti non rispettano determinati vincoli, verrà visualizzato un'errore sui campi in cui è segnalato l'errore.
  \end{enumerate}     
\end{itemize}
% subsub:uc2.2.4utente (end)

%UC2.2.5 (start)
\subsubsection{UC2.2.5 - Informazioni registrazione non valide}
\label{subsub:uc2.2.5utente}
\begin{itemize}
  \item \textbf{Caso d’uso:} UC2.2.5 
  \item \textbf{Titolo:} informazioni registrazione non valide
  \item \textbf{Attori primari:} utente non registrato
  \item \textbf{Precondizione:} l'utente si trova nella schermata di registrazione e non rispetta i vincoli imposti sui campi presenti.
  \item \textbf{Postcondizione:} l'applicazione mobile comunica all'utente l'errore.
  \item \textbf{Scenario Principale:} 
  \begin{enumerate}
    \item l'utente visualizza il messaggio d'errore, in quanto le informazioni inserite non sono valide.
  \end{enumerate}   
\end{itemize}
% subsub:uc2.2.5utente (end)

%UC3 (start)
\subsubsection{UC3 - Autenticazione}
\label{subsub:uc2utente}
%inserire diagramma UC3
\begin{itemize}
  \item \textbf{Caso d’uso:} UC3 
  \item \textbf{Titolo:} Autenticazione
  \item \textbf{Attori primari:} utente non autenticato
  \item \textbf{Precondizione:} l'utente è registrato al servizio.
  \item \textbf{Postcondizione:} l'utente è autenticato al servizio.
  \item \textbf{Scenario principale:} 
  \begin{enumerate}
    \item l'utente visualizza la schermata per l'autenticazione e inserisce email e password sugli appositi campi.
  \end{enumerate}  
  \item \textbf{Estensioni:} 
  \begin{enumerate}
    \item se la coppia email-password inserite non sono presenti all'interno del database di Stalker, allora verrà visualizzato un'errore.
  \end{enumerate}  
\end{itemize}
% subsub:uc3utente (end)

%UC3.1 (start)
\subsubsection{UC3.1 - Inserimento email}
\label{subsub:uc2.1utente}
\begin{itemize}
  \item \textbf{Caso d’uso:} UC3.1 
  \item \textbf{Titolo:} Inserimento email
  \item \textbf{Attori primari:} utente non autenticato
  \item \textbf{Precondizione:} l'utente si posiziona sul campo dell'email.
  \item \textbf{Postcondizione:} la email inserita è corretta.
  \item \textbf{Scenario principale:} 
  \begin{enumerate}
    \item l'utente seleziona il campo dell'email e la inserisce.
  \end{enumerate}  
  \item \textbf{Estensioni:} 
  \begin{enumerate}
    \item se la email inserita non è presente nel database, si visualizzerà un errore di credenziali errate (per motivi di sicurezza
    non si indica qual è il campo errato).
  \end{enumerate}  
\end{itemize}
% subsub:uc3.1utente (end)

%UC3.2 (start)
\subsubsection{UC3.2 - inserimento password}
\label{subsub:uc3.2utente}
\begin{itemize}
  \item \textbf{Caso d’uso:} UC3.2 
  \item \textbf{Titolo:} Inserimento password
  \item \textbf{Attori primari:} utente non autenticato
  \item \textbf{Precondizione:} l'utente si posiziona sul campo della password.
  \item \textbf{Postcondizione:} la password inserita è corretta.
  \item \textbf{Scenario principale:} 
  \begin{enumerate}
    \item l'utente seleziona il campo della password e la inserisce.
  \end{enumerate}  
  \item \textbf{Estensioni:} 
  \begin{enumerate}
    \item se la password inserita non è presente nel database, si visualizzerà un errore di credenziali errate (per motivi di sicurezza
    non si indica qual è il campo errato).
  \end{enumerate}  
\end{itemize}
% subsub:uc3.2utente (end)

%UC3.3 (start)
\subsubsection{UC3.3 - Informazioni autenticazione non valide}
\label{subsub:uc3.3utente}
\begin{itemize}
  \item \textbf{Caso d’uso:} UC3.3 
  \item \textbf{Titolo:} Informazioni autenticazione non valide
  \item \textbf{Attori primari:} utente non autenticato
  \item \textbf{Precondizione:} l'utente si trova nella schermata di autenticazione e non rispetta i vincoli imposti sui campi presenti.
  \item \textbf{Postcondizione:} l'applicazione mobile comunica all'utente l'errore.
  \item \textbf{Scenario principale:} 
  \begin{enumerate}
    \item l'utente visualizza il messaggio d'errore, in quanto le informazioni inserite non sono valide.
  \end{enumerate}  
\end{itemize}
% subsub:uc3.3utente (end)

%UC4 (start)
\subsubsection{UC4 - Recupero credenziali}
\label{subsub:uc4utente}
%inserire diagramma UC4
\begin{itemize}
  \item \textbf{Caso d’uso:} UC4 
  \item \textbf{Titolo:} Recupero credenziali
  \item \textbf{Attori primari:} utente non autenticato
  \item \textbf{Precondizione:} l'utente si trova nella schermata principale ed è registrato al servizio.
  \item \textbf{Postcondizione:} l'utente si trova nella schermaata del recupero credenziali.
  \item \textbf{Scenario principale:} 
  \begin{enumerate}
    \item l'utente seleziona la voce per il recupero credenziali e avviene tramite recupero password.
  \end{enumerate} 
\end{itemize}
% subsub:uc4utente (end)

%UC4.1 (start)
\subsubsection{UC4.1 - Recupero password}
\label{subsub:uc4.1utente}
\begin{itemize}
  \item \textbf{Caso d’uso:} UC4.1 
  \item \textbf{Titolo:} Recupero password
  \item \textbf{Attori primari:} utente non autenticato
  \item \textbf{Precondizione:} l'utente si trova nella schermata del recupero credenziali.
  \item \textbf{Postcondizione:} l'utente ha recuperato le credenziali e può nuovamente autenticarsi.
  \item \textbf{Scenario principale:} 
  \begin{enumerate}
    \item l'utente ha la possibilità di recuperare la password; questa procedura avviene in due passaggi:
    \begin{enumerate}
      \item reset della password vecchia;
      \item inserimento della nuova password e la conferma della nuova conferma.
    \end{enumerate}
  \end{enumerate} 
\end{itemize}
% subsub:uc4.1utente (end)

%UC5 (start)
\subsubsection{UC5 - Visualizzazione organizzazioni}
\label{subsub:uc5utente}
%inserire diagramma UC5
\begin{itemize}
  \item \textbf{Caso d’uso:} UC5 
  \item \textbf{Titolo:} Visualizzazione organizzazioni
  \item \textbf{Attori primari:} utente autenticato
  \item \textbf{Precondizione:} l'utente si è appena autenticato.
  \item \textbf{Postcondizione:} l'utente visualizza tutte le organizzazioni.
  \item \textbf{Scenario principale:} 
  \begin{enumerate}
    \item l'utente visualizza una lista di organizzazioni alla quale si può fare richiesta di monitoraggio.
  \end{enumerate}  
  \item \textbf{Inclusioni:} 
  \begin{enumerate}
    \item la lista delle organizzazioni si può aggiornare.
  \end{enumerate}
  \item \textbf{Estensioni:} 
  \begin{enumerate}
    \item in caso di rete mancante, non possono essere eseguite queste operazioni e quindi verrà notificato un errore.
  \end{enumerate}  
\end{itemize}
% subsub:uc5utente (end)

%UC5.1 (start)
\subsubsection{UC5.1 - Visualizzazione lista organizzazioni}
\label{subsub:uc5utente}
\begin{itemize}
  \item \textbf{Caso d’uso:} UC5 
  \item \textbf{Titolo:} Visualizzazione lista organizzazioni
  \item \textbf{Attori primari:} utente autenticato
  \item \textbf{Precondizione:} l'utente si trova nella schermata post-autenticazione.
  \item \textbf{Postcondizione:} l'utente visualizza la lista delle organizzazioni.
  \item \textbf{Scenario principale:} 
  \begin{enumerate}
    \item l'utente visualizza la lista delle organizzazioni.
  \end{enumerate}  
  \item \textbf{Inclusioni:} 
  \begin{enumerate}
    \item la lista delle organizzazioni può essere aggiornata.
  \end{enumerate}
  \item \textbf{Estensioni:} 
  \begin{enumerate}
    \item in caso di rete mancante, si visualizzerà un errore in caso di richiesta di visualizzazione della lista delle organizzazioni
    oppure di aggiornamento della lista delle organizzazioni.
  \end{enumerate}  
\end{itemize}
% subsub:uc5.1utente (end)

%UC5.1.1 (start)
\subsubsection{UC5.1.1 - Aggiornamento lista organizzazioni}
\label{subsub:uc5utente}
\begin{itemize}
  \item \textbf{Caso d’uso:} UC5 
  \item \textbf{Titolo:} Aggiornamento lista organizzazioni
  \item \textbf{Attori primari:} utente autenticato
  \item \textbf{Precondizione:} l'utente visualizza la lista delle organizzazioni.
  \item \textbf{Postcondizione:} l'utente ha aggiornato la lista delle organizzazioni.
  \item \textbf{Scenario principale:} 
  \begin{enumerate}
    \item l'utente ha la possibilità di aggiornare la lista delle organizzazioni, e viene avvisato mediante \glossario{notifica} 
    dell'applicazione mobile.
  \end{enumerate}  
\end{itemize}
% subsub:uc5.1.1utente (end)

%UC6 (start)
\subsubsection{UC6 - Visualizzazione errore rete mancante}
\label{subsub:uc6utente}
\begin{itemize}
  \item \textbf{Caso d’uso:} UC6
  \item \textbf{Titolo:} Visualizzazione errore rete mancante
  \item \textbf{Attori primari:} utente autenticato
  \item \textbf{Precondizione:} l'utente è autenticato, ma nel suo dispositivo non c'è alcun accesso alla rete.
  \item \textbf{Postcondizione:} l'applicazione mobile comunica all'utente l'errore.
  \item \textbf{Scenario principale:} 
  \begin{enumerate}
    \item l'utente visualizza il messaggio d'errore, in quanto non è possibile eseguire operazioni che necessitano della rete in mancanza di essa.
  \end{enumerate}  
\end{itemize}
% subsub:uc6utente (end)

%UC7 (start)
\subsubsection{UC7 - Collegamento organizzazione}
\label{subsub:uc7utente}
%inserire diagramma UC7
\begin{itemize}
  \item \textbf{Caso d’uso:} UC7
  \item \textbf{Titolo:} Collegamento organizzazione
  \item \textbf{Attori primari:} utente autenticato, in particolare utente non collegato
  \item \textbf{Precondizione:} l'utente si trova sulla lista delle organizzazioni.
  \item \textbf{Postcondizione:} l'utente ha selezionato un'organizzazione ed è collegato ad essa.
  \item \textbf{Scenario principale:} 
  \begin{enumerate}
    \item l'utente sceglie un'organizzazione alla quale collegarsi.
  \end{enumerate}  
  \item \textbf{Estensioni:} 
  \begin{enumerate}
    \item in caso di rete mancante, non è possibile collegarsi ad un'organizzazione.
  \end{enumerate}  
\end{itemize}
% subsub:uc7utente (end)

%UC7.1 (start)
\subsubsection{UC7.1 - Selezionamento di un'organizzazione}
\label{subsub:uc7utente}
\begin{itemize}
  \item \textbf{Caso d’uso:} UC7.1
  \item \textbf{Titolo:} Selezionamento di un'organizzazione
  \item \textbf{Attori primari:} utente autenticato, in particolare utente non collegato
  \item \textbf{Precondizione:} l'utente visualizza la lista delle organizzazioni.
  \item \textbf{Postcondizione:} l'utente è collegato ad un'organizzazione.
  \item \textbf{Scenario principale:} 
  \begin{enumerate}
    \item l'utente seleziona un'organizzazione dalla lista.
  \end{enumerate}  
  \item \textbf{Estensioni:} 
  \begin{enumerate}
    \item in caso di rete mancante, non è possibile collegarsi ad un'organizzazione.
  \end{enumerate}  
\end{itemize}
% subsub:uc7.1utente (end)

%UC8 (start)
\subsubsection{UC8 - Scollegamento organizzazione}
\label{subsub:uc8utente}
%inserire diagramma UC8
\begin{itemize}
  \item \textbf{Caso d’uso:} UC8
  \item \textbf{Titolo:} Scollegamento organizzazione
  \item \textbf{Attori primari:} utente autenticato, in particolare utente collegato
  \item \textbf{Precondizione:} l'utente è collegato ad un'organizzazione.
  \item \textbf{Postcondizione:} l'utente non è collegato ad alcuna organizzazione.
  \item \textbf{Scenario principale:} 
  \begin{enumerate}
    \item l'utente vuole scollegarsi dall'organizzazione corrente, e lo fa dalla schermata che specifica a quale organizzazione è collegato e le 
    sue informazioni relative.
  \end{enumerate}  
  \item \textbf{Estensioni:} 
  \begin{enumerate}
    \item se l'utente cerca di scollegarsi da un'organizzazione in caso di rete mancante, si visualizzerà un errore.
  \end{enumerate}  
\end{itemize}
% subsub:uc8utente (end)

%UC8.1 (start)
\subsubsection{UC8.1 - Uscita da un'organizzazione}
\label{subsub:uc8.1utente}
\begin{itemize}
  \item \textbf{Caso d’uso:} UC8.1
  \item \textbf{Titolo:} Uscita da un'organizzazione
  \item \textbf{Attori primari:} utente autenticato, in particolare utente collegato
  \item \textbf{Precondizione:} l'utente si trova sulla schermata specifica dell'utente collegato ad un'organizzazione.
  \item \textbf{Postcondizione:} l'utente non è collegato ad alcuna organizzazione.
  \item \textbf{Scenario principale:} 
  \begin{enumerate}
    \item l'utente esegue l'uscita dall'organizzazione tramite un'operazione specifica a questo requisito.
  \end{enumerate}  
  \item \textbf{Estensioni:} 
  \begin{enumerate}
    \item se l'utente cerca di scollegarsi da un'organizzazione in caso di rete mancante, si visualizzerà un errore.
  \end{enumerate}  
\end{itemize}
% subsub:uc8.1utente (end)

%UC9 (start)
\subsubsection{UC9 - Passaggio noto/incognito}
\label{subsub:uc9utente}
%inserire diagramma UC9
\begin{itemize}
  \item \textbf{Caso d’uso:} UC9
  \item \textbf{Titolo:} Passaggio noto/incognito
  \item \textbf{Attori primari:} utente autenticato, in particolare utente collegato
  \item \textbf{Precondizione:} l'utente si trova sulla schermata specifica dell'utente collegato ad un'organizzazione.
  \item \textbf{Postcondizione:} se l'utente era noto diventa incognito; vale anche il contrario.
  \item \textbf{Scenario principale:} 
  \begin{enumerate}
    \item l'utente ha la possibilità di cambiare il tipo di monitoraggio che l'organizzazione è tenuta a conoscere.
  \end{enumerate}  
  \item \textbf{Estensioni:} 
  \begin{enumerate}
    \item se viene eseguita questa operazione in mancanza di rete, verrà visualizzato un'errore. 
  \end{enumerate}  
\end{itemize}
% subsub:uc9utente (end)

%UC9.1 (start)
\subsubsection{UC9.1 - Scelta noto o incognito}
\label{subsub:uc9.1utente}
%inserire diagramma UC9.1
\begin{itemize}
  \item \textbf{Caso d’uso:} UC9.1
  \item \textbf{Titolo:} Scelta noto o incognito
  \item \textbf{Attori primari:} utente collegato, in particolare utente noto ed utente incognito
  \item \textbf{Precondizione:} l'utente si trova sulla schermata specifica dell'utente collegato ad un'organizzazione.
  \item \textbf{Postcondizione:} se l'utente era noto diventa incognito; vale anche il contrario.
  \item \textbf{Scenario principale:} 
  \begin{enumerate}
    \item l'utente è collegato ad un'organizzazione in cui la sua presenza è nota; l'utente ha la possibilità di passare in modalità incognito.
    \item l'utente è collegato ad un'organizzazione in cui la sua presenza è incognita; l'utente ha la possibilità di passare in modalità nota.
  \end{enumerate}
  \item \textbf{Estensioni:} 
  \begin{enumerate}
    \item se viene eseguita questa operazione in mancanza di rete, verrà visualizzato un'errore. 
  \end{enumerate}  
\end{itemize}
% subsub:uc9.1utente (end)

%UC9.1.1 (start)
\subsubsection{UC9.1.1 - Passaggio a incognito}
\label{subsub:uc9.1.1utente}
\begin{itemize}
  \item \textbf{Caso d’uso:} UC9.1.1
  \item \textbf{Titolo:} Passaggio a incognito
  \item \textbf{Attori primari:} utente collegato, in particolare utente noto
  \item \textbf{Precondizione:} l'utente deve essere collegato e noto.
  \item \textbf{Postcondizione:} l'utente è collegato ed incognito.
  \item \textbf{Scenario Principale:}
  \begin{enumerate}
    \item l'utente è noto in un'organizzazione e vuole diventare incognito: in questo modo la sua presenza è nota, ma non lo è la sua identità.
  \end{enumerate}
\end{itemize}
% subsub:uc9.1.1utente (end)

%UC9.1.2 (start)
\subsubsection{UC9.1.2 - Passaggio a noto}
\label{subsub:uc9.1.2utente}
\begin{itemize}
  \item \textbf{Caso d’uso:} UC9.1.2
  \item \textbf{Titolo:} Passaggio a noto
  \item \textbf{Attori primari:} utente collegato, in particolare utente incognito
  \item \textbf{Precondizione:} l'utente deve essere collegato ed incognito.
  \item \textbf{Postcondizione:} l'utente è collegato e noto.
  \item \textbf{Scenario Principale:}
  \begin{enumerate}
    \item l'utente è incognito in un'organizzazione e vuole diventare noto: in questo modo, sia la sua presenza che la sua identità sono note.
  \end{enumerate}
\end{itemize}
% subsub:uc9.1.2utente (end)

%UC10 (start)
\subsubsection{UC10 - Storico utente}
\label{subsub:uc10utente}
%inserire diagramma UC10
\begin{itemize}
  \item \textbf{Caso d’uso:} UC10
  \item \textbf{Titolo:} Storico utente
  \item \textbf{Attori primari:} utente autenticato
  \item \textbf{Precondizione:}  l'utente si è appena autenticato.
  \item \textbf{Postcondizione:} l'utente visualizza lo storico degli accessi.
  \item \textbf{Scenario principale:} 
  \begin{enumerate}
    \item l'utente ha la possibilità di vedere il personale storico degli accessi.
  \end{enumerate}  
  \item \textbf{Estensioni:} 
  \begin{enumerate}
    \item se l'utente cerca di visualizzare lo storico degli accessi in assenza di rete, si visualizzerà un errore.
  \end{enumerate}  
\end{itemize}
% subsub:uc10utente (end)

%UC10.1 (start)
\subsubsection{UC10.1 - Visualizzazione storico accessi}
\label{subsub:uc10.1utente}
\begin{itemize}
  \item \textbf{Caso d’uso:} UC10.1
  \item \textbf{Titolo:} Visualizzazione storico accessi
  \item \textbf{Attori primari:} utente autenticato
  \item \textbf{Precondizione:} l'utente si trova nella schermata post-autenticazione.
  \item \textbf{Postcondizione:} l'utente visualizza lo storico degli accessi.
  \item \textbf{Scenario principale:} 
  \begin{enumerate}
    \item l'utente accede ad una schermata per la visualizzazione dello storico personale degli accessi.
  \end{enumerate}
  \item \textbf{Estensioni:} 
  \begin{enumerate}
    \item se l'utente cerca di visualizzare lo storico degli accessi in assenza di rete, si visualizzerà un errore.
  \end{enumerate} 
\end{itemize}
% subsub:uc10.1utente (end)

%UC11 (start)
\subsubsection{UC11 - Tempo utente nell'organizzazione}
\label{subsub:uc11utente}
%inserire diagramma UC11
\begin{itemize}
  \item \textbf{Caso d’uso:} UC11
  \item \textbf{Titolo:} Tempo utente nell'organizzazione
  \item \textbf{Attori primari:} utente autenticato, in particolare utente collegato
  \item \textbf{Precondizione:} l'utente deve essere collegato ad un'organizzazione.
  \item \textbf{Postcondizione:} l'utente visualizza il tempo trascorso nll'organizzazione.
  \item \textbf{Scenario principale:} 
  \begin{enumerate}
    \item l'utente visualizza in tempo reale il tempo trascorso all'interno di un'organizzazione.
  \end{enumerate}  
  \item \textbf{Estensioni:} 
  \begin{enumerate}
    \item se l'utente cerca di visualizzare il tempo trascorso nell'organizzazione in assenza di rete, si visualizzerà un errore.
  \end{enumerate}  
\end{itemize}
% subsub:uc11utente (end)

%UC11.1 (start)
\subsubsection{UC11.1 - Visualizzazione tempo trascorso nell'organizzazione corrente}
\label{subsub:uc11utente}
\begin{itemize}
  \item \textbf{Caso d’uso:} UC11
  \item \textbf{Titolo:} Visualizzazione tempo trascorso nell'organizzazione corrente
  \item \textbf{Attori primari:} utente autenticato, in particolare utente collegato
  \item \textbf{Precondizione:} 
  \item \textbf{Postcondizione:}
  \item \textbf{Scenario principale:} 
  \begin{enumerate}
    \item l'utente accede ad una schermata per visualizzare in tempo reale il tempo trascorso all'interno dell'organizzazione corrente.
  \end{enumerate}  
  \item \textbf{Estensioni:} 
  \begin{enumerate}
    \item se l'utente cerca di visualizzare il tempo trascorso nell'organizzazione in assenza di rete, si visualizzerà un errore.
  \end{enumerate}  
\end{itemize}
% subsub:uc11.1utente (end)
