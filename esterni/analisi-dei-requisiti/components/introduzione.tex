\documentclass[../analisi-dei-requisiti.tex]{subfiles}

\begin{document}
\subsection{Scopo del documento}%
\label{sub:scopo_del_documento}
Il presente \glossario{documento} ha lo scopo di descrivere in maniera dettagliata i \glossario{requisiti} impliciti ed espliciti individuati per il \glossario{prodotto}. Tali requisiti sono stati identificati dall'analisi del \glossario{capitolato} C5 ed i successivi incontri con il \glossario{proponente}, \textit{Imola Informatica}.

\subsection{Scopo del prodotto}%
\label{sub:scopo_del_prodotto}
L'obiettivo del \glossario{progetto} è sviluppare un'\glossario{applicazione mobile} distribuita, seguendo il \glossario{modello client/server}.
Il \glossario{client} deve essere in grado di segnalare sia l'ingresso che l'uscita dell'\glossario{utente} dai \glossario{luoghi} (in modalità anonima o meno a seconda delle esigenze), i quali sono definiti dalle \glossario{organizzazioni}.
Il \glossario{server} deve fornire la possibilità di raccogliere ed analizzare i \glossario{dati} relativi alle organizzazioni.
In caso di utenti anonimi l'analisi riguarda solo una \glossario{stima} del numero totale di persone presenti in un dato momento.
In caso di utenti autenticati deve inoltre essere possibile effettuare \glossario{query} di monitoraggio specifiche.
In merito all'ottimizzazione della geolocalizzazione, è richiesto un \glossario{report} che esponga le scelte progettuali, le rispettive motivazioni e i test eseguiti per garantire la rilevazione sufficiente precisa della posizione, considerando le limitazioni dello \glossario{smartphone}.


\subsection{Glossario}%
\label{sub:glossario}
Al fine di rendere il documento più chiaro possibile, i termini che possono assumere un significato ambiguo sono evidenziati (i.e., \glossario{client}) e riportati in \textit{Glossario1.0.0.pdf} accompagnati da una definizione.

\subsection{Riferimenti}%
\label{sub:riferimenti}
\subsubsection{Normativi}%
\label{par:normativi}
\begin{itemize}
  \item \textit{Norme di progetto}
  \item \href{https://www.math.unipd.it/~tullio/IS-1/2019/Progetto/C5.pdf}{Capitolato d'appalto C5}
\end{itemize}

\subsubsection{Informativi}%
\label{par:informativi}
\begin{itemize}
  \item \href{https://www.math.unipd.it/~tullio/IS-1/2019/Dispense/C5a.pdf}{Seminario di presentazione del capitolato C5} ()
  \item Slide del corso di Ingegneria del Software
        \begin{itemize}
          \item \href{https://www.math.unipd.it/~tullio/IS-1/2019/Dispense/L08.pdf}{Slide di analisi dei requisiti, corso di Ingegneria del Software}
          \item \href{https://www.math.unipd.it/~tullio/IS-1/2019/Dispense/E03.pdf}{Diagrammi dei casi d'uso}
        \end{itemize}
  \item \href{https://www.openldap.org/}{Lightweight Directory Access Protocol}
  \item \href{https://www.ics.uci.edu/~fielding/pubs/dissertation/rest_arch_style.htm}{Representational state transfer}
  \item \href{https://grpc.io/docs/guides/}{gRPC}
\end{itemize}

\end{document}
