\documentclass[../analisi-dei-requisiti.tex]{subfiles}

\begin{document}
\subsection{Scopo del documento}%
\label{sub:scopo_del_documento}
Il presente \glossario{documento} ha lo scopo di descrivere in maniera dettagliata i \glossario{requisiti} impliciti ed espliciti individuati per il \glossario{prodotto}.
Tali requisiti sono stati identificati dall'analisi del \glossario{capitolato} C5 ed i successivi incontri con il \glossario{proponente}, \textit{Imola Informatica}.

\scopoDelProdottoEGlossario{}

\subsection{Riferimenti}%
\label{sub:riferimenti}
\subsubsection{Normativi}%
\label{par:normativi}
\begin{itemize}
  \item \textit{Norme di progetto (versione \versione)}.
  \item Capitolato d'appalto C5: \href{https://www.math.unipd.it/~tullio/IS-1/2019/Progetto/C5.pdf}{https://www.math.unipd.it/\textasciitilde tullio/IS-1/2019/Progetto/C5.pdf}.
\end{itemize}

\subsubsection{Informativi}%
\label{par:informativi}
\begin{itemize}
  \item Seminario di presentazione del capitolato C5: \href{https://www.math.unipd.it/~tullio/IS-1/2019/Dispense/C5a.pdf}{https://www.math.unipd.it/\textasciitilde tullio/IS-1/2019/Dispense/C5a.pdf}.
  \item Slide del corso di Ingegneria del Software
        \begin{itemize}
          \item Slide di analisi dei requisiti, corso di Ingegneria del Software, diapositive da 4 a 30: \href{https://www.math.unipd.it/~tullio/IS-1/2019/Dispense/L08.pdf}{https://www.math.unipd.it/\textasciitilde tullio/IS-1/2019/Dispense/L08.pdf}.
          \item Diagrammi dei casi d'uso, diapositive da 2 a 32: \href{https://www.math.unipd.it/~tullio/IS-1/2019/Dispense/E03.pdf}{https://www.math.unipd.it/\textasciitilde tullio/IS-1/2019/Dispense/E03.pdf}.
        \end{itemize}
  % \item \href{https://www.openldap.org/}{Lightweight Directory Access Protocol}
  % \item \href{https://www.ics.uci.edu/~fielding/pubs/dissertation/rest_arch_style.htm}{Representational state transfer}
  % \item \href{https://grpc.io/docs/guides/}{gRPC}
\end{itemize}

\end{document}
