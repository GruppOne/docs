\documentclass[../analisi-dei-requisiti.tex]{subfiles}

\begin{document}
\subsection{Scopo del documento}%
\label{sub:scopo_del_documento}
Il presente \glossario{documento} ha lo scopo di descrivere in maniera dettagliata i \glossario{requisiti} impliciti ed espliciti individuati per il \glossario{prodotto}. Tali requisiti sono stati identificati dall'analisi del \glossario{capitolato} C5 ed i successivi incontri con il \glossario{proponente}, \textit{Imola Informatica}.
% sub:scopo_del_documento (end)
\subsection{Scopo del prodotto}%
\label{sub:scopo_del_prodotto}
L'obiettivo del \glossario{progetto} è sviluppare un'\glossario{applicazione mobile} distribuita, seguendo il \glossario{modello server/client}.
Il \glossario{client} deve essere in grado di segnalare sia l'ingresso che l'uscita dell'\glossario{utente} dalle \glossario{aree d'interesse} (in modalità anonima o meno a seconda delle esigenze), le quali sono definite dalle \glossario{organizzazioni}.
Il \glossario{server} deve fornire la possibilità di raccogliere ed analizzare i \glossario{dati} relativi alle organizzazioni.
In caso di \glossario{utenti anonimi} l'analisi riguarda solo una \glossario{stima} del numero totale di persone presenti in un dato momento.
In caso di \glossario{utenti autenticati} deve inoltre essere possibile effettuare \glossario{query} di monitoraggio specifiche.
\subparagraph*{Errore sulla geolocalizzazione}%
\label{subp:errore_sulla_geolocalizzazione}
È richiesto un \glossario{report} che esponga le scelte progettuali, le rispettive motivazioni e i test eseguiti per garantire la rilevazione sufficiente precisa della posizione, considerando le limitazioni dello \glossario{smartphone}.
% subsub:errore_sulla_geolocalizzazione (end)
% sub:scopo_del_prodotto (end)
\subsection{Glossario}%
\label{sub:glossario}
Al fine di rendere il documento più chiaro possibile, i termini che possono assumere un significato ambiguo sono evidenziati (i.e., \glossario{client}) e riportati in \textit{Glossario1.0.0.pdf} accompagnati da una definizione.
% sub:glossario (end)
\subsection{Riferimenti}%
\label{sub:riferimenti}
\subsubsection{Normativi}%
\label{par:normativi}
\begin{itemize}
  \item \textit{NormeDiProgetto1.0.0.pdf}
  \item Capitolato d'appalto C5 (https://www.math.unipd.it/\textasciitilde tullio/IS-1/2019/Progetto/C5.pdf)
\end{itemize}
% subsub:normativi (end)
\subsubsection{Informativi}%
\label{par:informativi}
\begin{itemize}
  \item Seminario di presentazione del capitolato C5 (https://www.math.unipd.it/\textasciitilde tullio/IS-1/2019/Dispense/C5a.pdf)
  \item Slide del corso di Ingegneria del Software
        \begin{itemize}
          \item Analisi dei requisiti (https://www.math.unipd.it/\textasciitilde tullio/IS-1/2019/Dispense/L08.pdf)
          \item Diagrammi dei \glossario{casi d'uso} (https://www.math.unipd.it/\textasciitilde tullio/IS-1/2019/Dispense/E03.pdf)
        \end{itemize}
  \item Lightweight Directory Access Protocol (https://it.wikipedia.org/wiki/Lightweight\_Directory\_Access\_Protocol)
  \item Representational state transfer (https://en.wikipedia.org/wiki/Representational\_state\_transfer)
  \item gRPC (https://grpc.io/docs/guides/)
\end{itemize}
% subsub:informativi
% sub:riferimenti (end)
\end{document}
