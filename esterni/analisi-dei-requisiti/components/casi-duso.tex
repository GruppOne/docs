\documentclass[../analisi-dei-requisiti]{subfiles}

\begin{document}
La presente sezione ha lo scopo di descrivere in maniera dettagliata, attraverso il linguaggio \glossario{UML}, le funzionalità offerte da \glossario{Stalker}.

\subsection{Attori dei casi d'uso}%
\label{sub:attori_casi_duso}

\subsubsection{App}%
\label{subsub:utenti}

\begin{itemize}
  \item \textbf{\glossario{Utente non registrato}}: Si riferisce ad un \glossario{utente generico} che non ha ancora effettuato la registrazione alla piattaforma;
  \item \textbf{\glossario{Utente non autenticato}}: Si riferisce ad un \glossario{utente generico} che non ha ancora effettuato l'autenticazione alla piattaforma;
  \item \textbf{\glossario{Utente autenticato}}: Si riferisce ad un \glossario{utente non autenticato} che ha effettuato l'autenticazione alla piattaforma;
  \item \textbf{\glossario{Utente non collegato}}: Si riferisce ad un \glossario{utente autenticato} che ha non ha effettuato il collegamento ad un'organizzazione;
  \item \textbf{\glossario{Utente collegato}}: Si riferisce ad un \glossario{utente autenticato} che ha effettuato il collegamento ad un'organizzazione;
  \item \textbf{\glossario{Utente collegato incognito}}: Si riferisce ad un \glossario{utente collegato} che ha deciso di rimanere in modalità incognito all'interno di un'organizzazione;
  \item \textbf{\glossario{Utente collegato noto}}: Si riferisce ad un \glossario{utente collegato} che ha deciso di essere noto all'interno di un'organizzazione;
\end{itemize}
% subsub:utenti (end)

\subsubsection{Web App}%
\label{subsub:super_utenti}

\begin{itemize}
  \item \textbf{\glossario{Utente non autenticato}}: si riferisce ad un \glossario{utente} che non ha ancora effettuato l'accesso alla web app;
  \item \textbf{\glossario{Utente autenticato}}: si riferisce ad un \glossario{utente non autenticato} che ha affettuato l'accesso alla web app;
  \item \textbf{\glossario{Amministratore}}: Si riferisce ad un \glossario{utente} con privilegi su tutto il sistema, che gestisce le richieste di \emph{Gestore} e \emph{Owner}.
  \item \textbf{\glossario{Root}}: Si riferisce ad un \glossario{amministratore} con privilegi avanzati su tutto il sistema. Gestisce gli amministratori.
  \item \textbf{\glossario{Visualizzatore}}: Si riferisce ad un \glossario{utente} con privilegi di visualizzazione sugli utenti e organizzazione.
  \item \textbf{\glossario{Gestore}}: Si riferisce ad un \glossario{visualizzatore} con privilegi di richieste di modifiche all'organizzazione.
  \item \textbf{\glossario{Owner}}: Si riferisce ad un \glossario{gestore} con privilegi di gestione utenti e organizzazione. È il proprietario di un'organizzazione.
\end{itemize}
% subsub:super_utenti (end)

% sub:attori_casi_duso (end)

\subsection{Elenco casi d'uso app}%
\label{sub:casi_duso_app}
\subfile{components/casi-duso-app.tex}
% sub:casi_duso_app (end)
\subsection{Elenco casi d'uso web app}%
\label{sub:casi_duso_web_app}
\subfile{components/casi-duso-web-app.tex}
% sub:casi_duso_web_app (end)


\end{document}
