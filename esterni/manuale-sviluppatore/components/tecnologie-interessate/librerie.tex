\documentclass[../../../analisi-dei-requisiti.tex]{subfiles}

% Iniziare frasi con "linguaggio/strumento/libreria/framework utilizzato per server/web app/mobile app ..." !!

\begin{document}

\subsection{Librerie}%
\label{sub:librerie}
In questa sezione sono descritte tutte le librerie utilizzate per l'implementazione di Stalker.

\subsubsection{Angular Material}%
\label{subs:angular_material}
Web Application

Gli stili della web application sono basati sulla libreria Angular Material.
%TODO estendere

\subsubsection{Leaflet}%
\label{subs:leaflet}
Web Application

Per gestire mappe interattive, è stata utilizzata la libreria JavaScript Leaflet, una libreria che si può integrare con Angular che consente di implementare mappe basate su \glossarioLocale{OpenStreetMap}.

Leaflet permette di creare, modificare ed eliminare luoghi direttamente sulla mappa grazie al componente \textit{Leaflet Draw}, ed ottenere le informazioni necessarie grazie alla funzionalità di \glossarioLocale{reverse geocoding}, svincolando l'utente dall'onere di inserire manualmente l'insieme di coordinate, l'indirizzo e il nome associato al luogo selezionato.
%TODO estendere

\subsubsection{Chai}%
\label{subs:chai}
Web Application

Libreria di asserzioni BDD\%.
%TODO estendere

\subsubsection{Volley}%
\label{subs:volley}
Mobile Application

Libreria open source consigliata da Google come wrapper per le funzionalità di connessione a internet di basso livello.
%TODO estendere

\subsubsection{Lombok}%
\label{subs:lombok}
Mobile Application

Libreria open source che genera automaticamente attraverso delle annotazioni Java buona parte del codice boilerplate (come getter e setter).
%TODO estendere

\subsubsection{JJWT}%
\label{subs:jjwt}
Mobile Application

Acronimo di Java JSON Web Token, libreria open source per trasmettere informazioni tra due componenti in formato \glossarioLocale{JSON}, e quindi nello specifico serve a creare, codificare e ottenere informazioni dai token. Questo file JSON viene inviato ad ogni richiesta di login da web application e mobile application.
Riferimento: \href{https://github.com/jwtk/jjwt}.
%TODO estendere

\end{document}