\documentclass[../../../analisi-dei-requisiti.tex]{subfiles}

% Iniziare frasi con "linguaggio/strumento/libreria/framework utilizzato per server/web app/mobile app ..." !!

\begin{document}

\subsection{Framework}%
\label{sub:framework}
In questa sezione sono descritti tutti i framework utilizzati per l'implementazione di Stalker.

\subsubsection{Spring}%
\label{subs:spring}
Back End

Per sviluppare il back-end è stato utilizzato Spring, un framework concepito per la realizzazione di software enterprise su piattaforma Java.

Viene utilizzato in particolare il framework Spring Boot WebFlux, un progetto Spring disponibile dalla versione 5.0 costruito intorno al pattern Publisher/Subscriber (detto anche pattern Observer), che oltre a semplificare lo sviluppo delle applicazioni e fornire opzioni per fare build e deploy delle applicazioni in esecuzione, consente di supportare flussi reattivi completamente non bloccanti. L'elaborazione delle richieste asincrone avviene tramite un gestore eventi che non blocca alcuna richiesta entrante. 

La build è ottenuta mediante l'utilizzo di dipendenze Gradle che, nel momento in cui la build avviene correttamente, consente l'avvio di un embedded server che fungerà da intermediario tra web application/mobile application e strato di persistenza (MySQL/InfluxDB).
%TODO estendere

\subsubsection{Angular}%
\label{subs:angular}
Web Application

Per lo sviluppo della web application è stato scelto il framework Angular, che usa i linguaggi HTML e TypeScript.

Il pattern architetturale di riferimento è il Model-View-ViewModel, che consente di creare una struttura basata sui seguenti elementi:
\begin{description}
    \item[view] insieme di elementi di visualizzazione.
    \item[component] classe che definisce una view.
    \item[service] classe che incapsula la business logic interagendo con un modello.   
\end{description}
%TODO estendere

\subsubsection{Jasmine}%
\label{subs:jasmine}
Web Application

Framework di test.
%TODO estendere

\subsubsection{Protractor}%
\label{subs:protractor}
Web Application

Framework per i test e2e in grado di simulare interamente le interazioni dell’utente con la web-app.
%TODO estendere

\subsubsection{JUnit 4}%
\label{subs:junit4}
Mobile Application

Framework di test, a cui è stato appoggiato \glossarioLocale{Mockito} per sostituire le parti non sotto esame di ciascun test ed \glossarioLocale{Espresso} per simulare l’interazione utente con le interfacce grafiche. 
%TODO estendere

\subsubsection{Mockito}%
\label{subs:mockito}
Mobile Application

Framework di test open source
%TODO estendere

\subsubsection{Espresso}%
\label{subs:espresso}
Mobile Application

Framework di test open source
%TODO estendere

\end{document}