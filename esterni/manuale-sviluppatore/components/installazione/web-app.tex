\documentclass[../manuale-sviluppatore.tex]{subfiles}

\begin{document}

\subsection{Web application}%
\label{sub:web_application}


\subsubsection{Installazione delle dipendenze}%
\label{subs:installazione_delle_dipendenze}

Prima di effettuare qualsiasi operazione, l'utente deve posizionarsi nella root della cartella che contiene il codice della web application e avviare dalla CLI di Windows (preferibilmente Windows PowerShell) il seguente comando: \par\bigskip

\begin{center}
  \textit{npm install}
\end{center}
\par\bigskip

In questo modo saranno installate tutte le dipendenze necessarie per l'esecuzione della web application.

La procedura impiegherà qualche minuto.

\subsubsection{Avvio del server}%
\label{subs:avvio_del_server}

Per ottenere un server di sviluppo, utilizzare il seguente comando: \par\bigskip

\begin{center}
  \textit{ng serve}
\end{center}
\par\bigskip

Se la procedura è andata a buon fine, digitare sulla barra degli indirizzi di un browser web l'indirizzo \href{http://localhost:4200/}{http://localhost:4200/} e verificare che si visualizzi la pagina home del sito.

La procedura impiegherà qualche minuto.

Non serve che questo comando sia digitato più di una volta, in quanto le pagine web si aggiornano automaticamente al cambiamento dei file sorgente.

\subsubsection{Creazione di nuovi componenti}%
\label{subs:creazione_dei_nuovi_componenti}

Per generare un nuovo componente, eseguire il seguente comando: \par\bigskip

\begin{center}
  \textit{ng generate component component-name}
\end{center}
\par\bigskip

Un'alternativa di questo comando è: \par\bigskip

\begin{center}
  \textit{ng generate directive|pipe|service|class|guard|interface|enum|module}
\end{center}
\par\bigskip

\subsubsection{Build}%
\label{subs:build}

Per effettuare la build di questo progetto, eseguire il seguente comando: \par\bigskip

\begin{center}
  \textit{ng build}
\end{center}
\par\bigskip

Per effettuare una build di produzione, eseguire il seguente comando: \par\bigskip

\begin{center}
  \textit{ng build --prod}
\end{center}
\par\bigskip

\subsubsection{Configurazione}%
\label{subs:configurazione}

Per eseguire comandi con una configurazione specifica, eseguire il seguente comando: \par\bigskip

\begin{center}
  \textit{ng build|serve|test --configuration=your-configuration}
\end{center}
\par\bigskip

\textbf{your-configuration} corrisponde alla configurazione desiderata.

Le configurazioni possibili sono:
\begin{itemize}
  \item localhost.
  \item imola.
  \item production.
\end{itemize}

Per eseguire la configurazione di default, non serve aggiungere il flag \textit{--configuration}.

\subsubsection{Aiuto}%
\label{subs:aiuto}

Per ricevere informazioni sui comandi di Angular CLI, eseguire:
\begin{itemize}
  \item \textit{ng help}.
  \item oppure visitare il README di Angular CLI il seguente \href{https://github.com/angular/angular-cli/blob/master/README.md}{indirizzo}.
\end{itemize}

\end{document}
