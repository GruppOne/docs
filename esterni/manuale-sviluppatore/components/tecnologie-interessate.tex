\documentclass[../manuale-sviluppatore.tex]{subfiles}

\begin{document}
\subsection{Back End}%
\label{sub:back-end}
In questa sezione sono descritte le tecnologie scelte dopo una fase di analisi per lo sviluppo del back-end.

\subsubsection{InfluxDB}%
\label{sub:influxDB}
Abbiamo deciso di utilizzare InfluxDB, un Time Series Database open source, viste le sue alte performance e le sue elevate capacità di scrittura e di sostenere carichi di interrogazioni 
importanti. Usiamo questo database per salvare dati momentanei, non persistenti.
% TODO estendere

\subsubsection{MySql}%
\label{sub:mysql}
Abbiamo deciso di utilizzare MySql per il salvataggio di dati persistenti attraverso il modello relazionale, il quale garantisce affidabilità.
% TODO estendere

\subsubsection{Spring}%
\label{sub:spring}
Per sviluppare il back-end abbiamo deciso di utilizzare Spring, un framework concepito per la realizzazione di applicazioni enterprise, che soddisfa i requisiti
di performance, sicurezza e affidabilità richiesti. \\
Spring contiene molte librerie, che ci hanno consentito di risolvere in modo elegante più di qualche requisito.
Unico lato negativo di questo framework è la sua documentazione, che può risultare molto dispersiva.
% TODO estendere

\subsubsection{Open API}%
\label{sub:open_api}
Grazie alla specifica Open API abbiamo definito le interfacce per la comunicazione, andando quindi a semplificare la sincronizzazione tra documentazione e codice sorgente.
Open API è in grado di generare codice, documentazione e test case dato un file di interfaccia.
% TODO estendere


\newpage
\subsection{Web Application}%
\label{sub:back-end}
In questa sezione sono descritte le tecnologie scelte dopo una fase di analisi per lo sviluppo dell'applicazione web.

\subsubsection{Angular}%
\label{sub:angular}
Abbiamo scelto Angular come framework per lo sviluppo web, visto che consente di creare applicazioni moderne e molto avanzate in modo facile e veloce.
Alla base dell'architettura Model View ViewModel di Angular c'è TypeScript, necessario a chiunque voglia utilizzare Angular.\\
I principali tipi di classi in Angular sono i components, cioè la vista, e i servizi, che interagiscono con un modello.
% TODO estendere

\subsubsection{Angular Material}%
\label{sub:angular}
Gli stili dell'applicazione web sono basati sulla libreria Angular Material, che ci permette di ottenere dei risultati grafici eccellenti in poco tempo.
% TODO estendere


\subsubsection{Leaflet}%
\label{sub:angular}
Grazie alla libreria Leaflet e alla sua possibile integrazione in Angular, siamo riusciti a implementare delle mappe basate su OpenStreetMap che possano rendere
l'interfaccia dell'applicazione web friendly, andando a semplificare di molto l'utilizzo agli utenti inesperti.\\
 Leaflet permette di creare direttamente dalle mappe i luoghi selezionandoli, modificarli o eliminarli.
% TODO estendere

\end{document}
