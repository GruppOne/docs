\documentclass[../piano-di-qualifica.tex]{subfiles}
\appendToGraphicspath{../../../commons/img/}

\begin{document}

In questa sezione ci occupiamo di definire gli obiettivi di qualità dei processi da noi istanziati nel corso del progetto.
Prevediamo di mantenere il documento in continuo aggiornamento, espandendo progressivamente le sezioni interessate durante lo svolgimento delle diverse attività che non abbiamo ancora affrontato.

\subsection{Descrizione}%
\label{sub:descrizione}

La qualità di processo viene perseguita attraverso metriche e strumenti di valutazione per stabilire la bontà dei processi.
Dato il legame stretto tra processi e prodotti (essendo il prodotto l'output di un processo), il team sente l'esigenza di definire un buon sistema di qualità: migliorare un processo, infatti, significa anche migliorare la qualità del prodotto software risultante.
Il way of working dei processi software che GruppOne ha deciso di istanziare è definito all'interno delle \textit{Norme di progetto} insieme alle rispettive metriche di valutazione, mentre nei prossimi paragrafi si cercherà di strutturare l'organizzazione del sistema di qualità.
Nel complesso la qualità di processo si occupa di:
\begin{itemize}
  \item Presentare i processi istanziati e le ragioni di tali scelte.
  \item Controllare il processo e migliorarlo.
  \item Definire dei valori soglia delle metriche per avere una quantificazione oggettiva della qualità del processo in esame.
\end{itemize}


\subsection{Qualità di accertamento della qualità}%
\label{subsec:qualita_accertamento_qualita}

\subsubsection{Metriche}%
\label{subsec:metriche_aq}
Di seguito sono riportate le metriche utilizzate per misurare la qualità di accertamento della qualità, che possono essere visionate all'interno del documento \textit{Norme di progetto} alla sezione 3.4.5.



\rowcolors{2}{lightgray}{white!80!lightgray!100}
\renewcommand{\arraystretch}{2} % allarga le righe con dello spazio sotto e sopra
\begin{longtable}[H]{>{\centering\bfseries}m{3cm} >{}m{4cm} >{}m{5cm} >{\centering\arraybackslash}m{2cm} > {\centering\arraybackslash}m{2cm}}
  \rowcolor{darkgray!90!}
  \color{white}
  {\textbf{Codice}} & \color{white}{\textbf{Nome}} & \color{white}{\textbf{Descrizione} } & \color{white}{\textbf{Valore ammissibile}}  & \color{white}{\textbf{Valore ottimale}}   \\
  \endhead\rowcolor{white}%
  \multicolumn{3}{r}{\textit{Continua alla pagina seguente}}
  \endfoot%
  \endlastfoot%

  % - Applicazione mobile
  MPS-PME & Percentuale di metriche  & misura la percentuale di metriche che raggiungono risultati ottimali. &  10\% & 100\% \\

    \rowcolor{white}
    \caption{Tabella delle metriche di qualità di accertamento della qualità}%
    \label{tab:metriche_acc}
  \end{longtable}


\subsection{Qualità di gestione di processo}%
\label{subsec:qualita_processo}

\subsubsection{Metriche}%
\label{subsec:metriche_pr}

Di seguito sono riportate le metriche utilizzate per misurare la qualità di gestione di processo, che possono essere visionate all'interno del documento \textit{Norme di progetto} alla sezione 4.3.4.

\rowcolors{2}{lightgray}{white!80!lightgray!100}
\renewcommand{\arraystretch}{2} % allarga le righe con dello spazio sotto e sopra
\begin{longtable}[H]{>{\centering\bfseries}m{3cm} >{}m{4cm} >{}m{5cm} >{\centering\arraybackslash}m{2cm} > {\centering\arraybackslash}m{2cm}}
  \rowcolor{darkgray!90!}
  \color{white}
  {\textbf{Codice}} & \color{white}{\textbf{Nome}} & \color{white}{\textbf{Descrizione} } & \color{white}{\textbf{Valore ammissibile}}  & \color{white}{\textbf{Valore ottimale}}   \\
  \endhead\rowcolor{white}%
  \multicolumn{3}{r}{\textit{Continua alla pagina seguente}}
  \endfoot%
  \endlastfoot%

  % - Applicazione mobile
  MPS-DE & Discostamento economico  & misura quanto i costi effettivamente sostenuti nell'esecuzione del
  processo si discostano dai costi previsti. &   -5\%\(\leq\cdot\leq\)5\% &  0\% \\ % chktex 21

  MPS-DO &  Discostamento orario  & misura quanto il totale orario effettivamente lavorato durante l'esecuzione
  del processo si discosta dal totale orario previsto. & -5\%\(\leq\cdot\leq\)5\% &  0\% \\ % chktex 21

    \rowcolor{white}
    \caption{Tabella delle metriche di qualità di gestione di processo}%
    \label{tab:metriche_proc}
  \end{longtable}




\subsection{Qualità di verifica}%
\label{subsec:qualita_verifica}

\subsubsection{Metriche}%
\label{subsec:metriche_ver}

Di seguito sono riportate le metriche utilizzate per misurare la qualità di verifica, che possono essere visionate all'interno del documento \textit{Norme di progetto} alla sezione 3.5.5.

\rowcolors{2}{lightgray}{white!80!lightgray!100}
\renewcommand{\arraystretch}{2} % allarga le righe con dello spazio sotto e sopra
\begin{longtable}[H]{>{\centering\bfseries}m{3cm} >{}m{4cm} >{}m{5cm} >{\centering\arraybackslash}m{2cm} > {\centering\arraybackslash}m{2cm}}
  \rowcolor{darkgray!90!}
  \color{white}
  {\textbf{Codice}} & \color{white}{\textbf{Nome}} & \color{white}{\textbf{Descrizione} } & \color{white}{\textbf{Valore ammissibile}}  & \color{white}{\textbf{Valore ottimale}}   \\
  \endhead\rowcolor{white}%
  \multicolumn{3}{r}{\textit{Continua alla pagina seguente}}
  \endfoot%
  \endlastfoot%

  % - Applicazione mobile
  MPS-COC & Percentuale di copertura del codice  & misura la quantità di codice che viene esaminata attraverso i test di unità. &   >80\% &  100\%  \\

  MPS-TPA &  Percentuale di test passati  &  misura la percentuale di test superati con successo in una specifica fase del progetto fino alla data corrente. &   100\% &  100\% \\

  MPS-TNP &  Percentuale di test non passati  &  misura la percentuale di test non passati in una specifica fase del progetto fino alla data corrente. &  0\% &  0\% \\

    \rowcolor{white}
    \caption{Tabella delle metriche di qualità di verifica}%
    \label{tab:metriche_veri}
  \end{longtable}



\subsection{Qualità di gestione dei processi}%
\label{subsec:qualita_gestione_processi}

\subsubsection{Metriche}%
\label{subsec:metriche_gest_proc}


Di seguito sono riportate le metriche utilizzate per misurare la qualità di gestione dei processi, che possono essere visionate all'interno del documento \textit{Norme di progetto} alla sezione 4.2.6.

\rowcolors{2}{lightgray}{white!80!lightgray!100}
\renewcommand{\arraystretch}{2} % allarga le righe con dello spazio sotto e sopra
\begin{longtable}[H]{>{\centering\bfseries}m{3cm} >{}m{4cm} >{}m{5cm} >{\centering\arraybackslash}m{2cm} > {\centering\arraybackslash}m{2cm}}
  \rowcolor{darkgray!90!}
  \color{white}
  {\textbf{Codice}} & \color{white}{\textbf{Nome}} & \color{white}{\textbf{Descrizione} } & \color{white}{\textbf{Valore ammissibile}}  & \color{white}{\textbf{Valore ottimale}}   \\
  \endhead\rowcolor{white}%
  \multicolumn{3}{r}{\textit{Continua alla pagina seguente}}
  \endfoot%
  \endlastfoot%

  % - Applicazione mobile
  MPS-RNP & Rischi non preventivati  & misura il numero dei rischi non preventivati incontrati. &    <5 &  0 \\

    \rowcolor{white}
    \caption{Tabella delle metriche di qualità di gestione dei processi}%
    \label{tab:metriche_gest_proc}
  \end{longtable}



\end{document}
