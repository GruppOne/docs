\documentclass[../piano-di-qualifica.tex]{subfiles}
\appendToGraphicspath{../../../commons/img/}


\begin{document}

\section{Qualità di processo}%
\label{sec:qualità_di_processo}
In questa sezione si definisce la qualità di processo. Il documento è in continuo aggiornamento in quanto molti dei processi e delle attività trattate nelle \glossario{norme di progetto} non sono ancora stati affrontati.

\subsection{Descrizione}%
\label{sub:descrizione}

La qualità di processo offre delle metriche e degli strumenti di valutazione per stabilire la bontà dei processi software istanziati. 
Dato il legame stretto tra processi e prodotti (essendo il prodotto l`output di un processo), il team sente l'esigenza di definire un buon sistema di qualità: migliorare un processo, infatti, significa anche migliorare la qualità del prodotto software risultante. 
I processi software che GruppOne ha deciso di instanziare sono definiti all'interno delle norme di progetto insieme alle rispettive metriche di valutazione, mentre nei prossimi paragrafi si cercherà di strutturare l'organizzazione del sistema di qualità. 
Nel complesso la qualità di processo si occupa di:
\begin{itemize}
	\item Presentare i processi instanziati e le ragioni di tali scelte.
	\item Controllare il processo e migliorarlo.
	\item Definire de metriche per avere una quantificazione oggettiva della qualità del processo in esame.
\end{itemize}

\subsection{Processi primari}%
\label{sub:processi_primari}

\subsubsection{Processo di sviluppo}%
\label{subs:processo_di_sviluppo}
Il processo di sviluppo è sicuramente uno dei più articolati. Per tale ragione ogni attività del processo di sviluppo (e.g.analisi dei requisiti, progettazione architetturale) ha un paragrafo dedicato.

\paragraph{Analisi dei requisiti}%
\label{par:analisi_dei_requisiti}

\paragraph{Progettazione architetturale}%
\label{par:progettazione_architetturale}

\subsection{Processi di supporto}%
\label{sub:processi_di_supporto}

\subsubsection{Processo di documentazione}%
\label{subs:processo_di_documentazione}

\subsubsection{Processo di verifica}%
\label{subs:processo_di_verifica}

\subsection{Processi organizzativi}%
\label{sub:processi_organizzativi}

\subsubsection{Processo di gestione del personale}%
\label{processo_di_gestione_del_personale}

\subsubsection{Processo_di gestione dei rischi}%
\label{processo_di_gestione_dei_rischi}

% TODO scrivimi

% TODO q. di p. documentazione -> indice gulpease

\end{document}
