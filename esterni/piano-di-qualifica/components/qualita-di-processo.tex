\documentclass[../piano-di-qualifica.tex]{subfiles}
\appendToGraphicspath{../../../commons/img/}

\begin{document}

In questa sezione ci occupiamo di definire gli obiettivi di qualità dei processi da noi istanziati nel corso del progetto.
Prevediamo di mantenere il documento in continuo aggiornamento, espandendo progressivamente le sezioni interessate durante lo svolgimento delle diverse attività che non abbiamo ancora affrontato.

\subsection{Descrizione}%
\label{sub:descrizione}

La qualità di processo viene perseguita attraverso metriche e strumenti di valutazione per stabilire la bontà dei processi.
Dato il legame stretto tra processi e prodotti (essendo il prodotto l`output di un processo), il team sente l'esigenza di definire un buon sistema di qualità: migliorare un processo, infatti, significa anche migliorare la qualità del prodotto software risultante.
Il way of working dei processi software che GruppOne ha deciso di istanziare è definito all'interno delle \textit{Norme di progetto} insieme alle rispettive metriche di valutazione, mentre nei prossimi paragrafi si cercherà di strutturare l'organizzazione del sistema di qualità.
Nel complesso la qualità di processo si occupa di:
\begin{itemize}
  \item Presentare i processi istanziati e le ragioni di tali scelte.
  \item Controllare il processo e migliorarlo.
  \item Definire dei valori soglia delle metriche per avere una quantificazione oggettiva della qualità del processo in esame.
\end{itemize}

% \subsection{Qualità di sviluppo}%
% \label{subsec:qualita_sviluppo}

% \subsubsection{Metriche}%
% \label{par:metriche_sv}


% \begin{description}
% \item [[MPS-NFV \textasciitilde] Nomi di variabili o di titoli di metodi o classi che non rispettano le norme stabilite]: definito nelle \textit{Norme di progetto} alla sezione 2.3.4.2.
% \begin{itemize} \item Risultato: numero di nomi di variabili o di titoli di metodi o classi che non rispettano le norme stabilite. \item Valore ammissibile: 0\% \item Valore ottimale: 0\% \end{itemize}

% \item [[MPS-CNN \textasciitilde] Commenti non normati]: definito nelle \textit{Norme di progetto} alla sezione 2.3.4.3.
%   \begin{itemize} \item Risultato: numero di commenti che non rispettano le norme stabilite. \item Valore ammissibile: \leq{}  0 \item Valore ottimale: \leq{}  0 \end{itemize}

% \item [[MPS-ICC \textasciitilde] Classi che ereditano da una classe concreta]: definita nelle \textit{Norme di progetto} alla sezione 2.3.4.5.
%   \begin{itemize}  \item Risultato: numero di classi che ereditano da classi concrete. \item Valore ammissibile: \leq{}  0 \item Valore ottimale: \leq{}  0 \end{itemize}
% \end{description}


% \subsection{Qualità di risoluzione dei problemi}%
% \label{subsec:qualita_sviluppo}

% \subsubsection{Metriche}%
% \label{par:metriche_rp}



% \begin{description}
%   \item [[MPS-RDP \textasciitilde] Indice di velocità di risoluzione dei problemi]: definito nelle \textit{Norme di progetto} alla sezione 3.7.3.1.
%   \begin{itemize} \item Risultato: numero intero che rappresenta la quantità di giorni trascorsi tra l'apertura della issue e la sua chiusura. \item Valore ammissibile: 1\% \item Valore ottimale: 7\% \end{itemize}
% \end{description}


% \subsection{Qualità di gestione dei rischi}%
% \label{subsec:qualita_sviluppo}

% \subsubsection{Metriche}%
% \label{par:metriche_rp}

% \begin{description}
%   \item [[MPS-RDP \textasciitilde] Indice dei rischi incontrati ma non preventivati]: definito nelle \textit{Norme di progetto} alla sezione 4.2.2.1.
%   \begin{itemize} \item Risultato: numero intero che rappresenta il numero di rischi incontrati durante il progetto che non erano stati previsti. \item Valore ammissibile: 0\% \item Valore ottimale: 0\% \end{itemize}
% \end{description}

\subsection{Qualità di accertamento della qualità}%
\label{subsec:qualita_accertamento_qualita}

\subsubsection{Metriche}%
\label{subsec:metriche_aq}

\begin{description}
  \item [[MPS-PME \textasciitilde] Percentuale di metriche eccellenti]: definito nelle \textit{Norme di progetto} alla sezione 3.4.5.1.
        \begin{itemize} \item Risultato: misura la percentuale di metriche che raggiungono risultati ottimali. \item Valore ammissibile: 10\% \item Valore ottimale: 100\%  \end{itemize}
\end{description}

\subsection{Qualità di gestione di processo}%
\label{subsec:qualita_processo}

\subsubsection{Metriche}%
\label{subsec:metriche_pr}

\begin{description}
  \item [[MPS-DE \textasciitilde] Discostamento economico]: definito nelle \textit{Norme di progetto} alla sezione 4.3.4.1.
        \begin{itemize} \item Risultato: misura quanto i costi effettivamente sostenuti nell'esecuzione del
                processo si discostano dai costi previsti. \item Valore ammissibile: -5\% o 5\% \item Valore ottimale: 0\% \end{itemize}

  \item [[MPS-DO \textasciitilde] Discostamento orario]: definito nelle \textit{Norme di progetto} alla sezione 4.3.4.2.
        \begin{itemize} \item Risultato: misura quanto il totale orario effettivamente lavorato durante l'esecuzione
                del processo si discosta dal totale orario previsto. \item Valore ammissibile: 5\% \item Valore ottimale: 0\% \end{itemize}
\end{description}


\subsection{Qualità di verifica}%
\label{subsec:qualita_verifica}

\subsubsection{Metriche}%
\label{subsec:metriche_ver}

\begin{description}
  \item [[MPS-COC \textasciitilde] Percentuale di copertura del codice]: definito nelle \textit{Norme di progetto} alla sezione 3.5.5.1.
        \begin{itemize} \item Risultato: misura la quantità di codice che viene esaminata attraverso i test di unità. \item Valore ammissibile: >80\% \item Valore ottimale: 100\% \end{itemize}
  \item [[MPS-TPA \textasciitilde] Percentuale di test passati]: definito nelle \textit{Norme di progetto} alla sezione 3.5.5.2.
        \begin{itemize} \item Risultato: misura la percentuale di test superati con successo in una specifica fase del progetto fino alla data corrente. \item Valore ammissibile: 100\% \item Valore ottimale: 100\% \end{itemize}
  \item [[MPS-TNP \textasciitilde] Percentuale di test non passati]: definito nelle \textit{Norme di progetto} alla sezione 3.5.5.3.
        \begin{itemize} \item Risultato: misura la percentuale di test non passati in una specifica fase del progetto fino alla data corrente. \item Valore ammissibile: 0\% \item Valore ottimale: 0\% \end{itemize}

\end{description}



\subsection{Qualità di gestione dei processi}%
\label{subsec:qualita_gestione_processi}

\subsubsection{Metriche}%
\label{subsec:metriche_gest_proc}

\begin{description}
  \item [[MPS-RNP \textasciitilde] Rischi non preventivati]: definito nelle \textit{Norme di progetto} alla sezione 4.2.6.1.
        \begin{itemize} \item Risultato: misura il numero dei rischi non preventivati incontrati. \item Valore ammissibile: <5 \item Valore ottimale: 0 \end{itemize}
\end{description}


% TODO - copiare descrizioni metriche MISURABILI nelle norme


% \subsection{Processi primari}%
% \label{sub:processi_primari}

% \subsubsection{Processo di sviluppo}%
% \label{subs:processo_di_sviluppo}
% Il processo di sviluppo è sicuramente uno dei più articolati. Per tale ragione ogni attività del processo di sviluppo (e.g.\ analisi dei requisiti, progettazione architetturale) ha un paragrafo dedicato.

% \paragraph{Analisi dei requisiti}%
% \label{par:analisi_dei_requisiti}

% \subparagraph{Obiettivi}%
% \label{par:obiettivi}
% L'obiettivo dell'attività di analisi dei requisiti è di trasformare i requisiti dello stakeholder in requisiti tecnici che possano guidare il team alla progettazione del software.

% \subparagraph{Esiti}%
% \label{par:esiti}
% I risultati di un'analisi dei requisiti di qualità sono:
% \begin{itemize}
%   \item La definizione di una serie di requisiti che possano descrivere il problema.
%   \item Una classificazione dei requisiti in modo da poterne tracciare i cambiamenti.
%   \item La realizzazione dei diagrammi dei casi d'uso per iniziare la progettazione.
%   \item La valutazione delle richieste degli stakeholders per negoziare modifiche, se necessarie.
% \end{itemize}

% \subparagraph{Metriche}%
% \label{par:metriche}



% \subsection{Processi di supporto}%
% \label{sub:processi_di_supporto}


% \subsubsection{Processo di risoluzione dei problemi}%
% \label{subs:processo_di_risoluzione_dei_problemi}

% \paragraph{Obiettivi}%
% \label{par:obiettivi_prob}

% Lo scopo del processo di risoluzione dei problemi è di mantenere traccia delle problematiche riscontrate durante lo svolgimento del progetto e di offrire soluzioni.

% % \paragraph{Esiti}%
% % \label{par:esiti}

% \paragraph{Metriche}%
% \label{par:metriche_prob}

% I risultati di un processo di risoluzione dei problemi di qualità sono:
% \begin{itemize}

%   \item L'identificazione e l'analisi di problemi riscontrati durante lo svolgimento del progetto.
%   \item La definizione di strategie e procedure per risolverli.
% \end{itemize}

% \textbf{MPS-RDP}: l'indice di velocità di risoluzione dei problemi è definito nelle \textit{Norme di progetto} alla sezione 3.7.3.1.
% \begin{itemize}
%   \item Risultato: numero intero che rappresenta la quantità di giorni trascorsi tra l'apertura della issue e la sua chiusura.
%   \item Valore ottimale: 1.
%   \item Valore minimale: 7.
% \end{itemize}

% \paragraph{Processo di accertamento della qualità}%
% \label{par:processo_di_accertamento_della_qualita}

% \paragraph{Obiettivi}%
% \label{par:obiettivi_accertamento_della_qualita}

% Lo scopo del processo di accertamento della qualità è di garantire che i prodotti e i processi del ciclo di vita del software rispettino i requisiti prestabiliti e aderire ai piani esecutivi prefissati.

% \paragraph{Esiti}%
% \label{par:esiti}

% Il processo di accertamento della qualità implica che:
% \begin{itemize}
%   \item la qualità dei processi di fornitura sia assicurata attraverso il miglioramento continuo.
%   \item vengano soddisfatte tutte le richieste avanzate del proponente.
%   \item che non si consumino più risorse di quel che il fornitore ha a disposizione.
% \end{itemize}

% \paragraph{Metriche}%
% \label{par:metriche_prob}

% \textbf{MPS-PMS}: la metrica che definisce la percentuale di metriche soddisfatte è definita nelle \textit{Norme di progetto} alla sezione 3.4.4.1.
% \begin{itemize}
%   \item Risultato: percentuale compresa tra 0 e 100, che indica quante metriche sono state soddisfatte rispetto alle metriche totali.
%   \item Valore ottimale: compreso tra 75 e 100.
%   \item Valore minimale: non inferiore a 50.
% \end{itemize}

% \subsection{Processi organizzativi}%
% \label{sub:processi_organizzativi}

% \subsubsection{Processo di gestione dei rischi}%
% \label{sub:processo_di_gestione_dei_rischi}

% \paragraph{Obiettivi}%
% \label{par:obiettivi_rischi}

% L'obiettivo del processo di gestione dei rischi è di identificare, analizzare, monitorare e prevenire i rischi che possono verificarsi nel corso del ciclo di vita del software.

% \paragraph{Esiti}%
% \label{par:esiti_rischi}

% I risultati di un processo di gestione dei rischi di qualità sono:
% \begin{itemize}
%   \item La definizione dei rischi che si incontrano nello svolgimento del progetto.
%   \item La classificazione dei rischi in modo da determinarne la probabilità di verificarsi e la pericolosità.
%   \item La determinazione di strategie, soluzioni e metodi per prevenirli.
%   \item L'applicazione di procedure per frenare l'impatto negativo che potrebbero avere nel caso in cui si verificassero.
% \end{itemize}

% \paragraph{Metriche}%
% \label{par:metriche_rischi}

% \textbf{MPS-RNP}: l'indice dei rischi incontrati ma non preventivati è definito nelle \textit{Norme di progetto} alla sezione 4.2.2.1.
% \begin{itemize}
%   \item Risultato: numero intero che rappresenta il numero di rischi incontrati durante il progetto che non erano stati previsti.
%   \item Valore ottimale: 0.
%   \item Valore minimale: 0.
% \end{itemize}

% \subsubsection{Processo di gestione del personale}%
% \label{subs:processo_di_gestione_del_personale}

% \paragraph{Pianificazione}%
% \label{par:pianificazione}

% \subparagraph{Obiettivi}%
% \label{subp:obiettivi_pers}
% L'obiettivo dell'attività di pianificazione è di stabilire e organizzare le principali attività e compiti di progetto.

% \subparagraph{Esiti}%
% \label{subp:esiti_pers}
% I risultati di un'attività di pianificazione di qualità sono:
% \begin{itemize}
%   \item L'individuazione dello scopo del progetto.
%   \item L'identificazione delle risorse umani e temporali a disposizione e delle attività da svolgere.
%   \item La pianificazione di quanto stimato al punto precedente.
%   \item La definizione di un preventivo e di un consuntivo.
%   \item La stesura di un \textit{Piano di progetto}.
% \end{itemize}

% \paragraph{Metriche}%
% \label{par:metriche_pers}

% \textbf{MPS-VDC}: l'indice di varianza dei costi è definito nelle \textit{Norme di progetto} alla sezione 4.3.5.1.
% \begin{itemize}
%   \item Risultato: numero che rappresenta l'efficienza e la produttività.
%   \item Valore ottimale: maggiore di 0.
%   \item Valore minimale: maggiore o uguale a 0.
% \end{itemize}

% \textbf{MPS-VDS}: l'indice di varianza rispetto allo schedule è definito nelle \textit{Norme di progetto} alla sezione 4.3.5.2.
% \begin{itemize}
%   \item Risultato: numero che indica se si è in anticipo o in ritardo nella schedulazione rispetto a quanto pianificato.
%   \item Valore ottimale: maggiore di 0.
%   \item Valore minimale: maggiore o uguale a 0.
% \end{itemize}

\end{document}
