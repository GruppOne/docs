\documentclass[../piano-di-qualifica.tex]{subfiles}
\appendToGraphicspath{../../../commons/img/}


\begin{document}

In questa sezione ci occupiamo di definire gli obiettivi di qualità dei processi da noi istanziati nel corso del progetto.
Prevediamo di mantenere il documento in continuo aggiornamento, di pari passo con lo svolgimento delle diverse attività che non abbiamo ancora affrontato.
Per questo motivo, in alcune parti ci limiteremo a descrivere come prevediamo di garantire qualità invece di fissare a priori obiettivi che potrebbero essere non rilevanti.

\subsection{Descrizione}%
\label{sub:descrizione}

La qualità di processo viene perseguita attraverso metriche e strumenti di valutazione per stabilire la bontà dei processi.
Dato il legame stretto tra processi e prodotti (essendo il prodotto l`output di un processo), il team sente l'esigenza di definire un buon sistema di qualità: migliorare un processo, infatti, significa anche migliorare la qualità del prodotto software risultante.
Il way of working dei processi software che GruppOne ha deciso di istanziare è definito all'interno delle norme di progetto insieme alle rispettive metriche di valutazione, mentre nei prossimi paragrafi si cercherà di strutturare l'organizzazione del sistema di qualità.
Nel complesso la qualità di processo si occupa di:
\begin{itemize}
  \item Presentare i processi instanziati e le ragioni di tali scelte.
  \item Controllare il processo e migliorarlo.
  \item Definire dei valori soglia delle metriche per avere una quantificazione oggettiva della qualità del processo in esame.
\end{itemize}

\subsection{Processi primari}%
\label{sub:processi_primari}

\subsubsection{Processo di sviluppo}%
\label{subs:processo_di_sviluppo}
Il processo di sviluppo è sicuramente uno dei più articolati. Per tale ragione ogni attività del processo di sviluppo (e.g.\ analisi dei requisiti, progettazione architetturale) ha un paragrafo dedicato.

\paragraph{Analisi dei requisiti}%
\label{par:analisi_dei_requisiti}

\paragraph{Progettazione architetturale}%
\label{par:progettazione_architetturale}

\subsection{Processi di supporto}%
\label{sub:processi_di_supporto}

\subsubsection{Processo di documentazione}%
\label{subs:processo_di_documentazione}

\subsubsection{Processo di verifica}%
\label{subs:processo_di_verifica}

\subsection{Processi organizzativi}%
\label{sub:processi_organizzativi}

\subsubsection{Processo di gestione del personale}%
\label{sub:processo_di_gestione_del_personale}

\subsubsection{Processo di gestione dei rischi}%
\label{sub:processo_di_gestione_dei_rischi}

\end{document}
