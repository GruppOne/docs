\documentclass[../piano-di-qualifica.tex]{subfiles}
\appendToGraphicspath{../../../commons/img/}

\begin{document}

Per misurare la qualità di prodotto inoltre utilizziamo dei test secondo il \glossario{Modello a \textit{V}}, per il quale definiamo e sviluppiamo i test in parallelo alle attività di analisi, progettazione architetturale e di sviluppo e verifica incrementi.
Abbiamo definito i test dividendoli nelle seguenti categorie:
\begin{description}
  \item [Test di Accettazione (TA)].
  \item [Test di Sistema (TS)].
  \item [Test di Integrazione (TI)].
  \item [Test di Unità (TU)].
\end{description}

Inoltre classifichiamo i test in base al loro stato:
\begin{description}
  \item [S]: il test è stato soddisfatto.
  \item [NS]: il test non è stato ancora soddisfatto.
\end{description}

\subsubsection{Test di Accettazione}%
\label{subs:test_di_accettazione}

I test di accettazione vengono attuati dal proponente e dal committente, in sede di collaudo finale, e servono a dimostrare che il prodotto soddisfi tutti i requisiti.

\subsubsection{Test di Sistema}%
\label{subs:test_di_sistema}

I test di sistema hanno lo scopo di verificare che il prodotto soddisfi i requisiti richiesti, vengono eseguiti durante la fase di verifica e collaudo finale.
Essendo tutti i test derivanti da un gruppo di requisiti che hanno una determinata importanza nel prodotto software, abbiamo pensato di numerare i test di sistema nel modo seguente:
\begin{center}
  [TS][numero][tipo][priorità]
\end{center}

Dove: \textit{priorità} è un valore numerico da 1 a 3 ad indicare l'importanza del requisito che deve essere soddisfatto, \textit{numero} è un numero identificativo incrementale per tipologia che parte da 1 e \textit{tipo} può assumere i seguenti valori:
\begin{description}
  \item [F]: requisito funzionale.
  \item [P]: requisito prestazionale.
  \item [V]: requisito vincolo.
  \item [Q]: requisito qualità.
\end{description}

\rowcolors{2}{lightgray}{white!80!lightgray!100}
\renewcommand{\arraystretch}{2} % allarga le righe con dello spazio sotto e sopra
\begin{longtable}[H]{>{\centering\bfseries}m{3cm} >{}m{10cm} >{\centering\arraybackslash}m{3cm}}
  \rowcolor{darkgray!90!}
  \color{white}
  {\textbf{ID test}} & \color{white}{\textbf{Descrizione}}                                                                                                                                                                                              & \color{white}{\textbf{Esito}} \\
  \endhead
  \rowcolor{white}%
  \multicolumn{3}{r}{\textit{Continua alla pagina seguente}}
  \endfoot%
  \endlastfoot%

  % - Applicazione mobile

  TS001F1            & L'applicazione mobile deve permettere ad un nuovo utente di registrarsi. \newline
  L'utente deve:
  \begin{itemize}
    \item visualizzare e confermare l'EULA\@.
    \item inserire la propria email.
    \item inserire password.
    \item confermare la password.
    \item inserire i propri dati anagrafici.
    \item confermare la registrazione.
  \end{itemize}
                     & NS                                                                                                                                                                                                                                                               \\
  TS002F1            & Il sistema deve rifiutare la richiesta di registrazione se i dati inseriti non rispettano i vincoli imposti. \newline
  L'utente deve:
  \begin{itemize}
    \item inserire dati per la registrazione non validi.
    \item verificare l'impossibilità di proseguire con la registrazione.
  \end{itemize}
                     & NS                                                                                                                                                                                                                                                               \\
  TS003F1            & L'applicazione mobile deve permettere all'utente registrato di autenticarsi. \newline
  L'utente deve:
  \begin{itemize}
    \item inserire la propria email.
    \item inserire la propria password.
    \item confermare l'autenticazione.
  \end{itemize}
                     & NS                                                                                                                                                                                                                                                               \\
  TS004F1            & Il sistema deve rifiutare la richiesta di autenticazione se la combinazione di email e password non è presente nel database. \newline
  L'utente deve:
  \begin{itemize}
    \item inserire dati di accesso non validi.
    \item verificare l'impossibilità di proseguire con l'autenticazione.
  \end{itemize}
                     & NS                                                                                                                                                                                                                                                               \\
  TS005F1            & L'applicazione mobile deve permettere all'utente di recuperare la sua password. \newline
  L'utente deve:
  \begin{itemize}
    \item avviare la procedura di recupero password.
    \item inserire la propria email.
    \item avviare la procedura di reimpostazione password tramite il link ricevuto via mail.
    \item inserire una nuova password.
    \item confermare la nuova password.
    \item confermare la procedura.
    \item verificare che la password sia stata cambiata correttamente.
  \end{itemize}
                     & NS                                                                                                                                                                                                                                                               \\
  TS006F1            & Il sistema deve rifiutare la richiesta di reimpostazione password se la mail inserita all'inizio della procedura non è presente all'interno del database del sistema. \newline
  L'utente deve:
  \begin{itemize}
    \item avviare la procedura di recupero password.
    \item inserire una mail non presente nel database del sistema.
    \item visualizzare un messaggio di errore.
  \end{itemize}
                     & NS                                                                                                                                                                                                                                                               \\
  TS007F1            & Il sistema deve rifiutare la richiesta di reimpostazione password se la nuova password inserita non rispetta i vincoli imposti. \newline
  L'utente deve:
  \begin{itemize}
    \item avviare la procedura di recupero password.
    \item inserire la propria email.
    \item avviare la procedura di reimpostazione password tramite il link ricevuto via mail.
    \item inserire una nuova password non valida.
    \item confermare la nuova password.
    \item confermare la procedura.
    \item visualizzare un messaggio di errore.
  \end{itemize}
                     & NS                                                                                                                                                                                                                                                               \\
  TS008F1            & Il sistema deve rifiutare la richiesta di reimpostazione password se la nuova password e la conferma della password non sono valide. \newline
  L'utente deve:
  \begin{itemize}
    \item avviare la procedura di recupero password.
    \item inserire la propria email.
    \item avviare la procedura di reimpostazione password tramite il link ricevuto via mail.
    \item inserire una nuova password.
    \item confermare la nuova password in modo errato.
    \item confermare la procedura.
    \item visualizzare un messaggio di errore.
  \end{itemize}
                     & NS                                                                                                                                                                                                                                                               \\
  TS009F1            & L'applicazione mobile deve permettere all'utente autenticato di recuperare dal server una lista di organizzazioni a cui può collegarsi. \newline
  L'utente deve:
  \begin{itemize}
    \item visualizzare la lista di organizzazioni a cui può collegarsi.
  \end{itemize}
                     & NS                                                                                                                                                                                                                                                               \\
  TS010F1            & L'applicazione mobile deve permettere all'utente autenticato di visualizzare dal server una lista di organizzazioni a cui è collegato. \newline
  L'utente deve:
  \begin{itemize}
    \item visualizzare la lista di organizzazioni a cui è collegato.
  \end{itemize}
                     & NS                                                                                                                                                                                                                                                               \\
  TS011F1            & L'applicazione mobile deve permettere all'utente autenticato di visualizzare dal server una lista di organizzazioni a cui non è collegato. \newline
  L'utente deve:
  \begin{itemize}
    \item visualizzare la lista di organizzazioni a cui non è collegato.
  \end{itemize}
                     & NS                                                                                                                                                                                                                                                               \\
  TS012F1            & L'applicazione mobile deve permettere all'utente autenticato di visualizzare le informazioni di un'organizzazione in lista, che sia collegato a questa o non collegato. \newline
  L'utente deve:
  \begin{itemize}
    \item visualizzare il nome dell'organizzazione.
    \item visualizzare la descrizione dell'organizzazione.
    \item visualizzare lo stato, pubblico o privato, dell'organizzazione.
  \end{itemize}
                     & NS                                                                                                                                                                                                                                                               \\
  TS013F1            & L'applicazione mobile non recupera la lista di organizzazioni a cui può collegarsi l'utente autenticato, in quanto non c'è collegamento alla rete. \newline
  L'utente deve:
  \begin{itemize}
    \item visualizzare un messaggio d'errore.
  \end{itemize}
                     & NS                                                                                                                                                                                                                                                               \\
  TS014F1            & L'applicazione mobile deve permettere all'utente autenticato di aggiornare la lista delle organizzazioni a cui può collegarsi. \newline
  L'utente deve:
  \begin{itemize}
    \item inviare la richiesta di aggiornamento della lista di organizzazioni.
    \item visualizzare la lista aggiornata delle organizzazioni.
  \end{itemize}
                     & NS                                                                                                                                                                                                                                                               \\
  TS015F1            & L'applicazione mobile non riceve alcuna richiesta di aggiornare la lista di organizzazioni a cui può collegarsi l'utente autenticato, in quanto non c'è collegamento alla rete. \newline
  L'utente deve:
  \begin{itemize}
    \item visualizzare un messaggio d'errore.
  \end{itemize}
                     & NS                                                                                                                                                                                                                                                               \\
  TS016F1            & L'applicazione mobile deve permettere all'utente autenticato di collegarsi ad un'organizzazione pubblica. \newline
  L'utente deve:
  \begin{itemize}
    \item visualizzare la lista delle organizzazioni alla quale può collegarsi.
    \item selezionare un'organizzazione pubblica tra quelle presenti in lista.
    \item avviare la procedura di collegamento.
    \item visualizzare la pagina dell'organizzazione pubblica alla quale è ora collegato.
  \end{itemize}
                     & NS                                                                                                                                                                                                                                                               \\
  TS017F1            & L'applicazione mobile deve permettere all'utente autenticato di collegarsi ad un'organizzazione privata. \newline
  L'utente deve:
  \begin{itemize}
    \item visualizzare la lista delle organizzazioni alla quale può collegarsi.
    \item selezionare un'organizzazione privata tra quelle presenti in lista.
    \item visualizzare un form per l'inserimento dei dati di accesso al server LDAP dell'organizzazione privata selezionata.
    \item inserire lo username associato al server LDAP\@.
    \item inserire la password associata al server LDAP\@.
    \item avviare la procedura di collegamento.
    \item visualizzare la pagina dell'organizzazione privata alla quale è ora collegato.
  \end{itemize}
                     & NS                                                                                                                                                                                                                                                               \\
  TS018F1            & L'applicazione mobile non permette all'utente autenticato di collegarsi ad un'organizzazione privata, in quanto vengono inseriti erroneamente i dati d'accesso al server LDAP\@. \newline
  L'utente deve:
  \begin{itemize}
    \item visualizzare la lista delle organizzazioni alla quale può collegarsi.
    \item selezionare un'organizzazione privata tra quelle presenti in lista.
    \item visualizzare un form per l'inserimento dei dati di accesso al server LDAP dell'organizzazione privata selezionata.
    \item inserire lo username associato al server LDAP in modo errato.
    \item inserire la password associata al server LDAP in modo errato.
    \item avviare la procedura di collegamento.
    \item visualizzare un messaggio d'errore.
  \end{itemize}
                     & NS                                                                                                                                                                                                                                                               \\
  TS019F1            & L'applicazione mobile non è collegata alla rete, quindi il server non riceve alcuna richiesta di collegamento ad un'organizzazione a cui può collegarsi l'utente autenticato. \newline
  L'utente deve:
  \begin{itemize}
    \item visualizzare la lista delle organizzazione alle quale può collegarsi.
    \item selezionare un'organizzazione tra quelle presenti in lista.
    \item avviare la procedura di scollegamento.
    \item visualizzare un messaggio d'errore.
  \end{itemize}
                     & NS                                                                                                                                                                                                                                                               \\
  TS020F1            & L'applicazione mobile deve permettere all'utente autenticato di scollegarsi da un'organizzazione pubblica. \newline
  L'utente deve:
  \begin{itemize}
    \item visualizzare la pagina dell'organizzazione pubblica alla quale è ora collegato.
    \item avviare la procedura per lo scollegamento dall'organizzazione pubblica.
    \item visualizzare la lista delle organizzazioni alla quale può collegarsi.
  \end{itemize}
                     & NS                                                                                                                                                                                                                                                               \\
  TS021F1            & L'applicazione mobile deve permettere all'utente autenticato di scollegarsi ad un'organizzazione privata. \newline
  L'utente deve:
  \begin{itemize}
    \item visualizzare la pagina dell'organizzazione privata alla quale è ora collegato.
    \item avviare la procedura per lo scollegamento dall'organizzazione privata.
    \item visualizzare la lista delle organizzazioni alla quale può collegarsi.
  \end{itemize}
                     & NS                                                                                                                                                                                                                                                               \\
  TS022F1            & L'applicazione mobile non riceve alcuna richiesta di scollegamento da un'organizzazione a cui è collegato l'utente autenticato, in quanto non c'è collegamento alla rete. \newline
  L'utente deve:
  \begin{itemize}
    \item avviare la procedura di scollegamento.
    \item visualizzare un messaggio d'errore.
  \end{itemize}
                     & NS                                                                                                                                                                                                                                                               \\
  TS023F1            & L'applicazione mobile deve permettere all'utente autenticato, che è collegato ad un'organizzazione privata, di passare da noto ad incognito e viceversa. \newline
  L'utente deve:
  \begin{itemize}
    \item visualizzare la pagina dell'organizzazione privata alla quale è ora collegato.
    \item accertarsi che sia noto al sistema.
    \item avviare la procedura per il passaggio da noto ad incognito.
    \item accertarsi che sia incognito al sistema.
    \item avviare la procedura per il passaggio da incognito a noto.
    \item accertarsi che sia noto al sistema.
  \end{itemize}
                     & NS                                                                                                                                                                                                                                                               \\
  TS024F1            & L'applicazione mobile non permette all'utente autenticato, che è collegato ad un'organizzazione privata, di passare da noto ad incognito e viceversa, in quanto non c'è collegamento alla rete. \newline
  L'utente deve:
  \begin{itemize}
    \item visualizzare la pagina dell'organizzazione privata alla quale è ora collegato.
    \item accertarsi che sia noto al sistema, o al contrario incognito al sistema.
    \item avviare la procedura per il passaggio da noto ad incognito, o al contrario nel caso l'utente sia incognito al sistema.
    \item visualizzare un messaggio d'errore.
  \end{itemize}
                     & NS                                                                                                                                                                                                                                                               \\
  TS025F2            & L'applicazione mobile deve permettere all'utente autenticato di visualizzare il proprio storico degli accessi. \newline
  L'utente deve:
  \begin{itemize}
    \item visualizzare il proprio storico degli accessi.
  \end{itemize}
                     & NS                                                                                                                                                                                                                                                               \\
  TS026F2            & L'applicazione mobile non permette all'utente autenticato di visualizzare il proprio storico degli accessi, in quanto non c'è collegamento alla rete. \newline
  L'utente deve:
  \begin{itemize}
    \item visualizzare un messaggio d'errore.
  \end{itemize}
                     & NS                                                                                                                                                                                                                                                               \\
  TS027F2            & L'applicazione mobile deve permettere all'utente autenticato di visualizzare il tempo trascorso all'interno di un'organizzazione. \newline
  L'utente deve:
  \begin{itemize}
    \item visualizzare il proprio tempo trascorso all'interno di un'organizzazione.
  \end{itemize}
                     & NS                                                                                                                                                                                                                                                               \\
  TS028F2            & L'applicazione mobile non permette all'utente autenticato di visualizzare il tempo trascorso all'interno di un'organizzazione, in quanto non c'è collegamento alla rete. \newline
  L'utente deve:
  \begin{itemize}
    \item visualizzare un messaggio d'errore.
  \end{itemize}
                     & NS                                                                                                                                                                                                                                                               \\
  TS029F1            & L'applicazione mobile deve permettere all'utente autenticato di disconnettersi dall'applicazione. \newline
  L'utente deve:
  \begin{itemize}
    \item avviare la procedura di disconnessione.
    \item confermare la procedura.
  \end{itemize}
                     & NS                                                                                                                                                                                                                                                               \\
  TS030F1            & L'applicazione mobile non permette all'utente autenticato di disconnettersi dall'applicazione, in quanto non c'è collegamento alla rete. \newline
  L'utente deve:
  \begin{itemize}
    \item avviare la procedura di disconnessione.
    \item confermare la procedura.
    \item visualizzare un messaggio d'errore.
  \end{itemize}
                     & NS                                                                                                                                                                                                                                                               \\
  TS031F1            & L'applicazione mobile deve permettere all'utente autenticato di eliminare il suo account. \newline
  L'utente deve:
  \begin{itemize}
    \item avviare la procedura di eliminazione.
    \item confermare la procedura.
  \end{itemize}
                     & NS                                                                                                                                                                                                                                                               \\
  TS032F1            & L'applicazione mobile non permette all'utente autenticato di eliminare il suo account, in quanto non c'è collegamento alla rete. \newline
  L'utente deve:
  \begin{itemize}
    \item avviare la procedura di eliminazione.
    \item confermare la procedura.
    \item visualizzare un messaggio d'errore.
  \end{itemize}
                     & NS                                                                                                                                                                                                                                                               \\

  % - Web application

  TS033F1            & La web application deve permettere ad un'utente non autenticato di autenticarsi. \newline
  L'utente deve:
  \begin{itemize}
    \item compilare il campo email.
    \item compilare il campo password.
    \item confermare l'autenticazione.
    \item visualizzare la pagina principale della web application.
  \end{itemize}
                     & NS                                                                                                                                                                                                                                                               \\
  TS034F1            & La web application deve rifiutare la richiesta di autenticazione se i dati inseriti dall'utente non autenticato non rispettano i vincoli imposti, o se la combinazione di email e password non è presente nel database. \newline
  L'utente deve:
  \begin{itemize}
    \item inserire una combinazione email-password non valida.
    \item visualizzare un messaggio d'errore.
  \end{itemize}
                     & NS                                                                                                                                                                                                                                                               \\
  TS035F1            & La web application deve permettere all'utente autenticato di effettuare la disconnessione. \newline
  L'utente deve:
  \begin{itemize}
    \item avviare la procedura di disconnessione.
    \item confermare la procedura.
  \end{itemize}
                     & NS                                                                                                                                                                                                                                                               \\
  TS036F1            & La web application deve permettere all'owner di creare un'organizzazione. \newline
  L'owner deve:
  \begin{itemize}
    \item avviare la procedura di creazione di un'organizzazione.
    \item inserire il nome dell'organizzazione.
    \item inserire la descrizione dell'organizzazione.
    \item configurare i dettagli del server \glossario{LDAP} nel caso l'organizzazione che si vuole creare sia privata.
    \item confermare la procedura di creazione di un'organizzazione.
    \item verificare l'avvenuta creazione dell'organizzazione.
  \end{itemize}
                     & NS                                                                                                                                                                                                                                                               \\
  TS037F1            & La web application deve permettere all'owner di eliminare un'organizzazione in suo possesso. \newline
  L'owner deve:
  \begin{itemize}
    \item avviare la procedura di eliminazione di un'organizzazione.
    \item confermare la procedura di eliminazione.
    \item verificare che sia stata eliminata l'organizzazione.
  \end{itemize}
                     & NS                                                                                                                                                                                                                                                               \\
  TS038F1            & La web application deve permettere all'owner di modificare un'organizzazione in suo possesso. \newline
  L'owner deve:
  \begin{itemize}
    \item avviare la procedura di modifica di un'organizzazione.
    \item modificare il nome dell'organizzazione, se necessario.
    \item modificare la descrizione dell'organizzazione, se necessario.
    \item modificare i dettagli del server LDAP nel caso l'organizzazione che si vuole modificare sia privata, se necessario.
    \item confermare la procedura di modifica dell'organizzazione.
    \item verificare che le modifiche siano avvenute con successo.
  \end{itemize}
                     & NS                                                                                                                                                                                                                                                               \\
  TS039F1            & La web application deve permettere all'owner ed al gestore di aggiungere un luogo ad un'organizzazione. \newline
  L'owner ed il gestore devono:
  \begin{itemize}
    \item avviare la procedura di creazione di un luogo.
    \item inserire le coordinate geografiche specifiche al nuovo luogo.
    \item inserire l'indirizzo del nuovo luogo.
    \item confermare la procedura di creazione del luogo.
    \item verificare che sia stato aggiunto un nuovo luogo nell'organizzazione di riferimento con successo.
  \end{itemize}
                     & NS                                                                                                                                                                                                                                                               \\
  TS040F3            & La web application può permettere opzionalmente all'owner ed al gestore di aggiungere un luogo ad un'organizzazione, evidenziando l'area sulla mappa. \newline
  L'owner ed il gestore devono:
  \begin{itemize}
    \item avviare la procedura di creazione di un luogo.
    \item specificare l'area sulla mappa, che corrisponde al nuovo luogo da inserire.
    \item confermare la procedura di creazione del luogo.
    \item verificare che sia stato aggiunto un nuovo luogo nell'organizzazione di riferimento con successo.
  \end{itemize}
                     & NS                                                                                                                                                                                                                                                               \\
  TS041F1            & La web application deve permettere all'owner ed al gestore di modificare un luogo di un'organizzazione. \newline
  L'owner ed il gestore devono:
  \begin{itemize}
    \item avviare la procedura di modifica di un luogo.
    \item modificare le coordinate geografiche, se necessario.
    \item modificare l'indirizzo del luogo, se necessario.
    \item confermare la procedura di modifica del luogo.
    \item verificare che le modifiche siano avvenute con successo.
  \end{itemize}
                     & NS                                                                                                                                                                                                                                                               \\
  TS042F1            & La web application deve permettere all'owner ed al gestore di eliminare un luogo di un'organizzazione. \newline
  L'owner ed il gestore devono:
  \begin{itemize}
    \item avviare la procedura di eliminazione di un luogo.
    \item eliminare il luogo.
    \item confermare la procedura di eliminazione del luogo.
    \item verificare che l'eliminazione del luogo sia avvenuta con successo.
  \end{itemize}
                     & NS                                                                                                                                                                                                                                                               \\
  TS043F1            & La web application deve permettere all'owner, al gestore ed al visualizzatore di monitorare il numero di utenti in una specifica organizzazione. \newline
  L'owner, il gestore e il visualizzatore devono:
  \begin{itemize}
    \item avviare la procedura di monitoraggio.
    \item visualizzare il numero di utenti presenti all'interno di una specifica organizzazione.
  \end{itemize}
                     & NS                                                                                                                                                                                                                                                               \\
  TS044F1            & La web application deve permettere all'owner, al gestore ed al visualizzatore di monitorare il numero di utenti in un specifico luogo di una specifica organizzazione. \newline
  L'owner, il gestore e il visualizzatore devono:
  \begin{itemize}
    \item avviare la procedura di monitoraggio.
    \item visualizzare il numero di utenti presenti all'interno di uno specifico luogo di una specifica organizzazione.
  \end{itemize}
                     & NS                                                                                                                                                                                                                                                               \\
  TS045F2            & La web application deve permettere all'owner, al gestore ed al visualizzatore di visualizzare un report sotto forma tabellare di una specifica organizzazione. \newline
  L'owner, il gestore e il visualizzatore devono:
  \begin{itemize}
    \item avviare la procedura di visualizzazione del report.
    \item visualizzare gli accessi, le ore trascorse all'interno dei luoghi e quali luoghi sono più frequentati all'interno di un'organizzazione.
  \end{itemize}
                     & NS                                                                                                                                                                                                                                                               \\
  TS046F1            & La web application deve permettere all'owner, al gestore ed al visualizzatore di visualizzare le informazioni riguardo uno specifico utente. \newline
  L'owner, il gestore e il visualizzatore devono:
  \begin{itemize}
    \item avviare la procedura di monitoraggio dell'utente.
    \item visualizzare l'organizzazione in cui si trova l'utente.
    \item visualizzare il luogo in cui si trova l'utente.
  \end{itemize}
                     & NS                                                                                                                                                                                                                                                               \\
  TS047F1            & La web application deve permettere all'utente autenticato di diventare owner e ottenere la possibilità di creare una sua organizzazione. \newline
  L'utente deve:
  \begin{itemize}
    \item avviare la procedura per diventare owner.
    \item confermare la richiesta di diventare owner.
  \end{itemize}
                     & NS                                                                                                                                                                                                                                                               \\
  TS048F2            & La web application deve permettere all'owner di aggiungere un nuovo gestore dell'organizzazione. \newline
  L'owner deve:
  \begin{itemize}
    \item selezionare un'organizzazione di cui è owner.
    \item avviare la procedura di aggiunta di un gestore.
    \item inserire l'email dell'utente che deve diventare gestore.
    \item confermare l'aggiunta del gestore.
    \item verificare che l'utente indicato sia diventato gestore.
  \end{itemize}
                     & NS                                                                                                                                                                                                                                                               \\
  TS049F2            & La web application deve permettere all'owner di aggiungere un nuovo visualizzatore dell'organizzazione. \newline
  L'owner deve:
  \begin{itemize}
    \item selezionare un'organizzazione di cui è owner.
    \item avviare la procedura di aggiunta di un visualizzatore.
    \item inserire email dell'utente che deve diventare visualizzatore.
    \item confermare l'aggiunta del visualizzatore.
    \item verificare che l'utente indicato sia diventato visualizzatore.
  \end{itemize}
                     & NS                                                                                                                                                                                                                                                               \\
  TS050F1            & La web application deve permettere all'amministratore di rimuovere i privilegi di owner ad un utente che ne sia in possesso. \newline
  L'amministratore deve:
  \begin{itemize}
    \item selezionare un'organizzazione.
    \item avviare la procedura di rimozione dei privilegi di owner.
    \item inserire l'email dell'owner da rimuovere.
    \item confermare la rimozione dell'owner.
    \item verificare che l'utente indicato non sia più un owner.
  \end{itemize}
                     & NS                                                                                                                                                                                                                                                               \\
  TS051F2            & La web application deve permettere all'owner di rimuovere i privilegi di gestore ad un utente che ne sia in possesso. \newline
  L'owner deve:
  \begin{itemize}
    \item selezionare un'organizzazione.
    \item avviare la procedura di rimozione dei privilegi di gestore.
    \item inserire l'email del gestore da rimuovere.
    \item confermare la rimozione del gestore.
    \item verificare che l'utente indicato non sia più un gestore.
  \end{itemize}
                     & NS                                                                                                                                                                                                                                                               \\
  TS052F2            & La web application deve permettere all'owner di rimuovere i privilegi di visualizzatore ad un utente che ne sia in possesso. \newline
  L'owner deve:
  \begin{itemize}
    \item selezionare un'organizzazione.
    \item avviare la procedura di rimozione dei privilegi di visualizzatore.
    \item inserire l'email del visualizzatore da rimuovere.
    \item confermare la rimozione del visualizzatore.
    \item verificare che l'utente indicato non sia più un visualizzatore.
  \end{itemize}
                     & NS                                                                                                                                                                                                                                                               \\
  TS053F1            & La web application deve permettere all'amministratore di eliminare un account dal sistema. \newline
  L'amministratore deve:
  \begin{itemize}
    \item avviare la procedura di eliminazione dell'account utente.
    \item selezionare l'account da rimuovere.
    \item confermare la rimozione dell'account.
    \item verificare che l'utente eliminato non riesca più ad accedere al sistema.
  \end{itemize}
                     & NS                                                                                                                                                                                                                                                               \\

  % - Vincoli di sistema

  TS001P1            & Il sistema deve garantire un tracciamento con una precisione sufficiente a certificare la presenza della persona all’interno degli edifici.
                     & NS                                                                                                                                                                                                                                                               \\

  TS001V1            & Il sistema deve garantire che le comunicazioni tra applicazione e server avvengano solo al momento d’ingresso ed uscita dai luoghi designati.
                     & NS                                                                                                                                                                                                                                                               \\

  TS002V2            & Il proponente richiede che vengano utilizzati protocolli asincroni per le comunicazioni tra app e server.
                     & NS                                                                                                                                                                                                                                                               \\

  TS003V2            & Il proponente richiede l’utilizzo del design pattern Publisher/Subscriber.
                     & NS                                                                                                                                                                                                                                                               \\

  TS004V2            & Il proponente richiede che venga utilizzato un IAAS come Kubernetes o un PAAS come Openshift o Rancher per il rilascio delle componenti del server e per la gestione della scalabilità orizzontale.
                     & NS                                                                                                                                                                                                                                                               \\

  TS005V1            & Il server espone delle API REST attraverso cui utilizzare l’applicativo, o in alternativa alle API REST, utilizza il framework RPC\@.
                     & NS                                                                                                                                                                                                                                                               \\

  TS006V1            & L’applicazione utilizza tecnologie GPS\@.
                     & NS                                                                                                                                                                                                                                                               \\

  TS007V1            & L’applicazione è in grado di bilanciare il consumo della batteria e la necessità di aggiornare la posizione in background.
                     & NS                                                                                                                                                                                                                                                               \\

  TS001Q1            & Tutte le componenti applicative sono correlate dai test unitari e d’integrazione.
                     & NS                                                                                                                                                                                                                                                               \\

  TS002Q1            & Viene testato interamente il sistema tramite test end-to-end.
                     & NS                                                                                                                                                                                                                                                               \\

  TS003Q1            & Vengono effettuati test di carico che dimostrino il corretto funzionamento in ogni situazione: normale, di carico e di sovraccarico.
                     & NS                                                                                                                                                                                                                                                               \\

  TS004Q1            & Viene fornita una copertura dei test almeno dell’80\%, correlata di report.
                     & NS                                                                                                                                                                                                                                                               \\

  TS005Q2            & Tutte le comunicazioni tra app e server sono cifrate.
                     & NS                                                                                                                                                                                                                                                               \\

  TS006Q2            & Viene fornita un’analisi rispetto al carico massimo in numero di utenti e di quale sarebbe il servizio cloud più adatto per supportarlo, analizzando prezzo, stabilità del servizio ed assistenza.
                     & NS                                                                                                                                                                                                                                                               \\

  TS007Q1            & Viene garantita la privacy degli utenti e rispettare la normativa GDPR\@.
                     & NS                                                                                                                                                                                                                                                               \\
  \rowcolor{white}
  \caption{Tabella dei test di sistema}%
  \label{tab:test_sistema}
\end{longtable}

\subsubsection{Test di Integrazione}%
\label{subs:test_di_integrazione}

I test di integrazione verificano la corretta interazione e collaborazione tra un insieme di unità. I test verranno identificati nel seguente modo:
\begin{center}
  TI[codice]
\end{center}
Dove \textit{codice} rappresenta un numero progressivo che identifica il test di integrazione.

\rowcolors{2}{lightgray}{white!80!lightgray!100}
\renewcommand{\arraystretch}{2}
\begin{longtable}[H]{>{\centering\bfseries}m{3cm} >{}m{10cm} >{\centering\arraybackslash}m{3cm}}
  \rowcolor{darkgray!90!}
  \color{white}
  {\textbf{ID test}} & \color{white}{\textbf{Descrizione}}                                    & \color{white}{\textbf{Esito}} \\
  \endhead\rowcolor{white}%
  \multicolumn{3}{r}{\textit{Continua alla pagina seguente}}
  \endfoot%
  \endlastfoot%

  TI1                & Il sistema deve garantire l'integrazione tra API e web application.
                     & NS                                                                                                     \\

  TI2                & Il sistema deve garantire l'integrazione tra API e mobile application.
                     & NS                                                                                                     \\
  \rowcolor{white}
  \caption{Tabella dei test di integrazione}%
  \label{tab:test_integrazione}
\end{longtable}

%sub:test_di_integrazione (end)

\subsubsection{Test di Unità}%
\label{subs:test_di_unita}

I test di unità verificano la correttezza e il funzionamento di una singola unità del software. I test verranno identificati nel seguente modo:
\begin{center}
  TU[codice]
\end{center}
Dove \textit{codice} rappresenta un numero identificativo per l'unità a cui il test appartiene.
Il gruppo in questa fase non implementa ancora i test di unità.
%sub:test_di_unita (end)
\end{document}
