\documentclass[../piano-di-qualifica.tex]{subfiles}
\appendToGraphicspath{../../../commons/img/}

\begin{document}

Per misurare la qualità di prodotto inoltre utilizziamo dei test secondo il \glossario{Modello a \textit{V}}, per il quale definiamo e sviluppiamo i test in parallelo alle attività di analisi, progettazione architetturale e di sviluppo e verifica incrementi.
Abbiamo definito i test dividendoli nelle seguenti categorie:
\begin{description}
  \item [Test di Accettazione (TA)].
  \item [Test di Sistema (TS)].
  \item [Test di Integrazione (TI)].
  \item [Test di Unità (TU)].
\end{description}

Inoltre classifichiamo i test in base al loro stato:
\begin{description}
  \item [S]: il test è stato soddisfatto.
  \item [NS]: il test non è stato ancora soddisfatto.
\end{description}

\subsubsection{Test di Accettazione}%
\label{subs:test_di_accettazione}

I test di accettazione vengono attuati dal proponente e dal committente, in sede di collaudo finale, e servono a dimostrare che il prodotto soddisfi tutti i requisiti.
Sono i primi test che vengono creati e saranno gli ultimi ad essere eseguiti prima della consegna del prodotto.

I test di accettazione sono indicati nel modo seguente:
\begin{center}
  [TA][numero][tipo]
\end{center}
\textit{numero} è un numero identificativo incrementale e \textit{tipo} può assumere questi valori:
\begin{itemize}
  \item [WA]: Web Application.
  \item [MA]: Mobile Application.
\end{itemize}

\rowcolors{2}{lightgray}{white!80!lightgray!100}
\renewcommand{\arraystretch}{2} % allarga le righe con dello spazio sotto e sopra
\begin{longtable}[H]{>{\centering\bfseries}m{3cm} >{}m{10cm} >{\centering\arraybackslash}m{3cm}}
  \rowcolor{darkgray!90!}
  \color{white}
  {\textbf{ID test}} & \color{white}{\textbf{Descrizione}}                                                                                                                                                                                              & \color{white}{\textbf{Esito}} \\
  \endhead\rowcolor{white}%
  \multicolumn{3}{r}{\textit{Continua alla pagina seguente}}
  \endfoot%
  \endlastfoot%

  % - Applicazione mobile

  TA001MA          & L'applicazione mobile deve permettere ad un nuovo utente di registrarsi. \newline
  L'utente deve:
  \begin{itemize}
    \item visualizzare e confermare l'EULA\@.
    \item inserire la propria email.
    \item inserire password.
    \item confermare la password.
    \item inserire il nome della persona che si registra.
    \item inserire il cognome della persona che si registra.
    \item inserire la data di nascita.
    \item confermare la registrazione.
  \end{itemize}
                    & NS                                                                                                                                                                                                                                                               \\
  TA002MA           & Il sistema deve rifiutare la richiesta di registrazione se i dati inseriti non rispettano i vincoli imposti. \newline
  L'utente deve:
  \begin{itemize}
    \item inserire dati per la registrazione non validi.
    \item verificare l'impossibilità di proseguire con la registrazione.
  \end{itemize}
                    & NS                                                                                                                                                                                                                                                               \\
  TA003MA           & L'applicazione mobile deve permettere all'utente registrato di autenticarsi. \newline
  L'utente deve:
  \begin{itemize}
    \item inserire la propria email.
    \item inserire la propria password.
    \item confermare l'autenticazione.
  \end{itemize}
                    & NS                                                                                                                                                                                                                                                               \\
  TA004MA           & Il sistema deve rifiutare la richiesta di autenticazione se la combinazione di email e password non è presente nel database. \newline
  L'utente deve:
  \begin{itemize}
    \item inserire dati di accesso non validi.
    \item verificare l'impossibilità di proseguire con l'autenticazione.
  \end{itemize}
                    & NS                                                                                                                                                                                                                                                               \\
  TA005MA           & L'applicazione mobile deve permettere all'utente di recuperare la sua password. \newline
  L'utente deve:
  \begin{itemize}
    \item avviare la procedura di recupero password.
    \item inserire la propria email.
    \item avviare la procedura di reimpostazione password tramite il link ricevuto via mail.
    \item inserire una nuova password.
    \item confermare la nuova password.
    \item confermare la procedura.
    \item verificare che la password sia stata cambiata correttamente.
  \end{itemize}
                    & NS                                                                                                                                                                                                                                                               \\
  TA006MA           & Il sistema deve rifiutare la richiesta di reimpostazione password se la mail inserita all'inizio della procedura non è presente all'interno del database del sistema. \newline
  L'utente deve:
  \begin{itemize}
    \item avviare la procedura di recupero password.
    \item inserire una mail non presente nel database del sistema.
    \item visualizzare un messaggio di errore.
  \end{itemize}
                    & NS                                                                                                                                                                                                                                                               \\
  TA007MA           & Il sistema deve rifiutare la richiesta di reimpostazione password se la nuova password inserita non rispetta i vincoli imposti. \newline
  L'utente deve:
  \begin{itemize}
    \item avviare la procedura di recupero password.
    \item inserire la propria email.
    \item avviare la procedura di reimpostazione password tramite il link ricevuto via mail.
    \item inserire una nuova password non valida.
    \item confermare la nuova password.
    \item confermare la procedura.
    \item visualizzare un messaggio di errore.
  \end{itemize}
                    & NS                                                                                                                                                                                                                                                               \\
  TA008MA           & Il sistema deve rifiutare la richiesta di reimpostazione password se la nuova password e la conferma della password non sono valide. \newline
  L'utente deve:
  \begin{itemize}
    \item avviare la procedura di recupero password.
    \item inserire la propria email.
    \item avviare la procedura di reimpostazione password tramite il link ricevuto via mail.
    \item inserire una nuova password.
    \item confermare la nuova password in modo errato.
    \item confermare la procedura.
    \item visualizzare un messaggio di errore.
  \end{itemize}
                    & NS                                                                                                                                                                                                                                                               \\
  TA009MA           & L'applicazione mobile deve permettere all'utente autenticato di recuperare dal server una lista di organizzazioni a cui può collegarsi. \newline
  L'utente deve:
  \begin{itemize}
    \item visualizzare la lista di organizzazioni a cui può collegarsi.
  \end{itemize}
                    & NS                                                                                                                                                                                                                                                               \\
  TA010MA           & L'applicazione mobile deve permettere all'utente autenticato di visualizzare dal server una lista di organizzazioni a cui è collegato. \newline
  L'utente deve:
  \begin{itemize}
    \item visualizzare la lista di organizzazioni a cui è collegato.
  \end{itemize}
                    & NS                                                                                                                                                                                                                                                               \\
  TA011MA           & L'applicazione mobile deve permettere all'utente autenticato di visualizzare dal server una lista di organizzazioni a cui non è collegato. \newline
  L'utente deve:
  \begin{itemize}
    \item visualizzare la lista di organizzazioni a cui non è collegato.
  \end{itemize}
                    & NS                                                                                                                                                                                                                                                               \\
  TA012MA           & L'applicazione mobile deve permettere all'utente autenticato di visualizzare le informazioni di un'organizzazione in lista, che sia collegato a questa o non collegato. \newline
  L'utente deve:
  \begin{itemize}
    \item visualizzare il nome dell'organizzazione.
    \item visualizzare la descrizione dell'organizzazione.
    \item visualizzare lo stato, pubblico o privato, dell'organizzazione.
  \end{itemize}
                    & NS                                                                                                                                                                                                                                                               \\
  TA013MA           & L'applicazione mobile non recupera la lista di organizzazioni a cui può collegarsi l'utente autenticato, in quanto non c'è collegamento alla rete. \newline
  L'utente deve:
  \begin{itemize}
    \item visualizzare un messaggio d'errore.
  \end{itemize}
                    & NS                                                                                                                                                                                                                                                               \\
  TA014MA           & L'applicazione mobile deve permettere all'utente autenticato di aggiornare la lista delle organizzazioni a cui può collegarsi. \newline
  L'utente deve:
  \begin{itemize}
    \item inviare la richiesta di aggiornamento della lista di organizzazioni.
    \item visualizzare la lista aggiornata delle organizzazioni.
  \end{itemize}
                    & NS                                                                                                                                                                                                                                                               \\
  TA015MA           & L'applicazione mobile non riceve alcuna richiesta di aggiornare la lista di organizzazioni a cui può collegarsi l'utente autenticato, in quanto non c'è collegamento alla rete. \newline
  L'utente deve:
  \begin{itemize}
    \item visualizzare un messaggio d'errore.
  \end{itemize}
                    & NS                                                                                                                                                                                                                                                               \\
  TA016MA           & L'applicazione mobile deve permettere all'utente autenticato di collegarsi ad un'organizzazione pubblica. \newline
  L'utente deve:
  \begin{itemize}
    \item visualizzare la lista delle organizzazioni alla quale può collegarsi.
    \item selezionare un'organizzazione pubblica tra quelle presenti in lista.
    \item avviare la procedura di collegamento.
    \item visualizzare la pagina dell'organizzazione pubblica alla quale è ora collegato.
  \end{itemize}
                    & NS                                                                                                                                                                                                                                                               \\
  TA017MA           & L'applicazione mobile deve permettere all'utente autenticato di collegarsi ad un'organizzazione privata. \newline
  L'utente deve:
  \begin{itemize}
    \item visualizzare la lista delle organizzazioni alla quale può collegarsi.
    \item selezionare un'organizzazione privata tra quelle presenti in lista.
    \item visualizzare un form per l'inserimento dei dati di accesso al server LDAP dell'organizzazione privata selezionata.
    \item inserire lo username associato al server LDAP\@.
    \item inserire la password associata al server LDAP\@.
    \item avviare la procedura di collegamento.
    \item visualizzare la pagina dell'organizzazione privata alla quale è ora collegato.
  \end{itemize}
                    & NS                                                                                                                                                                                                                                                               \\
  TA018MA           & L'applicazione mobile non permette all'utente autenticato di collegarsi ad un'organizzazione privata, in quanto vengono inseriti erroneamente i dati d'accesso al server LDAP\@. \newline
  L'utente deve:
  \begin{itemize}
    \item visualizzare la lista delle organizzazioni alla quale può collegarsi.
    \item selezionare un'organizzazione privata tra quelle presenti in lista.
    \item visualizzare un form per l'inserimento dei dati di accesso al server LDAP dell'organizzazione privata selezionata.
    \item inserire lo username associato al server LDAP in modo errato.
    \item inserire la password associata al server LDAP in modo errato.
    \item avviare la procedura di collegamento.
    \item visualizzare un messaggio d'errore.
  \end{itemize}
                    & NS                                                                                                                                                                                                                                                               \\
  TA019MA           & L'applicazione mobile non è collegata alla rete, quindi il server non riceve alcuna richiesta di collegamento ad un'organizzazione a cui può collegarsi l'utente autenticato. \newline
  L'utente deve:
  \begin{itemize}
    \item visualizzare la lista delle organizzazione alle quale può collegarsi.
    \item selezionare un'organizzazione tra quelle presenti in lista.
    \item avviare la procedura di scollegamento.
    \item visualizzare un messaggio d'errore.
  \end{itemize}
                    & NS                                                                                                                                                                                                                                                               \\
  TA020MA           & L'applicazione mobile deve permettere all'utente autenticato di scollegarsi da un'organizzazione pubblica. \newline
  L'utente deve:
  \begin{itemize}
    \item visualizzare la pagina dell'organizzazione pubblica alla quale è ora collegato.
    \item avviare la procedura per lo scollegamento dall'organizzazione pubblica.
    \item visualizzare la lista delle organizzazioni alla quale può collegarsi.
  \end{itemize}
                    & NS                                                                                                                                                                                                                                                               \\
  TA021MA           & L'applicazione mobile deve permettere all'utente autenticato di scollegarsi ad un'organizzazione privata. \newline
  L'utente deve:
  \begin{itemize}
    \item visualizzare la pagina dell'organizzazione privata alla quale è ora collegato.
    \item avviare la procedura per lo scollegamento dall'organizzazione privata.
    \item visualizzare la lista delle organizzazioni alla quale può collegarsi.
  \end{itemize}
                    & NS                                                                                                                                                                                                                                                               \\
  TA022MA           & L'applicazione mobile non riceve alcuna richiesta di scollegamento da un'organizzazione a cui è collegato l'utente autenticato, in quanto non c'è collegamento alla rete. \newline
  L'utente deve:
  \begin{itemize}
    \item avviare la procedura di scollegamento.
    \item visualizzare un messaggio d'errore.
  \end{itemize}
                    & NS                                                                                                                                                                                                                                                               \\
  TA023MA           & L'applicazione mobile deve permettere all'utente autenticato, che è collegato ad un'organizzazione privata, di passare da noto ad incognito e viceversa. \newline
  L'utente deve:
  \begin{itemize}
    \item visualizzare la pagina dell'organizzazione privata alla quale è ora collegato.
    \item accertarsi che sia noto al sistema.
    \item avviare la procedura per il passaggio da noto ad incognito.
    \item accertarsi che sia incognito al sistema.
    \item avviare la procedura per il passaggio da incognito a noto.
    \item accertarsi che sia noto al sistema.
  \end{itemize}
                    & NS                                                                                                                                                                                                                                                               \\
  TA024MA           & L'applicazione mobile non permette all'utente autenticato, che è collegato ad un'organizzazione privata, di passare da noto ad incognito e viceversa, in quanto non c'è collegamento alla rete. \newline
  L'utente deve:
  \begin{itemize}
    \item visualizzare la pagina dell'organizzazione privata alla quale è ora collegato.
    \item accertarsi che sia noto al sistema, o al contrario incognito al sistema.
    \item avviare la procedura per il passaggio da noto ad incognito, o al contrario nel caso l'utente sia incognito al sistema.
    \item visualizzare un messaggio d'errore.
  \end{itemize}
                    & NS                                                                                                                                                                                                                                                               \\
  TA025MA           & L'applicazione mobile deve permettere all'utente autenticato di visualizzare il proprio storico degli accessi. \newline
  L'utente deve:
  \begin{itemize}
    \item visualizzare il proprio storico degli accessi.
  \end{itemize}
                    & NS                                                                                                                                                                                                                                                               \\
  TA026MA           & L'applicazione mobile non permette all'utente autenticato di visualizzare il proprio storico degli accessi, in quanto non c'è collegamento alla rete. \newline
  L'utente deve:
  \begin{itemize}
    \item visualizzare un messaggio d'errore.
  \end{itemize}
                    & NS                                                                                                                                                                                                                                                               \\
  TA027MA           & L'applicazione mobile deve permettere all'utente autenticato di visualizzare il tempo trascorso all'interno di un'organizzazione. \newline
  L'utente deve:
  \begin{itemize}
    \item visualizzare il proprio tempo trascorso all'interno di un'organizzazione.
  \end{itemize}
                    & NS                                                                                                                                                                                                                                                               \\
  TA028MA           & L'applicazione mobile non permette all'utente autenticato di visualizzare il tempo trascorso all'interno di un'organizzazione, in quanto non c'è collegamento alla rete. \newline
  L'utente deve:
  \begin{itemize}
    \item visualizzare un messaggio d'errore.
  \end{itemize}
                    & NS                                                                                                                                                                                                                                                               \\
  TA029MA           & L'applicazione mobile deve permettere all'utente autenticato di disconnettersi dall'applicazione. \newline
  L'utente deve:
  \begin{itemize}
    \item avviare la procedura di disconnessione.
    \item confermare la procedura.
  \end{itemize}
                    & NS                                                                                                                                                                                                                                                               \\
  TA030MA           & L'applicazione mobile non permette all'utente autenticato di disconnettersi dall'applicazione, in quanto non c'è collegamento alla rete. \newline
  L'utente deve:
  \begin{itemize}
    \item avviare la procedura di disconnessione.
    \item confermare la procedura.
    \item visualizzare un messaggio d'errore.
  \end{itemize}
                    & NS                                                                                                                                                                                                                                                               \\
  TA031MA           & L'applicazione mobile deve permettere all'utente autenticato di eliminare il suo account. \newline
  L'utente deve:
  \begin{itemize}
    \item avviare la procedura di eliminazione.
    \item confermare la procedura.
  \end{itemize}
                    & NS                                                                                                                                                                                                                                                               \\
  TA032MA           & L'applicazione mobile non permette all'utente autenticato di eliminare il suo account, in quanto non c'è collegamento alla rete. \newline
  L'utente deve:
  \begin{itemize}
    \item avviare la procedura di eliminazione.
    \item confermare la procedura.
    \item visualizzare un messaggio d'errore.
  \end{itemize}
                    & NS                                                                                                                                                                                                                                                               \\

  % - Web application

  TA033WA           & La web application deve permettere ad un'utente non autenticato di autenticarsi. \newline
  L'utente deve:
  \begin{itemize}
    \item compilare il campo email.
    \item compilare il campo password.
    \item confermare l'autenticazione.
    \item visualizzare la pagina principale della web application.
  \end{itemize}
                    & NS                                                                                                                                                                                                                                                               \\
  TA034WA           & La web application deve rifiutare la richiesta di autenticazione se i dati inseriti dall'utente non autenticato non rispettano i vincoli imposti, o se la combinazione di email e password non è presente nel database. \newline
  L'utente deve:
  \begin{itemize}
    \item inserire una combinazione email-password non valida.
    \item visualizzare un messaggio d'errore.
  \end{itemize}
                    & NS                                                                                                                                                                                                                                                               \\
  TA035WA           & La web application deve permettere all'utente autenticato di effettuare la disconnessione. \newline
  L'utente deve:
  \begin{itemize}
    \item avviare la procedura di disconnessione.
    \item confermare la procedura.
  \end{itemize}
                    & NS                                                                                                                                                                                                                                                               \\
  TA036WA           & La web application deve permettere all'owner di creare un'organizzazione. \newline
  L'owner deve:
  \begin{itemize}
    \item avviare la procedura di creazione di un'organizzazione.
    \item inserire il nome dell'organizzazione.
    \item inserire la descrizione dell'organizzazione.
    \item configurare i dettagli del server \glossario{LDAP} nel caso l'organizzazione che si vuole creare sia privata.
    \item confermare la procedura di creazione di un'organizzazione.
    \item verificare l'avvenuta creazione dell'organizzazione.
  \end{itemize}
                    & NS                                                                                                                                                                                                                                                               \\
  TA037WA           & La web application deve permettere all'owner di eliminare un'organizzazione in suo possesso. \newline
  L'owner deve:
  \begin{itemize}
    \item avviare la procedura di eliminazione di un'organizzazione.
    \item confermare la procedura di eliminazione.
    \item verificare che sia stata eliminata l'organizzazione.
  \end{itemize}
                    & NS                                                                                                                                                                                                                                                               \\
  TA038WA           & La web application deve permettere all'owner di modificare un'organizzazione in suo possesso. \newline
  L'owner deve:
  \begin{itemize}
    \item avviare la procedura di modifica di un'organizzazione.
    \item modificare il nome dell'organizzazione, se necessario.
    \item modificare la descrizione dell'organizzazione, se necessario.
    \item modificare i dettagli del server LDAP nel caso l'organizzazione che si vuole modificare sia privata, se necessario.
    \item confermare la procedura di modifica dell'organizzazione.
    \item verificare che le modifiche siano avvenute con successo.
  \end{itemize}
                    & NS                                                                                                                                                                                                                                                               \\
  TA039WA           & La web application deve permettere all'owner ed al gestore di aggiungere un luogo ad un'organizzazione. \newline
  L'owner ed il gestore devono:
  \begin{itemize}
    \item avviare la procedura di creazione di un luogo.
    \item inserire le coordinate geografiche specifiche al nuovo luogo.
    \item inserire l'indirizzo del nuovo luogo.
    \item confermare la procedura di creazione del luogo.
    \item verificare che sia stato aggiunto un nuovo luogo nell'organizzazione di riferimento con successo.
  \end{itemize}
                    & NS                                                                                                                                                                                                                                                               \\
  TA040WA           & La web application può permettere opzionalmente all'owner ed al gestore di aggiungere un luogo ad un'organizzazione, evidenziando l'area sulla mappa. \newline
  L'owner ed il gestore devono:
  \begin{itemize}
    \item avviare la procedura di creazione di un luogo.
    \item specificare l'area sulla mappa, che corrisponde al nuovo luogo da inserire.
    \item confermare la procedura di creazione del luogo.
    \item verificare che sia stato aggiunto un nuovo luogo nell'organizzazione di riferimento con successo.
  \end{itemize}
                    & NS                                                                                                                                                                                                                                                               \\
  TA041WA           & La web application deve permettere all'owner ed al gestore di modificare un luogo di un'organizzazione. \newline
  L'owner ed il gestore devono:
  \begin{itemize}
    \item avviare la procedura di modifica di un luogo.
    \item modificare le coordinate geografiche, se necessario.
    \item modificare l'indirizzo del luogo, se necessario.
    \item confermare la procedura di modifica del luogo.
    \item verificare che le modifiche siano avvenute con successo.
  \end{itemize}
                    & NS                                                                                                                                                                                                                                                               \\
  TA042WA           & La web application deve permettere all'owner ed al gestore di eliminare un luogo di un'organizzazione. \newline
  L'owner ed il gestore devono:
  \begin{itemize}
    \item avviare la procedura di eliminazione di un luogo.
    \item eliminare il luogo.
    \item confermare la procedura di eliminazione del luogo.
    \item verificare che l'eliminazione del luogo sia avvenuta con successo.
  \end{itemize}
                    & NS                                                                                                                                                                                                                                                               \\
  TA043WA           & La web application deve permettere all'owner, al gestore ed al visualizzatore di monitorare il numero di utenti in una specifica organizzazione. \newline
  L'owner, il gestore e il visualizzatore devono:
  \begin{itemize}
    \item avviare la procedura di monitoraggio.
    \item visualizzare il numero di utenti presenti all'interno di una specifica organizzazione.
  \end{itemize}
                    & NS                                                                                                                                                                                                                                                               \\
  TA044WA           & La web application deve permettere all'owner, al gestore ed al visualizzatore di monitorare il numero di utenti in un specifico luogo di una specifica organizzazione. \newline
  L'owner, il gestore e il visualizzatore devono:
  \begin{itemize}
    \item avviare la procedura di monitoraggio.
    \item visualizzare il numero di utenti presenti all'interno di uno specifico luogo di una specifica organizzazione.
  \end{itemize}
                    & NS                                                                                                                                                                                                                                                               \\
  TA045WA           & La web application deve permettere all'owner, al gestore ed al visualizzatore di visualizzare un report sotto forma tabellare di una specifica organizzazione. \newline
  L'owner, il gestore e il visualizzatore devono:
  \begin{itemize}
    \item avviare la procedura di visualizzazione del report.
    \item visualizzare gli accessi, le ore trascorse all'interno dei luoghi e quali luoghi sono più frequentati all'interno di un'organizzazione.
  \end{itemize}
                    & NS                                                                                                                                                                                                                                                               \\
  TA046WA           & La web application deve permettere all'owner, al gestore ed al visualizzatore di visualizzare le informazioni riguardo uno specifico utente. \newline
  L'owner, il gestore e il visualizzatore devono:
  \begin{itemize}
    \item avviare la procedura di monitoraggio dell'utente.
    \item visualizzare l'organizzazione in cui si trova l'utente.
    \item visualizzare il luogo in cui si trova l'utente.
  \end{itemize}
                    & NS                                                                                                                                                                                                                                                               \\
  TA047WA           & La web application deve permettere all'utente autenticato di diventare owner e ottenere la possibilità di creare una sua organizzazione. \newline
  L'utente deve:
  \begin{itemize}
    \item avviare la procedura per diventare owner.
    \item confermare la richiesta di diventare owner.
  \end{itemize}
                    & NS                                                                                                                                                                                                                                                               \\
  TA048WA           & La web application deve permettere all'owner di aggiungere un nuovo gestore dell'organizzazione. \newline
  L'owner deve:
  \begin{itemize}
    \item selezionare un'organizzazione di cui è owner.
    \item avviare la procedura di aggiunta di un gestore.
    \item inserire l'email dell'utente che deve diventare gestore.
    \item confermare l'aggiunta del gestore.
    \item verificare che l'utente indicato sia diventato gestore.
  \end{itemize}
                    & NS                                                                                                                                                                                                                                                               \\
  TA049WA           & La web application deve permettere all'owner di aggiungere un nuovo visualizzatore dell'organizzazione. \newline
  L'owner deve:
  \begin{itemize}
    \item selezionare un'organizzazione di cui è owner.
    \item avviare la procedura di aggiunta di un visualizzatore.
    \item inserire email dell'utente che deve diventare visualizzatore.
    \item confermare l'aggiunta del visualizzatore.
    \item verificare che l'utente indicato sia diventato visualizzatore.
  \end{itemize}
                    & NS                                                                                                                                                                                                                                                               \\
  TA050WA           & La web application deve permettere all'amministratore di rimuovere i privilegi di owner ad un utente che ne sia in possesso. \newline
  L'amministratore deve:
  \begin{itemize}
    \item selezionare un'organizzazione.
    \item avviare la procedura di rimozione dei privilegi di owner.
    \item inserire l'email dell'owner da rimuovere.
    \item confermare la rimozione dell'owner.
    \item verificare che l'utente indicato non sia più un owner.
  \end{itemize}
                    & NS                                                                                                                                                                                                                                                               \\
  TA051WA           & La web application deve permettere all'owner di rimuovere i privilegi di gestore ad un utente che ne sia in possesso. \newline
  L'owner deve:
  \begin{itemize}
    \item selezionare un'organizzazione.
    \item avviare la procedura di rimozione dei privilegi di gestore.
    \item inserire l'email del gestore da rimuovere.
    \item confermare la rimozione del gestore.
    \item verificare che l'utente indicato non sia più un gestore.
  \end{itemize}
                    & NS                                                                                                                                                                                                                                                               \\
  TA052WA           & La web application deve permettere all'owner di rimuovere i privilegi di visualizzatore ad un utente che ne sia in possesso. \newline
  L'owner deve:
  \begin{itemize}
    \item selezionare un'organizzazione.
    \item avviare la procedura di rimozione dei privilegi di visualizzatore.
    \item inserire l'email del visualizzatore da rimuovere.
    \item confermare la rimozione del visualizzatore.
    \item verificare che l'utente indicato non sia più un visualizzatore.
  \end{itemize}
                    & NS                                                                                                                                                                                                                                                               \\
  TA053WA           & La web application deve permettere all'amministratore di eliminare un account dal sistema. \newline
  L'amministratore deve:
  \begin{itemize}
    \item avviare la procedura di eliminazione dell'account utente.
    \item selezionare l'account da rimuovere.
    \item confermare la rimozione dell'account.
    \item verificare che l'utente eliminato non riesca più ad accedere al sistema.
  \end{itemize}
                    & NS                                                                                                                                                                                                                                                               \\

    \rowcolor{white}
    \caption{Tabella dei test di accettazione}%
    \label{tab:test_accettazione}
  \end{longtable}


\subsubsection{Test di Sistema}%
\label{subs:test_di_sistema}

I test di sistema hanno lo scopo di verificare che il prodotto soddisfi i requisiti richiesti, vengono eseguiti durante la fase di verifica e collaudo finale.
Essendo tutti i test derivanti da un gruppo di requisiti che hanno una determinata importanza nel prodotto software, abbiamo pensato di numerare i test di sistema nel modo seguente:
\begin{center}
  [TS][numero][tipo][priorità]
\end{center}

Dove: \textit{priorità} è un valore numerico da 1 a 3 ad indicare l'importanza del requisito che deve essere soddisfatto, \textit{numero} è un numero identificativo per tipologia che parte da 1 e \textit{tipo} può assumere i seguenti valori:
\begin{description}
  \item [F]: requisito funzionale.
  \item [P]: requisito prestazionale.
  \item [V]: requisito vincolo.
  \item [Q]: requisito qualità.
\end{description}

\rowcolors{2}{lightgray}{white!80!lightgray!100}
\renewcommand{\arraystretch}{2} % allarga le righe con dello spazio sotto e sopra
\begin{longtable}[H]{>{\centering\bfseries}m{3cm} >{}m{10cm} >{\centering\arraybackslash}m{3cm}}
  \rowcolor{darkgray!90!}
  \color{white}
  {\textbf{ID test}} & \color{white}{\textbf{Descrizione}}                                                                                                                                                                                              & \color{white}{\textbf{Esito}} \\
  \endhead\rowcolor{white}%
  \multicolumn{3}{r}{\textit{Continua alla pagina seguente}}
  \endfoot%
  \endlastfoot%

  % da associare uno ad uno con i requisiti dell'AdR
  % Le frasi iniziano con "Verifica che"

  TS001F1            & Si verifichi che un nuovo utente si possa registrare a Stalker. 
  & S               \\

  TS002F1            & Si verifichi che un nuovo utente visualizzi il EULA al momento della registrazione. 
  & NS               \\
  
  TS003F1            & Si verifichi che un nuovo utente possa confermare il EULA al momento della registrazione. 
  & NS               \\

  TS004F1            & Si verifichi che un utente possa effettuare l'accesso all'app Android. 
  & S               \\

  TS005F1            & Si verifichi che un utente possa effettuare il recupero password. 
  & S               \\

  TS006F1            & Si verifichi che un utente possa visualizzare la lista di organizzazioni a cui può collegarsi. 
  & S               \\

  TS007F1            & Si verifichi che un utente possa visualizzare la lista di organizzazioni a cui è collegato. 
  & S               \\

  TS008F1            & Si verifichi che un utente possa visualizzare le informazioni riguardati un'organizzazione in lista alla quale può collegarsi.
  & NS               \\

  TS009F1            & Si verifichi che un utente possa visualizzare le informazioni riguardati un'organizzazione in lista nella quale è collegato. 
  & NS               \\


  TS010F1            & Si verifichi che un utente possa aggiornare la lista delle organizzazioni. 
  & NS               \\

  TS011F1            & Si verifichi che un utente collegarsi ad un'organizzazione. 
  & NS               \\

  TS012F1            & Si verifichi che un utente possa scollegarsi da un'organizzazione. 
  & NS               \\

  TS013F1            & Si verifichi che un utente possa passare da noto ad incognito in un'organizzazione. 
  & NS               \\

  TS014F1            & Si verifichi che un utente possa passare da incognito a noto in un'organizzazione. 
  & NS               \\

  TS015F2            & Si verifichi che un utente possa visualizzare un suo storico degli accessi in un'organizzazione. 
  & NS               \\

  TS016F2            & Si verifichi che un utente possa visualizzare il tempo trascorso all'interno di un'organizzazione. 
  & NS               \\

  TS017F1            & Si verifichi che un utente autenticato possa disconnettersi dall'applicazione di Stalker. 
  & NS               \\

  TS018F1            & Si verifichi che un utente possa eliminare il suo account. 
  & NS               \\

  TS019F1            & Si verifichi che un utente visualizza un messaggio di errore nel caso mancasse una connessione ad internet. 
  & NS               \\

  TS020F1            & Si verifichi che un utente possa autenticarsi alla web application. 
  & S               \\

  TS021F1            & Si verifichi che un utente possa effettuare la disconnessione dalla web application. 
  & S               \\
  
  TS022F1            & Si verifichi che un owner possa creare un'organizzazione. 
  & NS               \\

  TS023F1            & Si verifichi che un owner possa eliminare una sua organizzazione. 
  & NS               \\

  TS024F1            & Si verifichi che un owner possa modificare una sua organizzazione. 
  & S               \\
  
  TS025F1            & Si verifichi che un gestore possa aggiungere un luogo ad un'organizzazione. 
  & S               \\

  TS026F1            & Si verifichi che un gestore possa modificare un luogo ad un'organizzazione. 
  & S               \\

  TS027F1            & Si verifichi che un gestore possa eliminare un luogo ad un'organizzazione. 
  & S               \\

  TS028F1            & Si verifichi che un visualizzatore possa visualizzare le informazioni di un'organizzazione. 
  & NS               \\

  TS029F1            & Si verifichi che un visualizzatore possa monitorare il numero dei dipendenti di tutta l'organizzazione.
  & NS               \\

  TS030F1            & Si verifichi che un visualizzatore possa monitorare il numero dei dipendenti in uno specifico luogo.
  & NS               \\

  TS031F2            & Si verifichi che un visualizzatore possa visualizzare un report degli accessi, delle ore trascorse nei vari luoghi, e dei luoghi più frequentati di un'organizzazione.
  & NS               \\

  TS032F1            & Si verifichi che un visualizzatore possa sapere se un dipendente si trova all’interno di un’organizzazione in cui opera.
  & NS               \\

  TS033F1            & Si verifichi che un visualizzatore possa sapere in che luogo si trova un dipendente all'interno di un'organizzazione.
  & NS               \\

  TS034F1            & Si verifichi che un utente abbia la possibilità di diventare owner, creando una sua organizzazione.
  & NS               \\

  TS035F1            & Si verifichi che un owner abbia la possibilità di aggiungere un nuovo gestore all'organizzazione.
  & NS               \\

  TS036F1            & Si verifichi che un owner abbia la possibilità di aggiungere un nuovo visualizzatore all'organizzazione.
  & NS               \\

  TS037F1            & Si verifichi che un amministratore possa togliere i privilegi di owner ad un utente.
  & NS              \\

  TS038F1            & Si verifichi che un owner possa togliere i privilegi di gestore ad un utente.
  & NS               \\

  TS039F1            & Si verifichi che un owner possa togliere i privilegi di visualizzatore ad un utente.
  & NS               \\

  TS040F1            & Si verifichi che un amministratore possa eliminare un account qualsiasi.
  & NS               \\

  % - Requisiti prestazionali

  TS041P1            & Si verifichi che il tracciamento garantisca precisione all’interno degli edifici, in modo da certificarne la presenza della persona.
                     & S               \\

  TS042V1            & Si verifichi che sia stato realizzato un server backend.
                     & S               \\

  TS043V1            & Si verifichi che la web app abbia un interfaccia grafica con cui gli utenti possano interagire.
                     & S               \\

  TS044V1            & Si verifichi che sia stata realizzata un applicazione Android per le versioni 5.0 e superiori.
                     & S               \\

  TS045V1            & Si verifichi che le comunicazioni tra applicazione e server avvengano solo al momento d'ingresso ed uscita dai luoghi designati.
                     & NS                                                                                                                                                                                                                                                             \\

  TS046V2            & Si verifichi che sia stato utilizzato il linguaggio Java (versione 8 o superiore), Python, o Node.js per lo sviluppo del server backend.
                     & S               \\

  TS047V2            & Si verifichi che vengano utilizzati protocolli asincroni per le comunicazioni tra app e server.
                     & S                                                                                                                                                                                                                                                               \\

  TS048V2            & Si verifichi l'utilizzo del design pattern Publisher/Subscriber.
                     & S                                                                                                                                                                                                                                                               \\

  TS049V2            & Si verifichi che venga utilizzato un IAAS come Kubernetes o un PAAS come Openshift o Rancher per il rilascio delle componenti del server e per la gestione della scalabilità orizzontale.
                     & NS           
                                                                                                                                                                                                                                                                  \\
  TS050V1            & Si verifichi che il server esponga API REST attraverso cui utilizzare l'applicativo, o in alternativa alle API REST, utilizza il framework RPC\@.
                     & S\\

  TS051V2            & Si verifichi che tutte le comunicazioni tra app e server siano cifrate.
                     & NS\\

  TS052V1            & Si verifichi che l'applicazione utilizzi tecnologie GPS\@.
                     & S \\

  TS053V1            & Si verifichi che l'applicazione sia in grado di bilanciare il consumo della batteria e la necessità di aggiornare la posizione in background.
                     & NS                                                                                                                                                                                                                                                               \\

  TS054V1            & Si verifichi che venga fornito un resoconto di tutte le scelte fatte e dei test effettuati per garantire il miglior rapporto raggiunto tra consumo della batteria e utilizzo del GPS, comprensivi di report.
                     & NS \\

  TS055Q1            & Si verifichi che tutte le componenti applicative siano correlate dai test unitari e d'integrazione.
                     & S                                                                                                                                                                                                                                                               \\

  TS056Q1            & Si verifichi che venga testato interamente il sistema tramite test end-to-end.
                     & S                                                                                                                                                                                                                                                               \\

  TS057Q1            & Si verifichi che vengano effettuati test di carico che dimostrino il corretto funzionamento in ogni situazione: normale, di carico e di sovraccarico.
                     & NS                                                                                                                                                                                                                                                               \\

  TS058Q1            & Si verifichi che venga fornita una copertura dei test almeno dell'80\%, correlata di report.
                     & S                                                                                                                                                                                                                                                               \\                                                                                                                                                                                                                                                       


  TS059Q1            & Si verifichi che sia stata fornita documentazione relativa alle scelte implementative e progettuali effettuate, correlate dalle relative motivazioni
                     & S                                                                                                                                                                                                                                                               \\                                                                                                                                                                                                                                                       

  TS060Q1            & Si verifichi che sia stata fornita documentazione relativa a problemi aperti ed eventuali soluzioni proposte da esplorare.
                     & S                                                                                                                                                                                                                                                               \\                                                                                                                                                                                                                                                       

  TS061Q2            & Si verifichi che venga fornita un'analisi rispetto al carico massimo in numero di utenti e di quale sarebbe il servizio cloud più adatto per supportarlo, analizzando prezzo, stabilità del servizio ed assistenza.
                     & NS                                                                                                                                                                                                                                                               \\

  TS062Q1            & Si verifichi che venga garantita la privacy degli utenti e rispettare la normativa GDPR\@.
                     & NS                                                                                                                                                                                                                                                               \\
  \rowcolor{white}
  \caption{Tabella dei test di sistema}%
  \label{tab:test_sistema}
\end{longtable}



% Tracciamento

\rowcolors{2}{lightgray}{white!80!lightgray!100}
\renewcommand{\arraystretch}{2} % allarga le righe con dello spazio sotto e sopra
\begin{longtable}[H]{>{\centering\bfseries}m{5cm} >{\centering\arraybackslash}m{5cm}}
  \rowcolor{darkgray!90!}
  \color{white}
  {\textbf{ID test}} & \color{white}{\textbf{Requisito}} \\
  \endhead\rowcolor{white}%

  \endfoot%
  \endlastfoot%
  TS001F1 & R001F1 \\
  TS002F1 & R002F1 \\
  TS003F1 & R003F1 \\
  TS004F1 & R017F1 \\
  TS005F1 & R021F1 \\
  TS006F1 & R029F1 \\
  TS007F1 & R030F1 \\
  TS008F1 & R031F1 \\
  TS009F1 & R032F1 \\
  TS010F1 & R040F1 \\
  TS011F1 & R041F1 \\
  TS012F1 & R047F1 \\
  TS013F1 & R048F1 \\
  TS014F1 & R049F1 \\
  TS015F2 & R050F2 \\
  TS016F2 & R051F2 \\
  TS017F1 & R052F1 \\
  TS018F1 & R053F1 \\
  TS019F1 & R055F1 \\
  TS020F1 & R056F1 \\
  TS021F1 & R060F1 \\
  TS022F1 & R061F1 \\
  TS023F1 & R066F1 \\
  TS024F1 & R067F1 \\
  TS025F1 & R071F1 \\
  TS026F1 & R075F1 \\
  TS027F1 & R078F1 \\
  TS028F1 & R080F1 \\
  TS029F1 & R081F1 \\
  TS030F1 & R082F1 \\
  TS031F2 & R083F2 \\
  TS032F1 & R085F1 \\
  TS033F1 & R086F1 \\
  TS034F1 & R087F1 \\
  TS035F1 & R089F2 \\
  TS036F1 & R091F2 \\
  TS037F1 & R093F1 \\
  TS038F1 & R095F2 \\
  TS039F1 & R097F2 \\
  TS040F1 & R099F1 \\
  TS041P1 & R101P1 \\
  TS042V1 & R102V1 \\
  TS043V1 & R103V1 \\
  TS044V1 & R104V1 \\
  TS045V1 & R105V1 \\
  TS046V2 & R106V2 \\
  TS047V2 & R107V2 \\
  TS048V2 & R108V2 \\
  TS049V2 & R109V2 \\
  TS050V1 & R110V1 \\
  TS051V2 & R111V2 \\
  TS052V1 & R112V1 \\
  TS053V1 & R113V1 \\
  TS054V1 & R114V1 \\
  TS055Q1 & R115Q1 \\
  TS056Q1 & R116Q1 \\
  TS057Q1 & R117Q1 \\
  TS058Q1 & R118Q1 \\
  TS059Q1 & R119Q1 \\
  TS060Q1 & R120Q1 \\
  TS061Q2 & R121Q2 \\
  TS062Q1 & R122Q1 \\


 




  \rowcolor{white}
  \caption{Tracciamento test di sistema \- requisito}%
  \label{tab:test_sistema_requisito}
\end{longtable}



\newpage
\subsubsection{Test di Integrazione}%
\label{subs:test_di_integrazione}

I test di integrazione verificano la corretta interazione e collaborazione tra un insieme di unità. I test verranno identificati nel seguente modo:
\begin{center}
  TI[codice]
\end{center}
Dove \textit{codice} rappresenta un numero progressivo che identifica il test di integrazione.

\rowcolors{2}{lightgray}{white!80!lightgray!100}
\renewcommand{\arraystretch}{2}
\begin{longtable}[H]{>{\centering\bfseries}m{3cm} >{}m{10cm} >{\centering\arraybackslash}m{3cm}}
  \rowcolor{darkgray!90!}
  \color{white}
  {\textbf{ID test}} & \color{white}{\textbf{Descrizione}}                                    & \color{white}{\textbf{Esito}} \\
  \endhead\rowcolor{white}%
  \multicolumn{3}{r}{\textit{Continua alla pagina seguente}}
  \endfoot%
  \endlastfoot%

  TI1                & Il sistema deve garantire l'integrazione tra API e web application.
                     & NS                                                                                                     \\

  TI2                & Il sistema deve garantire l'integrazione tra API e mobile application.
                     & NS                                                                                                     \\
  \rowcolor{white}
  \caption{Tabella dei test di integrazione}%
  \label{tab:test_integrazione}
\end{longtable}

%sub:test_di_integrazione (end)
\newpage
\subsubsection{Test di Unità}%
\label{subs:test_di_unita}

I test di unità verificano la correttezza e il funzionamento di una singola unità del software. I test verranno identificati nel seguente modo:
\begin{center}
  TU[tipo][codice]
\end{center}
Dove \textit{codice} rappresenta un numero identificativo per l'unità a cui il test appartiene e \textit{tipo} rappresenta il modulo di appartenenza con: 
\begin{description}
  \item [M]: Mobile app;
  \item [W]: Web app;
  \item [S]: Server; 
\end{description}




\rowcolors{2}{lightgray}{white!80!lightgray!100}
\renewcommand{\arraystretch}{2}
\begin{longtable}[H]{>{\centering\bfseries}m{3cm} >{}m{10cm} >{\centering\arraybackslash}m{3cm}}
  \rowcolor{darkgray!90!}
  \color{white}
  {\textbf{ID test}} & \color{white}{\textbf{Metodo}}                                    & \color{white}{\textbf{Esito}} \\
  \endhead\rowcolor{white}%
  \multicolumn{3}{r}{\textit{Continua alla pagina seguente}}
  \endfoot%
  \endlastfoot%

% Android App

% Test di unità 

  TUM001             & Verifica il ritorno di una lista vuota di organizzazioni. & S\\

  TUM002             & Verifica il ritorno di una lista non vuota di organizzazioni. & S\\

  TUM003             & Verifica il ritorno di una lista nulla di organizzazioni. & S\\

  TUM004             & Verifica il ritorno non nullo e non vuoto dato un punto all'interno del luogo. & S\\

  TUM005             & Verifica il ritorno non nullo e vuoto dato un punto all'interno del luogo, in assenza di organizzazioni. & S\\

  TUM006             & Verifica il ritorno non nullo e vuoto dato un punto all'esterno del luogo. & S\\

  TUM007             & Verifica il ritorno di un'istanza non nulla di CurrentSessionSingleton. & S\\

  TUM008             & Verifica l'uguaglianza dell'istanza ritornata di CurrentSessionSingleton. & S\\

  TUM009             & Verifica il ritorno a true dato un punto all'interno di un luogo di un'organizzazione. & S\\

  TUM010             & Verifica il ritorno a false dato un punto al margine di un luogo di un'organizzazione. & S\\

  TUM011             & Verifica il ritorno a false dato un punto esterno di un luogo di un'organizzazione. & S\\

  TUM012             & Verifica il ritorno a true dato un punto all'interno di un luogo. & S\\

  TUM013             & Verifica il ritorno a false dato un punto al margine di un luogo. & S\\

  TUM014             & Verifica il ritorno a false dato un punto esterno di un luogo. & S\\

  TUM015             & Verifica il ritorno di un'istanza non nulla di WebSingleton. & S\\

  TUM016             & Verifica l'uguaglianza dell'istanza ritornata di WebSingleton. & S\\

  TUM017             & Verifica l'aggiunta di una richiesta alla coda. & N\\ % TODO test ancora da completare

  TUM018             & Verifica l'entrata di un utente in un luogo. & N\\ % TODO test ancora da completare

  TUM019             & Verifica l'uscita di un utente da un luogo. & N\\ % TODO test ancora da completare

  TUM020             & Verifica il ritorno di una lista di organizzazioni. & N\\ % TODO test ancora da completare


  TUM021             & Verifica il ritorno di un valore true data email e password. & N\\ % TODO test ancora da completare

  TUM022             & Verifica il ritorno di un valore data email e password. & N\\ % TODO test ancora da completare

  TUM023             & Verifica il ritorno di una lista di organizzazioni per l'aggiornamento. & N\\ % TODO test ancora da completare

  TUM024             & Verifica il ritorno delle informazioni di un'organizzazione. & N\\ % TODO test ancora da completare

  TUM025             & Verifica la creazione di un componente organizzazione della lista. & N\\ % TODO test ancora da completare

  TUM026             & Verifica la corretta selezione di un componente organizzazione della lista. & N\\ % TODO test ancora da completare

  TUM027             & Verifica il ritorno del numero dei componenti organizzazione della lista. & N\\ % TODO test ancora da completare

  TUM028             & Verifica i permessi richiesti dall'applicazione. & N\\ % TODO test ancora da completare


  % Test di instrumentation

  TUM029             & Verifica la corretta creazione dell'activity App. & N\\ % TODO test ancora da completare

  TUM030             & Verifica il ritorno del context. & N\\ % TODO test ancora da completare

  TUM031             & Verifica la corretta assegnazione di una lista di organizzazioni. & N\\ % TODO test ancora da completare

 % TODO LOCATION NOTIFIER - mancano dei metodi da TUM032 a TUM035

  TUM036             & Verifica la corretta creazione dell'activity Login. & N\\ % TODO test ancora da completare

  TUM037             & Verifica la corretta creazione dell'activity Main. & N\\ % TODO test ancora da completare

  TUM038             & Verifica la corretta creazione dell'activity MainPage. & N\\ % TODO test ancora da completare

  TUM039             & Verifica la corretta creazione dell'activity Stalker. & N\\ % TODO test ancora da completare

  TUM040             & Verifica il ritorno non nullo di una coda. & N\\ % TODO test ancora da completare

  TUM041             & Verifica l'uguaglianza della coda del WebSingleton. & N\\ % TODO test ancora da completare

  TUM042             & Verifica l'aggiunta di una richiesta alla coda. & N\\ % TODO test ancora da completare


  \rowcolor{white}
  \caption{Tabella dei test di unità}%
  \label{tab:test_di_unità}
\end{longtable}








\rowcolors{2}{lightgray}{white!80!lightgray!100}
\renewcommand{\arraystretch}{2}
\begin{longtable}[H]{>{\centering\bfseries}m{3cm} >{}m{13cm}}
  \rowcolor{darkgray!90!}
  \color{white}
  {\textbf{ID test}}                    & \color{white}{\textbf{Componente}} \\
  \endhead\rowcolor{white}%

  \endfoot%
  \endlastfoot%

  % Android App

  % test di unità

  TUM001                & com.gruppone.stalker.CurrentSessionSingleton.zeroOrganizations\@() \\

  TUM002                & com.gruppone.stalker.CurrentSessionSingleton.zeroOrganizations\@() \\
  
  TUM003                & com.gruppone.stalker.CurrentSessionSingleton.zeroOrganizations\@() \\

  TUM004                & com.gruppone.stalker.CurrentSessionSingleton.getInsideId\@() \\

  TUM005                & com.gruppone.stalker.CurrentSessionSingleton.getInsideId\@() \\

  TUM006                & com.gruppone.stalker.CurrentSessionSingleton.getInsideId\@() \\

  TUM007                & com.gruppone.stalker.CurrentSessionSingleton.getInstance\@() \\

  TUM008                & com.gruppone.stalker.CurrentSessionSingleton.getInstance\@() \\

  TUM009                & com.gruppone.stalker.Organization.isInside\@() \\

  TUM010                & com.gruppone.stalker.Organization.isInside\@() \\

  TUM011                & com.gruppone.stalker.Organization.isInside\@() \\

  TUM012                & com.gruppone.stalker.Place.isInside\@() \\

  TUM013                & com.gruppone.stalker.Place.isInside\@() \\

  TUM014                & com.gruppone.stalker.Place.isInside\@() \\

  TUM015                & com.gruppone.stalker.WebSingleton.getInstance\@() \\

  TUM016                & com.gruppone.stalker.WebSingleton.getInstance\@() \\

  TUM017                & com.gruppone.stalker.WebSingleton.addToRequestQueue\@() \\

  TUM018                & com.gruppone.stalker.WebSingleton.locationUpdateInside\@() \\

  TUM019                & com.gruppone.stalker.WebSingleton.locationUpdateOutside\@() \\

  TUM020                & com.gruppone.stalker.WebSingleton.getOrganizationList\@() \\


  TUM021                & com.gruppone.stalker.LoginModel.login\@() \\

  TUM022                & com.gruppone.stalker.LoginViewModel.login\@() \\

  TUM023                & com.gruppone.stalker.MainPageModel.loadOrganizations\@() \\

  TUM024                & com.gruppone.stalker.MainPageViewModel.getOrgsLiveData\@() \\

  TUM025                & com.gruppone.stalker.OrganizationListAdapter.onCreateViewHolder\@() \\

  TUM026                & com.gruppone.stalker.OrganizationListAdapter.onBindViewHolder\@() \\

  TUM027                & com.gruppone.stalker.OrganizationListAdapter.getItemCount\@() \\

  TUM028                & com.gruppone.stalker.StalkerActivity.checkPermissions\@() \\

  % test di instrumentation

  TUM029                & com.gruppone.stalker.App.onCreate\@() \\

  TUM030                & com.gruppone.stalker.App.getAppContext\@() \\

  TUM031                & com.gruppone.stalker.CurrentSessionSingleton.setOrganizations\@() \\

 % TODO LOCATION NOTIFIER - mancano dei metodi da TUM032 a TUM035

  TUM036                & com.gruppone.stalker.LoginActivity.onCreate\@() \\

  TUM037                & com.gruppone.stalker.MainActivity.onCreate\@() \\

  TUM038                & com.gruppone.stalker.MainPageActivity.onCreate\@() \\

  TUM039                & com.gruppone.stalker.StalkerActivity.onResume\@() \\

  TUM040                & com.gruppone.stalker.WebSingleton.getRequestQueue\@() \\

  TUM041                & com.gruppone.stalker.WebSingleton.getRequestQueue\@() \\

  TUM042                & com.gruppone.stalker.WebSingleton.addToRequestQueue\@() \\



  \rowcolor{white}
  \caption{Tabella del tracciamento dei test di unità}%
  \label{tab:test_di_unità}
\end{longtable}








%sub:test_di_unita (end)
\end{document}
