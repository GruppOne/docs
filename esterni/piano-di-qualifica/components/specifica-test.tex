\documentclass[../piano-di-qualifica.tex]{subfiles}
\appendToGraphicspath{../../../commons/img/}

\begin{document}

\label{sub:test}
    Per misurare la qualità di prodotto inoltre utilizziamo dei test secondo il \glossario{Modello a \textit{V}} per il quale definiamo e sviluppiamo i test in parallelo alle attività di analisi, progettazione architetturale e di sviluppo e verifica incrementi.
    Abbiamo definito i test dividendoli nelle seguenti categorie:
    \begin{description}
      \item [Test di Accettazione (TA)]
      \item [Test di Sistema (TS)]
      \item [Test di Integrazione (TI)]
      \item [Test di Unità (TU)]
    \end{description}
    Inoltre classifichiamo i test in base al loro stato:
      \begin{description}
        \item [S]: il test è stato soddisfatto
        \item [NS]: il test non è stato ancora soddisfatto
      \end{description}
  \subsection{Test di Accettazione}%
  \label{subs:accettazione}
      I test di accettazione hanno lo scopo di verificare che il prodotto soddisfi i requisiti richiesti dal proponente, vengono eseguiti durante la fase di verifica e collaudo finale.
      Essendo tutti i test derivanti da un gruppo di requisiti che hanno una determinata importanza nel prodotto software abbiamo pensato di numerare i Test di Accettazione nel modo seguente:
      \begin{center}
          [TA][numero][tipo][priorità]
      \end{center}
      Dove: \textit{priorità} è un valore numerico ad indicare l'importanza del requisito che deve essere soddisfatto, con un numero da 1 a 3 \textit{numero} è un numero identificativo per il test e \textit{tipo} può assumere i seguenti valori:
      \begin{description}
        \item [F]: requisito funzionale
        \item [P]: requisito prestazionale
        \item [V]: requisito vincolo
        \item [Q]: requisito qualità
      \end{description}

      \newpage
      \begin{centering}
      \rowcolors{2}{lightgray}{white!80!lightgray!100}
      \renewcommand{\arraystretch}{2} % allarga le righe con dello spazio sotto e sopra
      \begin{longtable}[H]{>{\centering\bfseries}m{3cm} >{}p{10cm} >{\centering\arraybackslash}m{3cm}}
        \rowcolor{darkgray!90!}
        \color{white}
        {\textbf{ID test}} & \color{white}{\textbf{Descrizione}} & \color{white}{\textbf{Esito}} \\
        \endhead\rowcolor{white}%
        \multicolumn{3}{r}{\textit{Continua alla pagina seguente}}
        \endfoot{}%
        \endlastfoot{}

% - App

        TA001F1      & Al nuovo utente deve essere permesso di registrarsi. \newline
                        L'utente deve:
                        \begin{itemize}
                          \item visualizzare e confermare l'\glossario{EULA};
                          \item inserire password;
                          \item confermare la password;
                          \item inserire i propri dati anagrafici;
                          \item inserire la propria email;
                          \item confermare la registrazione.
                        \end{itemize}
                      & NS \\
        TA002F1      & Il sistema deve rifiutare la richiesta di registrazione se i dati inseriti non rispettano i vincoli imposti o se l'email è già presente nel database. \newline
                        L'utente deve:
                        \begin{itemize}
                          \item inserire dati di accesso non validi;
                          \item verificare l'impossibilità di proseguire con la registrazione.
                        \end{itemize}
                      & NS \\
        TA003F1      & Il sistema deve permettere all'utente registrato di autenticarsi. \newline
                        L'utente deve:
                        \begin{itemize}
                          \item inserire la propria email;
                          \item inserire la propria password;
                          \item confermare l'autenticazione.
                        \end{itemize}
                      & NS \\
        TA004F1      & Il sistema deve rifiutare la richiesta di autenticazione se i dati inseriti non rispettano i vincoli imposti, o se la combinazione di email e password non è presente nel database. \newline
                        L'utente deve:
                        \begin{itemize}
                          \item inserire dati di accesso non validi;
                          \item verificare l'impossibilità di proseguire con l'autenticazione.
                        \end{itemize}
                      & NS \\
        TA005F1      & Il sistema deve permettere all'utente di recuperare la sua password. \newline
                        L'utente deve:
                        \begin{itemize}
                          \item avviare la procedura di recupero password;
                          \item inserire la propria email;
                          \item il sistema deve inviare una mail contenente un link per la reimpostazione della password.
                        \end{itemize}
                      & NS \\
        TA006F1      & Il sistema deve permettere all'utente di reimpostare la sua password. \newline
                        L'utente deve:
                        \begin{itemize}
                          \item avviare la procedura di reimpostazione password tramite il link ricevuto via mail;
                          \item inserire una nuova password;
                          \item confermare la nuova password;
                          \item confermare la procedura;
                          \item verificare che la password sia stata cambiata correttamente;
                        \end{itemize}
                      & NS \\
          TA007F1    & Il sistema deve rifiutare la richiesta di reimpostazione password se la password scelta non rispetta i vincoli imposti. \newline
                        L'utente deve:
                        \begin{itemize}
                          \item avviare la procedura di reimpostazione password tramite il link ricevuto via mail;
                          \item inserire una nuova password non valida;
                          \item confermare la nuova password;
                          \item confermare la procedura;
                          \item visualizzare un messaggio di errore;
                        \end{itemize}
                      & NS \\
        TA008F1      & Il sistema deve permettere all'utente autenticato di recuperare dal server una lista di organizzazioni a cui può collegarsi e di aggiornarla. \newline
                        L'utente deve:
                        \begin{itemize}
                          \item visualizzare la lista di organizzazioni in cui può collegarsi;
                          \item aggiornare la lista;
                          \item visualizzare la lista aggiornata.
                        \end{itemize}
                      & NS \\
        TA009F1      & Il sistema deve permettere all'utente autenticato di recuperare dal server una lista di organizzazioni su cui è collegato e di aggiornarla. \newline
                      L'utente deve:
                      \begin{itemize}
                        \item visualizzare la lista di organizzazioni su cui è collegato;
                        \item aggiornare la lista;
                        \item visualizzare la lista aggiornata.
                      \end{itemize}
                    & NS \\

       TA010F1      & Il sistema deve permettere all'utente autenticato di visualizzare le informazioni di un'organizzazione in lista. \newline
                    L'utente deve:
                    \begin{itemize}
                      \item visualizzare nome dell'organizzazione;
                      \item visualizzare descrizione dell'organizzazione;
                      \item visualizzare stato dell'organizzazione (pubblica o privata).
                    \end{itemize}
                  & NS \\
       TA011F1      & Il sistema deve permettere all'utente autenticato di selezionare un'organizzazione dalla lista e collegarsi. \newline
                        L'utente deve:
                        \begin{itemize}
                          \item visualizzare la lista delle organizzazione alle quali può collegarsi;
                          \item selezionare un'organizzazione il lista;
                          \item avviare la procedura di collegamento.
                        \end{itemize}
                      & NS \\
       TA012F1      & Il sistema deve permettere all'utente autenticato e collegato ad una o più organizzazioni di scollegarsi da un'organizzazione. \newline
                        L'utente deve:
                        \begin{itemize}
                          \item visualizzare la lista delle organizzazione alle quali è collegato;
                          \item selezionare un'organizzazione il lista;
                          \item avviare la procedura di scollegamento.
                        \end{itemize}
                      & NS \\
        TAAFO012      & Il sistema deve permettere all'utente autenticato il passaggio da noto ad incognito e viceversa. \newline
                      L'utente deve:
                      \begin{itemize}
                        \item avviare la procedura di cambio di stato;
                        \item verificare l'effettivo cambio di stato.
                      \end{itemize}
                      & NS \\
        TAAFO013      & Il sistema deve permettere all'utente autenticato di visualizzare il proprio storico. \newline
                      L'utente deve:
                      \begin{itemize}
                        \item visualizzare il proprio storico degli accessi;
                        \item visualizzare il proprio tempo trascorso all'interno di ogni organizzazione.
                      \end{itemize}
                      & NS \\
        TAAFO014      & Il sistema deve permettere all'utente autenticato di disconnettersi dall'applicazione. \newline
                      L'utente deve:
                      \begin{itemize}
                        \item avviare la procedura di disconnessione;
                        \item confermare la procedura.
                      \end{itemize}
                      & NS \\
        TAAFO015      & Il sistema deve permettere all'utente di eliminare il suo account. \newline
                      L'utente deve:
                      \begin{itemize}
                        \item avviare la procedura di eliminazione;
                        \item confermare la procedura.
                      \end{itemize}
                      & NS \\
        TAAFO016      & Il sistema deve mostrare un messaggio d'errore e rifiutare le richieste da parte dell'app nel caso mancasse una connessione a internet. \newline
                      L'utente deve:
                      \begin{itemize}
                        \item disconnettere il dispositivo da internet;
                        \item effettuare un'azione che richieda la connessione ad internet;
                        \item visualizzare il messaggio di errore corrispondente.
                      \end{itemize}
                      & NS \\

% - WebApp

        TAFO017      & L'interfaccia web deve permettere all'utente non autenticato di autenticarsi. \newline
                      L'utente deve:
                      \begin{itemize}
                        \item compilare il campo email;
                        \item compilare il campo password;
                        \item confermare l'autenticazione.
                      \end{itemize}
                      & NS \\
        TAAFO018      & L'interfaccia web deve rifiutare la richiesta di autenticazione se i dati inseriti non rispettano i vincoli imposti, o se la combinazione di email e password non è presente nel database. \newline
                      L'utente deve:
                      \begin{itemize}
                        \item inserire una combinazione di dati non validi;
                        \item verificare la possibilità di proseguire l'autenticazione.
                      \end{itemize}
                      & NS \\
        TAWFO019      & L'interfaccia web deve permettere all'utente autenticato di effettuare la disconnessione. \newline
                      L'utente deve:
                      \begin{itemize}
                      \item avviare la procedura di disconnessione.
                      \end{itemize}
                      & NS \\
        TAWFO020      & L'interfaccia web deve permettere all'owner di creare un'organizzazione. \newline
                      L'owner deve:
                      \begin{itemize}
                      \item avviare la procedura di creazione di una organizzazione;
                      \item inserire il nome dell'organizzazione;
                      \item inserire la descrizione dell'organizzazione;
                      \item specificare i dettagli del server \glossario{LDAP}
                      \item verificare l'avvenuta creazione dell'organizzazione
                      \end{itemize}
                      & NS \\
        TAWFO021      & L'interfaccia web deve permettere all'owner di eliminare un'organizzazione in suo possesso. \newline
                      L'owner deve:
                      \begin{itemize}
                      \item avviare la procedura di eliminazione di una organizzazione;
                      \item confermare la procedura di eliminazione
                      \end{itemize}
                      & NS \\
        TAWFO022      & L'interfaccia web deve permettere all'owner di modificare l'organizzazione. \newline
                      L'owner deve:
                      \begin{itemize}
                      \item avviare la procedura di modifica di una organizzazione;
                      \item modificare i dati dell'organizzazione (nome, descrizione, configurazione del server LDAP).
                      \end{itemize}
                      & NS \\
        TAAFO024      & Il sistema deve inviare una richiesta di aggiornamento della lista delle organizzazioni a tutte le applicazioni mobile.         \newline
                      L'utente deve:
                      \begin{itemize}
                      \item visualizzare una notifica con la richiesta di aggiornamento qualora una delle organizzazioni venga modificata.
                      \end{itemize}
                      & NS \\
        TAWFO025      & L'interfaccia web deve permettere all'owner e al gestore di gestire i luoghi dell'organizzazione.         \newline
                      L'utente deve:
                      \begin{itemize}
                      \item avviare la procedura di inserimento di un nuovo luogo;
                      \item inserire le coordinate geografiche;
                      \item inserire l'indirizzo.
                      \end{itemize}
                      & NS \\
        TAWFF026      & L'interfaccia web può opzionalmente permettere all'owner e al
                      gestore di aggiungere un luogo all'organizzazione evidenziandone
                      l'area sulla mappa. \newline
                      L'owner e il gestore devono:
                      \begin{itemize}
                      \item avviare la procedura di inserimento di un nuovo luogo;
                      \item selezionare l'area sulla mappa.
                      \end{itemize}
                      & NS \\
        TAWFO027      & L'interfaccia web deve permettere all'owner e al gestore di eliminare luoghi dall'organizzazione. \newline
                      L'owner e il gestore devono:
                      \begin{itemize}
                      \item avviare la procedura di eliminazione di un nuovo luogo;
                      \item confermare l'eliminazione.
                      \end{itemize}
                      & NS \\
        TAWFO028      & L'interfaccia web deve permettere all'owner e al gestore di modificare i luoghi dell'organizzazione. \newline
                      L'owner e il gestore devono:
                      \begin{itemize}
                      \item avviare la procedura di modifica di un nuovo luogo;
                      \item modificare i dati del luogo (indirizzo e coordinate o area sulla mappa).
                      \end{itemize}
                      & NS \\
        TAWFO029      & L'interfaccia web deve permettere all'owner, gestore e visualizzatore di visualizzare una lista dei luoghi dell'organizzazione. \newline
                      L'owner, gestore e visualizzatore devono:
                      \begin{itemize}
                      \item selezionare la lista dei luoghi dell'organizzazione;
                      \item visualizzare la lista dei luoghi.
                      \end{itemize}
                      & NS \\
        TAWFO30      & L'interfaccia web deve permettere all'owner, gestore e visualizzatore di un'organizzazione di visualizzare le informazioni sull'organizzazione. \newline
                    L'owner, gestore e visualizzatore devono:
                    \begin{itemize}
                    \item selezionare un'organizzazione dalla lista;
                    \item visualizzare le informazioni dell'organizzazione.
                    \end{itemize}
                    & NS \\
        TAWFO31      & L'interfaccia web deve permettere all'owner, gestore e visualizzatore di un'organizzazione di visualizzare gli accessi di un dipendente. \newline
                  Gli amministratori dell'organizzazione devono:
                  \begin{itemize}
                  \item selezionare il dipendente dall'elenco della propria organizzazione;
                  \item visualizzare i dati di accesso del dipendente selezionato.
                  \end{itemize}
                  & NS \\
        TAWFO32      & L'interfaccia web deve permettere all'utente autenticato di diventare owner e ottenere la possibilità di creare una sua organizzazione. \newline
                  L'utente deve:
                  \begin{itemize}
                  \item avviare la procedura per diventare owner;
                  \item confermare la richiesta di diventare owner.
                  \end{itemize}
                  & NS \\
        TAWFO33      & Il sistema deve permettere all'owner di aggiungere un nuovo gestore dell'organizzazione.    \newline
                L'owner deve:
                \begin{itemize}
                \item selezionare un'organizzazione di cui è owner;
                \item avviare la procedura di aggiunta di un gestore;
                \item inserire email dell'utente da promuovere.
                \end{itemize}
                & NS \\
        TAWFO34      & Il sistema deve permettere all'owner di aggiungere un nuovo visualizzatore dell'organizzazione.    \newline
                L'owner deve:
                \begin{itemize}
                \item selezionare un'organizzazione di cui è owner;
                \item avviare la procedura di aggiunta di un visualizzatore;
                \item inserire email dell'utente da promuovere.
                \end{itemize}
                & NS \\
        TAWFO34      & Il sistema deve permettere all'amministratore di togliere i privilegi di owner ad un utente che ne sia in possesso.    \newline
                L'amministratore deve:
                \begin{itemize}
                \item selezionare un'organizzazione;
                \item avviare la procedura di rimozione owner;
                \item inserire email dell'owner da rimuovere;
                \item confermare la rimozione dell'owner.
                \end{itemize}
                & NS \\
        TAWFO35      & Il sistema deve permettere all'owner di togliere i privilegi di gestore ad un utente che ne sia in possesso. \newline
                L'owner deve:
                \begin{itemize}
                \item selezionare un'organizzazione;
                \item avviare la procedura di rimozione dei gestori;
                \item inserire email del gestore da rimuovere;
                \item confermare la rimozione del gestore.
                \end{itemize}
                & NS \\
        TAWFO36      & Il sistema deve permettere all'owner di togliere i privilegi di visualizzatore ad un utente che ne sia in possesso. \newline
                L'owner deve:
                \begin{itemize}
                \item selezionare un'organizzazione;
                \item avviare la procedura di rimozione dei visualizzatori;
                \item inserire email del visualizzatore da rimuovere;
                \item confermare la rimozione del visualizzatore.
                \end{itemize}
                & NS \\
        TAWFO37      & Il sistema deve permettere all'amministratore di eliminare un account dal sistema. \newline
                L'amministratore deve:
                \begin{itemize}
                \item avviare la procedura di rimozione utente;
                \item selezionare l'account da rimuovere;
                \item confermare la rimozione dell'account.
                \end{itemize}
                & NS \\
        \caption{Tabella dei test di accettazione}%
        \label{tab:test_accettazione}
      \end{longtable}
      %subs:accettazione (end)
    \end{centering}
    \subsubsection{Test di Sistema}%
  \label{subs:sistema}
    I test di sistema verificano il corretto funzionamento tra tutte le componenti del sistema. I test verranno identificati nel seguente modo:
    \begin{center}
      TS[codice]
    \end{center}
    Dove \textit{codice} rappresenta un numero identificativo per il Test di Sistema.
  %subs:sistema (end)
  \subsubsection{Test di Integrazione}%
  \label{subs:integrazione}
    I test di sistema verificano la corretta interazione e collaborazione tra un insieme di unità. I test verranno identificati nel seguente modo:
    \begin{center}
      TI[codice]
    \end{center}
    Dove \textit{codice} rappresenta un numero identificativo per il Test di Integrazione.
  %subs:integrazione (end)
  \subsubsection{Test di Unità}%
  \label{subs:unita}
    I test di sistema verificano la correttezza e il funzionamento di una singola unità del software. I test verranno identificati nel seguente modo:
    \begin{center}
      TU[codice]
    \end{center}
    Dove \textit{codice} rappresenta un numero identificativo per l'unità a cui il test appartiene.
  %subs:unita (end)
  \end{document}
