\documentclass[../piano-di-qualifica.tex]{subfiles}
\appendToGraphicspath{../../../commons/img/}

\begin{document}

Per misurare la qualità di prodotto inoltre utilizziamo dei test secondo il \glossario{Modello a \textit{V}}, per il quale definiamo e sviluppiamo i test in parallelo alle attività di analisi, progettazione architetturale e di sviluppo e verifica incrementi.
Abbiamo definito i test dividendoli nelle seguenti categorie:
\begin{description}
  \item [Test di Accettazione (TA)].
  \item [Test di Sistema (TS)].
  \item [Test di Integrazione (TI)].
  \item [Test di Unità (TU)].
\end{description}

Inoltre classifichiamo i test in base al loro stato:
\begin{description}
  \item [S]: il test è stato soddisfatto.
  \item [NS]: il test non è stato ancora soddisfatto.
\end{description}

Per visualizzare la percentuale dei test superati durante il periodo di progetto, consultare la sotto-sezione §~\ref{subs:test_non_superati}.

\subsection{Test di Accettazione}%
\label{subs:test_di_accettazione}

\subfile{tests/test-di-accettazione.tex}
\newpage

\subsection{Test di Sistema}%
\label{subs:test_di_sistema}

\subfile{tests/test-di-sistema.tex}

\newpage
\subsection{Test di Integrazione}%
\label{subs:test_di_integrazione}

\subfile{tests/test-di-integrazione.tex}

\newpage
\subsection{Test di Unità}%
\label{subs:test_di_unita}

\subfile{tests/test-di-unita.tex}


\end{document}
