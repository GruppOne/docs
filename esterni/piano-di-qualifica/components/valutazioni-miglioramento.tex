\documentclass[../piano-di-qualifica.tex]{subfiles}

\begin{document}
Lo scopo di questa sezione è riportare una valutazione riguardo il lavoro svolto, in modo da trattare
i problemi sorti e procedere ad un'efficiente risoluzione.
In questo modo, il gruppo sarà in grado di prevenire la loro ricomparsa oppure, in caso si ripresentino, di risolverli più velocemente.
I problemi sorti vengono sollevati dai membri stessi, in quanto non vi è una
persona esterna che possa dare una valutazione oggettiva.

Questa sezione è destinata ad aggiornarsi con l'avanzamento del lavoro e ad arricchirsi in modo
da migliorare progressivamente la qualità del lavoro.

\subsection{Valutazioni sull'organizzazione}

\rowcolors{2}{lightgray}{white!80!lightgray!100}
\begin{longtable}[H]{>{\centering\arraybackslash}m{3cm} >{\centering\arraybackslash}m{3cm} >{\centering\arraybackslash}m{5cm} >{\centering\arraybackslash}m{5cm}}
  \rowcolor{darkgray!90!}
  \color{white}{\textbf{Fase}} & \color{white}{\textbf{Dominio}} & \color{white}{\textbf{Problema}}                                                                                                                                         & \color{white}{\textbf{Soluzione}}                                                                                                                                                                                                        \\
  Prima versione del POC       & Incontri di gruppo              & Durante il periodo di progettazione il gruppo non ha potuto incontrarsi di persona a causa dell'epidemia di COVID-19.                                                    & Grazie a strumenti di videoconferenza e videochiamata abbiamo potuto continuare le nostre attività. I nostri incontri verranno eseguiti sempre in modalità telematica fino al termine del progetto didattico o delle misure ristrettive. \\
  Incremento 2                 & Technology baseline             & A causa di una mancata solida preparazione alla presentazione del Proof of Concept il gruppo si è dilungato in discorsi non richiesti, sforando il tempo a disposizione. & Il gruppo ha deciso di evitare questi eventi aumentando il tempo e la cura dedicati alla preparazione delle varie presentazioni.                                                                                                         \\
  \rowcolor{white}
  \caption{Tabella delle valutazioni sull'organizzazione del lavoro}%
  \label{tab:valutazioni_organizzazione}
\end{longtable}


\newpage
\subsection{Valutazioni sui ruoli}

\rowcolors{2}{lightgray}{white!80!lightgray!100}
\begin{longtable}[H]{>{\centering\arraybackslash}m{3cm} >{\centering\arraybackslash}m{3cm} >{\centering\arraybackslash}m{5cm} >{\centering\arraybackslash}m{5cm}}
  \rowcolor{darkgray!90!}
  \color{white}{\textbf{Fase}}  & \color{white}{\textbf{Ruolo}} & \color{white}{\textbf{Problema}}                                                                                                                                                                              & \color{white}{\textbf{Soluzione}}                                                                                                                                               \\
  Analisi preliminare           & Responsabile                  & Sono sorte diverse difficoltà da parte del responsabile nella suddivisione bilanciata delle ore tra i membri provocando sovraccarichi di lavoro.                                                              & Il gruppo ha deciso di incrementare l'attenzione nella suddivisione della mole di lavoro ai membri del gruppo.                                                                  \\
  Analisi preliminare           & Analista                      & Data l'inesperienza nell'analisi, sono sorti diversi problemi nell'individuazione dei requisiti e dei casi d'uso, causando rallentamenti.                                                                     & La soluzione intrapresa è stata quella di contattare il proponente cercando quindi di velocizzare l'individuazione dei requisiti, in modo da non ritardare la consegna in RR\@. \\
  Preparazione in entrata in RR & Responsabile                  & A causa di una disattenzione, è stato sbagliato il preventivo di progetto, andando sotto la soglia richiesta.                                                                                                 & Il gruppo ha deciso di attuare delle attività di verifica più consistenti, in modo da rilevare il prima possibile errori.                                                       \\
  Analisi preliminare           & Verificatore                  & La poca esperienza nel campo, ha portato ai verificatori uno studio molto consistente dei documenti su cui attuare le verifiche, provocando ritardi, oppure mancate correzioni, come quella sopra menzionata. & Il gruppo ha deciso di far prestare molta più attenzione ai verificatori, attuando una politica di ispezione attraverso una check-list.                                         \\
  Progettazione architetturale  & Programmatore                 & Sono sorte molte difficoltà nell'utilizzo delle tecnologie per colpa dell'inesperienza nel loro uso. Allungando di molto il tempo di autoapprendimento, ha portato il gruppo a saltare la seconda revisione.  & Il gruppo ha deciso di sfruttare meglio il tempo a disposizione cercando di massimizzare le ore produttive.                                                                     \\
  Progettazione architetturale  & Amministratore                & A causa della complessa continuous integration realizzata, i membri del gruppo si sono trovati in difficoltà causando rallentamenti e di conseguenza saltando la technology baseline.                         & Abbiamo deciso di limitare l'implementazione di nuovi strumenti di continuous integration, che provocano ritardi inaccettabili.                                                 \\
  \rowcolor{white}
  \caption{Tabella delle valutazioni sui ruoli}%
  \label{tab:valutazioni_ruoli}
\end{longtable}

\newpage
\subsection{Valutazioni sugli strumenti di lavoro}

\rowcolors{2}{lightgray}{white!80!lightgray!100}
\begin{longtable}[H]{>{\centering\arraybackslash}m{3cm} >{\centering\arraybackslash}m{3cm} >{\centering\arraybackslash}m{5cm} >{\centering\arraybackslash}m{5cm}}
  \rowcolor{darkgray!90!}
  \color{white}{\textbf{Fase}} & \color{white}{\textbf{Strumento}} & \color{white}{\textbf{Descrizione}}                                                                                                                                                                                                                              & \color{white}{\textbf{Soluzione}}                                                                                                                                                                                                                                                                                                                                 \\
  Analisi preliminare          & GitHub                            & Il primo problema che sorge spontaneo a tutti i membri del gruppo è come condividere globalmente il codice che viene sviluppato all'interno del gruppo.                                                                                                          & La scelta ricade su GitHub, in quanto è uno dei servizi di hosting per progetti software più utilizzati soprattutto nell'ambito di progetti open-source, ed è un implementazione del \glossario{VCS} Git, garantendo la piena compatibilità tra versione locale e versione remota del codice.                                                                     \\
  Analisi preliminare          & Latex                             & Molti membri del gruppo, non avendo mai utilizzato questo linguaggio, hanno dovuto apprendere il suo utilizzo.                                                                                                                                                   & Il gruppo ha deciso di impegnare alcune ore di autoapprendimento, data la rilevante importanza della documentazione ai fini del progetto.                                                                                                                                                                                                                         \\
  Analisi preliminare          & PlantUML                          & Gli analisti hanno scelto di utilizzare questo strumento che garantisce di implementare diagrammi UML versionabili all'interno dei documenti. Tuttavia lo strumento ha comportato diversi problemi di compatibilità, che abbiamo risolto impiegando diverse ore. & Il gruppo ha quindi deciso di valutare la complessità di implementazione degli strumenti prima di confermare il loro utilizzo.                                                                                                                                                                                                                                    \\
  Analisi preliminare          & Visual Studio Code                & I membri del gruppo hanno cercato un ambiente di sviluppo in grado di supportare la maggior parte delle attività del gruppo.                                                                                                                                     & La scelta è ricaduta su Visual Studio Code, in quanto ha una distribuzione eterogenea su tutti i sistemi operativi, e risulta uno strumento polivalente, in grado di gestire molteplici estensioni per qualsiasi evenienza, rendendo il suo utilizzo efficiente ed efficace.                                                                                      \\
  Analisi preliminare          & Sublime Text                      & Per il processo di documentazione, pochi membri del gruppo hanno deciso di utilizzare Sublime Text per la sua semplicità e leggerezza in termini di spazio sul disco.                                                                                            & È risultato che il problema non sussiste, in quanto se utilizzato solo per il processo di documentazione questo editor è pari ad altri editor. Non si può dire la stessa cosa se si parla di sviluppo backend e frontend, che risulta pressoché impossibile senza le funzionalità avanzate di Visual Studio Code ed IntelliJ IDEA\@.                              \\
  Progettazione architetturale & IntelliJ IDEA                     & Visual Studio Code si è rivelato essere un ottimo ambiente di sviluppo, ma è sorta la necessità di sviluppare il frontend dell'applicazione mobile con un ambiente di sviluppo integrato per la piattaforma Android.                                             & Vengono prese al vaglio due alternative: Android Studio oppure IntelliJ IDEA\@. Sebbene sia l'editor de facto per lo sviluppo di applicazioni Android, l'ambiente di sviluppo Android Studio è stato scartato a causa delle richieste minime troppo onerose, a vantaggio di IntelliJ IDEA che è più leggero e quindi utilizzabile da qualsiasi membro del gruppo. \\
  \rowcolor{white}
  \caption{Tabella delle valutazioni sugli strumenti di lavoro}%
  \label{tab:tabella_valutazioni_strumenti}
\end{longtable}

\end{document}
