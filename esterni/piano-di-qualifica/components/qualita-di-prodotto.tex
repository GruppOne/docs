\documentclass[../piano-di-qualifica.tex]{subfiles}
\appendToGraphicspath{../../../commons/img/}

% TODO usare https://en.wikipedia.org/wiki/ISO/IEC_9126?
% TODO usare https://www.praxiom.com/iso-90003.htm?
\begin{document}


\subsection{Specifica dei test}%
\label{sub:test}
    Per misurare la qualità di prodotto inoltre utilizziamo dei test secondo il \glossario{Modello a \textit{V}} per il quale definiamo e sviluppiamo i test in parallelo alle attività di analisi, progettazione architetturale e di sviluppo e verifica incrementi.
    Abbiamo definito i test dividendoli nelle seguenti categorie:
    \begin{description}
      \item [Test di Accettazione (TA):] vengono fatti per verificare che il prodotto soddisfi i requisiti richiesti dal proponente
      \item [Test di Sistema (TS):] vengono effettuati quando si mettono insieme tutte le componenti del software e se ne vuole testare compatibilità e interazioni
      \item [Test di Integrazione (TI):] questi test verificano la compatibilità e le interazioni tra diverse unità del software testate con successo solo singolarmente fino a quel momento tramite Test di Unità
      \item [Test di Unità (TU):] svolgono un'attività mirata di analisi sulle singole unità software, vengono eseguiti con il massimo grado possibile di parallelismo e garantiscono solo il funzionamento della singola unità a cui appartengono.
    \end{description}
    Inoltre classifichiamo i test in base al loro stato:
      \begin{description}
        \item [S]: il test è stato soddisfatto
        \item [NS]: il test non è stato ancora soddisfatto
      \end{description}
  \subsubsection{Test di Accettazione}
  \label{subs:accettazione}
      I test di accettazione hanno lo scopo di verificare che il prodotto soddisfi i requisiti richiesti dal proponente, vengono eseguiti durante la fase di verifica e collaudo finale.
      Essendo tutti i test derivanti da un gruppo di requisiti che hanno una determinata importanza nel prodotto software abbiamo pensato di numerare i Test di Accettazione nel modo seguente:
      \begin{center}
          [TA][requisito][importanza][codice]
      \end{center}
      Dove: \textit{importanza} è un valore numerico ad indicare l'importanza del requisito che deve essere soddisfatto, \textit{codice} è un codice numerico identificativo per il test e \textit{requisito} può assumere i seguenti valori:
      \begin{description}
        \item [F]: requisito funzionale 
        \item [P]: requisito prestazionale
        \item [D]: requisito dichiarativo
      \end{description}
  %subs:accettazione (end)
  \subsubsection{Test di Sistema}
  \label{subs:sistema}
    I test di sistema verificano il corretto funzionamento tra tutte le componenti del sistema. I test verranno identificati nel seguente modo:
    \begin{center}
      TS[codice]
    \end{center}
    Dove \textit{codice} rappresenta un numero identificativo per il Test di Sistema.
  %subs:sistema (end)
  \subsubsection{Test di Integrazione}
  \label{subs:integrazione}
    I test di sistema verificano la corretta interazione e collaborazione tra un isieme di unità. I test verranno identificati nel seguente modo:
    \begin{center}
      TI[codice]
    \end{center}
    Dove \textit{codice} rappresenta un numero identificativo per il Test di Integrazione.
  %subs:integrazione (end)
  \subsubsection{Test di Unità}
  \label{subs:unita}
    I test di sistema verificano la correttezza e il funzionamento di una singola unità del software. I test verranno identificati nel seguente modo:
    \begin{center}
      TU[codice]
    \end{center}
    Dove \textit{codice} rappresenta un numero identificativo per l'unità a cui il test appartiene.
  %subs:unita (end)
% sub:test (end)
\end{document}
