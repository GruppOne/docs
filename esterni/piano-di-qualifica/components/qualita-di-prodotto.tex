\documentclass[../piano-di-qualifica.tex]{subfiles}
\appendToGraphicspath{../../../commons/img/}

\begin{document}

Le caratteristiche dello standard ISO/IEC 25010:2011 che il gruppo ha ritenuto rilevanti per il prodotto vengono elencate qui sotto, accompagnate dai valori desiderabili delle metriche definite nelle \textit{Norme di Progetto} per misurare il grado di raggiungimento della qualità di prodotto.

\subsection{Qualità di documentazione}
Qualità di documentazione come:
\begin{itemize}
  \item Completa
  \item Leggibile
  \item Corretta
  \item Coerente
  \item Modulare.
\end{itemize}
\subsubsection{Metriche}%
\label{subsec:metriche_doc}

Di seguito sono riportate le metriche utilizzate per misurare la qualità di documentazione, che possono essere visionate all'interno del documento \textit{Norme di progetto} alla sezione 3.2.7.

\rowcolors{2}{lightgray}{white!80!lightgray!100}
\renewcommand{\arraystretch}{2} % allarga le righe con dello spazio sotto e sopra
\begin{longtable}[H]{>{\centering\bfseries}m{3cm} >{}m{4cm} >{}m{5cm} >{\centering\arraybackslash}m{2cm} > {\centering\arraybackslash}m{2cm}}
  \rowcolor{darkgray!90!}
  \color{white}
  {\textbf{Codice}} & \color{white}{\textbf{Nome}} & \color{white}{\textbf{Descrizione} } & \color{white}{\textbf{Valore ammissibile}}  & \color{white}{\textbf{Valore ottimale}}   \\
  \endhead\rowcolor{white}%
  \multicolumn{3}{r}{\textit{Continua alla pagina seguente}}
  \endfoot%
  \endlastfoot%

  % - Applicazione mobile
      MPR-IDG & Indice di Gulpease & indice compreso tra 0 e 100 che stabilisce la leggibilità del testo in base ala lunghezza delle parole. & \geq{}50 & \geq{}80 \\

      MPR-CO & Indice di correttezza ortografica & numero intero che rappresenta il numero di errori ortografici presenti nel testo. & 0 & 0 \\
 
    \rowcolor{white}
    \caption{Tabella delle metriche di qualità della documentazione}%
    \label{tab:metriche_doc}
  \end{longtable}



\subsection{Qualità di sviluppo software}


\subsubsection{Metriche}%
\label{subsec:metriche_svil_sw}

Di seguito sono riportate le metriche utilizzate per misurare la qualità di sviluppo software, che possono essere visionate all'interno del documento \textit{Norme di progetto} alla sezione 2.3.4.



\rowcolors{2}{lightgray}{white!80!lightgray!100}
\renewcommand{\arraystretch}{2} % allarga le righe con dello spazio sotto e sopra
\begin{longtable}[H]{>{\centering\bfseries}m{3cm} >{}m{4cm} >{}m{5cm} >{\centering\arraybackslash}m{2cm} > {\centering\arraybackslash}m{2cm}}
  \rowcolor{darkgray!90!}
  \color{white}
  {\textbf{Codice}} & \color{white}{\textbf{Nome}} & \color{white}{\textbf{Descrizione} } & \color{white}{\textbf{Valore ammissibile}}  & \color{white}{\textbf{Valore ottimale}}   \\
  \endhead\rowcolor{white}%
  \multicolumn{3}{r}{\textit{Continua alla pagina seguente}}
  \endfoot%
  \endlastfoot%

  % - Applicazione mobile
      MPR-CCE & Complessità ciclomatica eccessiva & numero intero indicante la analizzabilità, modificabilità e testabilità del prodotto. &  \leq{} 7 & \leq{} 3 \\

      MPR-CDU & Percentuale di duplicazione del codice & percentuale che indica il numero di parti di codice duplicate. & 20\% & 0\% \\
 
      MPR-SML & Numero di code smell & numero di code smell rilevati. & 40 & 0 \\

      MPR-BUG & Numero di bug rilevati &  numero di bug rilevati per un componente software. & 0 & 0 \\

      MPR-RSL & Tempo stimato di risoluzione bug & tempo stimato di risoluzione di un bug per un componente software. & <30min & <10min \\
    \rowcolor{white}
    \caption{Tabella delle metriche di qualità di sviluppo del software}%
    \label{tab:metriche_sv_sw}
  \end{longtable}



\end{document}
