\documentclass[../piano-di-qualifica.tex]{subfiles}
\appendToGraphicspath{../../../commons/img/}

% TODO usare https://en.wikipedia.org/wiki/ISO/IEC_9126?
% TODO usare https://www.praxiom.com/iso-90003.htm?
\begin{document}

Per valutare la qualità di prodotto il gruppo ha deciso di fare riferimento allo standard \glossario{ISO/IEC 25010}, che indica le caratteristiche di qualità del prodotto software e della qualità in uso. Le caratteristiche che il gruppo ha ritenuto rilevanti per il prodotto vengono elencate qui sotto, accompagnate dai valori desiderabili delle metriche definite nelle \textit{Norme di Progetto} per misurare il grado di raggiungimento della qualità di prodotto.

\subsection{Appropriatezza funzionale}%
\label{sub:appropriatezza_funzionale}
  Rappresenta il grado a cui un prodotto software fornisce funzionalità che rispettano requisiti espliciti o impliciti se usato in specifiche condizioni.
  \subsubsection{Obiettivi}
  \label{subs:obiettivi}
      \begin{description}
        \item [Correttezza]: Grado con cui il prodotto fornisce risultati corretti con il livello di precisione richiesto.
        \item [Completezza]: Grado con cui le funzionalità offerte dal prodotto coprono tutti i compiti specificati e gli obiettivi dell'utente
        \item [Appropriatezza]: Grado con cui le funzionalità del prodotto agevolano il raggiungimento di determinati obiettivi.
      \end{description}
  % subs:obiettivi (end)

  \subsubsection{Metriche}%
  \label{subs:metriche}
      \begin{description}
    \item [Percentuale di requisiti soddisfatti]: La completezza C del prodotto in relazione al numeri di requisiti soddisfatti verrà calcolata come \begin{equation} C = \frac{N_s}{N_{tot}}\\*100 \end{equation} con \(N_s\) numero di requisiti implementati e \(N_{tot}\) il numero di requisiti totali.
    \begin{itemize} \item Valore ammissibile: 100\% \item Valore ottimale: 100\% \end{itemize}
      \end{description}
  
  % subs:metriche (end)

% sub:appropriatezza_funzionale (end)

\subsection{Performance}%
\label{sub:performance}
 Questa caratteristica rappresenta la performance del prodotto relativamente alle risorse utilizzate in specifiche condizioni.
  \subsubsection{Obiettivi}
  \label{subs:obiettivi}
      \begin{description}
        \item[Risposta nel tempo]: Grado con cui i tempi di risposta e di elaborazione del prodotto rispettano i requisiti.
        \item[Utilizzo delle risorse]: Grado con cui la quantità e il tipo di risorse usate dal prodotto rispettano i requisiti.
        \item[Capacità]: Grado con cui il massimo limite di un parametro del prodotto rispetta i requisiti.
      \end{description}
  % subs:obiettivi (end)

  \subsubsection{Metriche}%
  \label{subs:metriche}
    \begin{description}
      \item [apdex]:
      \begin{itemize} \item Valore ammissibile: 0.6 \item Valore ottimale: 0.9 \end{itemize}
      \end{description}
  % subs:metriche (end)

% sub:performance (end)

\subsection{Sicurezza}%
\label{sub:sicurezza}
 Questa caratteristica indica il grado con cui il prodotto protegge e limita l'accesso a dati e informazioni in modo da rispettare i livelli di autorizzazione definiti.
  \subsubsection{Obiettivi}
  \label{subs:obiettivi}
      \begin{description}
        \item [Riservatezza]: Grado con cui il prodotto garantisce che i dati siano accessibili solamente a entità autorizzate.
        \item [Integrità]: Grado con cui il prodotto previene utilizzo e modifica non autorizzati di computer, programmi e dati.
        \item [Non-ripudiazione]: Grado con cui azioni o eventi che è provato siano avvenute non possano essere ripudiate successivamente
        \item [Autenticazione]: Grado con cui le azioni di un'entità possono essere collegate univocamente all'entità.
        \item [Autenticità]: Grado con cui l'identità di un soggetto o risorsa può essere confermata.
      \end{description}
  % subs:obiettivi (end)

  \subsubsection{Metriche}%
  \label{subs:metriche}

  % subs:metriche (end)

% sub:sicurezza (end)

\subsection{Mantenibilità}%
\label{sub:mantenibilita}
Questa caratteristica rappresenta il grado di efficacia e efficienza con cui il prodotto può essere migliorato, corretto o adattato a nuovi requisiti.
\subsubsection{Obiettivi}
\label{subs:obiettivi}
      \begin{description}
        \item [Modularità]: Grado con cui il prodotto è composto da parti indipendenti in modo che una modifica a una componente abbia un impatto minimo su tutte le altre.
        \item [Riusabilità]: Grado con cui una risorsa può essere utilizzata in più di un sistema.
        \item [Analizzabilità]: Grado di efficacia e di efficienza con cui è possibile identificare le cause di errori nel prodotto e le parti da modificare.
        \item [Modificabilità]: Grado con cui il prodotto può essere efficacemente ed efficientemente modificato senza introdurre regressioni.
        \item [Testabilità]: Grado di efficacia e di efficienza con cui possono essere stabiliti i criteri di test e con cui quei test possono essere eseguiti per verificare quei criteri.
      \end{description}
% subs:obiettivi (end)

\subsubsection{Metriche}%
\label{subs:metriche}
  \begin{description}
    \item [Semplicità delle funzioni]
      \begin{itemize} \item Valore ammissibile: 0.6 \item Valore ottimale: 0.9 \end{itemize}
    \item [Semplicità delle classi]
      \begin{itemize} \item Valore ammissibile: 0.6 \item Valore ottimale: 0.9 \end{itemize}
  \end{description}
% subs:metriche (end)

% sub:mantenibilita (end)

\subsection{Usabilità}%
\label{sub:usabilita}
Questa caratteristica rappresenta il grado con cui il prodotto può essere usato da un determinato utente per raggiungere obiettivi specifici con efficienza, efficacia e soddisfazione in un determinato contesto d'uso.
\subsubsection{Obiettivi}
\label{subs:obiettivi}
      \begin{description}
        \item [Apprendibilità]: Grado con cui il prodotto consente a un determinato utente di apprendere ad utilizzare il prodotto con efficienza, efficacia, libertà dai rischi, e soddisfazione.
        \item [Operabilità]: Grado con un il prodotto presenta delle caratteristiche che lo rendono di semplice utilizzo.
        \item [Protezione dagli errori]: Grado con cui il prodotto protegge l'utente dal commettere errori.
        \item [Accessibilità]: Grado con cui il prodotto risulta utilizzabile da persone con caratteristiche e capacità varie per raggiungere un determinato obiettivo.
      \end{description}
% subs:obiettivi (end)

\subsubsection{Metriche}%
\label{subs:metriche}
\item [Efficienza nel tempo]
    \begin{itemize} \item Valore ammissibile: 0.6 \item Valore ottimale: 0.9 \end{itemize}
% subs:metriche (end)

% sub:usabilita (end)

\subsection{Affidabilità}%
\label{sub:affidabilita}
 Questa caratteristica rappresenta il grado con cui il prodotto effettua specifici compiti sotto determinate condizioni per un periodo di tempo fissato.
\subsubsection{Obiettivi}
\label{subs:obiettivi}
      \begin{description}
        \item [Maturità]: Grado con cui il prodotto rispetta i requisiti di affidabilità in condizioni normali.
        \item [Disponibilità]: Grado con cui il prodotto risulta operativo e accessibile quando ne è richiesto l'uso.
        \item [Tolleranza agli errori]: Grado con cui il prodotto continua ad operare in maniera corretta nonostanze la presenza di errori software e hardware.
        \item [Recuperabilità]: Grado con cui il prodotto, in caso di un'interruzione o di un errore, riesce a recuperare i dati coinvolti e ristabilire il corretto funzionamento del sistema.
      \end{description}
% subs:obiettivi (end)

\subsubsection{Metriche}%
\label{subs:metriche}
\item [Numero di errori]
    \begin{itemize} \item Valore ammissibile: 0.6 \item Valore ottimale: 0.9 \end{itemize}
% subs:metriche (end)

% sub:affidabilita (end)

\subsection{Specifica dei test}%
\label{sub:test}
    Per misurare la qualità di prodotto inoltre utilizziamo dei test secondo il \glossario{Modello a \textit{V}} per il quale definiamo e sviluppiamo i test in parallelo alle attività di analisi, progettazione architetturale e di sviluppo e verifica incrementi.
    Abbiamo definito i test dividendoli nelle seguenti categorie:
    \begin{description}
      \item [Test di Accettazione (TA):] vengono fatti per verificare che il prodotto soddisfi i requisiti richiesti dal proponente
      \item [Test di Sistema (TS):] vengono effettuati quando si mettono insieme tutte le componenti del software e se ne vuole testare compatibilità e interazioni
      \item [Test di Integrazione (TI):] questi test verificano la compatibilità e le interazioni tra diverse unità del software testate con successo solo singolarmente fino a quel momento tramite Test di Unità
      \item [Test di Unità (TU):] svolgono un'attività mirata di analisi sulle singole unità software, vengono eseguiti con il massimo grado possibile di parallelismo e garantiscono solo il funzionamento della singola unità a cui appartengono.
    \end{description}
    Inoltre classifichiamo i test in base al loro stato:
      \begin{description}
        \item [S]: il test è stato soddisfatto
        \item [NS]: il test non è stato ancora soddisfatto
      \end{description}
  \subsubsection{Test di Accettazione}
  \label{subs:accettazione}
      I test di accettazione hanno lo scopo di verificare che il prodotto soddisfi i requisiti richiesti dal proponente, vengono eseguiti durante la fase di verifica e collaudo finale.
      Essendo tutti i test derivanti da un gruppo di requisiti che hanno una determinata importanza nel prodotto software abbiamo pensato di numerare i Test di Accettazione nel modo seguente:
      \begin{center}
          [TA][requisito][importanza][codice]
      \end{center}
      Dove: \textit{importanza} è un valore numerico ad indicare l'importanza del requisito che deve essere soddisfatto, \textit{codice} è un codice numerico identificativo per il test e \textit{requisito} può assumere i seguenti valori:
      \begin{description}
        \item [F]: requisito funzionale 
        \item [P]: requisito prestazionale
        \item [D]: requisito dichiarativo
      \end{description}
      \begin{centering}
      \rowcolors{2}{lightgray}{white!80!lightgray!100}
      \renewcommand{\arraystretch}{2} % allarga le righe con dello spazio sotto e sopra
      \begin{longtable}[H]{>{\centering\bfseries}m{3cm} >{}p{10cm} >{\centering\arraybackslash}m{3cm}}
        \rowcolor{darkgray!90!}
        \color{white}
        {\textbf{ID test}} & \color{white}{\textbf{Descrizione}} & \color{white}{\textbf{Esito}} \\
        \endhead
        \rowcolor{white}
        \multicolumn{3}{r}{\textit{Continua alla pagina seguente}}
        \endfoot
        \endlastfoot
        TAAFO001      & Al nuovo utente deve essere permesso di registrarsi. \newline 
                        L’utente deve: 
                        \begin{itemize} 
                          \item visualizzare e confermare l’\glossario{EULA};
                          \item inserire password;
                          \item confermare la password;
                          \item inserire i propri dati anagrafici;
                          \item inserire la proria email;
                          \item confermare la registrazione.
                        \end{itemize}
                      & NS \\   
        TAAFO002      & Il sistema deve rifiutare la richiesta di registrazione se i dati inseriti non rispettano i vincoli imposti o se l’email è già presente nel database. \newline 
                        L’utente deve:  
                        \begin{itemize} 
                          \item inserire dati di accesso non validi;
                          \item verificare l'impossibilità di proseguire con la registrazione.
                        \end{itemize}
                      & NS \\ 
        TAAFO003      & Il sistema deve permettere all’utente registrato di autenticarsi. \newline 
                        L’utente deve: 
                        \begin{itemize} 
                          \item inserire la propria email;
                          \item inserire la propria password;
                          \item confermare l'autenticazione.
                        \end{itemize}
                      & NS \\   
        TAAFO004      & Il sistema deve rifiutare la richiesta di autenticazione se i dati inseriti non rispettano i vincoli imposti, o se la combinazione di email e password non è presente nel database. \newline 
                        L’utente deve:  
                        \begin{itemize} 
                          \item inserire dati di accesso non validi;
                          \item verificare l'impossibilità di proseguire con l'autenticazione.
                        \end{itemize}
                      & NS \\ 
        TAAFO005      & Il sistema deve permettere all’utente di recuperare la sua password. \newline 
                        L’utente deve: 
                        \begin{itemize} 
                          \item avviare la procedura di recupero password;
                          \item inserire la propria email;
                          \item confermare la procedura;
                          \item verificare che la password venga ricevuta correttamente;
                        \end{itemize}
                      & NS \\   
        TAAFO006      & Il sistema deve permettere all’utente di reimpostare la sua password. \newline 
                        L’utente deve:  
                        \begin{itemize} 
                          \item avviare la procedura di reimpostazione password;
                          \item inserire la propria email;
                          \item inserire la vecchia password;
                          \item inserire una nuova password;
                          \item confermare la nuova password;
                          \item confermare la procedura;
                          \item verificare che la password sia stata cambiata correttamente;
                        \end{itemize}
                      & NS \\ 
          TAAFO007    & Il sistema deve rifiutare la richiesta di reimpostazione password se la password scelta non rispetta i vincoli imposti. \newline 
                        L’utente deve: 
                        \begin{itemize} 
                          \item avviare la procedura di reimpostazione password;
                          \item inserire la propria email;
                          \item inserire la vecchia password;
                          \item inserire una nuova password non valida;
                          \item confermare la nuova password;
                          \item confermare la procedura;
                          \item visualizzare un messaggio di errore;
                        \end{itemize}
                      & NS \\   
        TAAFO008      & Il sistema deve permettere all’utente autenticato di recuperare dal server una lista di organizzazioni a cui può collegarsi e di aggiornarla. \newline 
                        L’utente deve:  
                        \begin{itemize} 
                          \item autenticarsi;
                          \item visualizzare la lista di organizzazioni;
                          \item aggiornare la lista;
                          \item visualizzare la lista aggiornata.
                        \end{itemize}
                      & NS \\ 
        TAAFO009      & Il sistema deve permettere all’utente autenticato di selezionare una o più organizzazioni dalla lista e collegarsi alle organizzazioni selezionate. \newline 
                        L’utente deve: 
                        \begin{itemize} 
                          \item selezionare una o più organizzazioni dalla lista;
                          \item avviare la procedura di collegamento.
                        \end{itemize}
                      & NS \\   
        TAAFO010      & Il sistema deve permettere all’utente autenticato e collegato ad una o più organizzazioni di scollegarsi da una o più organizzazioni. \newline 
                        L’utente deve:  
                        \begin{itemize} 
                          \item selezionare una o più organizzazioni a cui è collegato;
                          \item avviare la procedura di scollegamento.
                        \end{itemize}
                      & NS \\ 
        TAAFO011      & Il sistema deve permettere all’utente autenticato di visualizzare tramite un filtro la lista delle organizzazioni a cui è collegato oppure quella delle organizzazioni da cui è scollegato. \newline 
                      L’utente deve:  
                      \begin{itemize} 
                        \item impostare il filtro;
                        \item visualizzare la lista richiesta.
                      \end{itemize}
                      & NS \\ 
        TAAFO012      & Il sistema deve permettere all’utente autenticato il passaggio da noto ad incognito e viceversa. \newline 
                      L’utente deve:  
                      \begin{itemize} 
                        \item avviare la procedura di cambio di stato;
                        \item verificare l'effettivo cambio di stato.
                      \end{itemize}
                      & NS \\ 
        TAAFO013      & Il sistema deve permettere all’utente autenticato di visualizzare il proprio storico. \newline 
                      L’utente deve:  
                      \begin{itemize} 
                        \item visualizzare il proprio storico degli accessi;
                        \item visualizzare il proprio tempo trascorso all'interno di ogni organizzazione.
                      \end{itemize}
                      & NS \\ 
        TAAFO014      & Il sistema deve permettere all’utente autenticato di disconnettersi dall’applicazione. \newline 
                      L’utente deve:  
                      \begin{itemize} 
                        \item avviare la procedura di disconnessione;
                        \item confermare la procedura.
                      \end{itemize}
                      & NS \\ 
        TAAFO015      & Il sistema deve permettere all’utente di eliminare il suo account. \newline 
                      L’utente deve:  
                      \begin{itemize} 
                        \item avviare la procedura di eliminazione;
                        \item confermare la procedura.
                      \end{itemize}
                      & NS \\ 
        TAAFO016      & Il sistema deve mostrare un messaggio d’errore e rifiutare le richieste da parte dell’app nel caso mancasse una connessione a internet. \newline 
                      L’utente deve:  
                      \begin{itemize} 
                        \item disconnettere il dispositivo da internet;
                        \item effettuare un'azione che richieda la connessione ad internet;
                        \item visualizzare il messaggio di errore corrispondente.
                      \end{itemize}
                      & NS \\ 
        TAAFO017      & L’interfaccia web deve permettere all’utente non autenticato di autenticarsi. \newline 
                      L’utente deve:  
                      \begin{itemize} 
                        \item compilare il campo email;
                        \item compilare il campo password;
                        \item confermare l'autenticazione.
                      \end{itemize}
                      & NS \\ 
        TAAFO018      & L’interfaccia web deve rifiutare la richiesta di autenticazione se i dati inseriti non rispettano i vincoli imposti, o se la combinazione di email e password non è presente nel database. \newline 
                      L’utente deve:  
                      \begin{itemize} 
                        \item inserire una combinazione di dati non validi;
                        \item verificare la possibilità di proseguire l'autenticazione.
                      \end{itemize}
                      & NS \\ 
        TAWFO019      & L’interfaccia web deve permettere all’utente autenticato di effettuare la disconnessione. \newline
        L’utente deve: 
        \begin{itemize} 
         \item avviare la procedura di disconnessione. 
        \end{itemize}
        & NS \\
        TAWFO020      & L’interfaccia web deve permettere all’owner di creare un’organizzazione. \newline
        L’owner deve: 
        \begin{itemize} 
         \item avviare la procedura di creazione di una organizzazione;
         \item inserire il nome dell'organizzazione;
         \item inserire la descrizione dell'organizzazione;
         \item specificare i dettagli del server \glossario{LDAP}
         \item verificare l'avvenuta creazione dell'organizzazione
        \end{itemize}
        & NS \\
        TAWFO021      & L’interfaccia web deve permettere all’owner di eliminare un’organizzazione in suo possesso. \newline
        L’owner deve: 
        \begin{itemize} 
         \item avviare la procedura di eliminazione di una organizzazione;
         \item confermare la procedura di eliminazione
        \end{itemize}
        & NS \\
        TAWFO022      & L’interfaccia web deve permettere all’owner di modificare l’organizzazione. \newline
        L’owner deve: 
        \begin{itemize} 
         \item avviare la procedura di modifica di una organizzazione;
         \item modificare uno o più dati dell'organizzazione (nome, descrizione, configurazione del server LDAP).
        \end{itemize}
        & NS \\
        TAWFO023      & L’interfaccia web deve permettere all’owner e al gestore di visualizzare una lista delle organizzazioni di cui fanno parte. \newline
        L’owner e il gestore devono: 
        \begin{itemize} 
         \item visualizzare la lista delle organizzazioni di cui fanno parte.
        \end{itemize}
        & NS \\
        TAAFO024      & Il sistema deve inviare una richiesta di aggiornamento della lista delle organizzazioni a tutte le applicazioni mobile.         \newline
        L’utente deve: 
        \begin{itemize} 
         \item visualizzare una notifica con la richiesta di aggiornamento qual'ora una delle organizzazioni venga modificata.
        \end{itemize}
        & NS \\
        TAWFO025      & L’interfaccia web deve permettere all’owner e al gestore di gestire i luoghi dell’organizzazione.         \newline
        L’utente deve: 
        \begin{itemize} 
         \item avviare la procedura di inserimento di un nuovo luogo;
         \item inserire le coordinate geografiche;
         \item inserire l'indirizzo.
        \end{itemize}
        & NS \\
        TAWFF026      & L’interfaccia web può opzionalmente permettere all’owner e al
        gestore di aggiungere un luogo all’organizzazione evidenziandone
        l’area sulla mappa. \newline
        L’owner e il gestore devono: 
        \begin{itemize} 
         \item avviare la procedura di inserimento di un nuovo luogo;
         \item selezionare l'area sulla mappa.
        \end{itemize}
        & NS \\
        TAWFO027      & L’interfaccia web deve permettere all’owner e al gestore di eliminare luoghi dall’organizzazione. \newline
        L’owner e il gestore devono: 
        \begin{itemize} 
         \item avviare la procedura di eliminazione di un nuovo luogo;
         \item confermare l'eliminazione.
        \end{itemize}
        & NS \\
        TAWFO028      & L’interfaccia web deve permettere all’owner e al gestore di modificare i luoghi dell’organizzazione. \newline
        L’owner e il gestore devono: 
        \begin{itemize} 
         \item avviare la procedura di modifica di un nuovo luogo;
         \item modificare i dati del luogo (indirizzo e coordinate o area sulla mappa).
        \end{itemize}
        & NS \\
        TAWFO029      & L’interfaccia web deve permettere all’owner e al gestore di visualizzare una lista dei luoghi dell’organizzazione. \newline
        L’owner e il gestore devono: 
        \begin{itemize} 
         \item selezionare la lista dei luoghi dell'organizzazione;
         \item visualizzare la lista dei luoghi.
        \end{itemize}
        & NS \\
        TAWFO30      & L’interfaccia web deve permettere agli amministratori di un’organizzazione di visualizzare le informazioni sull’organizzazione. \newline
        L’owner e il gestore devono: 
        \begin{itemize} 
         \item selezionare un'organizzazione dalla lista;
         \item visualizzare le informazioni dell'organizzazione.
        \end{itemize}
        & NS \\
        TAWFO31      & L’interfaccia web deve permettere agli amministratori di un’organizzazione di visualizzare gli accessi di un dipendente. \newline
        Gli amministratori dell'organizzazione devono: 
        \begin{itemize} 
         \item selezionare il dipendente dall'elenco della propria organizzazione;
         \item visualizzare i dati di accesso del dipendente selezionato.
        \end{itemize}
        & NS \\
        TAWFO32      & L’interfaccia web deve permettere all’utente autenticato di diventare owner e ottenere la possibilità di creare una sua organizzazione. \newline
        L'utente deve: 
        \begin{itemize} 
         \item avviare la procedura per diventare owner;
         \item confermare la richiesta di diventare owner.
        \end{itemize}
        & NS \\
        TAWFO33      & Il sistema deve permettere all’owner di aggiungere un nuovo gestore dell’organizzazione.    \newline
        L'owner deve: 
        \begin{itemize} 
         \item selezionare un'organizzazione di cui è owner;
         \item avviare la procedura di aggiunta di un gestore;
         \item inserire e-mail dell'utente da promuovere.
        \end{itemize}
        & NS \\
        TAWFO34      & Il sistema deve permettere all’owner di aggiungere un nuovo visualizzatore dell’organizzazione.    \newline
        L'owner deve: 
        \begin{itemize} 
         \item selezionare un'organizzazione di cui è owner;
         \item avviare la procedura di aggiunta di un visualizzatore;
         \item inserire e-mail dell'utente da promuovere.
        \end{itemize}
        & NS \\
        TAWFO34      & Il sistema deve permettere all’amministratore di togliere i privilegi di owner ad un utente che ne sia in possesso.    \newline
        L'amministratore deve: 
        \begin{itemize} 
         \item selezionare un'organizzazione;
         \item avviare la procedura di rimozione owner;
         \item inserire e-mail dell'owner da rimuovere;
         \item confermare la rimozione dell'owner.
        \end{itemize}
        & NS \\
        TAWFO35      & Il sistema deve permettere all’owner di togliere i privilegi di gestore ad un utente che ne sia in possesso. \newline
        L'owner deve: 
        \begin{itemize} 
         \item selezionare un'organizzazione;
         \item avviare la procedura di rimozione dei gestori;
         \item inserire e-mail del gestore da rimuovere;
         \item confermare la rimozione del gestore.
        \end{itemize}
        & NS \\
        TAWFO36      & Il sistema deve permettere all’owner di togliere i privilegi di visualizzatore ad un utente che ne sia in possesso. \newline
        L'owner deve: 
        \begin{itemize} 
         \item selezionare un'organizzazione;
         \item avviare la procedura di rimozione dei visualizzatori;
         \item inserire e-mail del visualizzatore da rimuovere;
         \item confermare la rimozione del visualizzatore.
        \end{itemize}
        & NS \\
        TAWFO37      & Il sistema deve permettere all’amministratore di eliminare un account dal sistema. \newline
        L'amministratore deve: 
        \begin{itemize} 
         \item avviare la procedura di rimozione utente;
         \item selezionare l'account da rimuovere;
         \item confermare la rimozione dell'account.
        \end{itemize}
        & NS \\


      \end{longtable}
      %subs:accettazione (end)
    \end{centering}
    \subsubsection{Test di Sistema}
  \label{subs:sistema}
    I test di sistema verificano il corretto funzionamento tra tutte le componenti del sistema. I test verranno identificati nel seguente modo:
    \begin{center}
      TS[codice]
    \end{center}
    Dove \textit{codice} rappresenta un numero identificativo per il Test di Sistema.
  %subs:sistema (end)
  \subsubsection{Test di Integrazione}
  \label{subs:integrazione}
    I test di sistema verificano la corretta interazione e collaborazione tra un isieme di unità. I test verranno identificati nel seguente modo:
    \begin{center}
      TI[codice]
    \end{center}
    Dove \textit{codice} rappresenta un numero identificativo per il Test di Integrazione.
  %subs:integrazione (end)
  \subsubsection{Test di Unità}
  \label{subs:unita}
    I test di sistema verificano la correttezza e il funzionamento di una singola unità del software. I test verranno identificati nel seguente modo:
    \begin{center}
      TU[codice]
    \end{center}
    Dove \textit{codice} rappresenta un numero identificativo per l'unità a cui il test appartiene.
  %subs:unita (end)
% sub:test (end)
\end{document}
