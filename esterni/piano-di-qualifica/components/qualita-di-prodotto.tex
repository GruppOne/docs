\documentclass[../piano-di-qualifica.tex]{subfiles}
\appendToGraphicspath{../../../commons/img/}

\begin{document}

Le caratteristiche dello standard ISO/IEC 25010:2011 che il gruppo ha ritenuto rilevanti per il prodotto vengono elencate qui sotto, accompagnate dai valori desiderabili delle metriche definite nelle \textit{Norme di Progetto} per misurare il grado di raggiungimento della qualità di prodotto.

\subsection{Qualità di documentazione}
Qualità di documentazione come:
\begin{itemize}
  \item Completa
  \item Leggibile
  \item Corretta
  \item Coerente
  \item Modulare
\end{itemize}
\subsubsection{Metriche}%
\label{par:metriche_doc}

\begin{description}
\item [[MPR-IDG \textasciitilde] Indice di Gulpease]: definito nelle \textit{Norme di progetto} alla sezione 3.2.7.1.
\begin{itemize} \item Risultato: indice compreso tra 0 e 100 che stabilisce la leggibilità del testo in base ala lunghezza delle parole. \item Valore ammissibile: maggiore di 50\% \item Valore ottimale: maggiore di 80\% \end{itemize}
\item [[MPR-IGF \textasciitilde] Indice di Gunning Fog]: definito nelle \textit{Norme di progetto} alla sezione 3.2.7.2.

\begin{itemize} \item Risultato: numero intero che indica il numero di anni minimi di educazione formale necessari a comprendere il testo. \item Valore ammissibile: \leq{}  inferiore a 12 \item Valore ottimale: \leq{}  compreso tra 9 e 10 \end{itemize}
  \item [[MPR-CO \textasciitilde] Indice di correttezza ortografica]: definito nelle \textit{Norme di progetto} alla sezione 3.2.7.3.
  \begin{itemize} \item Risultato: numero intero che rappresenta il numero di errori ortografici presenti nel testo. \item Valore ammissibile: \leq{}  0 \item Valore ottimale: \leq{}  0 \end{itemize}
\end{description}





\subsection{Qualità di prodotto software}

\subsubsection{Metriche}%
\label{subs:metriche_funzionale}
\begin{description}
\item [[MPR-ROS \textasciitilde] Percentuale di requisiti soddisfatti]: definito nelle \textit{Norme di progetto} alla sezione 2.3.4.1.
\begin{itemize} \item Valore ammissibile: 100\% \item Valore ottimale: 100\% \end{itemize}
\item [[MPR-CCE \textasciitilde] Complessità ciclomatica]: definito nelle \textit{Norme di progetto} alla sezione 2.3.4.4.
  \begin{itemize} \item Valore ammissibile: \leq{}  7 \item Valore ottimale: \leq{}  3 \end{itemize}
\end{description}

% TODO - copiare descrizioni sotto nelle NDP

% \subsection{Appropriatezza funzionale}%
% \label{sub:appropriatezza_funzionale}
%   Rappresenta il grado a cui un prodotto software fornisce funzionalità che rispettano requisiti espliciti o impliciti se usato in specifiche condizioni.
%   \subsubsection{Obiettivi}%
%   \label{subs:obiettivi_funzionale}
%       \begin{description}
%         \item [Correttezza]: Grado con cui il prodotto fornisce risultati corretti con il livello di precisione richiesto.
%         \item [Completezza]: Grado con cui le funzionalità offerte dal prodotto coprono tutti i compiti specificati e gli obiettivi dell'utente
%         \item [Appropriatezza]: Grado con cui le funzionalità del prodotto agevolano il raggiungimento di determinati obiettivi.
%       \end{description}
%   % subs:obiettivi_funzionale (end)

%   \subsubsection{Metriche}%
%   \label{subs:metriche_funzionale}
%       \begin{description}
%         \item [[MPR-RS \textasciitilde] Percentuale di requisiti soddisfatti]: La completezza C del prodotto in relazione al numeri di requisiti soddisfatti verrà calcolata come \begin{equation} C = \frac{N_s}{N_{tot}}\\*100 \end{equation} con \(N_s\) numero di requisiti implementati e \(N_{tot}\) il numero di requisiti totali.
%         \begin{itemize} \item Valore ammissibile: 100\% \item Valore ottimale: 100\% \end{itemize}
%         \item [[MPR-SOF \textasciitilde] Semplicità delle funzioni]: Possiamo monitorare la modularità, analizzabilità e testabilità del prodotto valutando il numero di parametri per ogni funzione, meno saranno più il suo compito sarà specifico.
%           \begin{itemize}   \item Valore ammissibile: \leq{}  6 \item Valore ottimale: \leq{}  3 \end{itemize}
%         \item [[MPR-SOC \textasciitilde] Semplicità delle classi]: Possiamo monitorare la modularità, analizzabilità e testabilità del prodotto valutando il numero di metodi per ogni classe, meno saranno più lo scopo di quella classe sarà specifico.
%           \begin{itemize} \item Valore ammissibile: \leq{}  10 \item Valore ottimale: \leq{}  5 \end{itemize}
%         \item [[MPR-CC \textasciitilde] Complessità ciclomatica]: Monitoriamo la analizzabilità, modificabilità e testabilità del prodotto calcolando la complessità ciclomatica (\(CC\)) secondo la seguente formula: \begin{equation} CC = e - n + p \end{equation}. Dove \(e\) è il numero di archi del grafo formato da tutti i possibili esiti del programma, \(n\) è il numero di nodi del grafo e \(p\) è il numero di \glossario{componenti connesse} del grafo.
%           \begin{itemize} \item Valore ammissibile: \leq{}  7 \item Valore ottimale: \leq{}  3 \end{itemize}
%       \end{description}

%   % subs:metriche_funzionale (end)

% % sub:appropriatezza_funzionale (end)

% % sub:sicurezza (end)

% \subsection{Manutenibilità}%
% \label{sub:manutenibilita}
% Questa caratteristica rappresenta il grado di efficacia e efficienza con cui il prodotto può essere migliorato, corretto o adattato a nuovi requisiti.

% \subsubsection{Obiettivi}%
% \label{subs:obiettivi_manutenibilita}
%       \begin{description}
%         \item [Modularità]: Grado con cui il prodotto è composto da parti indipendenti in modo che una modifica a una componente abbia un impatto minimo su tutte le altre.
%         \item [Riusabilità]: Grado con cui una risorsa può essere utilizzata in più di un sistema.
%         \item [Analizzabilità]: Grado di efficacia e di efficienza con cui è possibile identificare le cause di errori nel prodotto e le parti da modificare.
%         \item [Modificabilità]: Grado con cui il prodotto può essere efficacemente ed efficientemente modificato senza introdurre regressioni.
%         \item [Testabilità]: Grado di efficacia e di efficienza con cui possono essere stabiliti i criteri di test e con cui quei test possono essere eseguiti per verificare quei criteri.
%       \end{description}
% % subs:obiettivi_manutenibilita (end)

% \subsubsection{Metriche}%
% \label{subs:metriche_manutenibilita}
%   \begin{description}
%     \item [[MPR-SOF \textasciitilde] Semplicità delle funzioni]: Possiamo monitorare la modularità, analizzabilità e testabilità del prodotto valutando il numero di parametri per ogni funzione, meno saranno più il suo compito sarà specifico.
%       \begin{itemize}   \item Valore ammissibile: \leq{}  6 \item Valore ottimale: \leq{}  3 \end{itemize}
%     \item [[MPR-SOC \textasciitilde] Semplicità delle classi]: Possiamo monitorare la modularità, analizzabilità e testabilità del prodotto valutando il numero di metodi per ogni classe, meno saranno più lo scopo di quella classe sarà specifico.
%       \begin{itemize} \item Valore ammissibile: \leq{}  10 \item Valore ottimale: \leq{}  5 \end{itemize}
%     \item [[MPR-CC \textasciitilde] Complessità ciclomatica]: Monitoriamo la analizzabilità, modificabilità e testabilità del prodotto calcolando la complessità ciclomatica (\(CC\)) secondo la seguente formula: \begin{equation} CC = e - n + p \end{equation}. Dove \(e\) è il numero di archi del grafo formato da tutti i possibili esiti del programma, \(n\) è il numero di nodi del grafo e \(p\) è il numero di \glossario{componenti connesse} del grafo.
%       \begin{itemize} \item Valore ammissibile: \leq{}  7 \item Valore ottimale: \leq{}  3 \end{itemize}
%   \end{description}
% % subs:metriche_manutenibilita (end)

% % sub:manutenibilita (end)

\end{document}
