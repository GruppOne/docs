\documentclass[../piano-di-qualifica.tex]{subfiles}
\appendToGraphicspath{../../../commons/img/}

\begin{document}
\subsection{Scopo del documento}%
\label{sub:scopo_del_documento}
Questo documento ha lo scopo di mostrare le strategie di verifica e validazione adottate al fine di garantire la qualità di prodotto e di processo.
Per raggiungere questo obiettivo viene applicato un sistema di verifica continua sui processi in corso e sulle attività svolte.
In questo modo è quindi possibile rilevare e correggere il prima possibile eventuali anomalie, riducendo al minimo lo spreco delle risorse.

% sub:scopo_del_documento (end)

\scopoDelProdottoEGlossario{}

\subsection{Riferimenti}%
\label{sub:riferimenti}
\subsubsection{Normativi}%
\label{par:normativi}
\begin{itemize}
  \item \textit{Norme di progetto (versione \versione)}.
  \item Capitolato d'appalto C5: \href{https://www.math.unipd.it/~tullio/IS-1/2019/Progetto/C5.pdf}{https://www.math.unipd.it/\textasciitilde tullio/IS-1/2019/Progetto/C5.pdf}.
\end{itemize}
% subs:normativi (end)
\subsubsection{Informativi}%
\label{par:informativi}
\begin{itemize}
  \item Seminario di presentazione del capitolato C5: \href{https://www.math.unipd.it/~tullio/IS-1/2019/Dispense/C5a.pdf}{https://www.math.unipd.it/\textasciitilde tullio/IS-1/2019/Dispense/C5a.pdf}.
  \item ISO/IEC 25010:2011: \href{https://iso25000.com/index.php/en/iso-25000-standards/iso-25010}{https://iso25000.com/index.php/en/iso-25000-standards/iso-25010}.
  \item ISO/IEC 12207:1995, sezioni sul processo di accertamento della qualità, verifica e validazione: \linebreak\href{https://www.math.unipd.it/~tullio/IS-1/2009/Approfondimenti/ISO_12207-1995.pdf}{https://www.math.unipd.it/\textasciitilde tullio/IS-1/2009/Approfondimenti/ISO\_12207-1995.pdf}.
\end{itemize}
% subs:informativi
% sub:riferimenti (end)

\end{document}
