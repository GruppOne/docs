\documentclass[../piano-di-qualifica.tex]{subfiles}
\appendToGraphicspath{../../../commons/img/}

\begin{document}
\subsection{Scopo del documento}%
\label{sub:scopo_del_documento}
Questo documento ha lo scopo di mostrare le strategie di verifica e validazione adottate al fine di garantire la qualità di prodotto e di processo.
Per raggiungere questo obiettivo viene applicato un sistema di verifica continua sui processi in corso e sulle attività svolte.
In questo modo è quindi possibile rilevare e correggere il prima possibile eventuali anomalie, riducendo al minimo lo spreco delle risorse.

% sub:scopo_del_documento (end)
\subsection{Scopo del prodotto}%
\label{sub:scopo_del_prodotto}
L'obiettivo del \glossario{progetto} è sviluppare un'\glossario{applicazione mobile} distribuita, seguendo il \glossario{modello server/client}.
Il \glossario{client} deve essere in grado di segnalare sia l'ingresso che l'uscita dell'\glossario{utente} dalle \glossario{aree d'interesse} (in modalità anonima o meno a seconda delle esigenze), le quali sono definite dalle \glossario{organizzazioni}.
Il \glossario{server} deve fornire la possibilità di raccogliere ed analizzare i \glossario{dati} relativi alle organizzazioni.
In caso di \glossario{utenti anonimi} l'analisi riguarda solo una \glossario{stima} del numero totale di persone presenti in un dato momento.
In caso di \glossario{utenti autenticati} deve inoltre essere possibile effettuare \glossario{query} di monitoraggio specifiche.

\subparagraph*{Errore sulla geolocalizzazione}%
\label{subp:errore_sulla_geolocalizzazione}
È richiesto un \glossario{report} che esponga le scelte progettuali, le rispettive motivazioni e i test eseguiti per garantire la rilevazione sufficiente precisa della posizione, considerando le limitazioni dello \glossario{smartphone}.
% subs:errore_sulla_geolocalizzazione (end)
% sub:scopo_del_prodotto (end)
\subsection{Glossario}%
\label{sub:glossario}
Al fine di rendere il documento più chiaro possibile, i termini che possono assumere un significato ambiguo sono evidenziati (i.e., \glossario{client}) e riportati in \textit{Glossario1.0.0.pdf} accompagnati da una definizione.
% sub:glossario (end)
\subsection{Riferimenti}%
\label{sub:riferimenti}
\subsubsection{Normativi}%
\label{par:normativi}
\begin{itemize}
  \item \textit{Norme di progetto}
  \item \href{https://www.math.unipd.it/~tullio/IS-1/2019/Progetto/C5.pdf}{Capitolato d'appalto C5}
\end{itemize}
% subs:normativi (end)
\subsubsection{Informativi}%
\label{par:informativi}
\begin{itemize}
  \item \href{https://www.math.unipd.it/~tullio/IS-1/2019/Dispense/C5a.pdf}{Seminario di presentazione del capitolato C5}
  \item \href{https://iso25000.com/index.php/en/iso-25000-standards/iso-25010}{ISO 25010}
  \item \href{https://www.math.unipd.it/~tullio/IS-1/2009/Approfondimenti/ISO_12207-1995.pdf}{ISO 12207-1995}
\end{itemize}
% subs:informativi
% sub:riferimenti (end)

\end{document}
